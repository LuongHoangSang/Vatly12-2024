\section*{ÔN TẬP CHƯƠNG III (2)}
%%==========Bài 1
\begin{bt}%[9D1B10]
	Rút gọn các biểu thức sau 
	\begin{listEX}[2]
	\item $A=\sqrt{9 + 4\sqrt{5}} - \sqrt{9 - 4\sqrt{5}}$;
	\item $B=\sqrt{x^2 - 10x + 25} + x\text{ với }x<0$.
	\end{listEX}
	\loigiai
	{
	\begin{listEX}[2]
	\item $A=4$;
	\item $B=5$.
	\end{listEX}
	}
\end{bt}
%%==========Bài 2
\begin{bt}%[9D1B10]
	Tính
	\begin{listEX}[2]
	\item $(\sqrt{8} + \sqrt{18} + \sqrt{5})(\sqrt{50} - \sqrt{5})$;
	\item $\dfrac{\sqrt{3} - 1}{2} + \dfrac{\sqrt{3} + 2}{3} - \dfrac{2\sqrt{3} - 1}{4}$.
	\end{listEX}
	\loigiai
	{
	\begin{listEX}[2]
	\item $45$;
	\item $\dfrac{4\sqrt{3} + 5}{12}$.
	\end{listEX}
	}
\end{bt}
%%==========Bài 3
\begin{bt}%[9D1B10]
	Rút gọn các biểu thức sau
\begin{listEX}[2]
	\item $\sqrt{\dfrac{9}{16}:\dfrac{25}{36}} - \sqrt{\dfrac{49}{8}}:\sqrt{3\dfrac{1}{8}}$;
	\item $\sqrt{45{,}8^2 - 44{,}2^2} - \sqrt{6\left[(\sqrt{2} + 1)^2 + (\sqrt{2} - 1)^2\right]}$.
\end{listEX}
	\loigiai
	{
\begin{listEX}
	\item Ta có $\sqrt{\dfrac{9}{16}:\dfrac{25}{36}} - \sqrt{\dfrac{49}{8}}:\sqrt{3\dfrac{1}{8}}=\sqrt{\dfrac{9}{16}:\dfrac{25}{36}} - \sqrt{\dfrac{49}{8}:\dfrac{25}{8}}=\dfrac{3}{4}:\dfrac{5}{6} - \sqrt{\dfrac{49}{25}}=\dfrac{9}{10} - \dfrac{7}{5}= - \dfrac{1}{2}$.
	\item $\sqrt{45{,}8^2 - 44{,}2^2} - \sqrt{6\left[(\sqrt{2} + 1)^2 + (\sqrt{2} - 1)^2\right]} =\sqrt{1{,}6\cdot90} - \sqrt{6\cdot6}=4\cdot3 - 6=6$
\end{listEX}
	}
\end{bt}
%%==========Bài 4
\begin{bt}%[9D1B10]
	Rút gọn các biểu thức sau
	\begin{listEX}[2]
	\item $\dfrac{1}{34}\sqrt{\dfrac{165^2 - 124^2}{164}} + 4\sqrt{\dfrac{32}{176^2 - 112^2}}$;
	\item $\dfrac{5(\sqrt{6} - 1)}{\sqrt{6} + 1} + \dfrac{\sqrt{2} - \sqrt{3}}{\sqrt{2} + \sqrt{3}}$.
	\end{listEX}
	\loigiai
	{
\begin{listEX}[2]
\item \allowdisplaybreaks $\begin{aligned}[t]
	&\dfrac{1}{34}\sqrt{\dfrac{165^2 - 124^2}{164}} + 4\sqrt{\dfrac{32}{176^2 - 112^2}}\\
	=&\dfrac{1}{34}\sqrt{\dfrac{41 \cdot 289}{164}} + 4\sqrt{\dfrac{32}{64 \cdot 288}}\\
	=&\dfrac{1}{34}\sqrt{\dfrac{289}{4}} + 4\sqrt{\dfrac{1}{576}}\\
	=&\dfrac{1}{34}\cdot\dfrac{17}{2} + 4\cdot\dfrac{1}{24}=\dfrac{1}{4} + \dfrac{1}{6}=\dfrac{5}{12}.
 \end{aligned}$
	\item \allowdisplaybreaks $ 
	\begin{aligned}[t]
	&\dfrac{5(\sqrt{6} - 1)}{\sqrt{6} + 1} + \dfrac{\sqrt{2} - \sqrt{3}}{\sqrt{2} + \sqrt{3}}\\
	=&\dfrac{5(\sqrt{6} - 1)^2}{6 - 1} + \dfrac{(\sqrt{2} - \sqrt{3})^2}{2 - 3}\\
	=&7 - 2\sqrt{6} - 5 + 2\sqrt{6}=2.
 	\end{aligned}$
\end{listEX}
	}
\end{bt}
%%==========Bài 5
\begin{bt}%[9D1K10]
	Rút gọn biểu thức
	\begin{listEX}[2]
	\item $\dfrac{3 + 3\sqrt{5} - \sqrt{2} - \sqrt{10}}{6 + 2\sqrt{5}}$; 
	\item $\dfrac{x\sqrt{x} + x\sqrt{y} - y\sqrt{x} - y\sqrt{y}}{x - y + y\sqrt{x} - y\sqrt{y}}$.
	\end{listEX}
	\loigiai
	{
\begin{listEX}
	\item Ta có $\dfrac{3 + 3\sqrt{5} - \sqrt{2} - \sqrt{10}}{6 + 2\sqrt{5}}=\dfrac{3(1 + \sqrt{5}) - \sqrt{2}(1 + \sqrt{5})}{(1 + \sqrt{5})^2}=\dfrac{(1 + \sqrt{5})(3 - \sqrt{2})}{(1 + \sqrt{5})^2}=\dfrac{3 - \sqrt{2}}{1 + \sqrt{5}}$.
	\item 
	\allowdisplaybreaks $\begin{aligned}[t]
	&\dfrac{x\sqrt{x} + x\sqrt{y} - y\sqrt{x} - y\sqrt{y}}{x - y + y\sqrt{x} - y\sqrt{y}}=\dfrac{x(\sqrt{x} + \sqrt{y}) - y(\sqrt{x} + \sqrt{y})}{(\sqrt{x} - \sqrt{y})(\sqrt{x} + \sqrt{y}) + y(\sqrt{x} - \sqrt{y})}\\
	=&\dfrac{(\sqrt{x} + \sqrt{y})(x - y)}{(\sqrt{x} - \sqrt{y})(\sqrt{x} + \sqrt{y} + y)}=\dfrac{(\sqrt{x} + \sqrt{y})^2(\sqrt{x} - \sqrt{y})}{(\sqrt{x} - \sqrt{y})(\sqrt{x} + \sqrt{y} + y)}=\dfrac{(\sqrt{x} + \sqrt{y})^2}{\sqrt{x} + \sqrt{y} + y}.
	\end{aligned}$\\
	Điều kiện $x\geq 0 ; y\geq 0 ; x\neq y$.
\end{listEX}
	}
\end{bt}
%%==========Bài 6
\begin{bt}%[9D1K10]
Rút gọn biểu thức $P=\dfrac{2\sqrt{x} - 9}{x - 5\sqrt{x} + 6} - \dfrac{\sqrt{x} + 3}{\sqrt{x} - 2} - \dfrac{2\sqrt{x} + 1}{3 - \sqrt{x}}$.
	\loigiai
	{
	Điều kiện: $x\geq 0 ; x\neq 4 ; x\neq 9$. Khi đó ta có 
	\allowdisplaybreaks 
	\begin{eqnarray*}
	P&=&\dfrac{2\sqrt{x} - 9}{(\sqrt{x} - 2)(\sqrt{x} - 3)} - \dfrac{\sqrt{x} + 3}{\sqrt{x} - 2} - \dfrac{2\sqrt{x} + 1}{3 - \sqrt{x}}\\
	&=&\dfrac{2\sqrt{x} - 9 - (\sqrt{x} + 3)(\sqrt{x} - 3) + (2\sqrt{x} + 1)(\sqrt{x} - 2)}{(\sqrt{x} - 2)(\sqrt{x} - 3)}\\
	&=&\dfrac{2\sqrt{x} - 9 - x + 9 + 2x - 4\sqrt{x} + \sqrt{x} - 2}{(\sqrt{x} - 2)(\sqrt{x} - 3)}\\
	&=&\dfrac{x - \sqrt{x} - 2}{(\sqrt{x} - 2)(\sqrt{x} - 3)}=\dfrac{(\sqrt{x} + 1)(\sqrt{x} - 2)}{(\sqrt{x} - 2)(\sqrt{x} - 3)}=\dfrac{\sqrt{x} + 1}{\sqrt{x} - 3}.
	\end{eqnarray*}
	}
\end{bt}
%%==========Bài 7
\begin{bt}%[9D1K10]
	Cho biểu thức $P=\dfrac{2\sqrt{x}}{\sqrt{x} + 3} + \dfrac{\sqrt{x} + 1}{\sqrt{x} - 3} + \dfrac{3 - 11\sqrt{x}}{9 - x}$. 
	\begin{listEX}[2]
	\item Rút gọn $P$.
	\item Tính giá trị của $P$ với $x=\dfrac{7 + 4\sqrt{3}}{4}$.
	\end{listEX}
	\loigiai
	{
	\begin{listEX}
	\item Điều kiện: $x\geq 0 ; x\neq 9$. Khi đó ta có
	\allowdisplaybreaks 
	\begin{eqnarray*}
	P&=&\dfrac{2\sqrt{x}(\sqrt{x} - 3) + (\sqrt{x} + 1)(\sqrt{x} + 3) - 3 + 11\sqrt{x}}{(\sqrt{x} + 3)(\sqrt{x} - 3)}\\
	&=&\dfrac{2{x} - 6\sqrt{{x}} + {x} + 4\sqrt{{x}} + 3 - 3 + 11\sqrt{{x}}}{(\sqrt{{x}} + 3)(\sqrt{{x}} - 3)}\\
	&=&\dfrac{3{x} + 9\sqrt{{x}}}{(\sqrt{{x}} + 3)(\sqrt{{x}} - 3)}=\dfrac{3\sqrt{{x}}(\sqrt{{x}} + 3)}{(\sqrt{{x}} + 3)(\sqrt{{x}} - 3)}=\dfrac{3\sqrt{{x}}}{\sqrt{{x}} - 3}
	\end{eqnarray*}
	\item Ta có $x=\dfrac{7 + 4\sqrt{3}}{4}=\left(\dfrac{2 + \sqrt{3}}{2}\right)^2\Rightarrow\sqrt{x}=\dfrac{2 + \sqrt{3}}{2}$.\\ 
	Do đó
	\allowdisplaybreaks
	\begin{eqnarray*}
	P&=&\dfrac{3\cdot\dfrac{2 + \sqrt{3}}{2}}{\dfrac{2 + \sqrt{3}}{2} - 3}=\dfrac{6 + 3\sqrt{3}}{2}\cdot\dfrac{2}{\sqrt{3} - 4}\\
	&=&\dfrac{6 + 3\sqrt{3}}{\sqrt{3} - 4}=\dfrac{(6 + 3\sqrt{3})(\sqrt{3} + 4)}{(\sqrt{3} - 4)(\sqrt{3} + 4)}\\
	&=&\dfrac{6\sqrt{3} + 24 + 9 + 12\sqrt{3}}{3 - 16}=\dfrac{ - (33 + 18\sqrt{3})}{13}.
	\end{eqnarray*}
	\end{listEX}
	}
\end{bt}
%%==========Bài 8
\begin{bt}%[9D1K10]
	Cho biểu thức $P=\left(\dfrac{1}{\sqrt{x} + 3} + \dfrac{5}{\sqrt{x} - 3} - \dfrac{6}{9 - x}\right) :\dfrac{6}{\sqrt{x} + 2}$.
	\begin{listEX}
	\item Rút gọn $P$.
	\item Tính các giá trị nguyên của $x$ để $P$ có giá trị nguyên.
	\end{listEX}
	\loigiai
	{
	\begin{listEX}
	\item Điều kiện $x\geq 0 ; x\neq 9$. Khi đó ta có 
	\allowdisplaybreaks 
	\begin{align*}
	P&=\dfrac{(\sqrt{x} - 3) + 5(\sqrt{x} + 3) + 6}{(\sqrt{x} + 3)(\sqrt{x} - 3)}\cdot\dfrac{\sqrt{x} + 2}{6}\\
	&=\dfrac{\sqrt{x} - 3 + 5\sqrt{x} + 15 + 6}{(\sqrt{x} + 3)(\sqrt{x} - 3)}\cdot\dfrac{\sqrt{x} + 2}{6}\\
	&=\dfrac{6\sqrt{x} + 18}{(\sqrt{x} + 3)(\sqrt{x} - 3)}\cdot\dfrac{\sqrt{x} + 2}{6}\\
	&=\dfrac{6(\sqrt{x} + 3)}{(\sqrt{x} + 3)(\sqrt{x} - 3)}\cdot\dfrac{\sqrt{x} + 2}{6}\\
	&=\dfrac{\sqrt{x} + 2}{\sqrt{x} - 3}.
	\end{align*}
	\item Ta có $P=\dfrac{\sqrt{x} + 2}{\sqrt{x} - 3}=\dfrac{\sqrt{x} - 3 + 5}{\sqrt{x} - 3}=1 + \dfrac{5}{\sqrt{x} - 3}$. \\ 
$\begin{aligned}
P \text{ có giá trị nguyên }	&{\Rightarrow \dfrac{5}{\sqrt{x} - 3}\text{ có giá trị nguyên}}\\
&{\Rightarrow \sqrt{x} - 3\in \text{Ư}(5)}\\
&{\Rightarrow \sqrt{x} - 3\in\{\pm 1 ;\pm 5\}}.
\end{aligned}$\\
Ta có bảng sau
\begin{center}
\begin{tabular}{|c|c|c|c|c|}
	\hline 
	$\sqrt{x}-3$& $1$ & $-1$ &$5$& $-5$ \\ 
	\hline 
$\sqrt{x}$	& $4$ & $2$ & $8$ & $-2$ \\ 
	\hline 
$x$	& $16$ & $4$ & $64$ & $||$ \\ 
	\hline 
\end{tabular} 
\end{center}
Vậy khi $x\in\{4;16;64\}$ thì $P$ có giá trị nguyên.
\end{listEX}
	}
\end{bt}

%%==========Bài 9
\begin{bt}%[9D1K10]
	Cho biểu thức $P=\dfrac{2}{\sqrt{x y}}:\left(\dfrac{1}{\sqrt{x}} - \dfrac{1}{\sqrt{y}}\right)^2 - \dfrac{x + y}{x - 2\sqrt{x y} + y}$.\\ 
	Chứng minh rằng với mọi giá trị của $x$ và $y$ làm cho biểu thức $P$ có nghĩa thì giá trị của $P$ không phụ thuộc vào $x$ và $y$. 
	\loigiai
	{
	Điều kiện $x, y>0 ; x\ne y$. Khi đó ta có
	\allowdisplaybreaks 
	\begin{eqnarray*}
	P&=&\dfrac{2}{\sqrt{x y}}\colon \left(\dfrac{\sqrt{y}-\sqrt{x}}{\sqrt{x y}}\right)^2-\dfrac{x+y}{(\sqrt{x}-\sqrt{y})^2}\\
	&=&\dfrac{2}{\sqrt{x y}}\cdot\dfrac{(\sqrt{x y})^2}{(\sqrt{x}-\sqrt{y})^2}-\dfrac{x+y}{(\sqrt{x}-\sqrt{y})^2}\\
	&=&\dfrac{2\sqrt{x y}}{(\sqrt{x}-\sqrt{y})^2}-\dfrac{x+y}{(\sqrt{x}-\sqrt{y})^2}\\
	&=&\dfrac{-(x-2\sqrt{x y}+y)}{(\sqrt{x}-\sqrt{y})^2}=-\dfrac{(\sqrt{x}-\sqrt{y})^2}{(\sqrt{x}-\sqrt{y})^2}=-1.
	\end{eqnarray*}
	Vậy giá trị của $P$ không phụ thuộc vào $x$ và $y$.
	}
\end{bt}
%%==========Bài 10
\begin{bt}%[9D1B10]
	Cho biểu thức $P=\left(\dfrac{x + 3}{x - 9} + \dfrac{1}{\sqrt{x} + 3}\right) :\dfrac{\sqrt{x}}{\sqrt{x} - 3}$.
	\begin{listEX}[2]
	\item Rút gọn $P$.
	\item Chứng minh rằng $P>\dfrac{1}{3}$.
	\end{listEX}
	\loigiai
	{
\begin{listEX}
	\item Điều kiện $x>0 ; x\neq 9$. Khi đó ta có
	\allowdisplaybreaks 
	\begin{eqnarray*}
	P&=&\dfrac{x+3+\sqrt{x}-3}{(\sqrt{x}-3)(\sqrt{x}+3)} \cdot \dfrac{\sqrt{x}-3}{\sqrt{x}}\\
	&=&\dfrac{\sqrt{x}(\sqrt{x}+1)}{(\sqrt{x}-3)(\sqrt{x}+3)} \cdot \dfrac{\sqrt{x}-3}{\sqrt{x}}\\
	&=&\dfrac{\sqrt{x}+1}{\sqrt{x}+3}.
	\end{eqnarray*}
	\item Xét hiệu $P - \dfrac{1}{3}=\dfrac{\sqrt{x} + 1}{\sqrt{x} + 3} - \dfrac{1}{3} =\dfrac{3\sqrt{x} + 3 - \sqrt{x} - 3}{3(\sqrt{x} + 3)}=\dfrac{2\sqrt{x}}{3(\sqrt{x} + 3)}>0~(\text{vì }x>0)$.\\ Do đó $P>\dfrac{1}{3}$.
\end{listEX}
	}
\end{bt}
%%==========Bài 11
\begin{bt}%[9D1B10]
	Cho biểu thức $P=\left(\dfrac{1}{\sqrt{x} + 1} - \dfrac{x + 2}{x\sqrt{x} + 1}\right) :\dfrac{2}{\sqrt{x}}$
	\begin{listEX}
	\item Rút gọn $P.$
	\item Chứng minh rằng biểu thức $P$ luôn luôn âm với mọi giá trị của $x$ làm $P$ xác định.
	\end{listEX}
	\loigiai
	{
\begin{listEX}
	\item Điều kiện $x>0$. Khi đó ta có
	\allowdisplaybreaks 
	\begin{eqnarray*}
	P&=&\dfrac{(x-\sqrt{x}+1)-(x+2)}{(\sqrt{x}+1)(x-\sqrt{x}+1)} \colon \dfrac{2}{\sqrt{x}}\\
	&=&\dfrac{-(\sqrt{x}+1)}{(\sqrt{x}+1)(x-\sqrt{x}+1)} \cdot \dfrac{\sqrt{x}}{2}\\
	&=&\dfrac{-\sqrt{x}}{2(x-\sqrt{x}+1)}.
	\end{eqnarray*}
	\item Ta có $x>0$ nên $-\sqrt{x}<0 \Rightarrow x - \sqrt{x} + 1=\left(\sqrt{x} - \dfrac{1}{2}\right)^2 + \dfrac{3}{4}>0$.\\ Do đó $P<0$ với mọi $x>0$.
	\end{listEX}
}
\end{bt}
%%==========Bài 12
\begin{bt}%[9D1B10]
	Cho biểu thức $P=\dfrac{1}{\sqrt{x - 1} + \sqrt{x}} + \dfrac{1}{\sqrt{x - 1} - \sqrt{x}} + \dfrac{x\sqrt{x} + x}{\sqrt{x} + 1}$.
	\begin{listEX}
	\item Rút gọn $P$.
	\item Chứng minh rằng biểu thức $P$ luôn luôn không âm với mọi giá trị của $x$ làm $P$ xác định. 
	\end{listEX}
	\loigiai
	{
\begin{listEX}
	\item Điều kiện $x\ge 1$. Khi đó ta có 
	\begin{eqnarray*}
	P&=&\dfrac{\sqrt{x-1}-\sqrt{x}+\sqrt{x-1}+\sqrt{x}}{(\sqrt{x-1}+\sqrt{x})(\sqrt{x-1}-\sqrt{x})}+\dfrac{x(\sqrt{x}+1)}{\sqrt{x}+1}\\
	&=&\dfrac{2 \sqrt{x-1}}{-1}+x\\
	&=&x-2 \sqrt{x-1}.
	\end{eqnarray*}
	\item Ta có $P=x-2 \sqrt{x-1}=(x-1)-2 \sqrt{x-1}+1=\left(\sqrt{x-1}-1\right)^2 \geq 0$.\\
	Vậy $P$ luôn luôn không âm với mọi $x \ge 1$.
\end{listEX}
	}
\end{bt}
%%==========Bài 13
\begin{bt}%[9D1B1]
	Cho biểu thức $P=\dfrac{\sqrt{x}}{x + \sqrt{x}}:\left(\dfrac{1}{\sqrt{x}} + \dfrac{\sqrt{x}}{\sqrt{x} + 1}\right)$.
	\begin{listEX}[2]
	\item Rút gọn $P$.
	\item Tìm giá trị lớn nhất của $P$.
	\end{listEX}
	\loigiai
	{
	\begin{listEX}
	\item Điều kiện $x>0$. Khi đó ta có
	\allowdisplaybreaks 
	\begin{eqnarray*}
	P&=&\dfrac{\sqrt{x}}{\sqrt{x}(\sqrt{x}+1)} \colon \dfrac{\sqrt{x}+1+x}{\sqrt{x}(\sqrt{x}+1)}\\
	&=&\dfrac{\sqrt{x}}{\sqrt{x}(\sqrt{x}+1)} \cdot \dfrac{\sqrt{x}(\sqrt{x}+1)}{x+\sqrt{x}+1}\\
	&=&\dfrac{\sqrt{x}}{x+\sqrt{x}+1}.
	\end{eqnarray*}
	\item Ta có $P=\dfrac{\sqrt{x}}{x + \sqrt{x} + 1}=\dfrac{1}{\sqrt{x} + 1 + \dfrac{1}{\sqrt{x}}}$. \\ Xét biểu thức ở mẫu $\sqrt{x} + \dfrac{1}{\sqrt{x}} + 1\geq 2\sqrt{\sqrt{x}\cdot\dfrac{1}{\sqrt{x}}} + 1=3$.\\ Ta có $P=\dfrac{1}{\sqrt{x} + \dfrac{1}{\sqrt{x}} + 1}\leq\dfrac{1}{3}$. \\ Do đó $\max P=\dfrac{1}{3}$, đạt được khi $\sqrt{x}=\dfrac{1}{\sqrt{x}}\Rightarrow x=1$.
	\end{listEX}
	}
\end{bt}

%%==========Bài 14
\begin{bt}%[9D1K10]
	Cho biểu thức $P=\left(\dfrac{\sqrt{x} + 1}{\sqrt{x}} - \dfrac{2}{\sqrt{x} + 1}\right) :\left(\dfrac{\sqrt{x}}{2} + \dfrac{1}{2\sqrt{x}}\right)$.
\begin{listEX}
	\item Rút gọn $P.$
	\item Tính giá trị của $P$ khi $x=3-2\sqrt{2}$.
	\item Tìm $x$ để $P=1.$
\end{listEX}
\loigiai
{
	\begin{listEX}[3]
	\item $\dfrac{2}{\sqrt{x} + 1}$;
	\item $\sqrt{2}$;
	\item $x=1$.
	\end{listEX}
}
\end{bt}
%%==========Bài 15
\begin{bt}%[9D1K10]
	Cho biểu thức $P=\dfrac{3}{\sqrt{x}} + \left(\dfrac{x}{x - \sqrt{x}} + \dfrac{x + 1}{\sqrt{x}} - \dfrac{1}{\sqrt{x} - 1}\right)\cdot\dfrac{\sqrt{x}}{x + \sqrt{x} + 1}$.
\begin{listEX}
	\item Rút gọn $P$.
	\item Tìm các giá trị của $x$ để $P\ge 10$.
	\item Tìm các giá trị nguyên của 
	$x$ để $P$ có giá trị nguyên.
\end{listEX}
\loigiai
{
	\begin{listEX}[3]
	\item $\dfrac{\sqrt{x} + 3}{\sqrt{x}}$;
	\item $0<x\leq\dfrac{1}{9}$;
	\item $x\in\{1 ; 9\}$.
	\end{listEX}
}
\end{bt}

%%==========Bài 16
\begin{bt}%[9D1B10]
	Tìm $x$ biết 
	\begin{listEX}[2]
	\item $\sqrt{25(3x - 1)^2}=10$;
	\item $\dfrac{\sqrt{x} + 3}{\sqrt{x} - 3}=\dfrac{\sqrt{x} + 5}{\sqrt{x} - 2}$.
	\end{listEX}
	\loigiai
	{
	\begin{listEX}
	\item Ta có $\sqrt{25(3x - 1)^2}=10\Rightarrow 5|3x - 1|=10\Rightarrow |3x - 1|=2\Rightarrow \left[\begin{array}
	{l}{3x - 1=2}\\
	{3x - 1= - 2}
	\end{array}\right.\Rightarrow \left[\begin{array}
	{l}{x=1}\\
	{x= - \dfrac{1}{3}.}
	\end{array}\right.$
	\item $\dfrac{\sqrt{x}+3}{\sqrt{x}-3}=\dfrac{\sqrt{x}+5}{\sqrt{x}-2}$. \quad(*)\\
	Điều kiện: $x\geq 0 ; x\neq 4 ; x\neq 9$. Khi đó ta có
	\allowdisplaybreaks 
	\begin{eqnarray*}
	\text{(*)}&\Rightarrow & \left(\sqrt{x}+3\right)\left(\sqrt{x}-2\right)=\left(\sqrt{x}+5\right)\left(\sqrt{x}-3\right)\\
	&\Rightarrow & x+\sqrt{x}-6=x+2 \sqrt{x}-15\\
	&\Rightarrow & -\sqrt{x}=-9\\
	&\Rightarrow & \sqrt{x} = 9\\
	&\Rightarrow & x = 81 \text{ (thỏa mãn điều kiện).}
	\end{eqnarray*}
	\end{listEX}
	}
\end{bt}
%%==========Bài 17
\begin{bt}%[9D1K10]
	Tìm $x$ biết 
	\begin{listEX}[2]
	\item $5x - \sqrt{(2x - 1)^2}=2$;
	\item $\sqrt{x + 2\sqrt{x - 1}}=x$.
	\end{listEX}
	\loigiai
	{
	\begin{listEX}
	\item $5x - \sqrt{(2x-1)^2} = 2 \Rightarrow 5x - \left|2x-1\right| = 2$.
	\begin{itemize}
	\item Nếu $x \ge \dfrac{1}{2}$ thì
	\allowdisplaybreaks $\begin{aligned}[t]
	(1) &\Rightarrow 5x-(2x-1)=2\\
	&\Rightarrow 3 x=1 \Rightarrow x=\dfrac{1}{3} \text{ (loại).}
	\end{aligned}$
	\item Nếu $x<\dfrac{1}{2}$ thì 
	\allowdisplaybreaks $\begin{aligned}[t]
	(1) &\Rightarrow 5 x+(2 x-1)=2\\
	&\Rightarrow 7 x=3 \Rightarrow x=\dfrac{3}{7} \text{ (thỏa mãn).}
	\end{aligned}$
	\end{itemize}
	\item Điều kiện: $x \ge 1$. Khi đó ta có
	\allowdisplaybreaks 
	\begin{eqnarray*}
	&&\sqrt{x+2 \sqrt{x-1}}=x\\
	&\Rightarrow & \sqrt{x-1+2 \sqrt{x-1}+1}=x\\
	&\Rightarrow & \sqrt{(\sqrt{x-1}+1)^2}=x\\
	&\Rightarrow & |\sqrt{x-1}+1|=x\\
	&\Rightarrow & \sqrt{x-1}+1-x=0\\
	&\Rightarrow & \sqrt{x-1}(1-\sqrt{x-1})=0\\
	&\Rightarrow & \hoac{& \sqrt{x-1}=0 \\ & \sqrt{x-1} = 1} \Rightarrow \hoac{& x=1 \\ & x=2.} \text{ (thỏa mãn điều kiện).}
	\end{eqnarray*}
	\end{listEX}
	}
\end{bt}
%%==========Bài 18
\begin{bt}%[9D1G10]
	Chứng minh rằng $\dfrac{\sqrt{2} - \sqrt{1}}{2 + 1} + \dfrac{\sqrt{3} - \sqrt{2}}{3 + 2} + \cdots + \dfrac{\sqrt{100} - \sqrt{99}}{100 + 99}<\dfrac{9}{20}.$
	\loigiai
	{
	Xét dạng tổng quát của các số hạng $\dfrac{\sqrt{n + 1} - \sqrt{n}}{(n + 1) + n}$ trong đó $n \in \mathbb{N}^*$.\\
	Dễ thấy $\sqrt{n + 1} - \sqrt{n}>0 ;\, (n + 1) + n>0$.\\ Do đó 
	\[\dfrac{\sqrt{n + 1} - \sqrt{n}}{(n + 1) + n}=\dfrac{\sqrt{n + 1} - \sqrt{n}}{\sqrt{(2n + 1)^2}}=\dfrac{\sqrt{n + 1} - \sqrt{n}}{\sqrt{4n^2 + 4n + 1}}<\dfrac{\sqrt{n + 1} - \sqrt{n}}{\sqrt{4n^2 + 4n}}=\dfrac{\sqrt{n + 1} - \sqrt{n}}{2\sqrt{n(n + 1)}}=\dfrac{1}{2\sqrt{n}} - \dfrac{1}{2\sqrt{n + 1}}.\]
	Áp dụng bất đẳng thức này với $n$ lấy từ $1$ đến $99$ ta được
	\allowdisplaybreaks 
	\begin{align*}
	&\ \dfrac{\sqrt{2}-\sqrt{1}}{2+1}+\dfrac{\sqrt{3}-\sqrt{2}}{3+2}+\cdots+\dfrac{\sqrt{100}-\sqrt{99}}{100+99}<\dfrac{1}{2 \sqrt{1}}-\dfrac{1}{2 \sqrt{2}}+\dfrac{1}{2 \sqrt{2}}-\dfrac{1}{2 \sqrt{3}}+\cdots+\dfrac{1}{2 \sqrt{99}}-\dfrac{1}{2 \sqrt{100}}\\
	=&\ \dfrac{1}{2 \sqrt{1}}-\dfrac{1}{2 \sqrt{100}}=\dfrac{1}{2}-\dfrac{1}{20}=\dfrac{9}{20}.
	\end{align*}
	}
\end{bt}