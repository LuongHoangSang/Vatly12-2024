\setcounter{section}{0}
\section{PHƯƠNG TRÌNH QUY VỀ PHƯƠNG~TRÌNH~BẬC~NHẤT~MỘT~ẨN}
\subsection{Trọng tâm kiến thức}
\begin{tomtat}
\subsubsection{Phương trình tích}
\begin{boxdn}
	Muốn giải phương trình $\left(a_1x+b_1\right)\left(a_2x+b_2\right)=0$, ta giải hai phương trình $a_1x+b_1=0$ và $a_2x+b_2=0$, rồi lấy tất cả các nghiệm của chúng.
\end{boxdn}
\begin{note}
	Trong nhiều trường hợp, để giải một phương trình, ta biến đổi để đưa phương trình đó về dạng phương trình tích.
\end{note}
\subsubsection{Phương trình chứa ẩn ở mẫu quy về phương trình bậc nhất}
\begin{boxdn} Đối với phương trình chứa ẩn ở mẫu, điều kiện của ẩn sao cho các phân thức chứa trong phương trình đểu xác định gọi là điều kiên xác định của phương trình.
\end{boxdn}
\begin{note}
\begin{enumerate}[\bf --]
	\item Để tìm điều kiện xác định của phương trình chứa ấn ở mẫu, ta đặt điều kiện của ẩn để tất cả các mẫu thức chứa trong phương trình đều khác 0.
	\item Những giá trị của ẩn không thoả mãn điều kiện xác định thì không thể là nghiệm của phương trình.
\end{enumerate}
\end{note}
\begin{boxdl}
	\subsubsection*{Cách giải phương trình chứa ẩn ở mẫu}
	\begin{enumerate}[\it\bf Bước 1.]
		\item Tìm điều kiện xác định của phương trình.
		\item Quy đồng mẫu hai vế của phương trình rồi khử mẫu.
		\item Giải phương trình vừa tìm được.
		\item Kết luận. Trong các giá trị tìm được của ẩn ở Bước 3, giá trị nào thỏa mãn điều kiện xác định chính là nghiệm của phương trình đã cho.
	\end{enumerate}
\end{boxdl}
\end{tomtat}
%%%%%%%%%%%%%%%%%%%
\subsection{Các dạng bài tập}
%==============
\begin{dang}{Giải phương trình dạng tích}
\end{dang}
\begin{vd}
	Giải các phương trình sau:
	\begin{listEX}[4]
	\item $(2x+1)(3x-1)=0$;
	\item $(3x+1)(2-3x)=0$; 
	\item $(x+5)(3x-9)=0$;
	\item $3x\left(x+7\right)=0$.
	\end{listEX}
	\loigiai{
	\begin{listEX}
	\item
	Ta có $(2x+1)(3x-1)=0$ nên $2x+1=0$ hoặc $3x-1=0$.
	\begin{itemize}
	\item $2x+1=0$ hay $2x=-1$, suy ra $x=-\dfrac12$.
	\item $3x-1=0$ hay $3x=1$, suy ra $x=\dfrac13$.
	\end{itemize}
	Vậy phương trình đã cho có hai nghiệm là $x=-\dfrac12$ và $x=\dfrac13$.
	\item 
	Ta có $(3x+1)(2-3x)=0$ nên $3x+1=0$ hoặc $2-3x=0$.
	\begin{itemize}
	\item $3x+1=0$ hay $3x=-1$ suy ra $x=-\dfrac13$
	\item $2-4x=0$ hay $4x=2$ suy ra $x=\dfrac12$
	\end{itemize}
	Vậy phương trình có hai nghiệm $x=-\dfrac13$ và $x=\dfrac12$.
	\item 
	Ta có $(x+5)(3x-9)=0$ nên $x+5=0$ hoặc $3x-9=0$.
	\begin{itemize}
	\item $x+5=0$ suy ra $x=-5$
	\item $3x-9=0$ hay $3x=9$ suy ra $x=3$
	\end{itemize}
	Vậy phương trình đã cho có hai nghiệm là $x=-5$ và $x=3$.
	\item
	Ta có $3x\left(x+7\right)=0$ nên $3x=0$ hoặc $x+7=0$.
	\begin{itemize}
	\item $3x=0$ suy ra $x=0$
	\item $x+7=0$ suy ra $x=-7$
	\end{itemize}
	Vậy phương trình đã cho có hai nghiệm là $x=0$ và $x=-7$.
	\end{listEX}
	}
\end{vd}
\begin{vd}
	Giải các phương trình sau:
	\begin{listEX}[3]
	\item $\left(x-7\right)\left(5x+4\right)=0$;
	\item $\left(2x+9\right)\left(\dfrac{2}{3}x-5\right)=0$;
	\item $\left(x-5\right)\left(2x-4\right)=0$;
	\end{listEX}
	\loigiai{
	\begin{listEX}
	\item Ta có $\left(x-7\right)\left(5x+4\right)=0$ nên $x-7=0$ hoặc $5x+4=0$.
	\begin{itemize}
		\item $x-7=0$, suy ra $x=7$.
		\item $5x+4=0$ hay $5x=-4$, suy ra $x=-\dfrac{4}{5}$
	\end{itemize}
	Vậy phương trình đã cho có hai nghiệm là $x=7$ và $x=-\dfrac{4}{5}$.
	\item Ta có $\left(2x+9\right)\left(\dfrac{2}{3}x-5\right)=0$ nên $2x+9=0$ hoặc $\dfrac{2}{3}x-5=0$.
	\begin{itemize}
		\item $2x+9=0$ hay $2x=-9$, suy ra $x=-\dfrac{9}{2}$.
		\item $\dfrac{2}{3}x-5=0$ hay $\dfrac{2}{3}x=5$,suy ra $x=\dfrac{15}{2}$.
	\end{itemize}
	Vậy phương trình đã cho có hai nghiệm là $x=-\dfrac{9}{2}$ và $x=\dfrac{15}{2}$.
	\item Ta có $\left(x-5\right)\left(2x-4\right)=0$ nên $x-5=0$ hoặc $2x-4=0$.
	\begin{itemize}
		\item $x-5=0$, suy ra $x=5$.
		\item $2x-4=0$, hay $2x=4$, suy ra $x=2$.
	\end{itemize}
	Vậy phương trình đã cho có hai nghiệm là $x=5$ và $x=2$.
	\end{listEX}
	}
\end{vd}
\begin{vd}%[8D3B4]
	Giải các phương trình sau:
	\begin{listEX}[2]
	\item $(4x - 5)\left( \dfrac{6x - 1}{3} + 1 \right) = 0$; 
	\item $\left( \dfrac{2+x}{4} - \dfrac{x}{5}\right) \left(\dfrac{3x+5}{6} - \dfrac{13x - 1}{9}\right) = 0. $ 
	\end{listEX}
	\loigiai
	{
	\begin{listEX}
	\item 
	Ta có 
	$(4x - 5)\left( \dfrac{6x - 1}{3} + 1 \right) = 0$ nên $4x - 5=0$ hoặc $\dfrac{6x - 1}{3} + 1=0$.
	\begin{itemize}
		\item $4x - 5 = 0$ hay $4x = 5$, suy ra $x = \dfrac{5}{4}$.
		\item $\dfrac{6x - 1}{3} + 1= 0$  hay $6x + 2 = 0$, suy ra $x = -\dfrac{1}{3}$.
	\end{itemize}
	Vậy phương trình đã cho có hai nghiệm $x=\dfrac{5}{4}$ và $x=-\dfrac{1}{3}$.
	\item 
	Ta có
	$\left( \dfrac{2+x}{4} - \dfrac{x}{5}\right) \left(\dfrac{3x+5}{6} - \dfrac{13x - 1}{9}\right) = 0$ nên
	 $\dfrac{2+x}{4} - \dfrac{x}{5} = 0$ hoặc $\dfrac{3x + 5}{6} - \dfrac{13x - 1}{9} = 0$
 	\begin{itemize}
 		\item $\dfrac{2+x}{4} - \dfrac{x}{5} = 0$ hay $5(x + 2) - 4x = 0$ hay 
 		$x + 10 = 0$, suy ra $x = -10$.
 		\item $\dfrac{3x + 5}{6} - \dfrac{13x - 1}{9} = 0$ hay $9(3x + 5) - 6(13x - 1) = 0$
 	hay $-51 x + 51 = 0$, suy ra $x = 1$.
 	\end{itemize}
	Vậy phương trình đã cho có hai nghiệm $x=-10$ và $x=1$.	
	\end{listEX}
	}
\end{vd}
\begin{vd}
	\immini{Độ cao $h$ (mét) của một quả bóng gôn sau khi được đánh $t$ giây được cho bởi công thức $h=t\left(20-5t\right)$. Có thể tính được thời gian bay của quả bóng từ khi được đánh đến khi chạm đất không?}{\twemoji[scale=2]{1f3cc}}
	\loigiai{
	Quả bóng chạm đất khi $h(t)=0$, do đó ta giải phương trình: $t\left(20-t\right)=0$.
	\allowdisplaybreaks
	\begin{eqnarray*}
	&& t\left(20-t\right)=0 \\
	&& t=0 \text{ hoặc } 20-t=0	 \\
	&& t=0 \text{ hoặc } t=20.
	\end{eqnarray*}
	Vậy thời gian bay của quả bóng từ khi được đánh đến khi chạm đất là $20-0=20$ giây.
	}
\end{vd}
%==============
\begin{dang}{Giải phương trình đưa về dạng phương trình tích}
\end{dang}
\begin{vd}
	Giải các phương trình sau bằng cách đưa về phương trình tích:
	\begin{listEX}[2]
	\item $x^2+7x=0$;
	\item $\left(3x+2\right)^2-4x^2=0$;
	\item $2x\left(x+6\right)+5\left(x+6\right)=0$;
	\item $x\left(3x+5\right)-6x-10=0$;
	\end{listEX}
	\loigiai{
	\begin{listEX}
	\item 
	Biến đổi phương trình đã cho về phương trình tích như sau:
	\allowdisplaybreaks
	\begin{eqnarray*}
	&& x^2+7x=0 \\
	&& x\left(x+7\right)=0 \\
	&& x=0 \text{ hoặc } x+7=0 \\
	&& x=0 \text{ hoặc } x=-7.
	\end{eqnarray*}
	Vậy phương trình đã cho có hai nghiệm là $x=0$ và $x=-7$.
	\item 
	Biến đổi phương trình đã cho về phương trình tích như sau:
	\allowdisplaybreaks
	\begin{eqnarray*}
	&& \left(3x+2\right)^2-4x^2=0 \\
	&& \left(3x+2+2x\right)\left(3x+2-2x\right)=0 \\
	&& \left(5x+2\right)\left(x+2\right)=0 \\
	&& 5x+2=0 \text{ hoặc } x+2=0 \\
	&& x=-\dfrac{2}{5} \text{ hoặc } x=-2.
	\end{eqnarray*}
	Vậy phương trình đã cho có hai nghiệm là $x=-\dfrac{2}{5}$ và $x=-2$. 
	\item 
	Biến đổi phương trình đã cho về phương trình tích như sau:
	\allowdisplaybreaks
	\begin{eqnarray*}
	&& 2x\left(x+6\right)+5\left(x+6\right)=0\\
	&& \left(2x+5\right)\left(x+6\right)=0 \\
	&& 2x+5=0 \text{ hoặc } x+6=0	 \\
	&& x=-\dfrac{5}{2} \text{ hoặc } x=-6.
	\end{eqnarray*}	Vậy phương trình đã cho có hai nghiệm là $x=-\dfrac{5}{2}$ và $x=-6$.
	\item 
	Biến đổi phương trình đã cho về phương trình tích như sau:
	\allowdisplaybreaks
	\begin{eqnarray*}
	&& x\left(3x+5\right)-6x-10=0 \\
	&& x\left(3x+5\right)-2\left(3x+5\right)=0 \\
	&& \left(x-2\right)\left(3x+5\right)=0 \\
	&& x-2=0 \text{ hoặc } 3x+5=0	 \\
	&& x=2 \text{ hoặc } x=-\dfrac{5}{3}.
	\end{eqnarray*}	Vậy phương trình đã cho có hai nghiệm là $x=2$ và $x=-\dfrac{5}{3}$.
	\end{listEX}
	}
\end{vd}
\begin{vd}
	Giải các phương trình sau bằng cách đưa về phương trình tích:
	\begin{listEX}[2]
	\item $(2x-3)^2=(x+7)^2$;
	\item $x^2-9=3(x+3)$;
	\item $x^2-x=-2x+2$;
	\item $x^2-3x=2x-6$.
	\end{listEX}
	\loigiai{
	\begin{listEX}
	\item 
	Biến đổi phương trình đã cho về phương trình tích như sau:
	\begin{align*}
	&(2x-3)^2=(x+7)^2\\
	&(2x-3)^2-(x+7)^2=0\\
	&[(2x-3)-(x+7)][(2x-3)+(x+7)]=0\\
	&(x-10)(3x+4)=0
	\end{align*}
	Ta giải hai phương trình sau:
	\begin{itemize}
	\item $x-10=0$ suy ra $x=10$.
	\item $3x+4=0$ hay $3x=-4$, suy ra $x=-\dfrac{4}{3}$.
	\end{itemize}
	Vậy phương trình đã cho có hai nghiệm là $x=10$ và $x=-\dfrac{4}{3}$.
	\item 
	Biến đổi phương trình đã cho về phương trình tích như sau:
	\begin{align*}
	&x^2-9=3(x+3)\\
	&(x-3)(x+3)-3(x+3)=0\\
	&(x+3)[(x-3)-3]=0\\
	&[(x+3)(x-6)]=0.
	\end{align*}
	Ta giải hai phương trình sau
	\begin{itemize}
	\item $x+3=0$ suy ra $x=-3$.
	\item $x-6=0$ suy ra $x=6$.
	\end{itemize}
	Vậy phương trình đã cho có hai nghiệm là $x=-3$ và $x=6$.
	\item 
	Biến đổi phương trình đã cho về phương trình tích như sau:
	\begin{align*}
	&x^2-x=-2x+2\\
	&x^2-x+2x-2=0\\
	&x(-x-1)+2(x-1)=0\\
	&(x+2)(x-1)=0.
	\end{align*}
	Ta giải hai phương trình sau:
	\begin{itemize}
	\item $x+2=0$ suy ra $x=-2$.
	\item $x-1=0$ suy ra $x=1$.
	\end{itemize}
	Vậy phương trình đã cho có hai nghiệm là $x=-2$ và $x=1$.
	\item 
	Biến đổi phương trình đã cho về phương trình tích như sau:
	\begin{align*}
	&x^2-3x=2x-6\\
	&x^2-3x-2x+6=0\\
	&x(x-3)-2(x-3)=0\\
	&(x-3)(x-2)=0.
	\end{align*}
	\begin{itemize}
	\item $x-3=0$ suy ra $x=3$.
	\item $x-2=0$ suy ra $x=2$.
	\end{itemize}
	Vậy phương trình có hai nghiệm $x=3$ và $x=2$.
	\end{listEX}
	}
\end{vd}
\begin{vd}%[8D3K4]
		Giải các phương trình sau bằng cách đưa về phương trình tích:
	\begin{listEX}[4]
		\item $3 x^2- 11 x + 6 = 0$;
		\item $-2x^2 + 5x + 3 = 0$;
		\item $ x^3 + 2x - 3 = 0$;
		\item $x^3 + 8 = x^2- 4.$	
	\end{listEX}
	\loigiai{
		\begin{listEX}
			\item Biến đổi phương trình đã cho về phương trình tích như sau:
			$$\begin{aligned}[t]
				&3x^2- 11 x + 6 = 0\\
				&3 x^2- 9 x - 2 x + 6 = 0\\
				&3 x(x - 3) - 2(x - 3) = 0\\
				& (x - 3)(3 x - 2) = 0
			\end{aligned}$$
			\begin{itemize}
				\item $x - 3 = 0$, suy ra $x = 3$.
				\item $3x - 2 = 0$ hay $3x=2$, suy ra $x = \dfrac{2}{3}$.
			\end{itemize}
			Vậy phương trình có hai nghiệm $x=3$ và $x=\dfrac{2}{3}$.
			\item Biến đổi phương trình đã cho về phương trình tích như sau: 
			$$\begin{aligned}[t]
				&-2x^2 + 5x + 3 = 0\\
				\ & - 2 x^2 + 6 x - x + 3 = 0\\
				\ & - 2 x(x - 3) -(x - 3) = 0\\
				\ & (x - 3)(- 2x - 1) = 0
			\end{aligned}$$
			\begin{itemize}
				\item $x - 3 = 0$, suy ra $x = 3$. 
				\item $-2x - 1 = 0$ hay $-2x=1$, suy ra $x = -\dfrac{1}{2}$.
			\end{itemize}
			Vậy phương trình có hai nghiệm $x=3$ và $x=-\dfrac{1}{2}$.
			\item Biến đổi phương trình đã cho về phương trình tích như sau: 
			$$\begin{aligned}[t]
				& x^3 + 2x - 3 = 0\\
				\ & x^3- 1 + 2 x - 2 = 0\\
				\ & (x - 1)(x^2+ x + 1) + 2(x - 1)= 0\\
				\ & (x - 1)(x^2 + x + 1 + 2) = 0\\
				\ & (x - 1)(x^2 + x + 3) = 0.
			\end{aligned}$$
			Vì $x^2 + x + 3 = \left(x + \dfrac{1}{2} \right)^2 + \dfrac{11}{4} > 0 $ nên $x - 1 = 0$, suy ra $x = 1.$ \\
			Vậy phương trình có nghiệm $x =1$.
			\item 	Biến đổi phương trình đã cho về phương trình tích như sau:
			$$\begin{aligned}[t]
				&x^3 + 8 = x^2- 4\\
				\ &(x + 2)\left(x^2- 2 x + 4\right) -(x + 2)(x - 2) = 0\\
				\ & (x + 2)\left(x^2- 2 x + 4 - x + 2\right) = 0\\
				\ & (x + 2)\left(x^2- 3 x + 6\right) = 0.
			\end{aligned}$$
			Vì $x^2 - 3x + 6 = \left(x - \dfrac{3}{2}\right)^2 + \dfrac{15}{4} > 0$ nên $x + 2 = 0$, suy ra $x = -2$.\\
			Vậy phương trình có nghiệm $x=2.$
		\end{listEX}
	}
\end{vd}
%%=====Ví dụ 3
\begin{vd}
	\immini{Trong một khu đất có dạng hình vuông, người ta dành một mảnh đất, có dạng hình chữ nhật ở góc khu đất để làm bể bơi (Hình $1$). Biết diện tích bể bơi bằng $1\,250\,\mathrm{m^2}$. Tính độ dài cạnh khu đất đó.}
	{\begin{tikzpicture}[font=\footnotesize,scale=.7]
	\def\a{4}
	\path 	(0:0) coordinate (A)
	++(0:\a) coordinate (B)
	++(90:\a) coordinate (C)
	($(A)-(B)+(C)$) coordinate (D)
	($(D)!2/3!(C)$) coordinate (E)
	($(C)!2/3!(B)$) coordinate (F)
	($(F)-(C)+(E)$) coordinate (G)
	;
	\fill[color=teal!20](E)--(G)--(F)--(C)--cycle;
	\draw[thick] 	(A)--(B)--(C)--(D)--cycle (E)--(G)--(F);
	\draw[<->]($(D)+(90:2mm)$)--($(E)+(90:2mm)$)node[pos=0.5,sloped,above]{$50\text{m}$};
	\draw[<->]($(B)+(0:2mm)$)--($(F)+(0:2mm)$)node[pos=0.5,sloped,below]{$25\text{m}$};
	\draw[draw=none] (A)--(B)node[pos=0.5,sloped,below]{\bf Hình 1}
	(E)--(F)node[pos=0.5,sloped]{\bf Bể\\ bơi};
	\end{tikzpicture}}
	\loigiai{
	Gọi độ dài cạnh khu đất có dạng hình vuông là $x\,(\mathrm{m})$. Khi đó, mảnh đất dạng hình chữ nhật để làm bể bơi có các kích thước là $x-50\,(\mathrm{m}),\,(x>50)$ và $x-25\,(\mathrm{m})$. Do đó, diện tích của mảnh đất là $(x-50)(x-25)\,(\mathrm{m^2})$.\\
	Vì vậy, ta có phương trình $(x-50)(x-25)=1\,250$.\\
	Giải phương trình $(x-50)(x-25)=1\,250$
	\allowdisplaybreaks
	\begin{align*}
	&(x-50)(x-25)-1\,250=0\\
	&x^2-75x=0\\
	&x(x-75)=0\\
	&x=0 \text{ hoặc } x=75.
	\end{align*}
	Do $x>50$ nên $x=75$. Vậy độ dài cạnh khu đất là $75\,(\mathrm{m})$.
	}
\end{vd}
\begin{vd}
	Trong một khu vườn hình vuông có cạnh bằng $15$ m người ta làm một lối đi xung quanh vườn có bề rộng là $x$ (m). Để diện tích phần đất còn lại là $169$ m$^2$ thì bề rộng $x$ của lối đi là bao nhiêu?
	\loigiai{
	Phần đất còn lại vẫn là hình vuông có cạnh $15-2x$ (m) nên diện tích phần đất còn lại là $(15-2x)^2$.\\
	Do cạnh của hình vuông là một số dương nên $15-2x>0 x<\dfrac{15}2$.\\
	Theo bài ra ta có phương trình $(15-2x)^2=169$. Khi đó
	\begin{align*}
	&(15-2x)^2-13^2=0\\
	&(15-2x-13)(15-2x+13)=0\\
	&(2-2x)(28-2x)=0.
	\end{align*}
	\begin{itemize}
	\item $2-2x=0$ suy ra $x=1$
	\item $28-2x=0$ suy ra $x=14$ (loại).
	\end{itemize}
	Vậy lối đi rộng $1$ (m).
	}
\end{vd}
\begin{dang}{Giải phương trình chứa ẩn ở mẫu}
\end{dang}
\begin{vd} Giải các phương trình:
	\begin{listEX}[2]
	\item $\dfrac{x+6}{x+5}+\dfrac{3}{2}=2$;
	\item $\dfrac{2}{x-2}-\dfrac{3}{x-3}=\dfrac{3 x-20}{(x-3)(x-2)}$;
	\item $\dfrac2{x+1}+\dfrac1{x-2}=\dfrac{3}{(x+1)(x-2)}$;
	\item $\dfrac1{x-1}-\dfrac{4x}{x^3-1}=\dfrac{x}{x^2+x+1}$.
	\end{listEX}
	\loigiai{
	\begin{listEX}
	\item Điều kiện xác định: $x\neq-5$\\
	Ta có:\\
	$$\begin{aligned}
	& \dfrac{x+6}{x+5}+\dfrac{3}{2}=2\\
	& \dfrac{x+6}{x+5}=\dfrac{1}{2}\\
	& 2(x+6)=x+5\\
	& 2x + 12 = x+5\\
	& x = -7\, \text{(thỏa mãn điều kiện xác định).}
	\end{aligned}$$\\
	Vậy phương trình đã cho có nghiệm $x=-7$.
	\item Điều kiện xác định: $x\neq 2$ và $x \neq 3$.\\
	Ta có:
	\begin{align*}
	& \dfrac{2}{x-2}-\dfrac{3}{x-3}=\dfrac{3 x-20}{(x-3)(x-2)}\\
	& 2(x-3)-3(x-2)=3x-20\\
	& 2x - 6 - 3x + 6 = 3x - 20\\
	& -4x = -20\\
	& x = 5 \, \text{(thỏa mãn điều kiện xác định).}
	\end{align*}
	Vậy phương trình đã cho có nghiệm $x=5$.
	\item 
	Điều kiện xác định $x\ne-1$ và $x\ne2$.\\
	Ta có 
	\begin{align*}
	&\dfrac{2(x-2)+(x+1)}{(x+1)(x-2)}=\dfrac{3}{(x+1)(x-2)}\\
	&2(x-2)+(x+1)=3\\
	&2x-4+x+1=3\\
	&3x-3=3\\
	&3x=6\\
	&x=2.
	\end{align*}
	Giá trị $x=2$ không thỏa mãn ĐKXĐ. Vậy phương trình đã cho vô nghiệm.
	\item
	Điều kiện xác định $x\ne1$.\\
	Ta có:
	\begin{align*}
	&\dfrac{x^2+x+1}{(x-1)(x^2+x+1)}-\dfrac{4x}{(x-1)(x^2+x+1)}=\dfrac{x(x-1)}{(x-1)(x^2+x+1)}\\
	&x^2+x+1-4x=x(x-1)\\
	&x^2+x+1-4x=x^2-x\\
	&x^2+x+1-4x-x^2+x=0\\
	&-2x+1=0\\
	&-2x=-1\\
	&x=\dfrac12.
	\end{align*}
	Giá trị $x=\dfrac12$ thỏa mãn ĐKXĐ. Vậy phương trình có nghiệm $x=\dfrac12$.
	\end{listEX}
	}
\end{vd}
\begin{vd}
	Giải các phương trình
	\begin{listEX}[2]
	\item $\dfrac{x^2}{2-x}+\dfrac{3x-1}{3}=\dfrac{5}{3}$.
	\item $\dfrac{4}{x(x-1)}+\dfrac{3}{x}=\dfrac{4}{x-1}$.
	\item $\dfrac{x+3}{x-3}+\dfrac{x-2}{x}=2$.
	\item $\dfrac{3}{x-2}+\dfrac{2}{x+1}=\dfrac{2 x+5}{(x-2)(x+1)}$.
	\end{listEX}
	\loigiai{
	\begin{listEX}
	\item Điều kiện xác định $2-x\ne 0$ hay $x\ne 2$.
	\begin{align*}
	&\dfrac{x^2}{2-x}+\dfrac{3x-1}{3} = \dfrac{5}{3}\\
	&\dfrac{3x^2}{3(2-x)}+\dfrac{(3x-1)(2-x)}{3(2-x)} = \dfrac{5(2-x)}{3(2-x)}\\
	&3x^2+(3x-1)(2-x) = 5(2-x)\\
	&3x^2+6x-3x^2-2+x = 10-5x\\
	&7x-2 = 10-5x\\
	&12x = 12\\
	&x = 1.
	\end{align*}
	Ta thấy $x=1$ thỏa mãn điều kiện xác định của phương trình.\\
	Vậy phương trình đã cho có nghiệm $x=1$.
	\item Điều kiện xác định $x\ne 0$ và $x\ne 1$. 
	\begin{align*}
	&\dfrac{4}{x(x-1)}+\dfrac{3}{x} = \dfrac{4}{x-1}\\
	&\dfrac{4}{x(x-1)}+\dfrac{3(x-1)}{x(x-1)} = \dfrac{4x}{x(x-1)}\\
	&4+3(x-1) = 4x\\
	&4+3x-3 = 4x\\
	&3x+1 =4x\\
	&x = 1.
	\end{align*}
	Ta thấy $x=1$ không thỏa mãn điều kiện xác định của phương trình.\\
	Vậy phương trình đã cho vô nghiệm.
	\item Điều kiện xác định: $x \neq 3$ và $x \neq 0$.
	\begin{align*}
	&\dfrac{x+3}{x-3}+\dfrac{x-2}{x}=2 \\
	& \dfrac{(x+3) x}{x(x-3)}+\dfrac{(x-2)(x-3)}{x(x-3)}=\frac{2 x(x-3)}{x(x-3)} \\
	& (x+3) x+(x-2)(x-3)=2 x(x-3) \\
	& x^2+3 x+x^2-3 x-2 x+6=2 x^2-6 x \\
	& 4 x=-6 \\
	& x=-\dfrac{3}{2} \text { (thoả mãn điều kiện xác định). }
	\end{align*}
	Vây nghiệm của phương trình đã cho là $x=-\dfrac{3}{2}$.\\
	\item Điều kiện xác định: $x \neq 2$ và $x \neq-1$.
	\begin{align*}
	& \dfrac{3}{x-2}+\dfrac{2}{x+1}=\dfrac{2 x+5}{(x-2)(x+1)} \\
	& 3(x+1)+2(x-2)=2 x+5 \\
	& 3 x+3+2 x-4=2 x+5 \\
	& 3 x=6\\
	& x=2\, \text{(không thoả mãn điều kiện xác định).}
	\end{align*}
	Vậy phương trình đã cho vô nghiệm.
	\end{listEX}
	}
\end{vd}
%%=====Ví dụ 6
\begin{vd}
	Hai bạn Phong và Khang cùng hẹn nhau đạp xe đến một vị trí cách vị trí bạn Phong $6\,\mathrm{km}$ và cách vị trí bạn Khang $7\,\mathrm{km}$. Hai bạn cùng xuất phát và đến địa điểm đã hẹn cùng một lúc. Tính tốc độ của mỗi bạn, biết tốc độ của bạn Khang hơn tốc độ bạn Phong là $6\,\mathrm{km/h}$.
	\loigiai{
	Gọi tốc độ của bạn Phong là $x\,\mathrm{km/h},\,(x>0)$. Khi đó, tốc độ của bạn Khang là $x+2\,\mathrm{km/h}$.\\
	Thời gian đi của bạn Phong là $\dfrac{6}{x}$ (giờ).\\
	Thời gian đi của bạn Khang là $\dfrac{7}{x+2}$ (giờ).\\
	Do hai bạn cùng xuất phát và đến địa điểm đã hẹn cùng một lúc nên thời gian đi của hai bạn là như nhau. Ta có phương trình \[\dfrac{6}{x}=\dfrac{7}{x+2}.\]
	Giải phương trình
	\allowdisplaybreaks
	\begin{align*}
	&\dfrac{6}{x}=\dfrac{7}{x+2}\\
	&\dfrac{6}{x} = \dfrac{7}{x+2}\\
	&\dfrac{6(x+2)}{x(x+2)} = \dfrac{7x}{x(x+2)}\\
	&6(x+2) = 7x\\
	&6x+12 = 7x\\
	&x=12.\text{ (thỏa mãn }x>0).
	\end{align*}
	Vậy tốc độ của bạn Phong là $12\,\mathrm{km/h}$, tốc độ của bạn Khang là $14\,\mathrm{km/h}$.
	}
\end{vd}
%%=====Ví dụ 7
\begin{vd}
	Biết nồng độ muối của nước biển là $3{,}5\%$ và khối lượng riêng của nước biển là $1\,020$ g/ml. Từ $2$ lít nước biển như thế, người ta hòa tan thêm muối để được dung dịch có nồng độ muối là $20\%$. Tính khối lượng muối cần thêm.
	\loigiai{
	Khối lượng của $2$ lít nước biển là $1\,020\cdot 2=2\,040$ (g).\\
	Khối lượng muối trong $2$ lít nước biển là $2\,040\cdot 3{,}5\%=71{,}4$ (g).\\
	Gọi khối lượng muối cần hòa thêm vào $2$ lít nước biển như thế để được dung dịch có nồng độ muối là $20\%$ là $x$ (g) $(x>0)$. Ta có phương trình \[\dfrac{71{,}4+x}{2\,040}=\dfrac{20}{100}.\]
	Giải phương trình $\dfrac{71{,}4+x}{2\,040}=\dfrac{20}{100}$.\\
	\begin{align*}
	&\dfrac{71{,}4+x}{2\,040+x} = \dfrac{20}{100}\\
	&\dfrac{100\cdot(71{,}4+x)}{100\cdot(2\,040+x)} = \dfrac{20\cdot (2\,040+x)}{100\cdot (2\,040+x)}\\
	&100\cdot(71{,}4+x) = 20\cdot (2\,040+x)\\
	&7140+100x=40\,800 = 7\,140\\
	&80x = 33\,660\\
	&x = 420{,}75 \text{ (thỏa mãn }x>0).
	\end{align*}
	Vậy cần thêm $420{,}75$ (g) muối vào $2$ lít nước biển ban đầu để được dung dịch có nồng độ muối là $20\%$.
	}
\end{vd}
\begin{vd} Hai thành phố $A$ và $B$ cách nhau $120 \mathrm{~km}$. Một ô tô di chuyển từ $A$ đến $B$, rồi quay trở về $\mathrm{A}$ với tổng thời gian đi và về là 4 giờ 24 phút. Tính tốc độ lượt đi của ô tô, biêt tốc độ lượt về lớn hơn tốc độ lượt đi $20 \%$.
	\loigiai{
	Gọi $x$ là tốc độ lượt đi.\\
	Tốc độ lượt về bằng: $x+20 \% \cdot x=1{,}2x$.\\
	Thời gian 4 giờ 24 phút $=4{,}4$ giờ.\\
	Theo đề bài ta có: 
	\begin{align*}
	&t_1+t_2=4{,}4\\
	&\dfrac{120}{x}+\dfrac{120}{x+0{,}2x}=4,4\\
	&\dfrac{120}{x}+\dfrac{120}{1{,}2x}=4{,}4\\
	&120\cdot 1,2 + 120 = 4,4\cdot 1{,}2 x\\
	&5{,}28 x = 264\\
	&x = 50 \, \text{(thỏa mãn điều kiện).}
	\end{align*}
	Vậy tốc độ lượt đi là $50 \, \mathrm{km/h}$.
	}
\end{vd}
	\begin{dang}{\boldmath Biết phương trình có một trong các nghiệm là $x = x_0$. Tìm giá trị của tham số $m$}
	\begin{itemize}
		\item \textit{Bước 1}: Thay $x = x_0$ vào phương trình đã cho.
		\item \textit{Bước 2}: Giải phương trình với ẩn số là $m$.
	\end{itemize}
\end{dang}
%vd 1
\begin{vd}%[8D3K4]
	Cho phương trình $(x + m)^2 - (x - 3m)^2 = 0$ trong đó $m$ là một số cho trước.
	\begin{enumerate}
		\item Tìm các giá trị của $m$ để phương trình có một trong các nghiệm là $x = 2.$
		\item Với các giá trị của $m$ tìm được ở câu a), hãy giải phương trình đã cho.	
	\end{enumerate}
	\loigiai{
		\begin{enumerate}
			\item Thay $x = 2$ vào phương trình đã cho ta được phương trình 
			\begin{eqnarray*}
				&&(2 + m)^2-(2 - 3 m)^2= 0\\
				& & (2 + m - 2 + 3 m)(2 + m + 2 - 3 m)= 0\\
				& & 4 m(- 2 m + 4) = 0\\
				& & \hoac{&m = 0 \\ & - 2 m + 4 = 0}\\
				& & \hoac{&m = 0 \\ & m = 2.}
			\end{eqnarray*}
			\item \textit{Trường hợp 1}. Với $m = 0$ thì phương trình đã cho trở thành\\
			$$x^2 - x^2 = 0\Rightarrow  0x^2 = 0.$$
			Phương trình nghiệm đúng với mọi $x$.\\
			\textit{Trường hợp 2}. Với $m = 2$ thì phương trình đã cho trở thành 
			\begin{eqnarray*}
				& &(x + 2)^2 - (x + 6)^2 = 0 \\
				& & (x + 2 - x + 6)(x + 2 + x - 6) = 0\\
				& & 8(2x - 4) = 0\\
				& & x = 2.
			\end{eqnarray*}
		\end{enumerate}
	}
\end{vd}
\begin{vd}%[8D3K4]
	Cho phương trình $x^3 - x^2 - 9x - 9m = 0$ trong đó $m$ là một số cho trước. Biết $x = 3$ là một nghiệm của phương trình. Tìm tất cả các nghiệm còn lại.
	\loigiai{
		\allowdisplaybreaks
		Thay $x = 3$ vào phương trình ta được \\
		$$27 - 9 - 27 - 9m = 0 \Rightarrow m = -1.$$
		Với $m = -1$ thì phương trình đã cho trở thành
		\begin{eqnarray*}
			&& x^3- x^2- 9 x + 9 = 0\\
			& & x^2(x - 1) - 9(x - 1)= 0\\
			&  & (x - 1)(x^2- 9)= 0\\
			&  & (x - 1)(x + 3)(x - 3) = 0 \\
			&  & \hoac{&x = 1\\&x = -3\\ & x = 3.}
		\end{eqnarray*}
		Vậy tất cả các nghiệm còn lại là $x = 1, x = -3.$
	}	
\end{vd}

\begin{vd}%[8D3B5]
	Cho phương trình $\dfrac{x - 2}{x - 4} + \dfrac{x - 3}{x - 2} = \dfrac{m}{3}$ trong đó $m$ là một số cho trước. Biết $x = 5$ là một trong các nghiệm của phương trình, tìm các nghiệm còn lại.
	\loigiai{
		\begin{itemize}
			\item Thay $x = 5$ vào phương trình đã cho ta được
			$$\dfrac{3}{1} + \dfrac{2}{3} = \dfrac{m}{3}\Rightarrow  m = 11.$$
			\item Thay $m = 11$ vào phương trình đã cho ta được
			$$\dfrac{x - 2}{x - 4} + \dfrac{x - 3}{x - 2} = \dfrac{11}{3} \quad(1)$$\vspace{0.2cm}\\
			ĐKXĐ: $x \neq 4$; $x \neq 2$.
			\begin{align*}
				(1)\Rightarrow & 3(x - 2)^2 + 3(x - 3)(x - 4) = 11(x - 4)(x - 2)\\
				& 3(x^2 - 4x + 4) + 3(x^2 - 7x + 12) = 11(x^2 - 6x + 8)\\
				& 3x^2 - 12x + 12 + 3x^2 - 21x + 36 = 11x^2 - 66x + 88\\
				& 5x^2 - 33x + 40 = 0\\
				& (x - 5)(5x - 8) = 0\\
				& \hoac{&x = 5 \\&x = \dfrac{8}{5}} \text{(thỏa mãn ĐKXĐ).}
			\end{align*}
		\end{itemize}	
		Vậy nghiệm còn lại là $x = \dfrac{8}{5}$.
	}
\end{vd}
\begin{vd}%[8D3K5]
	Cho phương trình $\dfrac{2x + m}{x - 1} = \dfrac{5(x - 1)}{x + 1}$.\\
	Chứng minh rằng nếu $x = \dfrac{1}{3}$ là một nghiệm của phương trình thì phương trình còn có một nghiệm nguyên.
	\loigiai{
		\begin{itemize}
			\item Thay $x = \dfrac{1}{3}$ vào phương trình đã cho ta được
			$$\dfrac{\dfrac{2}{3} + m}{\dfrac{1}{3} - 1} = \dfrac{5\cdot \left( \dfrac{1}{3} - 1 \right) }{\dfrac{1}{3} + 1}\Rightarrow  \dfrac{- (2 + 3m)}{2} = \dfrac{-5}{2}\Rightarrow  m = 1.$$
			\item Thay $m = 1$ vào phương trình đã cho ta được
			$$\dfrac{2x + 1}{x - 1} = \dfrac{5(x - 1)}{x + 1} \quad(1)$$
			ĐKXĐ: $x \neq \pm 1$.
			\begin{align*}
				(1)\Rightarrow &(2x + 1)(x + 1) = 5(x - 1)^2\\
				& 2x^2 + 2x + x + 1 = 5x^2 - 10x + 5\\
				& 3x^2 - 13x + 4 = 0\\
				& (3x - 1)(x - 4) = 0\\
				& \hoac{&x = \dfrac{1}{3}\\&x = 4} \text{(thỏa mãn ĐKXĐ).}
			\end{align*}
		\end{itemize}	
		Vậy phương trình đã cho có nghiệm $x = \dfrac{1}{3}$ và nghiệm nguyên $x = 4$.
	}
\end{vd}
\begin{dang}{Tìm giá trị của biến để giá trị của hai biểu thức có mối liên quan nào đó}
\end{dang}
\begin{vd}
	Cho hai biểu thức $A = \dfrac{3}{3x + 1} + \dfrac{2}{1 - 3x}$, $B = \dfrac{x - 5}{9x^2 - 1}$ với giá trị nào của $x$ thì hai biểu thức $A$ và $B$ có cùng một giá trị?
	\loigiai{
		$$A = B\Rightarrow  	\dfrac{3}{3x + 1} + \dfrac{2}{1 - 3x} = \dfrac{x - 5}{9x^2 - 1}. \quad(1)$$
		ĐKXĐ: $x \neq \pm \dfrac{1}{3}$.
		\begin{align*}
			(1)\Rightarrow & \dfrac{3}{3x + 1} - \dfrac{2}{3x - 1} = \dfrac{x - 5}{9x^2 - 1}\\
			& 3(3x - 1) - 2(3x + 1) = x - 5\\
			& 9x - 3 - 6x - 2 - x + 5 = 0\\
			& 2x = 0\\
			& x = 0 \ \text{(thỏa mãn ĐKXĐ).}
		\end{align*}
		Vậy khi $x = 0$ thì $A = B$.
	}
\end{vd}
\begin{vd}%[8D3B5]
	Cho ba biểu thức $A = \dfrac{2}{5x - 2}$; $B = \dfrac{4}{1 - 5x}$; $C = \dfrac{2}{(5x - 2)(5x - 1)}$. \\
	Tìm các giá trị của $x$ để tổng $A + B$ có giá trị bằng giá trị của biểu thức $C$.
	\loigiai{
		$$A + B = C \Rightarrow \dfrac{2}{5x - 2} + \dfrac{4}{1 - 5x} = \dfrac{2}{(5x - 2)(5x - 1)} \quad(1) $$\\
		ĐKXĐ: $x \neq \dfrac{2}{5}$; $x \neq \dfrac{1}{5}$.
		\begin{align*}
			(1)\Rightarrow & \dfrac{2}{5x - 2} - \dfrac{4}{5x - 1} = \dfrac{2}{(5x - 2)(5x - 1)}\\
			& 2(5x - 1) - 4(5x - 2) = 2\\
			& 10x - 2 - 20x + 8 = 2\\
			& -10x = -4\\
			& x = \dfrac{2}{5}\ \text{(loại).} 
		\end{align*}
		Vậy không có giá trị nào của $x$ để tổng $A + B$ có giá trị bằng giá trị của biểu thức $C$.
	}
\end{vd}
\begin{vd}%[8D3K5]
	Cho hai biểu thức $P = \dfrac{x - 2}{x - 1} + \dfrac{x + 2}{x + 1}$ và $Q = \dfrac{2 - x^2}{1 - x^2}$. Với giá trị nào của $x$ thì giá trị của biểu thức $P$ bằng $2$ lần giá trị của biểu thức $Q$?
	\loigiai{
		$$ P = 2Q \Rightarrow \dfrac{x - 2}{x - 1} + \dfrac{x + 2}{x + 1} = \dfrac{2(2 - x^2)}{1 - x^2}. \quad(1)$$\\
		ĐKXĐ: $x \neq \pm 1$.
		\begin{align*}
			(1)\Rightarrow & \dfrac{x - 2}{x - 1} + \dfrac{x + 2}{x + 1} = \dfrac{2(x^2 - 2)}{x^2 - 1}\\
			&(x - 2)(x + 1) + (x + 2)(x - 1) = 2(x^2 - 2)\\
			& x^2 - x - 2 + x^2 + x - 2 = 2x^2 - 4\\
			& 0x = 0.
		\end{align*}	
		Phương trình này nghiệm đúng với mọi $x$.\\
		Vậy $P = 2Q$ với mọi giá trị của $x$ trừ $x = \pm 1$.
	}
\end{vd}
%%%%%%%%%%%%%%%%%%%%%%
\subsection{Bài tập vận dụng}
\begin{bt}
Giải các phương trình sau:
\begin{enumEX}{2}
\item $x(x-2)=0$;
\item $(2x+1)(3x-2)=0$;
\item $(x^2-4)+x(x-2)=0$;
\item $(2x+1)^2-9x^2=0$.
\end{enumEX}
\loigiai{
\begin{enumerate}
\item $x(x-2)=0$
\begin{itemize}
\item $x=0$
\item $x-2=0$ suy ra $x=2$
\end{itemize}
Vậy phương trình có hai nghiệm $x=0$ và $x=2$.
\item $(2x+1)(3x-2)=0$
\begin{itemize}
\item $2x+1=0$ suy ra $x=-\dfrac12$
\item $3x-2=0$ suy ra $x=\dfrac23$
\end{itemize}
Vậy phương trình có hai nghiệm $x=-\dfrac12$ và $x=\dfrac23$.
\item Ta có
\begin{align*}
	&(x^2-4)+x(x-2)=0\\
	&(x-2)(x+2)+x(x-2)=0\\
	&(x-2)(x+2+x)=0\\
	&(x-2)(2x+2)=0.
\end{align*}
\begin{itemize}
	\item $x-2=0$ suy ra $x=2$
	\item $2x+2=0$ suy ra $x=-1$
\end{itemize}
Vậy phương trình có hai nghiệm $x=2$ và $x=-1$.
\item Ta có
\begin{align*}
	&(2x+1)^2-9x^2=0\\
	&(2x+1)^2-(3x)^2=0\\
	&(2x+1-3x)(2x+1+3x)=0\\
	&(-x+1)(5x+1)=0.
\end{align*}
\begin{itemize}
	\item $-x+1=0$ suy ra $x=1$
	\item $5x+1=0$ suy ra $x=-\dfrac15$
\end{itemize}
Vậy phương trình có hai nghiệm $x=1$ và $x=-\dfrac15$.
\end{enumerate}
}
\end{bt}
\begin{bt} 
	Giải các phương trình:
	\begin{listEX}[2] 
	\item $5 x(2 x-3)=0$;
	\item $(2 x-5)(3 x+6)=0$;
	\item $\left(\dfrac{2}{3} x-1\right)\left(\dfrac{1}{2} x+3\right)=0$;
	\item $(2{,}5 t-7,5)(0{,}2 t+5)=0$.
	\end{listEX}
	\loigiai{
	\begin{listEX}
	\item 
	$5x(2x-3)=0$\\
	$\begin{aligned}
	&5x=0\,\, \text{ hoặc }\,\, 2x-3=0\\
	&x=0 \,\, \text{ hoặc } \,\, x=\dfrac{3}{2}.\\
	\end{aligned}$\\
	Vậy phương trình đã cho có hai nghiệm là $x=0$ và $x=\dfrac{3}{2}$.
	\item $(2x-5)(3x+6)=0$\\
	$\begin{aligned}
	&2x-5=0 \,\, \text{ hoặc }\,\, 3x+6=0\\
	&x=\dfrac{5}{2} \,\, \text{ hoặc }\,\, x= -2.\\
	\end{aligned}$\\
	Vậy phương trình đã cho có nghiệm là $x=\dfrac{5}{2}$ và $x=-2$.
	\item $\left(\dfrac{2}{3} x-1\right)\left(\dfrac{1}{2} x+3\right)=0$\\
	$\begin{aligned}
	&\left(\dfrac{2}{3} x-1\right)=0 \,\, \text{ hoặc }\,\, \left(\dfrac{1}{2} x+3\right)=0\\
	&x=\dfrac{3}{2} \,\, \text{ hoặc }\,\, x=-6.\\
	\end{aligned}$\\
	Vậy phương trình đã cho có nghiệm là $x=\dfrac{3}{2}$ và $x=-6$.
	\item $(2,5 t-7,5)(0,2 t+5)=0$\\
	$\begin{aligned}
	&2,5 t-7,5=0 \,\, \text{ hoặc }\,\, 0,2 t+5=0\\
	&t=3 \,\, \text{ hoặc }\,\, t=-25.\\
	\end{aligned}$\\
	Vậy phương trình đã cho có nghiệm là $t=3$ và $t=-25$.
	\end{listEX}
	}
\end{bt}
\begin{bt}
	Giải các phương trình
	\begin{enumEX}{2}
	\item $(9x-4)(2x+5) = 0$;
	\item $(1{,3}x+0{,}26)(0{,2}x-4) = 0$;
	\item $2x(x+3) - 5(x+3) = 0$;
	\item $x^2 - 4 + (x+2)(2x-1) = 0$.
	\end{enumEX}
	\loigiai{
	\begin{enumerate}
	\item Để giải phương trình đã cho, ta giải hai phương trình sau
	\begin{enumEX}[\itemCI]{2}
	\item $\begin{aligned}[t]
	9x-4 &= 0 \\
	x &= \dfrac{4}{9};
	\end{aligned}$
	\item $\begin{aligned}[t]
	2x+5 &= 0 \\
	x &= \dfrac{-5}{2}.
	\end{aligned}$
	\end{enumEX}
	Vậy phương trình đã cho có hai nghiệm là $x=\dfrac{4}{9}$ và $x=\dfrac{-5}{2}$.
	\item Để giải phương trình đã cho, ta giải hai phương trình sau
	\begin{enumEX}[\itemCI]{2}
	\item $\begin{aligned}[t]
	1{,}3x + 0{,}26 &= 0 \\
	x &= \dfrac{-1}{5};
	\end{aligned}$
	\item $\begin{aligned}[t]
	0{,}2x-4 &= 0 \\
	x &= 20.
	\end{aligned}$
	\end{enumEX}
	Vậy phương trình đã cho có hai nghiệm là $x=\dfrac{-1}{5}$ và $x=20$.
	\item $\begin{aligned}[t]
	2x(x+3) - 5(x+3) &= 0 \\
	(x+3)(2x-5) &=0.
	\end{aligned}$\\
	Để giải phương trình trên, ta giải hai phương trình sau
	\begin{enumEX}[\itemCI]{2}
	\item $\begin{aligned}[t]
	x + 3 &= 0 \\
	x &= -3;
	\end{aligned}$
	\item $\begin{aligned}[t]
	2x-5 &= 0 \\
	x &= \dfrac{5}{2}.
	\end{aligned}$
	\end{enumEX}
	Vậy phương trình đã cho có hai nghiệm là $x=-3$ và $x=\dfrac{5}{2}$.
	\item $\begin{aligned}[t]
	x^2 - 4 + (x+2)(2x-1) &= 0 \\
	x-2)(x+2) + (x+2)(2x-1) &= 0 \\
	(x+2)(x-2+2x-1) &= 0 \\
	(x+2)(3x-3) &=0\\
	\end{aligned}$\\
	Để giải phương trình trên, ta giải hai phương trình sau
	\begin{enumEX}[\itemCI]{2}
	\item $\begin{aligned}[t]
	x + 2 &= 0 \\
	x &= -2;
	\end{aligned}$
	\item $\begin{aligned}[t]
	3x-3 &= 0 \\
	x &= 1.
	\end{aligned}$
	\end{enumEX}
	Vậy phương trình đã cho có hai nghiệm là $x=-2$ và $x=1$.
	\end{enumerate}
	}
\end{bt}
%%=====Bài 2
%%=====Bài 2
\begin{bt}
	Giải các phương trình;
	\begin{enumEX}{2}
	\item $3x(x-4)+7(x-4)=0$
	\item $5x(x+6)-2x-12=0$
	\item $x^2-x-(5x-5)=0$
	\item $(3x-2)^2-(x+6)^2=0$
	\end{enumEX}
	\loigiai{
	\begin{enumEX}{1}
	\item Ta có: 
	$\begin{aligned}[t]
	&3x(x-4)+7(x-4)=0\\
	&(3x+7)(x-4)=0\\
	&3x+7=0 \text{ hoặc } x-4=0\\
	&x=\dfrac{-7}{3} \text{ hoặc } x=4.
	\end{aligned}$\\
	Vậy phương trình đã cho có hai nghiệm là $x=\dfrac{-7}{3}$ và $x=4$.
	\item Ta có: 
	$\begin{aligned}[t]
	&5x(x+6)-2x-12=0\\
	&5x(x+6)-2(x+6) = 0\\
	&(5x-2)(x+6) = 0\\
	&5x-2=0 \text{ hoặc } x+6=0\\
	&x=\dfrac{2}{5} \text{ hoặc } x=-6.
	\end{aligned}$\\
	Vậy phương trình đã cho có hai nghiệm là $x=\dfrac{2}{5}$ và $x=-6$.
	\item Ta có: 
	$\begin{aligned}[t]
	&x^2-x-(5x-5)=0\\
	&x(x-1)-5(x-1) = 0\\
	&(x-5)(x-1) = 0\\
	&x-5=0 \text{ hoặc } x-1=0\\
	&x=5 \text{ hoặc } x=1.\\
	\end{aligned}$\\
	Vậy phương trình đã cho có hai nghiệm là $x=5$ và $x=1$.
	\item Ta có: 
	$\begin{aligned}[t]
	&(3x-2)^2-(x+6)^2=0\\
	&(3x-2+x+6)(3x-2-x-6) = 0\\
	&(4x+4)(2x-8) = 0\\
	&4x+4=0 \text{ hoặc } 2x-8=0\\
	&x=-1 \text{ hoặc } x=4.
	\end{aligned}$\\
	Vậy phương trình đã cho có hai nghiệm là $x=-1$ và $x=4$.
	\end{enumEX}
	}
\end{bt}
\begin{bt}
	Giải các phương trình
	\begin{enumEX}{2}
	\item $\dfrac{1}{x} = \dfrac{5}{3(x+2)}$;
	\item $\dfrac{x}{2x-1} = \dfrac{x-2}{2x+5}$;
	\item $\dfrac{5x}{x-2} = 7 + \dfrac{10}{x-2}$;
	\item $\dfrac{x^2-6}{x} = x + \dfrac{3}{2}$.
	\end{enumEX}
	\loigiai{
	\begin{enumerate}
	\item Điều kiện xác định: $x \ne 0$ và $x \ne -2$.\\
	\begin{align*}
	&\dfrac{1}{x} = \dfrac{5}{3(x+2)} \\
	&\dfrac{3(x+2)}{3x(x+2)} = \dfrac{5x}{3x(x+2)} \\
	&3(x+2) = 5x \\
	&3x + 6 = 5x \\
	&2x = 6 \\
	&x=3.
	\end{align*}
	Ta thấy $x=3$ thỏa mãn điều kiện xác định của phương trình.\\
	Vậy phương trình đã cho có nghiệm $x=3$.
	\item Điều kiện xác định: $x \ne \dfrac{1}{2}$ và $x \ne \dfrac{-5}{2}$.\\
	\begin{align*}
	&\dfrac{x}{2x-1} = \dfrac{x-2}{2x+5} \\
	&\dfrac{x(2x+5)}{(2x-1)(2x+5)} = \dfrac{(x-2)(2x-1)}{(2x+5)(2x-1)} \\
	&x(2x+5) = (x-2)(2x-1) \\
	&2x^2 + 5x = 2x^2 - x - 4x + 2 \\
	&10x = 2 \\
	&x = \dfrac{1}{5}.	
	\end{align*}
	Ta thấy $x=\dfrac{1}{5}$ thỏa mãn điều kiện xác định của phương trình.\\
	Vậy phương trình đã cho có nghiệm $x=\dfrac{1}{5}$.
	\item Điều kiện xác định: $x \ne 2$.\\
	\begin{align*}
	&\dfrac{5x}{x-2} = 7 + \dfrac{10}{x-2} \\
	&\dfrac{5x}{x-2} = \dfrac{7(x-2)}{x-2} + \dfrac{10}{x-2} \\
	&5x = 7(x-2) + 10 \\
	&5x = 7x -14 + 10 \\
	&-2x = -4 \\
	&x = 2.
	\end{align*}
	Ta thấy $x=2$ không thỏa mãn điều kiện xác định của phương trình.\\
	Vậy phương trình đã cho vô nghiệm.
	\item Điều kiện xác định: $x \ne 0$.\\
	\begin{align*}
	&\dfrac{x^2-6}{x} = x + \dfrac{3}{2} \\
	&\dfrac{2(x^2-6)}{2x} = \dfrac{2x^2}{2x} + \dfrac{3x}{2x} \\
	&2(x^2-6) = 2x^2 + 3x \\
	&2x^2 -12 = 2x^2 + 3x \\
	&3x = -12 \\
	&x = -4.
	\end{align*}
	Ta thấy $x=-4$ thỏa mãn điều kiện xác định của phương trình.\\
	Vậy phương trình đã cho có nghiệm $x=-4$.
	\end{enumerate}
	}
\end{bt}
\begin{bt}
	Giải các phương trình sau:
	\begin{enumEX}{2}
	\item $\dfrac2{2x+1}+\dfrac1{x+1}=\dfrac3{(2x+1)(x+1)}$;
	\item $\dfrac1{x+1}-\dfrac x{x^2-x+1}=\dfrac{3x}{x^3+1}$.
	\end{enumEX}
	\loigiai{
	\begin{enumerate}
	\item Điều kiện xác định $x\ne-\dfrac12$ và $x\ne-1$.\\
	Quy đồng mẫu và khử mẫu ta được
	\begin{align*}
	&\dfrac{2(x+1)+(2x+1)}{(2x+1)(x+1)}=\dfrac3{(2x+1)(x+1)}\\
	&2(x+1)+(2x+1)=3\\
	&2x+2+2x+1=3\\
	&4x=0\\
	&x=0.
	\end{align*}
	Giá trị $x=0$ thỏa mãn ĐKXĐ. Vậy phương trình có nghiệm $x=0$.
	\item Điều kiện xác định $x\ne-1$.\\
	Quy đồng mẫu và khử mẫu ta được
	\begin{align*}
	&\dfrac{x^2-x+1-x(x+1)}{(x+1)(x^2-x+1}=\dfrac{3x}{(x+1)(x^2-x+1)}\\
	&x^2-x+1-x(x+1)=3x\\
	&x^2-x+1-x^2-x-3x=0\\
	&-5x+1=0\\
	&x=\dfrac15.
	\end{align*}
	Giá trị $x=\dfrac15$ thỏa mãn ĐKXĐ. Vậy phương trình đã cho có nghiệm $x=\dfrac15$.
	\end{enumerate}
	}
\end{bt}
%%=====Bài 3
\begin{bt}
	Giải các phương trình:
	\begin{enumEX}{2}
	\item $\dfrac{x+5}{x-3}+2=\dfrac{2}{x-3}$
	\item $\dfrac{3x+5}{x+1}+\dfrac{2}{x}=3$
	\item $\dfrac{x+3}{x-2}+\dfrac{x+2}{x-3}=2$
	\item $\dfrac{x+2}{x-2}-\dfrac{x-2}{x+2}=\dfrac{16}{x^2-4}$
	\end{enumEX}
	\loigiai{
	\begin{enumEX}{1}
	\item Điều kiện xác định: $x-3 \ne 0  x \ne 3$.\\
	Ta có: 
	\begin{align*}
	&\dfrac{x+5}{x-3}+2=\dfrac{2}{x-3}\\
	&\dfrac{x+5}{x-3}+\dfrac{2(x-3)}{x-3} = \dfrac{2}{x-3}\\
	&x+5+2(x-3) = 2\\
	&x+5+2x-6 = 2\\
	&3x = 3\\
	&x = 1\ (\text{thỏa mãn điều kiện xác định}).
	\end{align*}
	Vậy nghiệm của phương trình đã cho là $x=1$.
	\item Điều kiện xác định: $\heva{& x+1 \ne 0\\ & x \ne 0}  \heva{& x \ne -1\\ & x \ne 0.}$\\
	Ta có: 
	\begin{align*}
	&\dfrac{3x+5}{x+1}+\dfrac{2}{x}=3\\
	&\dfrac{(3x+5)x}{x(x+1)}+\dfrac{2(x+1)}{x(x+1)} = \dfrac{3x(x+1)}{x(x+1)}\\
	&(3x+5)x+2(x+1) = 3x(x+1)\\
	&3x^2+5x+2x+2 = 3x^2+3x\\
	&4x = -2\\
	&x = \dfrac{-1}{2}\ (\text{thỏa mãn điều kiện xác định}).
	\end{align*}
	Vậy nghiệm của phương trình đã cho là $x=\dfrac{-1}{2}$.
	\item Điều kiện xác định: $\heva{& x-2 \ne 0\\ & x-3 \ne 0}  \heva{& x \ne 2\\ & x \ne 3.}$\\
	Ta có: 
	\begin{align*}
	&\dfrac{x+3}{x-2}+\dfrac{x+2}{x-3}=2\\
	&\dfrac{(x+3)(x-3)}{(x-2)(x-3)}+\dfrac{(x+2)(x-2)}{(x-2)(x-3)} =\dfrac{2(x-2)(x-3)}{(x-2)(x-3)}\\
	&(x+3)(x-3)+(x+2)(x-2) = 2(x-2)(x-3)\\
	&x^2-9+x^2-4 = 2(x^2-5x+6)\\
	&x^2-9+x^2-4 = 2x^2-10x+12\\
	&10x = 25\\
	&x = \dfrac{5}{2}\ (\text{thỏa mãn điều kiện xác định}).
	\end{align*}
	Vậy nghiệm của phương trình đã cho là $x=\dfrac{5}{2}$.
	\item Điều kiện xác định: $\heva{& x-2 \ne 0\\ & x+2 \ne 0}  \heva{& x \ne 2\\ & x \ne -2.}$\\
	Ta có: 
	\begin{align*}
	&\dfrac{x+2}{x-2}-\dfrac{x-2}{x+2}=\dfrac{16}{x^2-4}\\
	&\dfrac{(x+2)(x+2)}{(x-2)(x-2)}-\dfrac{(x-2)(x-2)}{(x+2)(x-2)} = \dfrac{16}{(x+2)(x-2)}\\
	&(x+2)(x+2)-(x-2)(x-2) = 16\\
	&x^2+4x+4-x^2+4x-4 = 16\\
	&8x = 16\\
	&x = 2\ (\text{thỏa mãn điều kiện xác định}).
	\end{align*}
	Vậy nghiệm của phương trình đã cho là $x=2$.
	\end{enumEX}
	}
\end{bt}
\begin{bt}
	\immini{
	Bác An có một mảnh đất hình chữ nhật với chiều dài $14$ m và chiều rộng $12$ m. Bác dự định xây nhà trên mảnh đất đó và dành một phần diện tích đất để làm sân vườn như hình bên. Biết diện tích đất làm nhà là $100$ m$^2$. Hỏi $x$ bằng bao nhiêu mét?
	}{
	\begin{tikzpicture}[scale=.45, font=\scriptsize]
	\draw[thick](0,0)--(7,0)--(7,6)node[midway, right]{$12$}--(0,6)node[midway, above]{$14$}node[midway,below]{Sân vườn}--(0,0);
	\draw[stealth-stealth, thick](-.2, 6)--(-.2, 5)node[midway, left]{$x$};
	\draw[stealth-stealth, thick](5,-0.2)--(7,-.2)node[midway, below]{$x+2$};
	\draw[thick](5,0)--(5,5)--(0,5);
	\draw(3,3)node{Nhà};
	\end{tikzpicture}
	}
	\loigiai{
	Diện tích đất làm nhà là $(12-x)[14-(x+2)]=(12-x)(12-x)$ (m$^2$) với điều kiện $0<x<12$.\\
	Vì diện tích đất làm nhà là $100$ m$^2$ nên ta có phương trình
	\begin{align*}
	&(12-x)(12-x)=100\\
	&(12-x)^2-10^2=0\\
	&(12-x-10)(12-x+10)=0\\
	&(2-x)(22-x)=0.
	\end{align*}
	\begin{itemize}
	\item $2-x=0$ suy ra $x=2$
	\item $22-x=0$ suy ra $x=22$ (loại).
	\end{itemize}
	Vậy $x$ bằng $2$ mét.
	}
\end{bt}
%%=====Bài 6
\begin{bt}
	\immini{
	Một mảnh đất có dạng hình chữ nhật với chu vi bằng $52$ m. Trên mảnh đất đó, người ta làm một vườn rau có dạng hình chữ nhật với diện tích là $112$ m$^2$ và một lối đi xung quanh vườn rộng $1$ m (\textit{Hình $2$}). Tính các kích thước của mảnh đất đó.
	}
	{
	\begin{tikzpicture}[font=\footnotesize, scale=.5]
	\draw [thick] (0,0) rectangle (8,5);
	\fill [green!50] (0.5,0.5) rectangle (7.5,4.5);
	\draw [thick] (0.5,0.5) rectangle (7.5,4.5);
	\draw [thick, red] 	(4,0)--(4,0.5) (4,4.5)--(4,5)
	(0,2.5)--(0.5,2.5) (7.5,2.5)--(8,2.5);
	\draw [red]	(0.25,2.5)--(-.25,2.75)--(-1,2.75) node[midway,above]{$1$ m}
	(7.75,2.5)--(8.25,2.75)--(9,2.75) node[midway,above]{$1$ m}
	(4,0.25)--(4.25,-.25)--(5,-0.25) node[midway,below]{$1$ m}
	(4,4.75)--(4.25,5.25)--(5,5.25) node[midway,above]{$1$ m};
	\path (4,-1.5) node{\textit{Hình $2$}};
	\path (4,2.5) node[color=blue]{Vườn rau};	
	\end{tikzpicture}
	}
	\loigiai{
	Nửa chu vi của hình chữ nhật là $52 : 2 = 26$ (m).\\
	Gọi độ dài một cạnh của mảnh đất hình chữ nhật là $x$ (m) ($x > 0$).\\
	Suy ra độ dài cạnh còn lại là $26-x$ (m).\\
	Do lối đi được là xung quanh vườn và rộng $1$ m nên các kích thước của vườn rau lần lượt là $x-2$ m và $26-x-2 = 24-x$ m.\\
	Theo đề bài, diện tích của vườn rau là $112$ m$^2$ nên ta có phương trình $x-2)(24-x) = 112$.\\
	Giải phương trình 
	$$\begin{aligned}[t]
	&(x-2)(24-x) = 112 \\
	&24x - x^2 - 48 + 2x = 112 \\
	&x^2 - 26x + 160 = 0 \\
	&x^2 - 10x - 16x + 160 = 0 \\
	&x(x-10) - 16(x-10) = 0\\
	&(x-10)(x-16) = 0 \\
	&x-10 = 0 \text{ hoặc } x-16 =0 \\
	&x=10 \text{ hoặc } x = 16.
	\end{aligned}$$
	Với $x=10$ thì độ đài cạnh còn lại là $26-10 = 16$ m.\\
	Với $x=16$ thì độ đài cạnh còn lại là $26-16 = 10$ m.\\
	Vậy độ dài hai cạnh của mảnh đất là $10$ m và $16$ m.
	}
\end{bt}
\begin{bt}
Hai người cùng làm chung một công việc thì xong trong $8$ giờ. Hai người cùng làm được $4$ giờ thì người thứ nhất bị điều đi làm công việc khác. Người thứ hai tiếp tục làm việc trong $12$ giờ nữa thì xong công việc. Gọi $x$ là thời gian người thứ nhất làm một mình xong công việc (đơn vị tính là giờ, $x>0$).
\begin{enumerate}
\item Hãy biểu thị theo $x$:
\begin{itemize}
\item Khối lượng công việc mà người thứ nhất làm được trong $1$ giờ;
\item Khối lượng công việc mà người thứ hai làm được trong $1$ giờ.
\end{itemize}
\item Hãy lập phương trình theo $x$ và giải phương trình đó. Sau đó cho biết, nếu làm một mình thì mỗi người phải làm trong bao lâu mới xong cộng việc đó.
\end{enumerate}
\loigiai{
\begin{enumerate}
\item Ta có
\begin{itemize}
\item Khối lượng công việc mà người thứ nhất làm được trong $1$ giờ là $\dfrac1x$ (công việc).
\item Khối lượng công việc mà người thứ hai làm được trong $1$ giờ là $\dfrac18-\dfrac1x$ (công việc).
\end{itemize}
\item Vì người thứ hai chỉ làm $4$ giờ và người thứ nhất làm $4+12=16$ giờ nên ta có phương trình:
\begin{align*}
&16\cdot\dfrac1x+4\left(\dfrac18-\dfrac1x\right)=1\\
&\dfrac{16}{x}+\dfrac12-\dfrac4x=1\\
&\dfrac{12}{x}+\dfrac12=1\\
&\dfrac{12\cdot2+x}{2x}=\dfrac{2x}{2x}\\
&24+x=2x\\
&x=24.
\end{align*}
Vậy nếu làm một mình thì người thứ nhất cần $24$ giờ mới xong công việc đó.\\
Khi làm một mình thì trong $1$ giờ, người thứ hai làm được $\dfrac18-\dfrac1{24}=\dfrac1{12}$ công việc nên người thứ hai làm một mình cần $12$ giờ mới xong công việc đó.
\end{enumerate}
}
\end{bt}
%%=====Bài 3
\begin{bt}
	Một ca nô đi xuôi dòng từ địa điểm $A$ đến địa điểm $B$, rồi lại đi ngược dòng từ địa điểm $B$ trở về địa điểm $A$. Thời gian cả đi và về là $3$ giờ. Tính tốc độ của dòng nước. Biết tốc độ của ca nô khi nước yên lặng là $27$km/h và độ dài quãng đường $AB$ là $40$ km.
	\loigiai{
	Gọi tốc độ của dòng nước là $x$ km/h ($0 < x < 27$).\\
	Vận tốc của ca nô khi xuôi dòng là $27 + x$ (km/h). \\
	Vận tốc của ca nô khi ngược dòng là $27 - x$ (km/h). \\
	Thời gian ca nô xuôi dòng từ $A$ đến $B$ là $\dfrac{40}{27+x}$ (giờ).\\
	Thời gian ca nô ngược dòng từ $B$ về $A$ là $\dfrac{40}{27-x}$ (giờ).\\
	Vì thời gian cả đi và về là $3$ giờ nên ta có phương trình \\
	$$\begin{aligned}
	&\dfrac{40}{27+x} + \dfrac{40}{27-x}= 3 \\
	&\dfrac{40(27-x)}{(27+x)(27-x)} + \dfrac{40(27+x)}{(27-x)(27+x)}=\dfrac{3(27-x)(27+x)}{(27-x)(27+x)} \\
	&40(27-x) + 40(27+x) =3(27-x)(27+x) \\
	&40(27-x + 27+x) = 3(27^2 - x^2) \\
	&40 \cdot 2 \cdot 27 = 3 \cdot 27^2 - 3x^2 \\
	&3x^2 = 27 \\
	&x^2 = 9 \\
	&x = 3 \text{ hoặc } x = -3.
	\end{aligned}$$
	Ta thấy $x=3$ thỏa mãn điều kiện xác định; $x=-3$ không thỏa mãn điều kiện xác định.\\
	Vậy tốc độ của dòng nước là $3$ km/h.
	}
\end{bt}
%%=====Bài 4
\begin{bt}
	Một doanh nghiệp sử dụng than để sản xuất sản phẩm. Doanh nghiệp đó lập kế hoạch tài chính cho việc loại bỏ chất ô nhiễm trong khí thải theo dự kiến sau: Để loại bỏ $p\%$ chất ô nhiễm trong khí thải thì chi phí $C$ (triệu đồng) được tính theo công thức: $C=\dfrac{80}{100-p}$ với $0 \le p < 100$ (\textit{Nguồn: John W. Cell, Engineering Problems Illustrating Mathematics, MeGraw-Hill Book Company, Inc. New York and London, năm $1943$}). Với chi phí là $420$ triệu đồng thì doanh nghiệp loại bỏ được bao nhiêu phần trăm chất gây ô nhiễm trong khí thải (làm tròn kết quả đến hàng phần mười)?
	\loigiai{
	Theo đề bài ta có phương trình 
	$$\begin{aligned}[t]
	&420 =\dfrac{80}{100-p} \\
	&\dfrac{420(100-p)}{100-p} = \dfrac{80}{100-p} \\
	&420(100-p) = 80 \\
	&42000 - 420p = 80 \\
	&420p = 41920 \\
	&p = \dfrac{41920}{420} \approx 99{,}8 ~(\text{thỏa mãn } 0 \le p < 100).
	\end{aligned}$$
	Vậy với $420$ triệu đồng thì doanh nghiệp loại bỏ được $99{,}8\%$ chất gây ô nhiễm trong khí thải.
	}
\end{bt}
%%=====Bài 5
\begin{bt}
	Bạn Hoa dự định dùng hết số tiền $600$ nghìn đồng để mua một số chiếc áo đồng giá tặng các bạn có hoàn cảnh khó khăn. Khi đến cửa hàng, loại áo mà bạn Hoa dự định mua được giảm giá $30$ nghìn đồng/chiếc. Do vậy bạn hoa đã mua được số lượng áo gấp $1{,}25$ lần so với số lượng dự định. Tính giá tiền mỗi chiếc áo mà bạn Hoa đã mua.
	\loigiai{
	Gọi giá tiền ban đầu mỗi chiếc áo là $x$ (nghìn đồng) ($x > 30$).\\
	Số áo Hoa dự định mua được là $\dfrac{600}{x}$ (chiếc).\\
	Số tiền mỗi chiếc áo sau khi giảm giá là $x-30$ (nghìn đồng). \\
	Số áo Hoa mua được sau khi giảm giá là $\dfrac{600}{x-30}$ (chiếc).\\
	Theo đề bài số áo thực tế mua được gấp $1{,}25 = \dfrac{5}{4}$ lần số áo dự định mua được. Do đó, ta có phương trình 
	$$\begin{aligned}[t]
	&\dfrac{600}{x-30} = \dfrac{5}{4} \cdot \dfrac{600}{x} \\
	&\dfrac{600 \cdot 4 \cdot x}{4x(x-30)} =\dfrac{5 \cdot 600 \cdot (x-30)}{4x(x-30)} \\
	&2\ 400x = 3\ 000(x-30) \\
	&2\ 400x = 3\ 000x - 90\ 000 \\
	&600x = 90\ 000 \\
	&x = 150 ~(\text{thỏa mãn } x > 30).
	\end{aligned}$$
	Vậy giá tiền mỗi chiếc áo mà bạn Hoa đã mua là $150 - 30 = 120$ nghìn đồng.
	}
\end{bt}
%%=====Bài 4
\begin{bt}
	Một người đi xep đạp từ A đến B cách nhau $60\mathrm{~km}$. Sau $1$ giờ $40$ phút, một xe máy cũng đi từ A đến B và đến B sớm hơn xe đạp $1$ giờ. Tính tốc độ của mỗi xe, biết rằng tốc độ của xe máy gấp $3$ lần tốc độ của xe đạp.
	\loigiai{
	Gọi tốc độ của xe đạp là $x\mathrm{~(km/h)}$ ($x>0$).\\
	Vì tốc độ xe máy gấp $3$ lần tốc độ xe đạp\\
	nên tốc độ xe máy là: $3x\mathrm{~(km/h)}$.\\
	Thời gian người đi xe đạp từ A đến B là: $\dfrac{60}{x}$ (giờ).\smallskip\\
	Thời gian người đi xe máy từ A đến B là: $\dfrac{60}{3x}$ (giờ).\\
	Vì người đi xe máy đến sớm hơn người đi xe đạp $1$ giờ nên ta có phương trình:\\
	$$
	\begin{aligned}[t]
	&\dfrac{60}{x}-\dfrac{60}{3x} = 1+1+\dfrac{2}{3}\medskip\\
	&\dfrac{60}{x}-\dfrac{60}{3x} = \dfrac{8}{3}\medskip\\
	&\dfrac{180}{3x}-\dfrac{60}{3x} = \dfrac{8x}{3x}\\
	&180-60 = 8x\\
	&8x = 120\\
	&x = 15\ (\text{thỏa mãn điều kiện}).
	\end{aligned}$$
	Vậy tốc độ của người đi xe đạp là $15\mathrm{~(km/h)}$, tốc độ của người đi xe máy là $45\mathrm{~(km/h)}$.
	}
\end{bt}
%%=====Bài 5
\begin{bt}
	Một xí nghiệp dự đinh chia đều $12\ 600\ 000$ đồng để thưởng cho các công nhân tham gia hội thao nhân ngày thành lập xí nghiệp. Khi đến ngày hội thao chỉ có $80\%$ số công nhân tham gia, vì thế mỗi người tham gia hội thao được nhận thêm $105\ 000$ đồng. Tính số công nhân dự định tham gia lúc đầu.
	\loigiai{
	Gọi số công nhân dự định tham gia lúc đầu là $x$ ($x \in \mathbb{N^*}$).\\
	Số công nhân tham gia hội thao là: $0,8x$.\\
	Số tiền dự định lúc đầu mỗi công nhân nhận được là: $\dfrac{12\ 600\ 000}{x}$ (đồng).
	Số tiền mỗi công nhân nhận được khi tham gia hội thao là: $\dfrac{12\ 600\ 000}{0,8x}$.\\
	Vì mỗi công nhân tham gia hội thao được nhận thêm $105\ 000$ đồng nên ta có phương trình:
	$$
	\begin{aligned}[t]
	&\dfrac{12\ 600\ 000}{0,8x}-\dfrac{12\ 600\ 000}{x}=105\ 000\\
	&\dfrac{12\ 600\ 000}{0,8x}-\dfrac{12\ 600\ 000 \cdot 0,8}{0,8x} = \dfrac{105\ 000 \cdot 0,8x}{0,8x}\\
	&12\ 600\ 000 - 10\ 080\ 000 = 84\ 000x\\
	&84\ 000x = 2\ 520\ 000\\
	&x = 30\ (\text{thỏa mãn điều kiện}).\\
	\end{aligned}
	$$
	Vậy số công nhân dự định tham gia lúc đầu là $30$ công nhân.
	}
\end{bt}
%---------------
\begin{bt}%[8D3K5]
	Cho phương trình $\dfrac{1}{x + 1}-\dfrac{2 x^2- m}{x^3+ 1}=\dfrac{4}{x^2- x + 1}$. Biết $x = 0$ là một nghiệm của phương trình. Tìm các nghiệm còn lại.
	\loigiai{
		\begin{itemize}
			\item Thay $x = 0$ vào phương trình, ta được:
			$$\dfrac{1}{1}-\dfrac{0 - m}{0 + 1}=\dfrac{4}{0 - 0 + 1}\Rightarrow  1 + m = 4  m = 3.$$
			\item Thay $m = 3$ vào phương trình ta được:
			$$\begin{aligned}[t]
				&\dfrac{1}{x + 1}-\dfrac{2 x^2- 3}{x^3+ 1}=\dfrac{4}{x^2- x + 1} \hspace{0.2cm} (\text{ĐKXĐ}\ x \neq -1)\\
				\Rightarrow\ &x^2 - x + 1 - 2x^2 + 3 = 4(x + 1)\\
				\ & -x^2 - 5x = 0\\
				\ & \hoac{&x = 0\\&x = -5} \text{(thỏa mãn ĐKXĐ).}
			\end{aligned}$$
			Vậy $x = -5$ là nghiệm còn lại của phương trình.
		\end{itemize}
	}
\end{bt}
\begin{bt}%[8D3K5]
	Cho hai biểu thức $A =\dfrac{x + 1}{2 x - 3}$; $B =\dfrac{3 x}{x^2- 4}$, với giá trị nào của $x$ thì hai biểu thức $A$ và $B$ có cùng một giá trị?
	\loigiai{
		$$A = B\Rightarrow\dfrac{x + 1}{2 x - 3}=\dfrac{3 x}{x^2- 4}. $$
		ĐKXĐ: $x \neq \pm 2$ và $x \neq \dfrac{3}{2}$. Biến đổi phương trình về dạng:\\
		$$\begin{aligned}[t]
			&x^3- 5 x^2+ 5 x - 4 = 0\\
			\ & x^3- 4 x^2- x^2+ 4 x + x - 4 = 0\\
			\ &(x - 4)\left(x^2- x + 1\right) = 0\\
			\ &x = 4\ \text{(thỏa mãn ĐKXĐ)}.\\
		\end{aligned}$$
		Vậy $x = 4$ thì $A=B$.
	} 
\end{bt}
\subsection{CÂU HỎI TRẮC NGHIỆM}
\Opensolutionfile{ans}{ans/ans-D110}
\begin{ex}
	Nội dung câu hỏi
	\choice
	{Phương án A}
	{Phương án B}
	{Phương án C}
	{\True Phương án D}
	\loigiai{Nội dung lời giải}
\end{ex}
\begin{ex}
	Nội dung câu hỏi
	\choice
	{Phương án A}
	{Phương án B}
	{Phương án C}
	{\True Phương án A}
	\loigiai{Nội dung lời giải}
\end{ex}
\Closesolutionfile{ans}
