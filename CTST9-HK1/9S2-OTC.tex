\section*{ÔN TẬP CHƯƠNG II}
\subsection{Bài tập tự luận}
%%=====Bài 2.26
\begin{bt}
	Tìm $x$ sao cho:
	\begin{enumerate}
	\item[a)] Giá trị của biểu thức $2x+1$ không nhỏ hơn giá trị của biểu thức $3x-5$;
	\item[b)] Giá trị của biểu thức $2x+1$ không lớn hơn giá trị của biểu thức $3x-5$.
	\end{enumerate}
	\loigiai{
	\begin{enumerate}
	\item[a)] Giá trị của biểu thức $2x+1$ không nhỏ hơn giá trị của biểu thức $3x-5$,
	nên ta có $$2x+1\ge3x-5\Rightarrow x\le 6.$$
	\item[b)] Giá trị của biểu thức $2x+1$ không lớn hơn giá trị của biểu thức $3x-5$, nên ta có $$2x+1\le3x-5\Rightarrow x\ge 6.$$
	\end{enumerate}
	}
\end{bt}
\begin{bt}
	Cho bất đẳng thức $a>b$. Kết luận nào sau đây là \textbf{không} đúng?
	\begin{listEX}[4]
	\item $2a>2b$.
	\item $-a<-b$.
	\item $a-3 < b-3$.
	\item $a-b>0$.
	\end{listEX}
	\loigiai{
	Kết luận không đúng là c. Vì $a>b$ thì $a-3>b-3$.
	}
\end{bt}
%%%%%%%%%%% so sánh
\begin{bt}
	Cho $a < b$, hãy so sánh
	\begin{listEX}[2]
	\item $a + b + 5$ với $2b + 5$.
	\item $-2a - 3$ với $-(a + b) - 3$.
	\end{listEX}
	\loigiai{
	\begin{listEX}[2]
	\item Ta có $\begin{aligned}[t]
	a &< b\\
	a + b + 5 &< b + b + 5\\
	a + b + 5 &< 2b + 5.
	\end{aligned}$\\
	Vậy $a + b + 5 < 2b + 5$.
	\item Ta có $\begin{aligned}[t]
	a &< b\\
	-a &> -b\\
	-a -a &> -b -a\\
	-2a -3 &> -(a+b) -3.
	\end{aligned}$\\
	Vậy $-2a - 3 > -(a+b) -3$.
	\end{listEX}
	}
\end{bt}
%%%%%%%%%%% Chứng minh
\begin{bt}
	Cho $a>b$, chứng minh:
	\begin{enumEX}{4}
	\item $a-2>b-2$;
	\item $-5 a<-5 b$;
	\item $2 a+3>2 b+3$;
	\item $10-4 a<10-4 b$.
	\end{enumEX}
	\loigiai{
	\begin{enumerate}
	\item Ta có $a>b$. Cộng cả hai vế của bất đẳng thức với $-2$, ta được
	\[a+(-2)> b+(-2) \text{ hay } a-2>b-2.\]
	\item Ta có $a>b$. Nhân cả hai vế của bất đẳng thức với $-5<0$, ta được
	\[a \cdot (-5)< b \cdot (-5) \text{ hay } -5a<-5b.\]
	\item Ta có $a>b$. Nhân cả hai vế của bất đẳng thức với $2>0$, ta được
	\[a \cdot 2> b \cdot 2 \text{ hay } 2a>2b.\]
	Cộng cả hai vế của bất đẳng thức $2a>2b$ với $3$, ta được
	\[2a+3>2b+3.\]
	\item Ta có $a>b$. Nhân cả hai vế của bất đẳng thức với $-4<0$, ta được
	\[a \cdot (-4)< b \cdot (-4) \text{ hay } -4a<-4b.\]
	Cộng cả hai vế của bất đẳng thức $-4a<-4b$ với $10$, ta được
	\[-4a+10<-4b+10 \text{ hay } 10-4a<10-4b.\]
	\end{enumerate}
	}
\end{bt}
\begin{bt}
	Chứng minh:
	\begin{listEX}[2]
	\item Nếu $a>5$ thì $\dfrac{a-1}{2} - 2>0$;
	\item Nếu $b>7$ thì $4 - \dfrac{b+3}{5} < 2$;
	\end{listEX}
	\loigiai{
	\begin{enumerate}
	\item Ta có 
	$$\begin{aligned}[t]
	\dfrac{a-1}{2} - 2 & > 0\\
	\dfrac{a-1-4}{2} & > 0\\
	\dfrac{a-5}{2}& > 0\\
	\end{aligned}$$
	Vì $a>5$ nên $a-5>0$ và $2>0$, suy ra $\dfrac{a-5}{2}>0$.\\
	Vậy ta có điều phải chứng minh.
	\item Ta có
	$$\begin{aligned}[t]
	4 - \dfrac{b+3}{5} & < 2\\
	2- \dfrac{b+3}{5} & < 0\\
	\dfrac{10-b-3}{5} & < 0\\
	\dfrac{7-b}{5} & < 0.
	\end{aligned}$$
	Vì $b>7$ nên $7-b<0$ và $5>0$, suy ra $\dfrac{7-b}{5}<0$.\\
	Vậy ta có điều phải chứng minh.
	\end{enumerate}
	}
\end{bt}
\begin{bt}
	Cho $4{,}2<a<4{,}3$. Chứng minh $13{,}8<3a+1{,}2<14{,}1$.
	\loigiai{
	Ta có
	\begin{eqnarray*}
	4{,}2<&a&<4{,}3\\
	12{,}6<&3a&<12{,}9\\
	13{,}8<&3a +1{,}2 &<14{,}1.
	\end{eqnarray*}
	Vậy ta có điều phải chứng minh.
	}
\end{bt}
%%=====Bài 5
\begin{bt}
	Cho $a \geq 2$. Chứng minh:
	\begin{listEX}[2]
	\item $a^2 \geq 2a$;
	\item $(a+1)^2 \geq 4a+1$.
	\end{listEX}
	\loigiai{
	\begin{enumerate}
	\item Ta có
	$$\begin{aligned}[t]
	a^2 & \geq 2a\\
	a^2 -2a & \geq 0\\
	a(a-2) &\geq 0.
	\end{aligned}$$
	Vì $a \geq 2$ nên $a>0$ và $a-2 \geq 0$, suy ra $a(a-2)\geq0$ đúng.\\
	Vậy ta có điều phải chứng minh.
	\item Ta có
	$$\begin{aligned}[t]
	(a+1)^2 & \geq 4a+1\\
	a^2 + 2a +1 -4a - 1 & \geq 0\\
	a^2 - 2a & \geq 0\\
	a(a-2) &\geq 0.
	\end{aligned}$$
	Vì $a \geq 2$ nên $a>0$ và $a-2 \geq 0$, suy ra $a(a-2)\geq0$ đúng.\\
	Vậy ta có điều phải chứng minh.
	\end{enumerate}
	}
\end{bt}
\begin{bt}
	Chứng minh nửa chu vi của một tam giác lớn hơn độ dài mỗi cạnh của tam giác đó.
	\loigiai{
	Gọi độ dài 3 cạnh của một tam giác lần lượt là $a$, $b$, $c$ ($a$, $b$, $c>0$).\\
	Theo bất đẳng thức tam giác ta có: $a+b>c$, $b+c>a$, $c+a>b$.\\
	Nửa chu vi của tam giác: $p=\dfrac{a+b+c}{2} > \dfrac{a+a}{2} = a$.\\
	Tương tự, ta có $p>b$, $p>c$.\\
	Vậy ta có điều phải chứng minh.
	}
\end{bt}
\begin{bt}%Bài 4
	Cho $a>b$, chứng minh rằng:
	\begin{listEX}[2]
	\item $4 a+4>4 b+3$;
	\item $1-3 a<3-3 b$.
	\end{listEX}
	\loigiai{%
	\begin{enumerate}
	\item Từ $a>b$, ta có $4 a>4 b$. Suy ra $4 a+4<4 b+4$. \hfill(1)\\
	Vì $4>3$ nên $4b+4>4b+3$. \hfill(2)\\
	Theo tính chất bắc cầu, từ (1) và (2) suy ra $4 a+4>4b+3$.
	\item Từ $a>b$, ta có $-3 a<-3 b$. Suy ra $-3a+1<-3b+1$. \hfill(3)\\
	Vì $1<3$ nên $-3b+1>-3b+3$. \hfill(4)\\
	Theo tính chất bắc cầu, từ (3) và (4) suy ra $1-3a<3-3b$.
	\end{enumerate}	
	}
\end{bt}
\begin{bt}
	Cho $a<b$. Chứng minh rằng:
	\begin{listEX}[2]
	\item $2 a+1<2 b+2$;
	\item $-2 a-5>-2 b-7$.
	\end{listEX}
	\loigiai{%
	\begin{enumerate}
	\item Từ $a<b$, ta có $2 a<2 b$. Suy ra $2 a+1<2 b+1$. \hfill (1)\\
	Vì $1<2$ nên $2 b+1<2 b+2$. \hfill (2)\\
	Theo tính chất bắc cầu, từ (1) và (2) suy ra $2 a+1<2 b+2$.
	\item Từ $a<b$, ta có $-2 a>-2 b$. Suy ra $-2 a-5>-2 b-5$. \hfill (3)\\
	Vì $-5>-7$ nên $-2 b-5>-2 b-7$. \hfill (4) \\
	Theo tính chất bắc cầu, từ (3) và (4) suy ra $-2 a-5>-2 b-7$.
	\end{enumerate}
	}
\end{bt}
%%%%%%%%%%% Giải
\begin{bt}
	Giải các bất phương trình:
	\begin{listEX}[4]
	\item $5+7 x \leq 11$;
	\item $2{,}5 x-6>9+4 x$;
	\item $2 x-\dfrac{x-7}{3}<9$;
	\item $\dfrac{3 x+5}{2}+\dfrac{x}{5}-0{,}2 x \geq 4$.
	\end{listEX}
	\loigiai{
	\begin{listEX}[2]
	\item \allowdisplaybreaks
	$\begin{aligned}[t]
	5+7 x &\leq 11\\
	7x&\leq11-5\\
	7x&\leq6\\
	x&\leq6:7\\
	x&\leq\dfrac{6}{7}.
	\end{aligned}$\\
	Vậy bất phương trình đã cho có nghiệm $x\leq\dfrac{6}{7}.$
	\item \allowdisplaybreaks
	$\begin{aligned}[t]
	2{,}5 x-6&>9+4 x\\
	\dfrac{5}{2}x-4x&>9+6\\
	\dfrac{-3}{2}x&>15\\
	x&<15:\dfrac{-3}{2}\\
	x&<-10
	\end{aligned}$\\
	Vậy bất phương trình đã cho có nghiệm $x<-10$.
	\item \allowdisplaybreaks
	$\begin{aligned}[t]
	2 x-\dfrac{x-7}{3}&<9\\
	6x-(x-7)&<27\\
	6x-x+7&<27\\
	5x&<27-7\\
	5x&<20\\
	x&<20:5\\
	x&<4.
	\end{aligned}$\\
	Vậy bất phương trình đã cho có nghiệm $x<4$.
	\item \allowdisplaybreaks
	$\begin{aligned}[t]
	\dfrac{3 x+5}{2}+\dfrac{x}{5}-0{,}2 x &\geq 4\\
	\dfrac{3x+5}{2}+\dfrac{x}{5}-\dfrac{1}{5}x&\geq 4\\
	\dfrac{3x+5}{2}&\geq 4\\
	3x+5&\geq 8\\
	3x&\geq 8-5\\
	3x&\geq 3\\
	x&\geq 3:3\\
	x&\geq 1.
	\end{aligned}$\\
	Vậy bất phương trình đã cho có nghiệm $x\geq 1$.
	\end{listEX}
	}
\end{bt}
\begin{bt}
	Giải các bất phương trình sau:
	\begin{enumEX}{4}
	\item $3-0{,}2x<13$.
	\item $3<\dfrac{2x-2}{8}$.
	\item $\dfrac{1}{2}+\dfrac{x}{3}\ge\dfrac{1}{4}$.
	\item $\dfrac{2x-3}{3}\le\dfrac{3x-2}{4}$.
	\end{enumEX}
	\loigiai{
	\begin{enumEX}{2}
	\item 
	$\begin{aligned}[t]
	3-0{,}2x & <13\\
	-0{,}2x &<10\\
	x &>-50.
	\end{aligned}$\\
	Vậy bất phương trình đã cho có nghiệm $x>-50$.
	\item 
	$\begin{aligned}[t]
	3&<\dfrac{2x-2}{8}\\
	2x-2 & >24\\
	2x& >26\\
	x & >13.
	\end{aligned}$\\
	Vậy bất phương trình đã cho có nghiệm $x>13$
	\item
	$\begin{aligned}[t]
	\dfrac{1}{2}+\dfrac{x}{3}&\ge\dfrac{1}{4}\\
	\dfrac{x}{3} & \ge\dfrac{-1}{4}\\
	x & \ge\dfrac{-3}{4}.
	\end{aligned}$\\
	Vậy bất phương trình đã cho có nghiệm $x \ge\dfrac{-3}{4}$.
	\item 
	$\begin{aligned}[t]
	\dfrac{2x-3}{3}&\le\dfrac{3x-2}{4}\\
	4\left(2x-3\right) &\le 3\left(3x-2\right)\\
	8x-12 &\le 9x-6\\
	-x &\le 6\\
	x &\ge -6.
	\end{aligned}$\\
	Vậy bất phương trình đã cho có nghiệm $x\geq -6$.	
	\end{enumEX}
	}
\end{bt}
\begin{bt}
	Giải các bất phương trình
	\begin{listEX}[2]
	\item $2x + 3(x + 1) > 5x - (2x - 4)$.
	\item $(x + 1)(2x - 1) < 2x^2 - 4x + 1$.
	\end{listEX}
	\loigiai{
	\begin{listEX}[2]
	\item Ta có $\begin{aligned}[t]
	2x + 3(x + 1) &> 5x - (2x - 4)\\
	5x + 3 &> 3x + 4\\
	2x &> 1\\
	x &> \dfrac{1}{2}.
	\end{aligned}$\\
	Vậy bất phương trình đã cho có nghiệm $x > \dfrac{1}{2}$.
	\item Ta có $\begin{aligned}[t]
	(x + 1)(2x - 1) &< 2x^2 - 4x + 1\\
	2x^2 - x &< 2x^2 - 4x + 1\\
	3x &< 1\\
	x &< \dfrac{1}{3}.
	\end{aligned}$\\
	Vậy bất phương trình đã cho có nghiệm $x < \dfrac{1}{3}$.
	\end{listEX}
	}
\end{bt}
%%=====Bài 2.28
%%=====Bài 2.29
%%=====Bài 2.30
%%=====Bài 2.31
\begin{bt}
	Thanh tham dự một kì kiểm tra năng lực tiếng Anh gồm $4$ bài kiểm tra nghe, nói, đọc và viết. Mỗi bài kiểm tra có điểm là số nguyên từ $0$ đến $10$. Điểm trung bình của ba bài kiểm tra nghe, nói, đọc của Thanh là $6{,}7$. Hỏi bài kiểm tra viết của Thanh cần được bao nhiêu điểm để điểm trung bình cả $4$ bài kiểm tra được từ $7{,}0$ trở lên? Biết điểm trung bình được tính gần đúng đến chữ số thập phân thứ nhất.
	\loigiai{
	Gọi $x$ (điểm) là số điểm của bài kiểm tra viết của Thanh ($0 \leq x \leq 10$).\\
	Vì điểm trung bình của ba bài kiểm tra nghe, nói, đọc của Thanh là $6{,}7$ nên tổng điểm ba bài kiểm tra này là $6{,}7\cdot 3 = 20{,}1$.\\
	Điểm trung bình của bốn bài kiểm tra là $\dfrac{20{,}1 + x}{4}$.\\
	Để điểm trung bình cả $4$ bài kiểm tra được từ $7{,}0$ trở lên thì
	\begin{eqnarray*}
	\dfrac{20{,}1 + x}{4} &>& 7{,}0\\
	20{,}1 + x &>& 28\\
	x &>& 7{,}9.
	\end{eqnarray*}
	Vậy bài kiểm tra viết của Thanh cần trên $7{,}9$ điểm để điểm trung bình cả $4$ bài kiểm tra được từ $7{,}0$ trở lên.
	}
\end{bt}
%%=====Bài 2.32
\begin{bt}
	Để lập đội tuyển năng khiếu về bóng rổ của trường, thầy thể dục đưa ra quy định tuyển chọn như sau: mỗi bạn dự tuyển sẽ được ném $15$ quả bóng vào rổ, quả bóng vào rổ được cộng $2$ điểm; quả bóng ném ra ngoài bị trừ $1$ điểm. Nếu bạn nào có số điểm từ $15$ điểm trở lên thì sẽ được chọn vào đội tuyển. Hỏi một học sinh muốn được chọn vào đội tuyển thì phải ném ít nhất bao nhiêu quả vào rổ?
	\loigiai{
	Gọi $x$ (quả) là số quả bóng được ném vào rổ ($0 < x \leq 15$).\\
	Theo đề bài, ta có
	\begin{eqnarray*}
	x\cdot 2 + (15-x)\cdot (-1) &\geq& 15\\
	3x - 15 &\geq& 15\\
	3x &\geq& 30\\
	x &\geq& 10 \textrm{ (nhận).}
	\end{eqnarray*}
	Vậy một học sinh muốn được chọn vào đội tuyển thì phải ném ít nhất $10$ quả vào rổ.
	}
\end{bt}
%%%%%%%%%%%%%
%%=====Bài 6
%%=====Bài 7
%%=====Bài 8
%%=====Bài 9
\begin{bt}
	Trong cuộc thi ``Đố vui để học'', mỗi thí sinh phải trả lời $12$ câu hỏi của ban tổ chức. Mỗi câu hỏi gồm bốn phương án, trong đó chỉ có một phương án đúng. Với mỗi câu hỏi, nếu trả lời đúng thì được cộng $5$ điểm, trả lời sai bị trừ $2$ điểm. Khi bắt đầu cuộc thi mỗi thí sinh có sẵn $20$ điểm. Thí sinh nào đạt từ $50$ điểm trở lên sẽ được vào vòng tiếp theo. Hỏi thí sinh phải trả lời đúng ít nhất bao nhiêu câu thì được vào vòng thi tiếp theo?
	\loigiai{
	Gọi $x$ là số câu trả lời đúng, $\left(12-x\right)$ là số câu trả lời sai $\left(0\le x\le 12{,}\,x\in\mathbb{N}\right)$.\\
	Để thí sinh được vào vòng tiếp theo thì ta có
	\begin{eqnarray*}
	&20+5x-2\left(12-x\right)&\ge50\\
	&20+5x-24+2x&\ge50\\
	&7x-4 &\ge50\\
	&x &\ge\dfrac{54}{7}\approx 7{,}714.
	\end{eqnarray*}
	Vậy thí sinh muốn vào vòng tiếp theo cần trả lời đúng $8$ câu hỏi trở lên.
	}
\end{bt}
\begin{bt}
	Để đổi từ độ Fahrenheit (độ $F$) sang độ Celsius (độ $C$), người ta dùng công thức sau:
	\[C=\dfrac{5}{9}(F-32).\]
	\begin{enumerate}
	\item Giả sử nhiệt độ ngoài trời của một ngày mùa hè ít nhất là $95^{\circ} F$. Hỏi nhiệt độ ngoài trời khi đó ít nhất là bao nhiêu độ $C$ ?
	\item Giả sử nhiệt độ ngoài trời của một ngày mùa hè ít nhất là $36^{\circ} C$. Hỏi nhiệt độ ngoài trời khi đó ít nhất là bao nhiêu độ $F$ ?
	\end{enumerate}
	\loigiai{
	\begin{enumerate}
	\item Ta có $C=\dfrac{5}{9}(F-32)\Rightarrow F=\dfrac{9}{5}C+32$.\\
	Gọi $x$ là nhiệt độ (theo độ $C$) ngoài trời khi đó, ta có bất phương trình
	\allowdisplaybreaks
	\begin{eqnarray*}
	\dfrac{9}{5}x+32&\geq&95\\
	\dfrac{9}{5}x&\geq&95-32\\
	\dfrac{9}{5}x&\geq&63\\
	x&\geq&63:\dfrac{9}{5}\\
	x&\geq&35.
	\end{eqnarray*}
	Vậy nhiệt độ ngoài trời khi đó ít nhất là $35$ độ $C$.
	\item Gọi $y$ là nhiệt độ ngoài trời của ngày mùa hè (theo độ $F$), ta có bất phương trình.
	\allowdisplaybreaks
	\begin{eqnarray*}
	\dfrac{5}{9}(y-32)&\geq&36\\
	y-32&\geq&36:\dfrac{5}{9}\\
	y-32&\geq&\dfrac{324}{5}\\
	y&\geq&\dfrac{324}{5}+32\\
	y&\geq&\dfrac{484}{5}\\
	y&\geq&96{,}8.
	\end{eqnarray*}
	Vậy nhiệt độ ngoài trời khi đó ít nhất là $96{,}8$ độ $F$.
	\end{enumerate}
	}
\end{bt}
\begin{bt}
	Một nhà máy sản xuất xi măng mỗi ngày đều sản xuất được $100$ tấn xi măng. Lượng xi măng tồn trong kho của nhà máy là $300$ tấn. Hỏi nhà máy đó cần ít nhất bao nhiêu ngày để có thể xuất đi $15~300$ tấn xi măng (tính cả lượng xi măng tồn trong kho)?
	\loigiai{
	Gọi $x$ là số ngày cần để nhà máy đó có thể xuất đi $15~300$ tấn xi măng ($x\in\mathbb{N^*}$).\\
	Theo đề bài, ta có bất phương trình sau
	\allowdisplaybreaks
	\begin{eqnarray*}
	300+100x&\geq&15~300\\
	100x&\geq&15~300-300\\
	100x&\geq&15~000\\
	x&\geq&15~000:100\\
	x&\geq&150.
	\end{eqnarray*}
	Vậy nhà máy đó cần ít nhất $150$ ngày để có thể xuất đi $15~300$ tấn xi măng.
	}
\end{bt}
\begin{bt}
	Đến ngày $31 / 12 / 2022$, gia đình bác Hoa đã tiết kiệm được số tiền là $250$ triệu đồng. Sau thời điểm đó, mỗi tháng gia đình bác Hoa đều tiết kiệm được $10$ triệu đồng. Gia đình bác Hoa dự định mua một chiếc ô tô tải nhỏ để vận chuyển hàng hoá với giá tối thiểu là $370$ triệu đồng. Hỏi sau ít nhất bao nhiêu tháng gia đình bác Hoa có thể mua được chiếc ô tô tải đó bằng số tiền tiết kiệm được?
	\loigiai{
	Gọi $x$ là số tháng để gia đình bác Hoa tiết kiệm đủ tiền mua chiếc ô tô tải.\\
	Theo đề bài, ta có bất phương trình
	\allowdisplaybreaks
	\begin{eqnarray*}
	250+10x&\geq&370\\
	10x&\geq&370-250\\
	10x&\geq&120\\
	x&\geq&120:10\\
	x&\geq&12.
	\end{eqnarray*}
	Vậy sau ít nhất $12$ tháng gia đình bác Hoa có thể mua được chiếc ô tô tải đó bằng số tiền tiết kiệm được.
	}
\end{bt}
\begin{bt}
	Chỉ số khối cơ thể, thường được biết đến với tên viết tắt $\mathrm{BMI}$ (tiếng Anh là Body Mass Index) cho phép đánh giá thể trạng của một người là gầy, bình thường hay béo. Chỉ số khối cơ thể của một người được tính theo công thức sau
	\[\mathrm{BMI} =\dfrac{m}{h^2}\]
	trong đó $m$ là khối lượng cơ thể tính theo kilôgam, $h$ là chiều cao tính theo mét.\\
	Dưới đây là bảng đánh giá thể trạng ở người lớn theo $\mathrm{BMI}$ đối với khu vực châu Á -- Thái Bình Dương
	\begin{center}
	\begin{tabular}{|l|l|}
	\hline \multicolumn{1}{|c|}{ Nam } & \multicolumn{1}{|c|}{ Nữ } \\
	\hline $\mathrm{BMI} < 20$: Gầy & $\mathrm{BMI} < 18$: Gầy \\
	$20 \leq \mathrm{BMI} < 25$: Bình thường & $18 \leq \mathrm{BMI} <23$: Bình thường \\
	$25 \leq \mathrm{BMI} < 30$: Béo phì độ I (nhẹ) & $23 \leq \mathrm{BMI} <30$: Béo phì độ I (nhẹ) \\
	$30 \leq \mathrm{BMI} < 40$: Béo phì độ II (trung bình) & $30 \leq \mathrm{BMI} <40$: Béo phì độ II (trung bình) \\
	$40 \leq \mathrm{BMI}$: Béo phì độ III (nặng) & $40 \leq \mathrm{BMI}$: Béo phì độ III (nặng) \\
	\hline
	\end{tabular}
	\end{center}
	\begin{enumerate}
	\item Giả sử một người đàn ông có chiều cao $1{,}68~\mathrm{m}$. Hãy lập bảng về chỉ số cân nặng của người đó dựa theo bảng đánh giá thể trạng trên.
	\item Giả sử một người phụ nữ có chiều cao $1{,}6~\mathrm{m}$. Hãy lập bảng về chỉ số cân nặng của người đó dựa theo bảng đánh giá thể trạng trên.
	\end{enumerate}
	\loigiai{
	\begin{enumerate}
	\item Ta có $\mathrm{BMI}=\dfrac{m}{h^2}\Rightarrow m=\mathrm{BMI}\cdot h^2$. Từ đó ta lập được bảng về thể trạng dựa trên chỉ số cân nặng của một người đàn ông cao $1{,}68\mathrm{~m}$ như sau
	\begin{center}
	\begin{tabular}{|c|c|}
	\hline
	Cân nặng ($\mathrm{kg}$)&Thể trạng\\
	\hline
	$m<56{,}448$&Gầy\\
	\hline
	$56{,}448\leq m<70{,}56$&Bình thường\\
	\hline
	$70{,}56\leq m<$&Béo phì độ I (nhẹ)\\
	\hline
	$84{,}672\leq m<112{,}896$&Béo phì độ II (trung bình)\\
	\hline
	$112{,}896\leq m$&Béo phì độ III (nặng)\\
	\hline
	\end{tabular}
	\end{center}
	\item Ta lập được bảng về thể trạng dựa trên chỉ số cân nặng của một người phụ nữ $1{,}6\mathrm{~m}$ như sau
	\begin{center}
	\begin{tabular}{|c|c|}
	\hline
	Cân nặng ($\mathrm{kg}$)&Thể trạng\\
	\hline
	$m<46{,}08$&Gầy\\
	\hline
	$46{,}08\leq m<58{,}88$&Bình thường\\
	\hline
	$58{,}88\leq m<76{,}8$&Béo phì độ I (nhẹ)\\
	\hline
	$76{,}8\leq m<102{,}4$&Béo phì độ II (trung bình)\\
	\hline
	$102{,}4\leq m$&Béo phì độ III (nặng)\\
	\hline
	\end{tabular}
	\end{center}
	\end{enumerate}
	}
\end{bt}
%%%%%%%%%%%%%%%%
\subsection{Bài tập trắc nghiệm}
\Opensolutionfile{ans}[ans/9T2-OTC]
%%=====Bài 2.21
\begin{ex}
	Nghiệm của bất phương trình $-2x + 1 < 0$ là
	\choice
	{$x < \dfrac{1}{2}$}
	{\True $x > \dfrac{1}{2}$}
	{$x \leq \dfrac{1}{2}$}
	{$x \geq \dfrac{1}{2}$}
	\loigiai{
	Ta có $\begin{aligned}[t]
	-2x + 1 &< 0 \\
	-2x &< -1\\
	x &> \dfrac{1}{2}.
	\end{aligned}$\\
	Vậy nghiệm của bất phương trình $-2x + 1 < 0$ là $x > \dfrac{1}{2}$.
	}
\end{ex}
%%=====Bài 2.24
\begin{ex}
	Nghiệm của bất phương trình $1 - 2x \geq 2 - x$ là
	\choice
	{$x > \dfrac{1}{2}$}
	{$x < \dfrac{1}{2}$}
	{\True $x \leq -1$}
	{$x \geq -1$}
	\loigiai{
	Ta có $\begin{aligned}[t]
	1 - 2x &\geq 2 - x\\
	x &\leq -1.
	\end{aligned}$\\
	Vậy nghiệm của bất phương trình $1 - 2x \geq 2 - x$ là $x \leq -1$.
	}
\end{ex}
%%=====Bài 2.25
\begin{ex}
	Cho $a > b$. Khi đó, ta có
	\choice
	{$2a > 3b$}
	{$2x > 2b + 1$}
	{\True $5a + 1 > 5b + 1$}
	{$-3a < -3b - 3$}
	\loigiai{
	Ta có $\begin{aligned}[t]
	a &> b\\
	5a &> 5b\\
	5a + 1 &> 5b + 1.
	\end{aligned}$
	}
\end{ex}
\begin{ex}
	Bất đẳng thức $n \leq 3$ có thể được phát biểu là
	\choice
	{$n$ lớn hơn $3$}
	{$n$ nhỏ hơn $3$}
	{$n$ không nhỏ hơn $3$}
	{\True $n$ không lớn hơn $3$}
	\loigiai{
	Ta có \lq\lq$n \leq 3$\rq\rq \, nghĩa là \lq\lq$n$ nhỏ hơn hoặc bằng $3$\rq\rq \, hay \lq\lq$n$ không lớn hơn $3$\rq\rq.
	}
\end{ex}
%%=====Câu 2
\begin{ex}
	Cho các số thực $x$, $y$, $z$ biết $x<y$. Khẳng định nào sau đây {\bf sai}?
	\choice
	{$x+z<y+z$}
	{\True $x z<y z$ nếu $z$ âm}
	{$x z<y z$ nếu $z$ dương}
	{$x-z<y-z$}
	\loigiai{
	Khi nhân hai vế của một bất đẳng thức với cùng một số âm thì được một bất đẳng thức mới ngược chiều với bất đẳng thức đã cho.\\
	Do đó $x<y$ thì $xz>yz$ nếu $z$ âm.
	}
\end{ex}
%%=====Câu 3
\begin{ex}
	Hệ thức nào sau đây là bất đẳng thức?
	\choice
	{$1-\mathrm{x}=0$}
	{$x^2-5 x+6=0$}
	{\True $y^2 \geq 0$}
	{$x=y$}
	\loigiai{
	Hệ thức $y^2 \geq 0$ là bất đẳng thức vì có chứa dấu \lq\lq$\geq$\rq\rq.
	}
\end{ex}
%%=====Câu 4
\begin{ex}
	Bất phương trình $3 x-5>4 x+2$ có nghiệm là
	\choice
	{$x>-7$}
	{\True $x<-7$}
	{$x<7$}
	{$x \leq-7$}
	\loigiai{
	Ta có $\begin{aligned}[t]
	3 x-5&>4 x+2\\
	3x-4x&>2+5\\
	-x&>7\\
	x&<-7.
	\end{aligned}$\\
	Vậy nghiệm của bất phương trình $3 x-5>4 x+2$ là $x<-7$.
	}
\end{ex}
%%=====Câu 5
\begin{ex}
	Bất phương trình $2 x-1 \leq x+4$ có nghiệm là
	\choice
	{\True $x \leq 5$}
	{$x \geq 5$}
	{$x \leq-5$}
	{$x<5$}
	\loigiai{
	Ta có $\begin{aligned}[t]
	2 x-1 &\leq x+4\\
	2x-x&\leq 4+1\\
	x&\leq 5.
	\end{aligned}$\\
	Vậy nghiệm của bất phương trình $2 x-1 \leq x+4$ là $x \leq 5$.
	}
\end{ex}
\Closesolutionfile{ans}