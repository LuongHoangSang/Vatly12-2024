\setcounter{section}{2}
\section{GIẢI HỆ HAI PHƯƠNG TRÌNH BẬC NHẤT HAI ẨN}
\subsection{Trọng tâm kiến thức}
\begin{tomtat}
	\subsubsection{Phương pháp thế}
\begin{boxdn}
	Để giải hệ phương trình bằng phương pháp thế, ta thực hiện các bước như sau:
	\begin{itemize}
	\item \textbf{Bước 1:} Từ một phương trình của hệ, biểu diễn một ẩn theo ẩn kia rổi thế vào phương trình còn lại của hệ để được phương trình chỉ còn chứa một ẩn.
	\item \textbf{Bước 2:} Giải phương trình một ẩn vừa nhận được, từ đó suy ra nghiệm của hệ đã cho.
	\end{itemize}
\end{boxdn}
\begin{note}
	Tuỳ theo hệ phương trình, ta có thể lựa chọn cách biểu diễn $x$ theo $y$ hoặc biểu diên $y$ theo $x$. 
\end{note}
\subsubsection{Giải hệ phương trình bằng phương pháp cộng đại số}
\begin{boxdn}
	Để giải hệ phương trình bằng phương pháp cộng đại số, ta thực hiện các bước như sau:
	\begin{itemize}
	\item \textbf{Bước 1:} Nhân hai vế của mỗi phương trình với một số thích hợp (nếu cần) sao cho các hệ số của một ấn nào đó trong hai phương trình của hệ bằng nhau hoặc đối nhau.
	\item \textbf{Bước 2:} Cộng hay trừ từng vế hai phương trình của hệ để được một phương trình một ẩn và giải phương trình đó.
	\item \textbf{Bước 3:} Thế giá trị của ẩn tìm được ở Bước 2 vào một trong hai phương trình của hệ đã cho đế tìm giá tri của ẩn còn lại. Kết luận nghiệm của hệ.
	\end{itemize}
\end{boxdn}
\begin{note}	
	Trường hợp trong hệ phương trình đã cho không có hai hệ số của cùng một ẩn bằng nhau hay đối nhau, ta có thể đưa về trường hợp đã xét bằng cách nhân hai vế của mỗi phương trình với một số thích hợp (khác $0$).
\end{note}
\subsubsection{Giải bài toán bằng cách lập hệ phương trình}
\begin{boxdn}
	Để giải bài toán bằng cách lập hệ hai phương trình bậc nhât hai ẩn, ta thực hiện như sau:
	\begin{itemize}
	\item \textbf{Bước 1:} Lập hệ phương trình.
		\begin{itemize}
		\item Chọn hai ẩn biểu thị hai đại lượng chư biết và đặt điều kiện thích hợp cho các ẩn.
		\item Biểu diễn các đại lượng liên quan theo các ẩn và các đại lượng đã biết.
		\item Lập hệ hai phương trình bậc nhất hai ấn biểu thị mối quan hệ giữa các đại lượng.
		\end{itemize}
	\item \textbf{Bước 2:} Giải hệ phương trình nhận được.
	\item \textbf{Bước 3:} Kiểm tra nghiệm tìm được ở Bước 2 có thoả mãn điều kiện của ẩn hay không, rồi trả lời bài toán.
	\end{itemize}
\end{boxdn}
\end{tomtat}
%%%%%%%%%%%%%%%%%%%%
\subsection{Các dạng bài tập}
\begin{dang}{Giải hệ phương trình bằng phương pháp thế}
\end{dang}
\begin{vd}
	Giải các hệ phương trình sau bằng phương pháp thế
	\begin{listEX}[4]
	\item $\heva{& 2x-y=3 \\ & x+2y=4}$;
	\item $\heva{& x-3y=2 \\ & -2x+5y=1}$;
	\item $\heva{& 4x+y=-1 \\ & 7x+2y=-3}$;
	\item $\heva{& x-y=-2 \\ & 2x-2y=8}$;
	\item $\heva{& -2x+y=3 \\ & 4x-2y=-4}$;
	\item $\heva{& -x+y=-2 \\ & 3x-3y=6}$;
	\item $\heva{& x+3y=-1 \\ & 3x+9y=-3}$;
	\item $\heva{&3x+y=3\\ &-2x-3y=5}$.
	\end{listEX}
	\loigiai{
	\begin{listEX}
	\item
	$\heva{& 2x-y=3 \\ & x+2y=4}$\\
	Từ phương trình thứ nhất của hệ ta có $y=2 x-3$. Thế vào phương trình thứ hai của hệ, ta được $x+2(2 x-3)=4$ hay $5 x-6=4$, suy ra $x=2$.\\
	Từ đó $y=2 \cdot 2-3=1$. Vậy hệ phương trình đã cho có nghiệm là $(2;1)$.
	\item
	$\heva{& x-3y=2 \\ & -2x+5y=1}$$;$\\
	Từ phương trình thứ nhất của hệ ta có $x=2+3y$. Thế vào phương trình thứ hai của hệ, ta được: $-2(2+3y)+5y=1$ hay $-4-y=1$, suy ra $y=-5$.\\
	Từ đó, $x=2+3.(-5)=-13$. Vậy hệ phương trình có nghiệm là $(-13;-5)$\\
	\item $\heva{& 4x+y=-1 \\ & 7x+2y=-3}.$\\
	Từ phương trình thứ nhất của hệ ta có: $y=-1-4x$. Thế vào phương trình thứ hai của hệ, ta được: $7x+2(-1-4x)=-3$ hay $-x=-1$, suy ra $x=1$.\\
	Từ đó, $y=-1-4.(1)=-5$. Vậy hệ phương trình có nghiệm là $(1;-5)$.
	\item $\heva{& x-y=-2 \\ & 2x-2y=8}$\\
	Từ phương trình thứ nhất của hệ ta có: $y=x+2$. Thế vào phương trình thứ hai của hệ, ta được: $2x-2(x+2)=8$ hay $0x-4=8$ $\hspace{1cm} (1)$. \\
	Do không có giá trị nào của $x$ thỏa mãn hệ thức $(1)$ nên hệ phương trình đã cho vô nghiệm.
	\item
	$\heva{& -2x+y=3 \\ & 4x-2y=-4}$\\
	Từ phương trình thứ nhất của hệ ta có: $y=2x+3$. Thế vào phương trình thứ hai của hệ, ta được: $4x-2(2x+3)=-4$ hay $0x-6=-4$ $\hspace{1cm} (1)$. \\
	Do không có giá trị nào của số $x$ thỏa mãn hệ thức $(1)$ nên hệ phương trình đã cho vô nghiệm.
	\item 
	$\heva{& -x+y=-2 \\ & 3x-3y=6}$\\
	Từ phương trình thứ nhất của hệ ta có: $y=x-2$. $\hspace{1cm} (1)$\\
	Thế vào phương trình thứ hai của hệ, ta được: $3x-3(x-2)=6$ hay $0x+6=6$\\
	hay $0x=0$. \hspace{1cm}(2)\\
	Ta thấy mọi giá trị của $x$ đều thỏa mãn $(2)$.\\
	Với mỗi giá trị tùy ý của $x$ , giá trị tương ứng của y được tính bởi $(1)$.\\
	Vậy hệ phương trình đã cho có nghiệm là $(x;x-2)$ với $x\in\mathbb{R}$ tùy ý.	
	\item 
	$\heva{& x+3y=-1 \\ & 3x+9y=-3}$\\
	Từ phương trình thứ nhất của hệ ta có: $x=-1-3y$. $\hspace{1cm} (1)$\\
	Thế vào phương trình thứ hai của hệ, ta được: $3(-1-3y)+9y=-3$ hay $0y-3=-3$\\
	hay $0y=0$. \hspace{1cm}(2)\\
	Ta thấy mọi giá trị của $y$ đều thỏa mãn $(2)$.\\
	Với mỗi giá trị tùy ý của $y$ , giá trị tương ứng của x được tính bởi $(1)$.\\
	Vậy hệ phương trình đã cho có nghiệm là $(-1-3y;y)$ với $y\in\mathbb{R}$ tùy ý.	
	\item 
	$\heva{&3x+y=3&&(1)\\ &-2x-3y=5.&&(2)}$.\\
	Từ phương trình $(1)$, ta có $y=3-3x$ $(3)$.\\
	Thay $y=3-3x$ vào phương trình $(2)$, ta được: $-2x-3(3-3x)=5$.\\
	Giải phương trình này, ta được $x=2$.\\
	Thay $x=2$ vào phương trình $(3)$, ta được $y=-3$.\\
	Vậy hệ phương trình có nghiệm duy nhất là $(2;-3)$.
	\end{listEX}
	}
\end{vd}
%%=====Ví dụ 2
\begin{vd}
	Giải các hệ phương trình sau bằng phương pháp thế: 
	\begin{listEX}[3]
	\item $\heva{&3x+12y=-5\\&x+4y=3}$;
	\item $\heva{&-2x+4y=5\\&-x+2y=1}$;
	\item $\heva{&12x-4y=-16\\&3x-y=-4}$;
	\item $\heva{&x-3y=4\\&-2x+6y=-8}$;
	\item $\heva{&2x+y=5\\&3x-2y=11}$;
	\item $\heva{&x-3y=2\\&-2x+5y=1}$.
	\end{listEX}	
	\loigiai{
	\begin{listEX}[2]
	\item
	Đặt $\heva{&3x+12y=-5 &(1)\\&x+4y=3 &(2)}$\\
	Từ phương trình (2), ta có $x=3-4y$.\hfill(3)\\
	Thay vào phương trình (1), ta được
	\allowdisplaybreaks
	\begin{eqnarray*}
	3\cdot(3-4y)+12y&=&-5\\
	9-12y+12y&=&-5\\
	0y&=&-14\ (\text{vô nghiệm}).
	\end{eqnarray*}
	Vậy hệ phương trình đã cho vô nghiệm.
	\item
	Đặt $\heva{&-2x+4y=5 &(1)\\&-x+2y=1 &(2).}$\\
	Từ phương trình (2), ta có $x=2y-1$.\hfill(3)\\
	Thay vào phương trình (1), ta được
	\allowdisplaybreaks
	\begin{eqnarray*}
	-2\cdot (2y-1)+4y&=&5\\
	-4y+2+4y&=&5\\
	0y&=&3\ (\text{vô nghiệm}).
	\end{eqnarray*}
	Vậy hệ phương trình đã cho vô nghiệm.
	\item
	Đặt $\heva{&12x-4y=-16 &(1)\\&3x-y=-4 &(2)}$\\
	Từ phương trình (2), ta có $y=3x+4$.\hfill(3)\\
	Thay vào phương trình (1), ta được
	\allowdisplaybreaks
	\begin{eqnarray*}
	12x-4\cdot (3x+4)&=&-16\\
	12x-12x-16&=&-16\\
	0x&=&0\ (\text{vô số nghiệm}).
	\end{eqnarray*}
	Vậy hệ phương trình đã cho có vô số nghiệm.
	\item
	Đặt $\heva{&x-3y=4 &(1)\\&-2x+6y=-8 &(2)}$\\
	Từ phương trình (1), ta có $x=3y+4$.\hfill(3)\\
	Thay vào phương trình (2), ta được
	\allowdisplaybreaks
	\begin{eqnarray*}
	-2\cdot (3y+4)+6y&=&-8\\
	-6y-8+6y&=&-8\\
	0y&=&0\ (\text{vô số nghiệm}).
	\end{eqnarray*}
	Vậy hệ phương trình đã cho có vô số nghiệm.
	\item
	Đặt $\heva{&2x+y=5 &(1)\\&3x-2y=11 &(2)}$\\
	Từ phương trình (1), ta có $y=5-2x$.\hfill(3)\\
	Thay vào phương trình (2), ta được
	\allowdisplaybreaks
	\begin{eqnarray*}
	3x-2\cdot (5-2x)&=&11\\
	3x-10+4x&=&11\\
	7x&=&21\\
	x&=&3.
	\end{eqnarray*}
	Thay giá trị $x=3$ vào phương trình (3), ta có
	\[y=5-2\cdot 3=-1. \]
	Vậy hệ phương trình đã cho có nghiệm duy nhất $(x;y)=(3;-1)$.
	\item
	Đặt $\heva{&x-3y=2 &(1)\\&-2x+5y=1 &(2)}$\\
	Từ phương trình (1), ta có $x=3y+2$.\hfill(3)\\
	Thay vào phương trình (2), ta được
	\allowdisplaybreaks
	\begin{eqnarray*}
	-2\cdot (3y+2)+5y&=&1\\
	-6y-4+5y&=&1\\
	-y&=&5\\
	y&=&-5.
	\end{eqnarray*}
	Thay giá trị $y=-5$ vào phương trình (3), ta có
	\[x=3\cdot (-5)+2=-13. \]
	Vậy hệ phương trình đã cho có nghiệm duy nhất $(x;y)=(-13;-5)$.
	\end{listEX}
	}
\end{vd}
\begin{vd}%[9D3B2]
	Giải các phương trình sau bằng phương pháp thế
	\begin{enumEX}{2}
	\item $\begin{cases}
	{5x\sqrt{3} + y = 2\sqrt{2}}\\
	{x\sqrt{6} - y\sqrt{2}= 2;}
	\end{cases}$
	\item $\begin{cases}
	{\sqrt{2}\cdot x - \sqrt{3}\cdot y = 1}\\
	{x + \sqrt{3}\cdot y =\sqrt{2}.}
	\end{cases}$
	\end{enumEX}
	\loigiai{
	\begin{enumerate}
	\item Từ phương trình thứ nhất của hệ ta có $y = 2\sqrt{2} - 5x\sqrt{3}$. Thay vào phương trình thứ hai ta được
	\[x\sqrt{6} - (2\sqrt{2} - 5\sqrt{3}\cdot x)\cdot\sqrt{2}= 2\Rightarrow x =\dfrac{\sqrt{6}}{6}.\]
	Từ đó $y = 2\sqrt{2} - 5\cdot\dfrac{\sqrt{6}}{6}\cdot\sqrt{3}=\dfrac{ - \sqrt{2}}{2}$.\\
	Vậy hệ có nghiệm duy nhất là $\left(\dfrac{\sqrt{6}}{6};\dfrac{ - \sqrt{2}}{2}\right)$.
	\item Từ phương trình thứ hai của hệ ta có $x =\sqrt{2} - \sqrt{3}\cdot y$. Thay vào phương trình thứ nhất ta được
	\[\sqrt{2}\cdot(\sqrt{2} - \sqrt{3}\cdot y) - \sqrt{3}y = 1\Rightarrow y =\dfrac{1}{\sqrt{6} + \sqrt{3}}.\]
	Từ đó $x =\sqrt{2} - \sqrt{3}\cdot\dfrac{1}{\sqrt{6} + \sqrt{3}}= 1$.\\
	Vậy hệ có nghiệm duy nhất là $\left(1 ;\dfrac{1}{\sqrt{6} + \sqrt{3}}\right)$.
	\end{enumerate}
	}
\end{vd}
%------------------
\begin{dang}{Giải hệ phương trình bằng phương pháp cộng đại số}
\end{dang}
\begin{vd}
	Giải các hệ phương trình sau bằng phương pháp cộng đại số:
	\begin{listEX}[4]
	\item $\heva{& -2x+5y=12 \\ & 2x+3y=4}$;
	\item $\heva{& 5x-7y=9 \\ & 5x-3y=1}$;
	\item $\heva{& -4x+3y=0 \\ & 4x-5y=-8}$;
	\item $\heva{& 4x+3y=0 \\ & x+3y=9}$;
	\item $\heva{& 3x+2y=7 \\ & 2x-3y=-4}$;
	\item $\heva{& 4x+3y=6 \\ & -5x+2y=4}$;
	\item $\heva{& 3x-5y=2 \\ & -6x+10y=-4}$;
	\item $\heva{& -0{,}5x+0{,}5y=1 \\ & -2x+2y=8}$.
	\end{listEX}
	\loigiai{
	\begin{listEX}
	\item
	Cộng từng vế của hai phương trình ta được $8y=16$, suy ra $y=2$.\\
	Thế $y=2$ vào phương trình thứ hai ta được $2x+3.2=4$, hay $2x=-2$, suy ra $x=-1$.\\
	Vậy hệ phương trình đã cho có nghiệm là $(-1;2)$.
	\item
	Trừ từng vế của hai phương trình ta được $(5x-5x)+(-7y+3y)=9-1$, hay $-4y=8$, suy ra $y=-2$.\\
	Thế $y=-2$ vào phương trình thứ nhất, ta được $5x-7.(-2)=9$ hay $5x+14=9$, suy ra $x=-1$.\\
	Vậy hệ phương trình có nghiệm $(-1;-2)$.
	\item
	Cộng từng vế của hai phương trình ta được $-2y=-8$, suy ra $y=4$.\\
	Thế $y=4$ vào phương trình thứ nhất, ta được $-4x+3.4=0$, hay $-4x=-12$, suy ra $x=3$.\\
	Vậy hệ phương trình có nghiệm $(3;4)$.
	\item
	Trừ từng vế của hai phương trình, ta được $3x=-9$, suy ra $x=-3$.\\
	Thế $x=-3$ vào phương trình thứ nhất, ta được $4.(-3)+3y=0$, hay $3y=12$, suy ra $y=4$.\\
	Vậy hệ phương trình có nghiệm $(-3;4)$.
	\item
	Nhân hai vế của phương trình thứ nhất với $3$ và nhân hai vế của phương trình thứ $2$, ta được:\\
	\centerline{$\heva{& 9x+6y=21\\ & 2x-6y=-8}$}.\\
	Cộng từng vế hai phương trình của hệ mới, ta được $13x=13$ hay $x=1$.
	Thế $x=1$ vào phương trình thứ nhất của hệ đã cho, ta có $3.1+2y=7$, suy ra $y=2$. \\
	Vậy nghiệm của hệ phương trình có nghiệm là $(1;2)$.
	\item
	Nhân cả hai vế của phương trình thứ nhất cho $5$ và nhân của hai vế của phương trình thứ hai cho $4$ ta được hệ $\heva{& 20x+15y=30 \\& -20x+8y=16}$.\\
	Cộng từng vế hai phương trình của hệ mới, ta được $23y=46$, suy ra $y=2$. \\
	Thế $x=2$ vào phương trình thứ nhất của hệ đã cho, ta có: $4x+3.2=6$, hay $4x=0$, suy ra $x=0$.\\
	Vậy hệ phương trình đã cho có nghiệm là $(0;2)$.
	\item 
	Chia cả hai vế của phương trình thứ hai cho $2$, ta được hệ $\heva{& 3x-5y=2 \\& -3x+5y=-2}$.\\
	Cộng từng vế hai phương trình của hệ mới ta được $0x+0y=0$. Hệ thức này luôn thỏa mãn với các giá trị tùy ý của $x$ và $y$.\\
	Với giá trị tùy ý của $x$, giá trị của $y$ tính được nhờ hệ thức $3x-5y=2$, suy ra $y=\dfrac{3}{5}x-\dfrac{2}{5}$.\\
	Vậy hệ phương trình đã cho có nghiệm là $(x;\dfrac{3}{5}x-\dfrac{2}{5})$ với $x\in\mathbb{R}$
	\item
	Nhân cả hai vế của phương trình thứ nhất cho 4, ta được hệ $\heva{& -2x+2y=4\\& -2x+2y=8}$.\\
	Cộng từng vế của hai phương trình của hệ mới ta được $0x+0y=12$. Không tìm được bất kì giá trị $x$, $y$ nào thỏa mãn hệ thức này.\\
	Vậy hệ phương trình vô nghiệm.
	\end{listEX}
	}
\end{vd}
\begin{vd}
	Giải các hệ phương trình sau bằng phương pháp thế: 
	\begin{listEX}[3]
	\item $\heva{&3x+6y=-9\\&3x+4y=-5}$;
	\item $\heva{&3x+2y=5\\&5x+2y=7}$;
	\item $\heva{&3x+2y=4 &(1)\\&-2x+3y=-7 &(2)}$.
	\end{listEX}
	\loigiai{
	\begin{listEX}
	\item 
	Đặt $\heva{&3x+6y=-9 &(1)\\&3x+4y=-5 &(2)}$\\
	Trừ từng vế hai phương trình (1) và (2), ta nhận được phương trình
	\[2y=-4,\ \text{tức là}\ y=-2. \]
	Thế $y=-2$ vào phương trình (2), ta được phương trình
	\allowdisplaybreaks
	\begin{eqnarray*}
	3x+4\cdot (-2)&=&-5\\
	3x-8&=&-5\\
	3x&=&3\\
	x&=&1.
	\end{eqnarray*}
	Vậy hệ phương trình đã cho có nghiệm duy nhất $(x;y)=(1;-2)$.
	\item
	Đặt $\heva{&3x+2y=5 &(1)\\&5x+2y=7 &(2)}$\\
	Trừ từng vế hai phương trình (1) và (2), ta nhận được phương trình
	\[-2x=-2,\ \text{tức là}\ x=1. \]
	Thế $x=1$ vào phương trình (1), ta được phương trình
	\allowdisplaybreaks
	\begin{eqnarray*}
	3\cdot 1+2y&=&5\\
	3+2y&=&5\\
	2y&=&2\\
	y&=&=1.
	\end{eqnarray*}
	Vậy hệ phương trình đã cho có nghiệm duy nhất $(x;y)=(1;1)$.
	\item
	Đặt $\heva{&3x+2y=4 &(1)\\&-2x+3y=-7 &(2)}$\\
	Nhân hai vế của phương trình (1) với $2$ và nhân hai vế của phương trình (2) với $3$, ta được hệ phương trình $\heva{&6x+4y=8 &(3)\\&-6x+9y=-21 &(4)}$\\
	Cộng từng vế hai phương trình (3) và (4), ta nhận được phương trình
	\allowdisplaybreaks
	\begin{eqnarray*}
	13y&=&-13\\
	y&=&-1.
	\end{eqnarray*}
	Thế giá trị $y=-1$ vào phương trình (1), ta được phương trình:
	\allowdisplaybreaks
	\begin{eqnarray*}
	3x+2\cdot (-1)&=&4\\
	3x-2&=&4\\
	3x&=&6\\
	x&=&2.
	\end{eqnarray*}
	Vậy hệ phương trình đã cho có nghiệm $(x;y)=(2;-1)$.
	\end{listEX}
	}
\end{vd}
\begin{vd}%[9D3K4]
	Giải các hệ phương trình sau:
	\begin{enumEX}{2}
	\item $\left\{\begin{aligned}&\left(\sqrt{3}+1\right) x+\left(\sqrt{3}-1\right) y=\sqrt{3}\\&2 \sqrt{3} \cdot x-2 y=3 \sqrt{3}+1\end{aligned} \right.$ 
	\item $\left\{\begin{aligned}&x \sqrt{3}-y \sqrt{2}=1\\&x \sqrt{2}+y \sqrt{3}=\sqrt{3}\end{aligned} \right.$ 
	\end{enumEX}
	\loigiai{
	\begin{listEX}[2]
	\item $\begin{aligned}[t] 
	&\left\{\begin{aligned}&\left(\sqrt{3}+1\right) x+\left(\sqrt{3}-1\right) y=\sqrt{3}\\&2 \sqrt{3} \cdot x-2 y=3 \sqrt{3}+1\end{aligned} \right.\\
	& \left\{\begin{aligned}&\left(4+2 \sqrt{3}\right) x+2 y=3+\sqrt{3}\\&2 \sqrt{3} \cdot x-2 y=3 \sqrt{3}+1\end{aligned} \right. \\
	&\left\{\begin{aligned}&\left(4+4 \sqrt{3}\right) x=4+4 \sqrt{3}\\&2 \sqrt{3} \cdot x-2 y=3 \sqrt{3}+1\end{aligned} \right. \\ 
	& \left\{\begin{aligned}&x=1\\&y=-\dfrac{\sqrt{3}+1}{2};\end{aligned} \right. 
	\end{aligned}$\\
	Vậy hệ có nghiệm duy nhất $\left(1; -\dfrac{\sqrt{3}+1}{2}\right)$. 
	\item $\begin{aligned}[t] 
	&\left\{\begin{aligned}&x \sqrt{3}-y \sqrt{2}=1\\&x \sqrt{2}+y \sqrt{3}=\sqrt{3}\end{aligned} \right.\\
	&\left\{\begin{aligned}&3 x-y \sqrt{6}=\sqrt{3}\\&2 x+y \sqrt{6}=\sqrt{6}\end{aligned} \right. 
	\\
	&\left\{\begin{aligned}&5 x=\sqrt{3}+\sqrt{6}\\&x \sqrt{2}+y \sqrt{3}=\sqrt{3}\end{aligned} \right. \\ &\left\{\begin{aligned}&x=\dfrac{\sqrt{3}+\sqrt{6}}{5}\\&y=\dfrac{3-\sqrt{2}}{5}.\end{aligned} \right.
	\end{aligned}$\\
	Vậy hệ có nghiệm duy nhất là $\left(\dfrac{\sqrt{3}+\sqrt{6}}{5};\dfrac{3-\sqrt{2}}{5} \right)$.
	\end{listEX}
	}
\end{vd}
%==================
\begin{dang}{Sử dụng MTCT để giải hệ phương trình}
\end{dang}
\begin{vd}
	Tìm nghiệm của mỗi hệ phương trình sau bằng máy tính cầm tay:
	\begin{listEX}[3]
	\item $\heva{&2x+5y=-4\\&-3x+y=-11}$;
	\item $\heva{&x-3y=2\\&-2x+5y=1}$;
	\item $\heva{&3x-2y=1\\&-6x+y=3}.$
	\end{listEX} 
	\loigiai{
	\begin{listEX}
	\item
	Sử dụng loại máy tính phù hợp (chẳng hạn đối với CASIO FX570-VN PLUS), ấn liên tiếp các phím sau:
	\begin{center}
	\key{\_} \key{5} \key{1}
	\key{2} \key{=} \key{5} \key{=} \key{z}\key{4} \key{=}
	\key{z}\key{3} \key{=}\key{1} \key{=}\key{z}\key{11}\key{=}\key{=}
	\end{center}
	Màn hình hiện ra kết quả như hình sau: $X=3$\\
	Ấn \key{=}, kết quả như hình sau: $Y=-2$\\
	Vậy hệ phương trình có nghiệm duy nhất là $(3;-2)$.
	\item
	Sử dụng loại máy tính phù hợp (chẳng hạn đối với CASIO FX570-VN PLUS), ấn liên tiếp các phím sau:
	\begin{center}
	\key{\_} \key{5} \key{1} \key{1} \key{=} \subk \key{3} \key{=} \key{2} \key{=} \subk \key{2} \key{=} \key{5} \key{=} \key{1} \key{=} \key{=}
	\end{center}
	Ta thấy trên màn hình hiện ra $x=-13$.\\
	Ấn tiếp phím \key{=} ta thấy trên màn hình hiện ra $y=-5$.\\
	Vậy nghiệm của hệ phương trình là $(x;y)=(-13;-5)$.
	\item
	Sử dụng loại máy tính phù hợp (chẳng hạn đối với CASIO FX570-VN PLUS), ấn liên tiếp các phím sau:
	\begin{center}
	\key{\_} \key{5} \key{1} \key{3} \key{=} \subk \key{2} \key{=} \key{1} \key{=} \subk \key{6} \key{=} \key{1} \key{=} \key{3} \key{=} \key{=}
	\end{center}
	Ta thấy trên màn hình hiện ra $x=-\dfrac{7}{9}$.\\
	Ấn tiếp phím \key{=} ta thấy trên màn hình hiện ra $y=-\dfrac{5}{3}$.\\
	Vậy nghiệm của hệ phương trình là $(x;y)=\left(\dfrac{-7}{9};\dfrac{-5}{3}\right)$.
	\end{listEX}
	}
\end{vd}
\begin{vd}
	Dùng MTCT thích hợp để tìm nghiệm của các hệ phương trình sau
	\begin{listEX}[3]
	\item $\heva{& 2x+3y=-4 \\ & -3x-7y=13;}$
	\item $\heva{& 2x+3y=1 \\ & -x-1{,}5y=1;}$
	\item $\heva{& 8x-2y-6=0 \\ & 4x-y-3=0.}$
	\end{listEX}
	\loigiai{
	\begin{listEX}[1]
	\item $x=\dfrac{11}{5}$; $y=-\dfrac{14}{5}$.
	\item Hệ phương trình vô nghiệm.
	\item Hệ phương trình vô số nghiệm.
	\end{listEX}
	}
\end{vd}
\begin{vd}
	Thực hiện lần lượt các yêu cầu sau để tính số mililít dung dịch acid $HCl$ nồng độ $20\%$ và số mililít dung dịch acid $HCl$ nồng độ $5\%$ cần dùng để pha chế $2$ lít dung dịch acid $HCl$ nổng độ $10\%$.
	\begin{listEX}[1]
	\item Gọi $x$ là số mililít dung dịch acid HCl nồng độ $20\%$, $y$ là số mililít dung dịch acid HCl nồng độ $5\%$ cần lấy. Hãy biểu thị qua $x$ và $y$
	\begin{itemize}
	\item Thể tích của dung dịch acid HCl $10 \%$ nhận được sau khi trộn lẫn hai dung dịch acid ban đầu.
	\item Tổng số gam acid HCl nguyên chất có trong hai dung dịch acid này.
	\end{itemize}
	\item Sử dụng kết quả ở câu $a$, hãy lập một hệ hai phương trình bậc nhất với hai ẩn là $x$, $y$.\\ Dùng MTCT giải hệ phương trình này để tính số mililít cần lấy của mỗi dung dịch acid HCl ở trên.
	\end{listEX}
	\loigiai{
	\begin{listEX}[1]
	\item Thể tích của dung dịch acid $HCl 10 \%$ nhận được sau khi trộn lẫn hai dung dịch acid ban đầu là $x+y$.\\
	Tổng số gam acid HCl nguyên chất có trong hai dung dịch acid là $20\%x+5\%y$
	\item Theo câu $a$, ta có hệ $\heva{& x+y=2000 \\ & 20\%x+5\%y=2000\cdot10\%}$.\\
	Ta có $x=\dfrac{2000}{9}$ và $y=\dfrac{16000}{9}$ nên thể tích dung dịch HCL $10\%$ ban đầu là $\dfrac{2000}{9}$ (ml) và thể tích dung dịch HCL $5\%$ ban đầu là $\dfrac{16000}{9}$ (ml). 
	\end{listEX}
	}
\end{vd}
%===================
\begin{dang}{Giải hệ bằng phương pháp đặt ẩn phụ}
%	\begin{itemize}
%	\item Đặt điều kiện để hệ có nghĩa.
%	\item Đặt ẩn phụ và điều kiện của ẩn phụ.
%	\item Giải hệ theo các ẩn phụ đã đặt.
%	\item Trở lại ẩn đã cho để tìm nghiệm của hệ.
%	\end{itemize}
\end{dang}
\begin{vd}%[9D3K2]
	Giải các hệ phương trình
	\begin{enumEX}{3}
	\item $\begin{cases}
	{\dfrac{1}{x} + \dfrac{1}{y}= 2}\\
	{\dfrac{3}{x} - \dfrac{4}{y}= - 1;}
	\end{cases}$
	\item $\heva{&\dfrac{6}{x - 1}- \dfrac{5}{y - 2} = 7\\ &\dfrac{3}{x - 1} + \dfrac{2}{y - 2} = - 1}$.
	\item $\begin{cases}
	{\dfrac{3}{2x - y} - \dfrac{6}{x + y}= - 1}\\
	{\dfrac{1}{2x - y} - \dfrac{1}{x + y}= 0;}
	\end{cases}$
	\end{enumEX}
	\loigiai{
	\begin{enumerate}
	\item Điều kiện $x\neq 0 ; y\neq 0$.\\
	Đặt $\dfrac{1}{x}= u ;\dfrac{1}{y}= v$, ta có hệ phương trình\\
	$\begin{aligned}[t]
	&\begin{cases}
	{u + v = 2}\\
	{3u - 4v = - 1.}
	\end{cases}\\
	&\begin{cases}
	{v= 2 - u}\\
	{3u - 4(2 - u) = - 1}
	\end{cases}\\
	&\begin{cases}
	{v = 2 - u}\\
	{u = 1}
	\end{cases}\\
	&\begin{cases}
	{u = 1}\\
	{v = 1.}
	\end{cases}
	\end{aligned}$\\	
	Trở lại ẩn $x$, $y$ ta có
	$\begin{cases}
	{\dfrac{1}{x}= 1}\\
	{\dfrac{1}{y}= 1}
	\end{cases}$, do đó $\begin{cases}
	{x = 1}\\
	{y = 1.}
	\end{cases}$\\
	Vậy hệ có nghiệm duy nhất $(1;1)$.
	\item
	Điều kiện $\heva{& x\neq 1 \\ &y\neq 2}\quad (*)$\\
	Đặt $u = \dfrac{1}{x - 1}; v = \dfrac{1}{y - 2}$, ta có hệ phương trình\\
	\allowdisplaybreaks
	$\begin{aligned}[t]
	&\heva{&6u - 5v = 7\\ &3u + 2v = - 1}\\
	&\heva{&6u - 5v = 7\\ &6u + 4v = - 2}\\
	&\heva{&6u - 5v = 7\\ &9v = - 9}\\
	&\heva{&u = 2\\ &v = - 1}
	\end{aligned} $\\
	Khi $\heva{&u = 2\\ &v = - 1}$ suy ra 
	$\heva{& \dfrac{1}{x - 1} = 2\\ &\dfrac{1}{y - 2} = - 1}\Rightarrow \heva{&x - 1 = \dfrac{1}{2}\\ &y - 2 = - 1}\Rightarrow \heva{& x = \dfrac{3}{2} \\ & y = 1}$\\
	So sánh với điều kiện $(*)$ thấy thỏa mãn. Vậy hệ phương trình có nghiệm duy nhất $\left(\dfrac{3}{2}; 1\right)$.
	\item Điều kiện $y\neq 2x ; y\neq - x$.\\
	Đặt $\dfrac{1}{2x - y}= u ,\dfrac{1}{x + y}= v$, ta có hệ phương trình\\
	$\begin{aligned}[t]
	&\begin{cases}
	{3u - 6v = - 1}\\
	{u - v = 0}
	\end{cases}\\
	&\begin{cases}
	{u =\dfrac{1}{3}}\\
	{v =\dfrac{1}{3}.}
	\end{cases}
	\end{aligned}$\\
	Trở lại ẩn $x$, $y$ ta có
	$\begin{cases}
	{\dfrac{x}{x + 1}= 5}\\
	{\dfrac{y}{y - 3}= 2}
	\end{cases}$, do đó $\begin{cases}
	{x = - \dfrac{5}{4}}\\
	{y = 6.}
	\end{cases}$\\
	Vậy hệ có nghiệm duy nhất $\left( - \dfrac{5}{4}; 6\right)$.
	\end{enumerate}
	}
\end{vd}
%==================
\begin{dang}{Xác định đường thẳng, giao điểm giữa các đường thẳng}
\end{dang}
\begin{vd}
	Xác định $a$, $b$ để đồ thị hàm số $y=ax+b$ đi qua hai điểm:
	\begin{listEX}[3]
	\item $A(2;-2)$ và $B(-1;3)$;
	\item $A(2;1)$ và $B(1;2)$;
	\item $A(3;-6)$ và $B(-2;4)$.
	\end{listEX}	 
	\loigiai{
	\begin{listEX}
	\item
	Vì đồ thị hàm số $y=ax+b$ đi qua điểm $A(2;-2)$ nên ta có phương trình sau $2a+b=-2$ $(1)$.\\
	Vì đồ thị hàm số $y=ax+b$ đi qua điểm $B(-1;3)$. nên ta có phương trình sau $-a+b=3$ $(2)$.\\
	Từ $(1)$ và $(2)$ ta có hệ phương trình 
	$\begin{aligned}[t]
	&\heva{& 2a+b=-2\\ & -a+b=3}\\
	&\heva{& 3a=-5\\ &-a+b=3}\\
	&\heva{&a=-\dfrac{5}{3}\\ &b=3+a}\\
	&\heva{&a=-\dfrac{5}{3}\\ &b=\dfrac{4}{3}.}
	\end{aligned}$\\
	Vậy hệ phương trình có nghiệm duy nhất là $\left(-\dfrac{5}{3};\dfrac{4}{3}\right)$.\\
	Vậy khi $a=-\dfrac{5}{3}$, $b=\dfrac{4}{3}$ thì đồ thị hàm số $y=ax+b$ đi qua hai điểm $A(2;-2)$ và $B(-1;3)$.
	\item Hai điểm $A(2;1)$ và $B(1;2)$ thuộc đường thẳng $y=ax+b$ nên ta có hệ phương trình ẩn $a$, $b$ là 
	\begin{align*}
	&\left\{\begin{aligned}&2 a+b=1\\&a+b=2\end{aligned} \right.\\ 
	&\left\{\begin{aligned}&a=-1\\&a+b=2\end{aligned} \right.\\ 
	&\left\{\begin{aligned}&a=-1\\&b=3.\end{aligned} \right.
	\end{align*}
	Vậy với $a=-1$; $b=3$ thì đồ thị $y=ax+b$ đi qua $A(2;1)$ và $B(1;2)$.
	\item Hai điểm $A(3;-6)$ và $B(-2;4)$ thuộc đường thẳng $y=ax+b$ nên ta có hệ phương trình ẩn $a$, $b$ là
	\begin{align*}
	&\left\{\begin{aligned}&3 a+b=-6\\&-2 a+b=4\end{aligned} \right. \\
	&\left\{\begin{aligned}&5 a=-10\\&-2 a+b=4\end{aligned} \right. \\
	&\left\{\begin{aligned}&a=-2\\&b=0.\end{aligned} \right.
	\end{align*}
	Vậy với $a=-2$; $b=0$ thì đồ thị $y=ax+b$ đi qua $A(3;-6)$ và $B(-2;4)$.	
	\end{listEX}
	}
\end{vd}
\begin{vd}%[9D3K2]
	Xác định tọa độ giao điểm của hai đường thẳng
	\begin{listEX}[2]
	\item $(d):2x-y=3$ và $(d'):x+2y=4$;
	\item $(d):2x+y=2$ và $(d'):x+\dfrac{1}{2}y=1$.
	\end{listEX}
	\loigiai{
	\begin{enumerate}
	\item Toạ độ giao điểm $M$ của $(d)$ và $(d')$ là nghiệm của hệ phương trình
	\[\begin{cases}
	{2x - y = 3}\\
	{x + 2y = 4.}
	\end{cases}\]
	Giải hệ phương trình ta được nghiệm là $(2;1)$.\\
	Vậy tọa độ giao điểm là $(2;1)$.
	\item Toạ dộ giao điểm $M$ của $(d)$ và $(d')$ là nghiệm của hệ phương trình
	\begin{align*}
	&\begin{cases}
	{2x + y = 2}\\
	{x + \dfrac{1}{2}y = 1}
	\end{cases}\\
	&\begin{cases}
	{y = 2 - 2x}\\
	{x + \dfrac{1}{2}(2 - 2x) = 1}
	\end{cases}\\
	& \begin{cases}
	{y = 2 - 2x}\\
	{0 \cdot x = 0}
	\end{cases}\\
	&\begin{cases}
	{y = 2 - 2x}\\
	{x\in \mathbb{R}.}
	\end{cases}
	\end{align*}	
	Hệ đã cho có vô số nghiệm nên $(d)$ trùng với $(d')$.\\
	Tọa độ giao điểm $N(x ; 2 - 2x),x\in \mathbb{R}$.
	\end{enumerate}
	}
\end{vd}
\begin{vd}%[9D3K2]
	Tìm $m$ để ba đường thẳng sau đồng quy
	\begin{align*}
	&\left(d_1\right) : 2x - y = 0\\
	&\left(d_2\right) : x + y = 3\\
	&\left(d_3\right) : 2x - 3y = m.
	\end{align*}
	\loigiai{
	Tọa độ giao điểm $M$ của $(d_1)$ và $(d_2)$ là nghiệm của hệ
	$\begin{cases}
	{2x - y = 0}\\
	{x + y = 3.}
	\end{cases}$\\
	Giải hệ phương trình ta được nghiệm là $(1;2)$.\\
	Để $(d_1)$,$(d_2)$ và $(d_3)$ đồng quy thì $(d_3)$ phải đi qua $(1;2)$. Do đó
	\[2\cdot1 - 3\cdot2 = m \text{ suy ra } m = - 4.\]
	Vậy với $m=-4$ thì ba đường thẳng đồng quy.
	}
\end{vd}
%====================
\begin{dang}{Xác định tham số $m$ để hệ phương trình thỏa mãn điều kiện về nghiệm số}
%	\begin{itemize}
%	\item Giải hệ phương trình tìm $x$, $y$ theo tham số $m$.
%	\item Với điều kiện về nghiệm số của đề bài để tìm $m$.
%	\end{itemize}
\end{dang}
\begin{vd}%[9D3K2]
	Cho hệ phương trình
	\[\heva{&(a + 1) x - a y = 5 &(1) \\ &x + a y = a^2 + 4a &(2)}\]
	Tìm giá trị của $a\in \mathbb{Z}$ để cho hệ có nghiệm $(x;y)$ với $x,y\in\mathbb{Z}$.
	\loigiai{
	Từ phương trình $(2)$ ta có $x = a^2 + 4a - a y$. Thay vào $(1)$, ta được
	\begin{align}
	(a + 1)\cdot\left(a^2 + 4a - a y\right) - a y = 5\Rightarrow a(a + 2) y = a^3 + 5a^2 + 4a - 5. \tag{3}
	\end{align}
	\begin{itemize}
	\item Nếu $a=0$ hoặc $a=-2$ thì phương trình $(3)$ vô nghiệm.
	\item Điều kiện để hệ có nghiệm duy nhất là $\mathrm{a}\neq 0 ;\mathrm{a}\neq - 2$. Khi đó
	\[y =\dfrac{a^3 + 5a^2 + 4a - 5}{a(a + 2)}\Rightarrow x =\dfrac{a^2 + 4a + 5}{a + 2}.\]
	\item Trước hết ta tìm $a\in \mathbb{Z}$ để $x\in \mathbb{Z}$.
	\[x =\dfrac{(a + 2)^2 + 1}{a + 2}= a + 2 + \dfrac{1}{a + 2}.\]
	Để $x\in \mathbb{Z}$ thì $a+2\in$ Ư$(1)$ hay $a + 2 =\pm 1\Rightarrow a = - 3 ; a = - 1$.
	\item Với $a=-3$ thì $y =\dfrac{( - 3)^3 + 5\cdot( - 3)^2 + 4 .( - 3) - 5}{ - 3 .( - 3 + 2)}=\dfrac{1}{3}\notin\mathbb{Z}$.
	\item Với $a=-1$ thì $y =\dfrac{( - 1)^3 + 5\cdot( - 1)^2 + 4\cdot( - 1) - 5}{ - 1\cdot( - 1 + 2)}= 5\in \mathbb{Z}$.
	\end{itemize}
	Vậy với $a=-1$ thì hệ có nghiệm nguyên là $(2;5)$.
	}
\end{vd}
\begin{vd}%[9D3G4]
	Cho hệ phương trình $\heva{&x+my=2\\&mx-2y=1.}$
	\begin{enumerate}
	\item Tìm số nguyên $m$ để hệ có nghiệm duy nhất $(x;y)$ mà $x>0$, $y<0$.
	\item Tìm số nguyên $m$ để hệ có nghiệm duy nhất $(x;y)$ mà $x$, $y$ là các số nguyên.
	\end{enumerate}
	\loigiai{
	\begin{enumerate}
	\item $\bullet$ Với $m=0$ thì hệ có nghiệm $\left(2;-\dfrac{1}{2} \right)$ thỏa mãn đề bài.\\
	$\bullet$ Với $m \ne 0$ thì $\left\{\begin{aligned}&m x+m^2 y=2 m\\&m x-2 y=1\end{aligned} \right. \Rightarrow \left\{\begin{aligned}&\left(m^2+2 \right) y=2 m-1\\&m x-2 y=1\end{aligned} \right. \Rightarrow \left\{\begin{aligned}&y=\dfrac{2 m-1}{m^2+2}\\&x=\dfrac{m+4}{m^2+2}.\end{aligned} \right.$ \\
	Ta có $\left\{\begin{aligned}&x > 0\\&y < 0\end{aligned} \right. \Rightarrow \left\{\begin{aligned}&\dfrac{2 m-1}{m^2+2} < 0\\&\dfrac{m+4}{m^2+2} > 0\end{aligned} \right. \Rightarrow \left\{\begin{aligned}&2 m-1 < 0\\&m+4 > 0\end{aligned} \right. \Rightarrow-4 < m < \dfrac{1}{2}.$ \\
	Vì $m \in \mathbb{Z}$ nên $m \in \{-3;-2;-1;0\}$.\\
	Vậy với $m \in \{-3;-2;-1;0\}$ thì hệ có nghiệm duy nhất thỏa mãn $x>0$; $y<0$.
	\item $\bullet$ Với $m=0$ thì hệ có nghiệm $\left(2;-\dfrac{1}{2} \right)$ không thỏa mãn đề bài. \\
	$\bullet$ Với $m \ne 0$, thì hệ có nghiệm duy nhất $\left(\dfrac{m+4}{m^2+2}; \dfrac{2 m-1}{m^2+2} \right)$.\\
	Trước hết tìm $m \in \mathbb{Z}$ để $x \in \mathbb{Z}$ thì $m+4 \,\vdots\, m^2+2$\\
	$\Rightarrow m^2+4 m \,\vdots\, m^2+2 \Rightarrow 4 m-2 \,\vdots\, m^2+2$ \\
	$\Rightarrow 4(m+4)-(4 m-2) \,\vdots\, m^2+2 \Rightarrow 18 \,\vdots\, m^2+2$ \\
	mà $m^2+2>2$ nên $m^2+2 \in \{3;6;9;18\} \Rightarrow m^2 \in \{1;4;7;16\}$.\\
	Vì $m\in \mathbb{Z}$ nên $m \in \{\pm 1; \pm 2; \pm 4 \}$.
	\end{enumerate}
	}
\end{vd}
\begin{vd}
	Với giá trị nào của $m$ thì hai phương trình sau có nghiệm chung
	\begin{center}
	$2x^2 + mx - 1 = 0$ và $mx^2 - x + 2 = 0$.
	\end{center}
	\loigiai{Để thỏa mãn bài toán khi và chỉ khi hệ phương trình $$\heva{&2x^2 + mx - 1 = 0 \\ & mx^2 - x + 2 = 0}\quad (*)$$
	có nghiệm.\\
	Đặt $y = x^2$, điều kiện $y\geq 0$.\\
	Khi đó hệ phương trình $(*)$ trở thành $\heva{&mx + 2y - 1 = 0\quad (1) \\ &- x + my + 2 = 0\quad (2)}$.\\
	Xét phương trình $(2)$ suy ra $x = my + 2\quad (3)$. Thay vào phương trình $(1)$ ta có
	$$m\left(my + 2\right) + 2y - 1 = 0\Rightarrow \left(m^2 + 2\right)y = 1 - 2m\Rightarrow y = \dfrac{1- 2m}{m^2 + 2}.$$
	Thay $y = \dfrac{1- 2m}{m^2 + 2}$ vào $(3)$ ta có
	$$x = m\cdot \left(\dfrac{1- 2m}{m^2 + 2}\right) + 2\Rightarrow x = \dfrac{m\left(1- 2m\right) + 2\left(m^2 + 2\right)}{m^2 + 2}\Rightarrow x = \dfrac{m + 4}{m^2 + 2}.$$ 	
	So sánh với điều kiện suy ra $1 - 2m\geq 0\Rightarrow m\leq \dfrac{1}{2}$.\\
	Do $y = x^2$ nên 
	$$\left(\dfrac{m + 4}{m^2 + 2}\right)^2 = \dfrac{1 -2m}{m^2 + 2}\Rightarrow \left(m + 4\right)^2 = \left(1 - 2m\right)\left(m^2 + 2\right)$$	
	$$m^3 + 6m + 7= 0\Rightarrow \left(m + 1\right) \left(m^2 - m + 1\right) = 0\quad (3)$$
	Do $m^2 - m + 1 = \left(m - \dfrac{1}{2}\right)^2 + \dfrac{3}{4} > 0\quad \forall m$. Nên $(3)\Rightarrow m + 1 = 0\Rightarrow m = - 1$.\\
	So sánh với điều kiện thỏa mãn.\\
	Vậy $m = - 1$ hai phương trình có nghiệm chung.
	}
\end{vd}
%===================
\begin{dang}{Toán về quan hệ giữa các số}
	%	\begin{itemize}
	%	\item Tổng của $a$ và $b$ là $m$ thì ta có $a+b=m$.
	%	\item Số $a$ lớn hơn số $b$ là $m$ thì ta có $a-b=m$.
	%	\item Số $a$ bằng $k$ lần số $b$ thì ta có $a=k\cdot b$.
	%	\end{itemize}
\end{dang}
\begin{vd}Tìm hai số tự nhiên có tổng bằng $1\,006$, biết rằng nếu lấy số lớn chia cho số nhỏ thì được thương là $2$ và số dư là $124$.
	\loigiai{Gọi hai số cẩn tìm là $x$ và $y$, trong đó $x<y$. Số dư trong phép chia $y$ cho $x$ là $124$ nên $x>124$. Vậy điều kiện của hai ẩn là $x, y \in \mathbb{N}$ và $124<x<y$.\\Tổng hai số bằng $1006$ nên ta có phương trình $x+y=1006$.\\
	Khi chia $y$ cho $x$ ta được thương là $2$, dư $124$ nên ta có phương trình $y=2 x+124$.\\
	Do đó, ta có hệ phương trình $\heva{&x+y=1006\qquad (1) \\& y=2 x+124.\qquad (2)}$
	\\Giải hệ phương trình\\
	Từ (2) thế $y=2 x+124$ vào (1), ta được $3 x+124=1006$ hay $3 x=882$, suy ra $x=294$. Từ đó ta được $y=2 \cdot 294+124=712$.\\
	Các giá trị $x=294$ và $y=712$ thoả mãn các điều kiện của ẩn. Vậy hai số cần tìm là $294$ và $712$.
	\\
	}
\end{vd}
\begin{vd}%[9D3K5]
	Tỉ số của hai số là $3 \div 4$. Nếu giảm số lớn đi $100$ và tăng số nhỏ thêm $200$ thì tỉ số mới là $5 \div 3$. Tìm hai số đó.
	\loigiai{
	Gọi số bé là $x$ và số lớn là $y$ $(y>x)$. \\
	Tỉ số của hai số là $3\div 4$ thì ta có $\dfrac{x}{y}=\dfrac{3}{4}$ \hfill(1) \\
	Nếu giảm số lớn đi $100$ và tăng số nhỏ thêm $200$ thì tỉ số mới là $5 \div 3$, ta có phương trình
	\[\dfrac{x+100}{y+200}=\dfrac{5}{3} \tag{2}\]
	Từ (1), (2) ta có hệ phương trình
	\[\left\{\begin{aligned}&\dfrac{x}{y}=\dfrac{3}{4}\\&\dfrac{x+200}{y-100}=\dfrac{5}{3}\end{aligned} \right. \Rightarrow \left\{\begin{aligned}&4 x-3 y=0\\&3 x-5 y=-1100\end{aligned} \right. \Rightarrow \left\{\begin{aligned}&x=300\\&y=400.\end{aligned} \right.\]
	Vậy hai số cần tìm là $300$ và $400$.
	}
\end{vd}
%%=====Ví dụ 6
\begin{vd}
	Một trường trung học cơ sở mua $500$ quyển vở để làm phần thưởng cho học sinh. Giá bán của mỗi quyển vở loại thứ nhất, loại thứ hai lần lượt là $8\,000$ đồng, $9\,000$ đồng. Hỏi nhà trường đã mua mỗi loại bao nhiêu quyển vở? Biết rằng số tiền nhà trường đã dùng để mua $500$ quyển vở đó là $4\,200\,000$ đồng.
	\loigiai{
	Gọi số quyển vở loại thứ nhất, loại thứ hai lần lượt là $x, y\ (x \in \mathbb{N}, y \in \mathbb{N})$.\\
	Theo giả thiết, ta có phương trình: $x+y=500$.\\
	Mặt khác, ta có phương trình: $8\,000 x+9\,000 y=4\,200\,000$, tức là $8 x+9 y=4200$.\\
	Ta có hệ phương trình: $\heva{&x+y=500 &(1)\\&8 x+9 y=4\,200 &(2)}$\\
	Nhân phương trình (1) cho $8$, ta được $\heva{&8x+8y=4\,000 &(3)\\&8 x+9 y=4\,200 &(2)}$\\
	Trừ từng vế phương trình (3) cho phương trình (2), ta được $y=200$.\\
	Thay giá trị $y=200$ vào phương trình (1), ta được:
	\[x+200=500,\ \text{tức là}\ x=300. \]
	Do đó hệ phương trình đã cho có nghiệm duy nhất $(x;y)=(300;200)$.\\
	Vậy nhà trường đã mua $300$ quyển vở loại thứ nhất và $200$ quyển vở loại thứ hai.
	}
\end{vd}
%%=====Luyện tập 7
\begin{vd}
	Một nhóm khách vào cửa hàng bán trà sữa. Nhóm khách đó đã mua $6$ cốc trà sữa gồm trà sữa trân châu và trà sữa phô mai. Giá mỗi cốc trà sữa trân châu, trà sữa phô mai lần lượt là $33\,000$ đồng, $28\,000$ đồng. Tổng số tiền nhóm khách thanh toán cho cửa hàng là $188\,000$ đồng. Hỏi nhóm khách đó mua bao nhiêu cốc trà sữa mỗi loại?
	\loigiai{
	Gọi $x,y$ (cốc) lần lượt là số cốc trà sữa trân châu và trà sữa phô mai mà nhóm khách đã mua ($x,y\in \mathbb{N},x<6,y<6$).\\
	Vì nhóm khách đã mua $6$ cốc trà sữa nên ta có phương trình: $x+y=6$.\\
	Lại vì nhóm khách thanh toán cho cửa hàng là $188\,000$ đồng nên ta lại có phương trình
	\[33\,000x+28\,000y=188\,000\ \text{hay}\ 33x+28y=188. \]
	Do đó, ta có hệ phương trình $\heva{&x+y=6 &(1)\\&33x+28y=188 &(2)}$\\
	Từ phương trình (1), ta có $x=6-y$.\hfill(3)\\
	Thế (3) và (2), ta được
	\allowdisplaybreaks
	\begin{eqnarray*}
	33\cdot (6-y)+28y&=&188\\
	198-33y+28y&=&188\\
	10&=&5y\\
	y&=&2.
	\end{eqnarray*}
	Thay giá trị $y=2$ vào phương trình (3), ta có $x=6-2=4$.\\
	Do đó hệ phương trình có nghiệm duy nhất $(x;y)=(4;2)$.\\
	Vậy nhóm khách đó đã mua $4$ cốc trà sữa trân châu và $2$ cốc trà sữa phô mai.
	}
\end{vd}
%%=====Ví dụ 7
\begin{vd}
	Hai lớp $9\mathrm{A}$ và $9\mathrm{B}$ có tổng số $82$ học sinh. Trong dip tết trồng cây năm $2022$, mỗi học sinh lớp $9\mathrm{A}$ trông được $3$ cây, mỗi học sinh lợ $9\mathrm{B}$ trồng được $4$ cây nên cả hai lớp trồng được tổng sổ $288$ cây. Gọi $x$, $y$ lần lượt là số học sinh lớp $9\mathrm{A}$ và lớp $9\mathrm{B}$ $\left(x\in\mathbb{N}*,\mathrm{y}\in\mathbb{N}*\right)$.
	\begin{enumerate}
	\item Từ dữ liệu đã cho, lập hai phương trình bậc nhất hai ẩn biểu thi số học sinh của hai lớp và số cây trồng.
	\item Giải hệ hai phương trình bậc nhất hai ẫn và cho biết mỗi lớp có bao nhiêu học sinh.
	\end{enumerate}
	\loigiai{
	Trong ví dụ, ta tìm được số học sinh của mỗi lớp bằng cách lập và giải hệ hai phương trình bậc nhất hai ẩn.
	}
\end{vd}
Tổng quát, để giải bài toán bằng cách lập hệ hai phương trình bậc nhât hai ẩn, ta thực hiện như sau:
\begin{itemize}
	\item Bước 1: Lập hệ phương trình.
	\begin{itemize}
	\item Chọn hai ẩn biểu thị hai đại lượng chư biết và đặt điều kiện thích hợp cho các ẩn.
	\item Biểu diễn các đại lượng liên quan theo các ẩn và các đại lượng đã biết.
	\item Lập hệ hai phương trình bậc nhất hai ấn biểu thị mối quan hệ giữa các đại lượng.
	\end{itemize}
	\item Bước 2: Giải hệ phương trình nhận được.
	\item Bước 3: Kiểm tra nghiệm tìm được ở Bước 2 có thỏ mãn điều kiện của ẩn hay không, rồi trả lời bài toán.
\end{itemize}
\begin{vd}
	Hai ngăn của một kệ sách có tổng cộng $400$ cuốn sách. Nếu chuyển $80$ cuốn sách từ ngăn thứ nhất sang ngăn thứ hai thi số sách ở ngăn thứ hai gấp $3$ lần số sách ở ngăn thứ nhất. Tính số sách ở mỗi ngăn lúc đầu.
	\loigiai{
	Gọi $x$, $y$ lần lượt là số sách ở ngăn thứ nhất, ngăn thứ hai lúc đầu $\left(x\in\mathbb{N}*,y\in\mathbb{N}*\right)$. Tống số sách ở hai ngăn là $400$ cuốn, nên ta có phương trình $x+y=400$ $(1)$.\\
	Sau khi chuyển thì số sách ở ngăn thứ hai gấp $3$ lần số sách ở ngăn thứ nhất, nên ta có phương trình $y+80=3(x-80)$ $(2)$.\\
	Từ $(1)$ và $(2)$, ta có hệ phương trình $\heva{
	&x+y=400\\
	&y+80=3(x-80).\\}$\\
	Giải hệ phương trình ta được $\heva{
	&x=180\\ 
	&y=220\\}$ (thoả mãn).\\
	Vậy lúc đầu ngăn thứ nhất có $180$ cuốn sách, ngăn thứ hai có $220$ cuốn sách.
	}
\end{vd}
\begin{vd} 
	Một mảnh vườn hình chữ nhật có chu vi $64$ m. Nếu tăng chiều dài thêm $2$ m và tăng chiều rộng thêm $3$ m thì diện tích tăng thêm $88~\mathrm{m}^2$. Tính chiều dài, chiều rộng của mảnh vườn đó.
	\loigiai{
	Gọi chiều dài, chiều rộng ban đầu hình chữ nhật lần lượt là $x$ m, $y$ m.\\
	Điều kiện $0<x,y<32$.\\
	Vì mảnh vườn hình chữ nhật có chu vi $64$ m nên ta có phương trình $$2(x+y)=64 \Rightarrow 2x+2y=64 \quad(1).$$
	Diện tích mảnh vườn lúc đầu là $xy$ m$^2$.\\
	Khi tăng chiều dài thêm $2$ m và tăng chiều rộng thêm $3$ m thì diện tích tăng thêm $88$ m$^2$, ta được phương trình là
	$$(x+2)(y+3)=xy+88\Rightarrow xy+3x+2y=xy+88\Rightarrow 3x+2y=88\quad (2).$$
	Từ (1) và (2) ta được hệ phương trình $\heva{& x+y=32\\ &3x+2y=88.}$\\
	Ta có \allowdisplaybreaks
	\begin{eqnarray*}
	&&\heva{& 2x+2y=64\\ & 3x+2y=88}\\
	&&\heva{& x=24\\ & 2x+2y=64}\\
	&&\heva{& x=24\\ & 2y=64-2x}\\
	&&\heva{& x=24\\ & 2y=64-2\cdot 24}\\
	&&\heva{& x=24\\ & 2y=16}\\
	&&\heva{& x=24\\ & y=8.}
	\end{eqnarray*}
	Vậy chiều dài ban đầu mảnh vườn là $24$ m, chiều rộng ban đầu mảnh vườn là $8$ m.
	}
\end{vd}
\begin{vd}%[9D3K5]
	Một thửa ruộng hình chữ nhật, nếu tăng chiều dài thêm $2$ m, chiều rộng thêm $3$ m thì diện tích tăng thêm $100$ m$^2$. Nếu giảm chiều dài và chiều rộng đi $2$ m thì diện tích giảm đi $68$ m$^2$. Tính diện tích của thửa ruộng đó.
	\loigiai{
	Gọi chiều dài thửa ruộng là $x$ (m) ($x>0$) và chiều rộng $y$ (m) ($y>0$).\\
	Nếu tăng chiều dài thêm $2$ m, chiều rộng thêm $3$ m thì diện tích tăng thêm $100$ m$^2$. \\
	Ta có phương trình $(x+2) \cdot (y+3) = xy+100$ \hfill (1)\\
	Nếu giảm chiều dài và chiều rộng đi $2$ m thì diện tích giảm đi $68$ m$^2$. \\
	Ta có phương trình $(x-2)\cdot (y-2) = xy - 68$ \hfill (2)\\
	Từ (1), (2) ta có hệ phương trình 
	$\left\{\begin{aligned}&(x+2)(y+3)=x y+100\\&(x-2)(y-2)=x y-68\end{aligned} \right. \Rightarrow \left\{\begin{aligned}&3 x+2 y=94\\&-2 x-2 y=-72\end{aligned} \right. \Rightarrow \heva{&x=22 \\&y=14.}$\\
	Vậy diện tích thửa ruộng là $22 \cdot 14 = 308$ $\left(\text{m}^2 \right)$. 	
	}
\end{vd}
\begin{vd}%[9D3K5]
	Tháng thứ nhất hai tổ sản xuất được $900$ chi tiết máy. Tháng thứ hai tổ I vượt mức $15 \%$ và tổ thứ II vượt mức $10 \%$ so với tháng thứ nhất. Vì vậy hai tổ đã sản xuất được $1010$ chi tiết máy. Hỏi tháng thứ nhất mỗi tổ sản xuất được bao nhiêu chi tiết máy?
	\loigiai{
	Gọi số chi tiết máy của tổ I sản xuất được trong tháng thứ nhất là $x$ (chi tiết, $x \in \mathbb{N}$).\\
	Gọi số chi tiết máy của tổ II sản xuất được trong tháng thứ nhất là $y$ (chi tiết, $y \in \mathbb{N}$).\\
	Tháng thứ nhất hai tổ sản xuất được $900$ chi tiết máy có phương trình: $x+y=900$ \hfill (1) \\	
	Tháng thứ hai tổ I vượt mức $15 \%$ và tổ II vượt mức $10\%$ nên đã sản xuất được $1010$ chi tiết máy, ta có phương trình 
	\[\dfrac{115x}{100} +\dfrac{110y}{100} =100 \tag{2} \]
	Từ (1), (2) ta được hệ $\heva{&x+y=900 \\ &\dfrac{115x}{100}+\dfrac{110y}{100}=1010} \Rightarrow \heva{&x=400 \\&y=500}$ (thỏa mãn điều kiện). \\
	Vậy trong tháng thứ nhất tổ I sản xuất được $400$ chi tiết, tổ II sản xuất được $500$ chi tiết máy.
	}
\end{vd}
\begin{vd}%[9D3K5]
	Hai kho chứa $450$ tấn hàng. Nếu chuyển $50$ tấn từ kho I sang kho II thì số hàng kho II bằng $\dfrac{4}{5}$ số hàng kho I. Tính số hàng mỗi kho.
	\loigiai{
	Gọi số hàng trong kho I là $x$ (tấn, $x>0$); số hàng trong kho II là y (tấn, $y>0$).	\\
	Hai kho chứa $450$ tấn hàng nên ta có phương trình:
	\[x+y=450 \tag{1}\]
	Nếu chuyển $50$ tấn từ kho I sang kho II thì số hàng kho II bằng $\dfrac{4}{5}$ số hàng kho I ta có phương trình \[y+50=\dfrac{4}{5}\cdot (x-50) \tag{2}\]
	Từ (1), (2) ta có hệ phương trình $\heva{&x+y=450 \\&y+50=\dfrac{4}{5}\cdot (x-50)} \Rightarrow \heva{&x=300 \\&y=150}$. \\
	Vậy trong kho I có $300$ tấn hàng, kho II có $150$ tấn hàng.
	}
\end{vd}
\begin{dang}{Toán liên quan đến chữ số}
	Viết số dưới dạng thập phân 
	\begin{align*}
	\overline{a b}&=10 \cdot a+b\\
	\overline{a b c}&=100 \cdot a+10 \cdot b+c
	\end{align*}
	với điều kiện
	\begin{align*}
	&0<a \leq 9;~a \in \mathbb{N}\\
	&0 \leq b, c \leq 9;~b,\, c \in \mathbb{N}
	\end{align*}
\end{dang}
\begin{vd}%[9D3K5]
	Tìm số tự nhiên có hai chữ số biết rằng tổng các chữ số của nó bằng $10$ và nếu viết số đó theo thứ tự ngược lại thì được số mới nhỏ hơn số ban đầu $18$ đơn vị.
	\loigiai{
	Gọi chữ số hàng chục là $x$, chữ số hàng đơn vị là $y$: $0<x;\, y\leq 9$; $x,\,y \in \mathbb{N}$. \\
	Theo đề bài, ta có hệ phương trình
	\[\heva{&x+y=10 \\ &\overline{xy}=\overline{yx}+18} \Rightarrow \heva{&x+y=10 \\ &10x+y=10y+x + 18} \Rightarrow \heva{&x+y=10 \\&x-y=2} \Rightarrow \heva{&x=6 \\ &y=4} \text{ (thỏa mãn điều kiện).}\]
	Vậy số cần tìm là $64$.
	}
\end{vd}
\begin{vd}%[9D3K5]
	Một số tự nhiên có hai chữ số. Tỉ số giữa chữ số hàng chục và chữ số hàng đơn vị là $\dfrac{2}{3}$. Nếu viết thêm chữ số $1$ xen vào giữa thì được số mới lớn hơn số đã cho là $370$ đơn vị. Tìm số đã cho.
	\loigiai{
	Gọi chữ số hàng chục là $x$, chữ số hàng đơn vị là $y$.\\
	Điều kiện: $0<x;\, y\leq 9,\ x,\,y \in \mathbb{N}$. \\
	Theo đề bài ta có hệ phương trình
	\[\heva{&\dfrac{x}{y}=\dfrac{2}{3}\\& \overline{x 1 y}=\overline{x y}+370 } \Rightarrow \heva{&3x=2y \\&100x+10+y=10x+y+370} \Rightarrow \heva{&x=4 \\&y=6} \text{ (thỏa mãn điều kiện).}\]
	Vậy số đã cho là $46$.
	}
\end{vd}
\begin{vd}%[9D3B5]
	Tìm số có hai chữ số, biết rằng tổng của chữ số hàng đơn vị và hai lần chữ số hàng chục bằng $10$. Ngoài ra, nếu đổi chữ số hàng chục và hàng đơn vị cho nhau thì sẽ được số mới nhỏ hơn số ban đầu $18$ đơn vị. 
	\loigiai{ 
	Gọi số có hai chữ số là $\overline{xy} = 10x + y$, với $x$,$y\in\mathbb{N}$, $1 \le x$, $y \le 9$.	
	\\Với giả thiết:
	\begin{itemize}
	\item Tổng của chữ số hàng đơn vị và hai lần chữ số hàng chục bằng $10$, ta được: $$ 2x + y = 10. \quad (1)$$
	\item Nếu đổi chỗ chữ số hàng chục và hàng đơn vị cho nhau thì sẽ được số mới $(\overline{xy} = 10y + x)$ nhỏ hơn số ban đầu $18$ đơn vị, ta được:
	$$ \overline{xy}- \overline{yx} = 18 \Rightarrow (10x + y) -(10y + x) = 18 \Rightarrow x - y = 2. \quad (2)$$
	\end{itemize}
	Từ $(1)$ và $(2)$, ta có hệ phương trình: $\heva{&2x + y = 10 \\ & x - y = 2} \Rightarrow \heva{& x = 4 \\ &y = 2}$, thỏa mãn điều kiện.
	\\ Vậy, số cần tìm là $42$. 
	}
\end{vd}
\begin{vd}%[9D3B5]
	Tìm một số có hai chữ số. Biết rằng chữ số hàng chục lớn hơn chữ số hàng đơn vị $6$ đơn vị. Nếu viết xen chữ số $0$ vào giữa chữ số hàng chục và chữ số hàng đơn vị thì số tự nhiên đó tăng $720$ đơn vị.
	\loigiai{
	Gọi số có hai chữ số là $\overline{xy}=10x+y$, với $x$, $y \in \mathbb{N}$, $1 \le x \le 9$, $0 \le y \le 9$.\\
	Với giả thiết
	\begin{itemize}
	\item Chữ số hàng chục lớn hơn chữ số hàng đơn vị $6$ đơn vị, ta được:
	$$ x-y=6. \quad (1) $$
	\item Khi viết xen chữ số $0$ vào giữa chữ số hàng đơn vị (được số $\overline{x0y}=100x+y$) thì số tự nhiên đó tăng $720$ đơn vị, ta được:
	$$ (100x+y)-(10x+y)=720 \Rightarrow x=8. \quad (2) $$
	\end{itemize}	
	Từ $(1)$ và $(2)$, ta có hệ phương trình $\heva{&x-y=6\\&x=8} \Rightarrow \heva{&x=8\\&y=2}.$\\
	Vậy số cần tìm là $82$.
	}
\end{vd}
\begin{dang}{Toán làm chung công việc}
	\begin{itemize}
	\item Toán chung công việc có ba đại lượng tham gia là toàn bộ công việc, phần việc trong một đơn vị thời gian; thời gian.
	\item Nếu một đội làm xong công việc trong $x$ ngày thì một ngày đội đó làm được $\dfrac{1}{x}$ công việc.
	\item Xem toàn bộ công việc là $1$.
	\end{itemize}
\end{dang}
\begin{vd}Hai đội công nhân cùng làm một đoạn đường trong $24$ ngày thì xong. Mỗi ngày, đội I làm được nhiều gấp rưỡi đội II. Hỏi nếu làm một mình thì mổi đội làm xong đoạn đường đó trong bao lâu? (Giả sử năng suất của mổi đội là không đổi).
	\loigiai{Gọi $x$ là số ngày để đội I hoàn thành công việc nếu làm riêng một mình; $y$ là số ngày để đội II hoàn thành công việc nếu làm riêng một mình. Điều kiện $x>0$ và $y>0$.
	\\Mỗi ngày đội I làm được $\dfrac{1}{x}$ (công việc) và đội II làm được $\dfrac{1}{y}$ (công việc).\\
	Mỗi ngày đội I làm được nhiểu gấp rưỡi đội II nên ta có phương trình $\dfrac{1}{x}=1,5 \cdot \dfrac{1}{y}$ hay $ \dfrac{1}{x}=\dfrac{3}{2} \cdot \dfrac{1}{y}$.\\
	Hai đội làm chung trong $24$ ngày thì xong công việc nên mỗi ngày, hai đội làm chung thì được $\dfrac{1}{24}$ (công việc). Ta có phương trình
	$$
	\dfrac{1}{x}+\dfrac{1}{y}=\dfrac{1}{24}.
	$$
	Từ (1) và (2), ta có hệ phương trình
	(I) $\heva{&\dfrac{1}{x}=\dfrac{3}{2} \cdot \dfrac{1}{y} \\& \dfrac{1}{x}+\dfrac{1}{y}=\dfrac{1}{24}.}$
	\\ Đặt $u=\dfrac{1}{x}$ và $v=\dfrac{1}{y}$ thì ta có hệ phương trình bậc nhất hai ẩn mới là $u$ và $v$
	(II) $\heva{&u=\dfrac{3}{2}v\qquad (3) \\& u+v=\dfrac{1}{24}.\qquad (4) }$
	\\Thế $u=\dfrac{3}{2} v$ vào phương trình (4), ta được $\dfrac{3}{2} v+v=\dfrac{1}{24}$, hay $\dfrac{5}{2} v=\dfrac{1}{24}$, suy ra $v=\dfrac{1}{60}$. 
	\\Do đó $u=\dfrac{3}{2} v=\dfrac{3}{2} \cdot \dfrac{1}{60}=\dfrac{1}{40}$.
	Từ đó, ta có
	$$
	u=\dfrac{1}{x}=\dfrac{1}{40} \text { suy ra } x=40 ; \quad v=\dfrac{1}{y}=\dfrac{1}{60} \text { suy ra } y=60 .
	$$
	Các giá trị tìm được của $x$ và $y$ thoả mãn điều kiện của ẩn. \\
	Trả lời: Nếu làm một mình thì đội I làm xong đoạn đường đó trong $40$ ngày, còn đội II làm xong trong $60$ ngày.
	}
\end{vd}
\begin{vd}%[9D3K5]
	Hai công nhân cùng làm một công việc trong $18$h thì xong. Nếu người thứ nhất là $6$h và người thứ hai là $12$h thì chỉ hoàn thành $50 \%$ công việc. Hỏi nếu làm riêng thì mỗi người hoàn thành công việc trong bao lâu?
	\loigiai{
	Gọi thời gian người thứ nhất làm một mình thì hoàn thành công việc là $x$ (giờ, $x>0$); người thứ hai làm một mình hoàn thành công việc là $y$ (giờ, $y>0$).\\
	Trong $1$h người thứ nhất làm được $\dfrac{1}{x}$ công việc, người thứ hai làm được $\dfrac{1}{y}$ công việc.\\
	Hai người làm chung $18$h thì xong, ta có phương trình:
	\[\dfrac{1}{x} +\dfrac{1}{y}=\dfrac{1}{18} \tag{1}\]
	Nếu người thứ nhất làm trong $6$h và người thứ hai làm trong $12$h thì hoàn thành $50\%$ công việc, ta có phương trình:
	\[\dfrac{6}{x} + \dfrac{12}{y} =50\% \tag{2} \]
	Từ (1), (2) ta có hệ phương trình:
	\[\heva{&\dfrac{1}{x}+\dfrac{1}{y}=\dfrac{1}{18}\\&\dfrac{6}{x}+\dfrac{12}{y}=50\%} \Rightarrow \heva{&x=36\\&y=36} \text{ (thỏa mãn điều kiện).} \]
	Vậy nếu làm riêng thì người thứ nhất hoàn thành công việc trong $36$h và người thứ hai hoàn thành công việc trong $36$h.
	}
\end{vd}
\begin{vd}
	Nếu hai vòi nước cùng chảy vào một bể không có nước thì bể sẽ đầy trong 1 giờ 20 phút. Nếu mở riêng vòi thứ nhất trong $ 10 $ phút và vòi thứ hai trong $ 12 $ phút thì chỉ được $\dfrac{2}{15}$ bể nước. Hỏi nếu mở riêng từng vòi thì thời gian để mỗi vòi chảy đẩy bể nước là bao nhiēu phút?
	\loigiai{Gọi $ x, y $ (phút) lần lượt là thời gian mỗi vòi chảy đầy bể nếu mở riêng từng vòi.\\
	Trong một giờ, vòi thứ nhất chảy được $ \dfrac{1}{x} $ (bể) và vòi thứ hai chảy được $ \dfrac{1}{y} $ (bể). \\
	Hai vòi cùng chảy vào bể thì sau $ 1 $ giờ $ 20 $ phút $ =80 $ phút thì đầy bể nên $ \dfrac{80}{x}+\dfrac{80}{y}=1 $.\\
	Nếu vòi thứ nhất chảy trong $ 10 $ phút và vòi thứ hai chảy trong $ 12 $ phút thì chỉ chảy được $\dfrac{2}{15}$ bể nên $ \dfrac{10}{x}+\dfrac{12}{y}=\dfrac{2}{15} $.\\
	Ta có hệ phương trình $ \heva{&\dfrac{80}{x}+\dfrac{80}{y}=1\\&\dfrac{10}{x}+\dfrac{12}{y}=\dfrac{2}{15}.} $\\
	Đặt ẩn phụ $ u=\dfrac{1}{x} $ và $ v=\dfrac{1}{y} $, ta đưa về hệ phương trình $ \heva{&80u+80v=1\\&10u+12v=\dfrac{2}{15}} $, giải hệ phương trình ta được $ \heva{&u=\dfrac{1}{120}\\&v=\dfrac{1}{240}} $, tức là $ \heva{&\dfrac{1}{x}=\dfrac{1}{120}\\& \dfrac{1}{y}=\dfrac{1}{240}.}$ Suy ra $ \heva{x=120\\y=240.} $
	\\Vậy nếu làm riêng thì người thứ nhất hoàn thành xong công việc trong $ 120 $ phút và người thứ hai hoàn thành công việc trong $ 240 $ phút.}
\end{vd}
\begin{vd}%[9D3K5]
	Hai voi nước cùng chảy vào một bể không có nước thì sau $1$h$30$ phút sẽ đầy bể. Nếu mở vòi I chảy trong $15$ phút rồi khóa lại và mở vòi thứ II chảy trong $20$ phút thì được $\dfrac{1}{5}$ bể. Hỏi nếu mỗi vòi chảy riêng thì bao lâu đầy bể?
	\loigiai{
	Ta có $1$h$30$ phút $=\dfrac{3}{2}$h; $15$ phút $=\dfrac{1}{4}$h; $20$ phút $=\dfrac{1}{3}$h. \\
	Gọi thời gian vòi I chảy một mình đầy bể là $x$ (h; $x>0$); \\
	Thời gian vòi II chảy một mình đầy bể là $y$ (h; $y>0$). \\
	Hai vòi cùng chảy thì sau $1$h$30$ phút đầy bể, ta có phương trình 
	\[\dfrac{1}{x}+\dfrac{1}{y}=\dfrac{2}{4} \tag{1} \]
	Vòi I chảy trong $15$ phút và vòi II chảy trong $20$ phút thì được $\dfrac{1}{5}$ bể ta có phương trình\[\dfrac{1}{4x} + \dfrac{1}{3y} = \dfrac{1}{5} \tag{2}\]
	Từ (1), (2) ta có hệ phương trình 
	\[\heva{&\dfrac{1}{x}+\dfrac{1}{y}=\dfrac{2}{3} \\&\dfrac{1}{4x}+\dfrac{1}{3y}=\dfrac{1}{5}} \Rightarrow \heva{&x=\dfrac{15}{4} \\&y=\dfrac{5}{2}} \text{ (thỏa mãn điều kiện).} \]
	Vậy nếu chảy riêng thì vòi I chảy đầy bể trong $\dfrac{15}{4}$h, vòi II chảy đầy bể trong $\dfrac{5}{2}$h.
	}
\end{vd}
\begin{vd}%[9D3K5]
	Hai vòi nước cùng chảy vào một bể nước cạn (không có nước) thì sau $4\dfrac{4}{5}$ giờ đầy bể. Nếu lúc đầu chỉ mở vòi thứ nhất và $9$ giờ sau mới mở thêm vòi thứ hai thì sau $\dfrac{6}{5}$ giờ nữa mới đầy bể. Hỏi nếu ngay từ đàu chỉ mở vòi thứ hai thì sau bao lâu sẽ đầy bể.
	\loigiai{
	Gọi $x$ và $y$ là thời gian để vòi thứ nhất và vòi thứ hai chảy một mình thì đầy bể ($x>0$, $y>0$, đơn vị giờ).\\
	Do đó:
	\begin{itemize}
	\item Trong $1$ giờ vòi thứ nhất chảy được $\dfrac{1}{x}$ phần của bể.
	\item Trong $1$ giờ vòi thứ hai chảy được $\dfrac{1}{y}$ phần của bể.
	\item Trong $1$ giờ cả hai vòi chảy được $\dfrac{1}{x}+\dfrac{1}{y}$ phần của bể.
	\end{itemize}	
	Hai vòi cùng chảy thì trong $4\dfrac{4}{5}=\dfrac{24}{5}$ giờ sẽ đầy bể, nên mỗi giờ hai vòi chảy được $1 : \dfrac{24}{5}=\dfrac{5}{24}$ (bể). Do đó, ta có phương trình $\dfrac{1}{x}+\dfrac{1}{y}=\dfrac{5}{24}. \quad (1)$\\
	Nếu lúc đầu chỉ mở vòi thứ nhất và $9$ giờ sau mới mở thêm vòi thứ hai thì sau $\dfrac{6}{5}$ giờ nữa mới đầy bể. Do đó, ta có phương trình $\dfrac{9}{x}+\dfrac{5}{24} \cdot \dfrac{6}{5}=1. \quad (2)$\\
	Từ $(1)$ và $(2)$, ta có hệ phương trình 
	\begin{align*}
	\heva{&\dfrac{1}{x}+\dfrac{1}{y}=\dfrac{5}{24}\\&\dfrac{9}{x}+\dfrac{5}{24} \cdot \dfrac{6}{5}=1} \Rightarrow \heva{&\dfrac{1}{x}+\dfrac{1}{y}=\dfrac{5}{24}\\&\dfrac{9}{x}+\dfrac{1}{4} =1} \Rightarrow \heva{&\dfrac{1}{x}+\dfrac{1}{y}=\dfrac{5}{24}\\&36+x=4x} \Rightarrow \heva{&y=8\\&x=12.}
	\end{align*}
	Vậy nếu vòi thứ hai chảy một mình thì sau $8$ giờ sẽ đầy bể.
	}
\end{vd}
\begin{vd}%[9D3K5]
	Hai vòi nước cùng chảy vào một bể không có nước thì sau $1$ giờ $20$ phút sẽ đầy. Nếu mở vòi thứ nhất chảy trong $10$ phút và vòi thứ hai chảy trong $12$ phút thì đầy $\dfrac{2}{15}$ bể. Hỏi mỗi vòi chảy một mình thì sau bao lâu mới đầy bể?
	\loigiai{
	Ta có thể lựa chọn một trong hai cách trình bày sau:\\
	\textit{Cách 1: Thiết lập ẩn thông qua giá trị cần tìm.}\\
	Gọi $x$ là thời gian để vòi I chảy một mình cho đầy bể, điều kiện $x>0$. Suy ra, mỗi giờ vòi I chảy được $\dfrac{1}{x}$ bể.\\
	Gọi $y$ là thời gian để vòi II chảy một mình cho đầy bể, điều kiện $y>0$. Suy ra, mỗi giờ vòi II chảy được $\dfrac{1}{y}$ bể.\\
	Ta thực hiện đổi đơn vị:
	\begin{center}
	$1$ giờ $20$ phút $=1+\dfrac{20}{60}=\dfrac{4}{3}$ giờ; \quad $10$ phút $=\dfrac{1}{6}$ giờ; \quad $12$ phút $=\dfrac{1}{5}$ giờ.
	\end{center}
	Với giả thiết:
	\begin{itemize}
	\item Hai vòi nước cùng chảy vào một bể không có nước thì sau $1$ giờ $20$ phút sẽ đầy, ta được $$\dfrac{4}{3} \left( \dfrac{1}{x}+\dfrac{1}{y} \right)=1 \Rightarrow \dfrac{1}{x}+\dfrac{1}{y}=\dfrac{3}{4}. \quad (1) $$
	\item Nếu mở vòi thứ nhất chảy trong $10$ phút và vòi thứ hai chảy trong $12$ phút thì đầy $\dfrac{2}{15}$, ta được $$\dfrac{1}{6}\cdot \dfrac{1}{x}+\dfrac{1}{5}\cdot \dfrac{1}{y}=\dfrac{2}{15}\Rightarrow \dfrac{1}{6x}+\dfrac{1}{5y}=\dfrac{2}{15}. \quad (2)$$
	\end{itemize}
	Từ $(1)$ và $(2)$, ta có hệ phương trình:
	$$ \heva{&\dfrac{1}{x}+\dfrac{1}{y}=\dfrac{3}{4}\\& \dfrac{1}{6x}+\dfrac{1}{5y}=\dfrac{2}{15}} \Rightarrow \heva{&\dfrac{6}{6x}+\dfrac{5}{5y}=\dfrac{3}{4}\\&\dfrac{1}{6x}+\dfrac{1}{5y}=\dfrac{2}{15}} \quad \text{(I)} $$
	Đặt $\heva{&u=\dfrac{1}{6x}\\&v=\dfrac{1}{5y}}$. Khi đó, hệ có dạng
	$$ \heva{&6u+5v=\dfrac{3}{4}\\&u+v=\dfrac{2}{15}} \Rightarrow \heva{&u=\dfrac{1}{12}\\&v=\dfrac{1}{20}} \Rightarrow \heva{&\dfrac{1}{6x}=\dfrac{1}{12}\\& \dfrac{1}{5y}=\dfrac{1}{20}} \Rightarrow \heva{&x=2\\&y=4.} $$
	Vậy vòi I chảy trong $2$ giờ sẽ đầy bể, vòi II chảy trong $4$ giờ sẽ đầy bể.\\
	\textit{Cách 2: Thiết lập ẩn thông qua giá trị trung gian.}\\
	Giả sử mỗi giờ vòi I chảy được $x$ phần bể, điều kiện $x>0$.\\
	Giả sử mồi giờ vòi II chảy được $y$ phần bể, điều kiện $y>0$.\\
	Với giả thiết:
	\begin{itemize}
	\item Hai vòi nước cùng chảy vào một bể không có nước thì sau $1$ giờ $20$ phút sẽ đầy, ta được $$\dfrac{4}{3}(x+y)=1 \Rightarrow 4x+4y=3. \quad (3)$$
	\item Nếu mở vòi thứ nhất chảy trong $10$ phút và vòi thứ hai chảy trong $12$ phút thì đầy $\dfrac{2}{15}$ bể, ta được $$\dfrac{1}{6}\cdot x+\dfrac{1}{5}\cdot y=\dfrac{2}{15} \Rightarrow 5x+6y=4. \quad (4)$$
	\end{itemize}
	Từ $(3)$ và $(4)$, ta có hệ phương trình $\heva{&4x+4y=3\\&5x+6y=4} \Rightarrow \heva{&x=\dfrac{1}{2}\\&y=\dfrac{1}{4}}.$\\
	Vậy vòi I chảy trong $2$ giờ sẽ đầy bể, vòi II chảy trong $4$ giờ sẽ đầy bể.
	}
\end{vd}
\begin{dang}{Toán chuyển động}
	\begin{itemize}
	\item Toán chuyển động có ba đại lượng tham gia: vận tốc ($v$), thời gian ($t$), quãng đường ($S$).
	\item $v=\dfrac{S}{t}$, $S=v \cdot t$, $t=\dfrac{S}{v}$.
	\end{itemize}
\end{dang}
\begin{vd}Một chiếc xe khách đi từ Thành phố Hồ Chí Minh đến Cần Thơ, quãng đường dài $170$ km. Sau khi xe khách xuất phát $1$ giờ $40$ phút, một chiếc xe tải bắt đầu đi từ Cần Thơ về Thành phố Hồ Chí Minh và gặp xe khách sau đó $40$ phút. Tính vận tốc của mỗi xe, biết rằng mỗi giờ xe khách đi nhanh hơn xe tải $15$ km.
	\loigiai{Gọi $x$ (km/h) là vận tốc của xe tải và $y$ (km/h) là vận tốc xe khách $(x, y>0)$. 
	\\Do mỗi giờ xe khách đi nhanh hơn xe tải $15$ km nên $x=y+15$.
	\\Do sau khi xe khách xuất phát $1$ giờ $40$ phút, một chiếc xe tải bắt đầu đi từ Cần Thơ về Thành phố Hồ Chí Minh và gặp xe khách sau đó $40$ phút nên tổng quảng đường hai xe là $170$. Từ đó, ta có phương trình $\dfrac{2}{3}y+\left(2+\dfrac{1}{3} \right)x=170$.
	\\Từ đó, ta có hệ phương trình $\heva{& x=y+15\\&\dfrac{2}{3}y+\left(2+\dfrac{1}{3} \right)x=170.}$
	\\Giải hệ phương trình trên, ta có nghiệm là $\heva{&x=60\\&y=45.}$
	\\Vậy vận tốc của xe khách $60$ (km/h), vận tốc của xe tải $45$ (km/h).
	}
\end{vd}
\begin{vd}%[9D3K5]
	Một ca nô đi từ A đến B với vận tốc và thời gian dự định. Nếu ca nô tăng vận tốc thêm $3$ km/h thì thời gian rút ngắn được $2$ giờ. Nếu ca nô giảm vận tốc đi $3$ km/h thì thời gian tăng $3$ giờ. Tính vận tốc và thời gian dự định của ca nô.
	\loigiai{
	Gọi vận tốc dự định của ca nô là $x$ (km/h; $x>3$) và thời gian dự định đi từ A đến B là $y$ (giờ; $y>0$).\\
	Nếu ca nô tăng vận tốc thêm $3$ km/h thì thời gian rút ngắn được $2$ giờ, ta có phương trình \[(x+3)\cdot (y-2)=x\cdot y \tag{1}\]
	Nếu ca nô giảm vận tốc đi $3$ km/h thì thời gian tăng thêm $3$ giờ, ta có phương trình \[(x-3)\cdot (y+3)=x \cdot y \tag{2}\]
	Từ (1), (2) ta có hệ phương trình 
	\[\heva{&(x+3) \cdot(y-2)=x y\\&(x-3) \cdot(y+3)=x y} \Rightarrow \heva{&x=15 \\&y=12} \text{ (thỏa mãn điều kiện).}\]
	Vậy vận tốc dự định của ca nô là $15$ km/h
	và thời gian dự định của ca nô là $12$ giờ. 
	}
\end{vd}
\begin{vd}%[9D3K5]
	Một ca nô chạy trên sông trong $8$ giờ xuôi dòng được $81$km và ngược dòng $105$ km. Một lần khác, ca nô chạy trên sông trong $4$ giờ xuôi dòng $54$ km và ngược dòng $42$ km. Tính vận tốc riêng của ca nô và vận tốc dòng nước. (Biết vận tốc riêng của ca nô; vận tốc dòng nước không đổi).
	\loigiai{
	Gọi vận tốc riêng của ca nô là $x$ (km/h; $x>0$)
	và vận tốc dòng nước là $y$ (km/h; $x>y>0$). \\
	Suy ra vận tốc xuôi dòng của ca nô là $(x+y)$ (km/h) và vận tốc ngược dòng của ca nô là $(x-y)$ (km/h).\\
	Ca nô chạy trong $8$ giờ xuôi dòng được $81$ km và ngược dòng được $105$ km, ta có phương trình 
	\[\dfrac{81}{x+y} + \dfrac{105}{x-y} = 8 \tag{1}\]
	Ca nô chạy trong $4$ giờ xuôi dòng dược $54$ km và ngược dòng được $42$ km, ta có phương trình 
	\[\dfrac{54}{x+y}+\dfrac{42}{x-y}=4 \tag{2}\]
	Từ (1), (2) ta có hệ phương trình $\left\{\begin{aligned}&\dfrac{81}{x+y}+\dfrac{105}{x-y}=8\\&\dfrac{54}{x+y}+\dfrac{42}{x-y}=4.\end{aligned} \right.$ \\
	Đặt $u = \dfrac{1}{x+y}$; $v=\dfrac{1}{x-y}$, ta có $\left\{\begin{aligned}&81 u+105 v=8\\&54 u+42 v=4.\end{aligned} \right.$ \\
	Giải hệ ta có $\left\{\begin{aligned}&u=\dfrac{1}{27}\\&v=\dfrac{1}{21}\end{aligned} \right. \Rightarrow \left\{\begin{aligned}&x+y=27\\&x-y=21\end{aligned} \right. \Rightarrow \left\{\begin{aligned}&x=24\\&y=3\end{aligned} \text{ (thỏa mãn điều kiện).}\right.$ \\
	Vậy vận tốc riêng của ca nô là $24$ km/h và vận tốc dòng nước là $3$ km/h.
	}
\end{vd}
\begin{vd}%[9D3K5]
	Một xe máy đi từ A đến B trong thời gian đã định. Nếu đi với vận tốc $45$ km/h sẽ tới B chậm mất nữa giờ. Nếu đi với vận tốc $60$ km/h sẽ tới B sớm $45$ phút. Tính quãng đường AB và thời gian dự định.
	\loigiai{
	Ta có $45$ phút $=\dfrac{3}{4}$ giờ. \\
	Gọi quãng đường AB là $x$ (km; $x>0$) và thời gian dự định đi từ A đến B là $y$ (h, $y>0$).\\
	Nếu đi với vận tốc $45$ km/h sẽ tới B chậm nữa giờ, ta có phương trình \[x=45 \cdot \left(y+\dfrac{1}{2} \right) \tag{1} \]
	Nếu đi với vận tốc $60$ km/h sẽ tới B sớm $45$ phút, ta có phương trình
	\[y=60 \cdot \left(y-\dfrac{3}{4} \right) \tag{2} \]
	Từ (1), (2) ta có hệ phương trình $\heva{&x=45 \cdot \left(y+\dfrac{1}{2} \right) \\ &y=60 \cdot \left(y-\dfrac{3}{4} \right)} \Rightarrow \heva{&x=225 \\&x=4{,}5}$.\\
	Vậy quảng đường AB dài $225$ km và thời gian dự định đi từ A đến B hết $4{,}5$ giờ.	
	}
\end{vd}
\begin{vd}%[9D3B5]
	Lúc $7$ giờ một người đi xe máy khởi hành từ $A$ với vận tốc $40$km/h. Sau đó, lúc $8$ giờ $30$ phút, một người khác cũng đi xe máy từ $A$ đuổi theo với vận tốc $60$km/h. Hỏi hai người gặp nhau lúc mấy giờ?
	\loigiai{
	Ta thực hiện đổi đơn vị: $8$ giờ $30$ phút $=8+\dfrac{30}{60}=\dfrac{17}{2}$ (giờ).\\
	Gọi $x$ là thời gian hai người gặp nhau, điều kiện $x>\dfrac{17}{2}.$\\
	Gọi $y$ là độ dài quãng đường từ $A$ tới điểm gặp nhau, điều kiện $y>0.$\\
	Với giả thiết:
	\begin{itemize}
	\item Người thứ nhất đi với vận tốc $40$km/h và xuất phát lúc $7$ giờ, ta được:
	$$ 40(x-7)=y \Rightarrow 40x-y=280. \quad (1) $$
	\item Người thứ hai đi với vận tốc $60$km/h và xuất phát lúc $8$ giờ $30$ phút, ta được:
	$$ 60 \left( x-\dfrac{17}{2} \right) =y \Rightarrow 60x-y=510. \quad (2) $$
	\end{itemize}	
	Từ $(1)$ và $(2)$, ta có hệ phương trình $\heva{&40x-y=280\\&60x-y=510}\Rightarrow \heva{&x=11\dfrac{1}{2}=11\text{ giờ } 30 \text{ phút}\\&y=180}.$\\
	Vậy họ gặp nhau lúc $11$ giờ $30$ phút.
	}
\end{vd}
\begin{vd}%[9D3K5]
	Hai người ở hai địa điểm $A$ và $B$ cách nhau $3.6$km, khởi hành cùng một lúc, đi ngược chiều và gặp nhau ở một địa điểm cách $A$ là $2$km. Nếu cả hai cùng giữ nguyên vận tốc như trong trường hợp trên, nhưng người đi chậm xuất phát trước người kia $6$ phút thì họ sẽ gặp nhau ở chính giữa quãng đường. Tính vận tốc của mỗi người.
	\loigiai{
	Đổi $6$ phút $=\dfrac{1}{10}$ giờ.\\
	Gọi $x$ là vận tốc của người đi nhanh hơn ($x>0$, đơn vị km/h).\\
	Gọi $y$ là vận tốc của người đi chậm hơn ($y>0$, đơn vị km/h).\\
	Hai người khởi hành cùng một lúc, đi ngược chiều nhau và gặp nhau ở một địa điểm cách $A$ là $2$km (nghĩa là cách $B$ là $1.6$km). Lúc đó
	\begin{itemize}
	\item Người đi nhanh mất $\dfrac{2}{x}$ (h).
	\item Người đi chậm mất $\dfrac{1.6}{y}$ (h).
	\end{itemize}	
	Do đó, ta có phương trình $\dfrac{2}{x}=\dfrac{1.6}{y}. \quad (1)$\\
	Nếu cả hai cùng giữ nguyên vận tốc như trong trường hợp trên, nhưng người đi chậm xuất phát trước người kia $6$ phút thì họ sẽ gặp nhau chính giữa quãng đường. Lúc đó:
	\begin{itemize}
	\item Người đi nhanh mất $\dfrac{1.8}{x}$ (h).
	\item Người đi chậm mất $\dfrac{1.8}{y}+\dfrac{1}{10}$ (h).
	\end{itemize}
	Do đó, ta có phương trình $\dfrac{1.8}{x}=\dfrac{1.8}{y}+\dfrac{1}{10}. \quad (2)$\\
	Từ $(1)$ và $(2)$ ta có hệ $\heva{&\dfrac{2}{x}=\dfrac{1.6}{y}\\&\dfrac{1.8}{x}=\dfrac{1.8}{y}+\dfrac{1}{10}} \Rightarrow \heva{&x=2.5\\&y=2}.$\\
	Vậy vận tốc của người đi nhanh là $2.5$km/h và vận tốc của người đi chậm là $2$km/h.
	}
\end{vd}
\begin{vd}%[9D3G5]
	Hai cano cùng khởi hành từ bến $A$ và $B$ cách nhau $85$km, đi ngược chiều nhau. Sau $1$ giờ $40$ phút thì gặp nhau. Tính vận tốc riêng của mỗi cano. Biết rằng cano đi xuôi lớn hơn vận tốc riêng của cano đi ngược $9$km/h và vận tốc nước là $3$km/h.
	\loigiai{
	Ta thực hiện đổi đơn vị: $1$ giờ $40$ phút $=1+\dfrac{40}{60}=\dfrac{5}{3}$ giờ.\\
	Gọi $x$ là vận tốc riêng của cano đi xuôi dòng, điều kiện $x>0$. Do đó, khi đi xuôi dòng nó đi với vận tốc $(x+3)$km/h.\\
	Gọi $y$ là vận tốc riêng của cano đi ngược dòng, điều kiện $y>3$. Do đó, khi đi ngược dòng nó đi với vận tốc $(y-3)$km/h.\\
	Với giả thiết:
	\begin{itemize}
	\item Vận tốc riêng của cano đi xuôi lớn hơn vận tốc riêng của cano đi ngược $9$km/h, ta được: $x-y=9. \quad (1)$
	\item Sau $1$ giờ $40$ phút hai cano gặp nhau, ta được:
	$$ \dfrac{5}{3}\left[ (x+3)+(y-3) \right]=85 \Rightarrow x+y=51. \quad (2) $$
	\end{itemize}
	Từ $(1)$ và $(2)$ ta có hệ phương trình $\heva{&x-y=9\\&x+y=51}\Rightarrow \heva{&x=30\\&y=21}.$
	\\ Vậy vận tốc riêng của cano đi xuôi bằng $30$km/h, vận tốc riêng của cano đi ngược bằng $21$km/h.
	}	
\end{vd}
\begin{vd}%[9D3K5]
	Hai vật chuyển động đều trên một đường tròn đường kính $20$cm, xuất phát cùng một lúc, từ cùng một điểm. Nếu chuyển động cùng chiều thì cứ $20$ giây chúng lại gặp nhau. Nếu chuyển động ngược chiều thì cứ $4$ giây chúng lại gặp nhau. Tính vận tốc của mỗi vật.
	\loigiai{
	Gọi $x$ và $y$ là vận tốc của các vật ($x$, $y>0$, đơn vị cm/s).
	\begin{itemize}
	\item Nếu chuyển động cùng chiều thì cứ $20$ giây chúng lại gặp nhau.\\
	Do đó, ta có $\dfrac{20\pi}{x-y}=20.$
	\item Nếu chuyển động ngược chiều thì cứ $4$ giây chúng lại gặp nhau.\\
	Do đó, ta có $\dfrac{20\pi}{x+y}=4.$
	\end{itemize}	
	Ta có hệ phương trình $\heva{&\dfrac{20\pi}{x-y}=20\\&\dfrac{20\pi }{x+y}=4}\Rightarrow \heva{&20\pi =20x-20y\\&20\pi =4x+4y} \Rightarrow \heva{&x=3\pi \\&y=2\pi}.$\\
	Vậy vận tốc vật thứ nhất là $3\pi$cm/s và vận tốc vật thứ hai là $2\pi$cm/s.
	}
\end{vd}
\begin{dang}{Toán có nội dung lí, hóa}
	%	Nắm vững các công thức lí, hóa liên quan.
\end{dang}
\begin{vd}
	Tìm các hệ số $x$, $y$ để cân bằng phương trình phản ứng hóa học:
	\[x \mathrm{Fe}_3 \mathrm{O}_4+\mathrm{O}_2 \to y \mathrm{Fe}_2 \mathrm{O}_3. \]
	\loigiai{
	Theo định luật bảo toàn nguyên tố đối với $\mathrm{Fe}$ và $\mathrm{O}$, ta có: $\heva{&3 x=2 y &(1)\\&4 x+2=3 y &(2)}$\\
	Từ phương trình (1), ta suy ra $y=\dfrac{3}{2}x$.\hfill(3)\\
	Thế (3) và (2), ta được
	\allowdisplaybreaks
	\begin{eqnarray*}
	4x+2&=&3\cdot \dfrac{3}{2}x\\
	4x+2&=&\dfrac{9}{2}x\\
	2&=&\dfrac{1}{2}x\\
	x&=&4.
	\end{eqnarray*}
	Thay giá trị $x=4$ vào phương trình (3), ta có
	\[y=\dfrac{3}{2}\cdot 4=6. \]
	Do đó, hệ phương trình đã cho có nghiệm duy nhất $(x; y)=(4; 6)$.\\
	Vậy ta có phương trình sau cân bằng: $4 \mathrm{Fe}_3 \mathrm{O}_4+\mathrm{O}_2 \to 6 \mathrm{Fe}_2 \mathrm{O}_3$.
	}
\end{vd}
\begin{vd}
	Cân bằng phương trình hoá học sau bằng phương pháp đại số.
	$$\mathrm{P}+\mathrm{O}_2\rightarrow\mathrm{P}_2\mathrm{O}_5.$$
	\loigiai{
	Gọi $x$, $y$ lần lượt là hệ số của $\mathrm{P}$ và $\mathrm{O}_2$ thoả mãn cân bằng phương trình hoá học
	$$x\mathrm{P}+y\mathrm{O}_2\rightarrow\mathrm{P}_2\mathrm{O}_5$$
	Cân bằng số nguyên tử $\mathrm{P}$, số nguyên tử $\mathrm{O}$ ở hai vế, ta được hệ
	$$\heva{
	&x=2\\
	&2 y=5.\\}$$
	Giải hệ phương trình này, ta được $x=2$, $y=\dfrac{5}{2}$.\\
	Đưa các hệ số tim được vào phương trình hoá học, ta có
	$$2\mathrm{P}+\dfrac{5}{2}\mathrm{O}_2\rightarrow\mathrm{P}_2\mathrm{O}_5$$
	Do các hệ số của phương trình hoá học phải là số nguyên nên nhân hai vế của phương trình hoá học trên với $2$, ta được
	$$4\mathrm{P}+5\mathrm{O}_2\rightarrow2\mathrm{P}_2\mathrm{O}_5.$$
	}
\end{vd}
\begin{vd}
	Cân bằng phương trình hoá học sau bằng phương pháp đại số.
	$$\mathrm{NO}+\mathrm{O}_2 \rightarrow \mathrm{NO}_2.$$
	\loigiai{
	Gọi $x$, $y$ lần lượt là hệ số của $\mathrm{NO}$ và $\mathrm{O}_2$ thỏa mãn cân bằng phương trình hóa học
	$$x\mathrm{NO}+y\mathrm{O}_2\rightarrow \mathrm{NO}_2.$$
	Cân bằng số nguyên tử $\mathrm{NO}$, số nguyên tử $\mathrm{O}_2$ ở hai vế, ta được hệ
	$$\heva{&x=1\\ &x+2y=2.}$$
	Giải hệ phương trình trên ta được $\heva{&x=1\\ &y=\dfrac{1}{2}.}$\\
	Đưa các hệ số tìm được vào phương trình hóa học, ta có
	$$\mathrm{NO}+\dfrac{1}{2}\mathrm{O}_2\rightarrow \mathrm{NO}_2.$$
	Do các hệ số của phương trình hoá học phải là số nguyên nên nhân hai vế của phương trình hoá học trên với $2$, ta được
	$$2\mathrm{NO}+\mathrm{O}_2\rightarrow 2\mathrm{NO}_2.$$
	}
\end{vd}
\begin{vd}%[9D3K5]
	Có hai loại quặng chứa $75$ \% sắt và $50$ \% sắt. Tính khối lượng của mỗi loại quặng đem trộn để được $25$ tấn quặng chứa $66$ \% sắt.
	\loigiai{
	Gọi khối lượng quặng chứa $75$\% sắt và $50$ \% sắt lần lượt là $x$, $y$ (tấn; $x,~y>0$). \\
	Theo đề bài ta có hệ phương trình $\heva{&x+y=25 \\ &\dfrac{75x}{100}+\dfrac{50y}{100} =\dfrac{66}{100}\cdot 25} \Rightarrow \heva{&x=16 \\&y=9} \text{ (thỏa mãn điều kiện)}$. \\
	Vậy đem $16$ tấn loại quặng chứa $75$ \% sắt; $9$ tấn quặng chứa $50$ \% sắt.
	}
\end{vd}
\begin{vd}%[9D3K5]
	Người ta cho thêm $1$ kg nước vào dung dịch A thì được dung dịch B có nồng độ axit là $20$ \%. Sau đó lại cho thêm $1$ kg axit vào dung dịch B thì được dung dịch C có nồng độ axit là $33\dfrac{1}{3}$\%. Tính nồng độ axit trong dung dịch A.
	\loigiai{
	Gọi khối lượng axit trong dung dịch A là $x$ (kg; $x>0$) và khối lượng nước trong dung dịch A là $y$ (kg; $y>0$). \\
	Cho thêm $1$ kg nước vào dung dịch A thì được dung dịch B có nồng độ axit là $20$ \% ta có phương trình
	\[\dfrac{x}{x+y+1}= 20\% \tag{1}\]
	Cho thêm $1$ kg axit vào dung dịch B thì được dung dịch C có nồng độ axit là $33\dfrac{1}{3}$\% ta có phương trình \[\dfrac{x+1}{x+y+2}=33\dfrac{1}{3}\% \tag{2}\]
	Từ (1), (2) ta có hệ phương trình $\heva{&\dfrac{x}{x+y+1}= 20 \\ &\dfrac{x+1}{x+y+2}=33\dfrac{1}{3}} \Rightarrow \heva{&x=1 \\&y=3} \text{( thỏa mãn điều kiện).}$ \\
	Vậy nồng độ axit trong dung dịch A là $\dfrac{1}{3+1} = \dfrac{1}{4} = 25$\%.
	}
\end{vd}
%%%%%%%%%%%%%%%%%%%%
\subsection{Bài tập vận dụng}
\begin{bt}
	Giải các hệ phương trình sau bằng phương pháp thế
	\begin{listEX}[3]
	\item $\heva{& x-y=3 \\ & 3x-4y=2;}$
	\item $\heva{& 7x-3y=13 \\ & 4x+y=2;}$
	\item $\heva{& 0{,}5x-1{,}5y=1 \\ & -x+3y=2.}$
	\end{listEX}
	\loigiai
	{
	\begin{listEX}[1]
	\item Từ phương trình thứ nhất ta có $x=y+3$. Thế vào phương trình thứ hai, ta được $3(y+3)-4y=2$ hay $-y+7=0$, suy ra $y=7$.\\
	Từ đó $x=7+3=10$. Vậy hệ phương trình đã cho có nghiệm là $(10;7)$.
	\item Từ phương trình thứ hai ta có $y=2-4x$. Thế vào phương trình thứ nhất, ta được $7x-3(2-4x)=13$ hay $19x-19=0$, suy ra $x=1$.\\
	Từ đó $y=2-4\cdot 1=-2$. Vậy hệ phương trình đã cho có nghiệm là $(1;-2)$.
	\item Từ phương trình thứ hai ta có $x=3y-2$. Thế vào phương trình thứ nhất, ta được $0{,}5\cdot(3y-2)-1{,}5y=1$ hay $0y=0$.\\
	Ta thấy mọi giá trị của $y$ đều thoả mãn phương trình $0y=0$.\\
	Vậy với giá trị tuỳ ý của $y$, giá trị tương ứng của $x$ được tính bởi phương trình $x=3y-2$.\\
	Vậy hệ phương trình đã cho có nghiệm là $(3y-2,y)$ với $y\in\mathbb{R}$ tuỳ ý.
	\end{listEX}
	}
\end{bt}
\begin{bt}
	Giải các hệ phương trình sau bằng phương pháp thế
	\begin{listEX}[3]
	\item $ \heva{&x-2 y =0 \\& 3 x+2 y =8.} $
	\item $ \heva{&-\dfrac{3}{4} x+\dfrac{1}{2} y =-2 \\& \dfrac{3}{2} x-y =4.} $
	\item $ \heva{&4 x-2 y =1 \\& -2 x+y =0.} $
	\end{listEX}
	\loigiai{\begin{listEX}
	\item $ \heva{&x-2 y =0 \\ &3 x+2 y=8} $\\
	$ \heva{&x =2y \\ &3.2y+2 y =8} $\\
	$ \heva{&x =2y \\ &8 y =8} $\\
	$ \heva{&x =2 \\ & y =1.} $\\
	Vậy hệ phương trình có nghiệm duy nhất là $ \heva{&x =2 \\ & y =1.} $
	\item $ \heva{&-\dfrac{3}{4} x+\dfrac{1}{2} y =-2 \\ &\dfrac{3}{2} x-y =4} $\\
	$ \heva{&-\dfrac{3}{4} x+\dfrac{1}{2}.\left(\dfrac{3}{2} x-4\right) =-2 \\& y=\dfrac{3}{2} x-4} $\\
	$ \heva{&0x=0 \\& y=\dfrac{3}{2} x-4.} $\\
	Phương trình $ 0x=0 $ nghiệm đúng với mọi $ x \in \mathbb{R} $.\\ Vậy hệ phương trình có vô số nghiệm.\\
	Các nghiệm của hệ được viết như sau $ \heva{&x\in \mathbb{R}\\&y=\dfrac{3}{2} x-4.} $
	\item $ \heva{&4 x-2 y =1 \\& -2 x+y =0} $\\
	$ \heva{&4 x-2 .2x =1 \\&y=2x} $\\
	$ \heva{&0x =1 \\& -2 x+y =0.} $\\
	Phương trình $ 0x=1 $ vô nghiệm.\\
	Vậy hệ phương trình vô nghiệm.
	\end{listEX}
	}
\end{bt}
\begin{bt}
	Giải các hệ phương trình sau bằng phương pháp cộng đại số
	\begin{listEX}[3]
	\item $\heva{& 3x+2y=6 \\ & 2x-2y=14;}$
	\item $\heva{& 0{,}3x+0{,}5y=3 \\ & 1{,}5x-2y=1{,}5;}$
	\item $\heva{& -2x+6y=8 \\ & 3x-9y=-12.}$
	\end{listEX}
	\loigiai
	{
	\begin{listEX}[1]
	\item Cộng từng vế của phương trình ta được $5x=20$, suy ra $x=4$.\\
	Thế $x=4$ vào phương trình thứ nhất ta được $3\cdot4+2y=6$, hay $2y=-6$, suy ra $y=-3$.\\
	Vậy hệ phương trình đã cho có nghiệm là $(4;-3)$.
	\item Nhân hai vế của phương trình thứ nhất cho $4$, ta có hệ $\heva{& 1{,}2x+2y=12 \\ & 1{,}5x-2y=1{,}5.}$\\
	Cộng từng vế của phương trình ta được $2{,}7x=13{,}5$, suy ra $x=5$.\\
	Thế $x=5$ vào phương trình thứ hai, ta được $1{,}5\cdot5-2y=1{,}5$, hay $-2y=-6$, suy ra $y=3$.\\
	Vậy hệ phương trình đã cho có nghiệm là $(5;3)$.
	\item Nhân hai vế của phương trình thứ nhất cho $\dfrac{3}{2}$, ta có hệ $\heva{& -3x+9y=12 \\ & 3x-9y=-12.}$\\
	Cộng từng vế của phương trình ta được $0x+0y=0$. Hệ thức này luôn thoả mãn với các giá trị tuỳ ý của $x$ và $y$.\\
	Với giá trị tuỳ ý của $y$, giá trị của $x$ được tính nhờ hệ thức $3x-9y=-12$, suy ra $x=3y-4$.\\
	Vậy hệ phương trình đã cho có nghiệm là $(3y-4;y)$ với $y\in\mathbb{R}$.
	\end{listEX}
	}
\end{bt}
\begin{bt}
	Giải các hệ phương trình:
	\begin{listEX}[2]
	\item $\heva{
	&2x-3y=-5\\ 
	&x+3y=11.\\}$
	\item $\heva{
	&3x+2y=7\\ 
	&2x+3y=3.\\}$
	\end{listEX}
	\loigiai{
	\begin{listEX}
	\item Cộng từng vế hai phương trình của hệ, ta được $3x=6$. Suy ra $x=2$.\\
	Thay $x=2$ vào phương trình thứ hai của hệ, ta được $2+3y=11$. Do đó $y=3$.\\
	Vậy hệ phương trình có nghiệm duy nhất là $(2;3)$.
	\item Nhân hai vế của phương trình thứ nhất với $2$, nhân hai vế của phương trình thứ hai với $-3$, ta được $\heva{
	&6x+4y=14\\
	&-6x-9y=-9.\\}$\\
	Cộng từng vế hai phương trình của hệ, ta được $-5y=5$. Suy ra $y=-1$.\\
	Thay $y=-1$ vào phương trình $3x+2y=7$, ta được $3x+2\cdot(-1)=7$. Do đó $x=3$.\\
	Vậy hệ phương trình có nghiệm duy nhất là $(3;-1)$.
	\end{listEX}
	}
\end{bt}
\begin{bt}
	Giải các hệ phương trình:
	\begin{listEX}[2]
	\item $\heva{
	&2x-5y=-14\\ 
	&2x+3y=2.\\}$
	\item $\heva{
	&4x+5y=15\\ 
	&6x-4y=11.\\}$
	\end{listEX}
	\loigiai{
	\begin{listEX}
	\item Ta có \allowdisplaybreaks
	\begin{eqnarray*}
	&&\heva{
	&2x-5y=-14\\ 
	&2x+3y=2}\\
	&&\heva{
	&2x-5y=-14\\ 
	&-8y=-16}\\
	&&\heva{
	& 2x=-14+5y\\
	&y=2}\\
	&&\heva{
	& x=\dfrac{-14+5\cdot 2}{2}\\
	& y=2}\\
	&&\heva{& x=-2\\ &y=2.}
	\end{eqnarray*}
	Vậy hệ phương trình có nghiệm duy nhất là $(-2;2)$.
	\item Ta có 
	\allowdisplaybreaks
	\begin{eqnarray*}
	&&\heva{
	&4x+5y=15\\ 
	&6x-4y=11}\\
	&&\heva{
	&12x+15y=45\\
	&12x-8y=22}\\
	&&\heva{
	&4x+5y=15\\
	&23y=23}\\
	&&\heva{
	&4x=15-5y\\
	&y=1}\\
	&&\heva{
	&x=\dfrac{15-5\cdot 1}{4}\\
	&y=1}\\
	&&\heva{&x=\dfrac{5}{2}\\ &y=1.}
	\end{eqnarray*}
	Vậy hệ phương trình có nghiệm duy nhất là $\left(\dfrac{5}{2};1\right)$.
	\end{listEX}}
\end{bt}
%%=====Bài 1
\begin{bt}
	Giải các hệ phương trình:
	\begin{listEX}[4]
	\item $\heva{&3x+y=3\\&2x-y=7;}$
	\item $\heva{&x-y=3\\&3x-4y=2;}$
	\item $\heva{&4x+5y=-2\\&2x-y=8;}$
	\item $\heva{&3x+y=3\\&-3y=5.}$
	\end{listEX}
	\loigiai{
	\begin{listEX}[2]
	\item $\heva{&3x+y=3\\&2x-y=7}\\\heva{&y=3-3x\\&2x-(3-3x)=7}\\\heva{&y=3-3x\\&5x=10}\\\heva{&y=-3\\&x=2.}$\\
	Vậy hệ phương trình có nghiệm duy nhất là $(2;-3)$.
	\item $\heva{&x-y=3\\&3x-4y=2}\\\heva{&x=3+y\\&3(3+y)-4y=2}\\\heva{&x=3+y\\&-y=-7}\\\heva{&x=10\\&y=7.}$\\
	Vậy hệ phương trình có nghiệm duy nhất là $(10;7)$.
	\item $\heva{&4x+5y=-2\\&2x-y=8}\\\heva{&4x+5(2x-8)=-2\\&y=2x-8}\\\heva{&14x=38\\&y=2x-8}\\\heva{&x=\dfrac{19}{7}\\&y=\dfrac{-18}{7}.}$\\
	Vậy hệ phương trình có nghiệm duy nhất là $\left(\dfrac{19}{7};\dfrac{-18}{7}\right)$.
	\item $\heva{&3x+y=3\\&-3y=5}\\\heva{&3x+y=3\\&y=\dfrac{-5}{3}}\\\heva{&x=\dfrac{14}{9}\\&y=\dfrac{-5}{3}.}$\\
	Vậy hệ phương trình có nghiệm duy nhất là $\left(\dfrac{14}{9};\dfrac{-5}{3}\right)$.
	\end{listEX}
	}
\end{bt}
\begin{bt}
	Giải các hệ phương trình sau bằng phương pháp cộng đại số
	\begin{listEX}[4]
	\item $ \heva{&2 x+y =4\\&x-y =2.} $
	\item $ \heva{&4 x+5 y=11\\&2 x-3 y=0.} $
	\item $ \heva{&12 x+18 y=-24\\&-2 x-3 y =4.} $
	\item $ \heva{&x-3 y =5\\&-2 x+6 y =10.} $
	\end{listEX}
	\loigiai{\begin{listEX}
	\item Cộng từng vế hai phương trình của hệ, ta được $ 3x=6 $. Suy ra $ x=2 $.\\
	Thay $ x=2 $ vào phương trình thứ hai của hệ, ta được $ 2-y=2 $. Do đó $ y=0 $.\\
	Vậy hệ phương trình có nghiệm duy nhất là $ (2;0) $.
	\item Nhân hai vế của phương trình thứ hai với $ -2 $, ta được
	$ \heva{&4 x+5 y=11\\&-4 x+6y=0.} $\\
	Cộng từng vế hai phương trình của hệ, ta được $ 11y=11 $. Suy ra $ y=1 $.\\
	Thay $ y=1 $ vào phương trình thứ nhất của hệ, ta được $ 4x+5=11 $. Do đó $ x=\dfrac{3}{2} $.\\
	Vậy hệ phương trình có nghiệm duy nhất là $ \left(\dfrac{3}{2};1\right) $.
	\item Nhân hai vế của phương trình thứ hai với $ 6 $, ta được
	$ \heva{&12 x+18 y=-24\\&-12 x-18 y =24.} $\\
	Cộng từng vế hai phương trình của hệ, ta được $ 0=0 $ nghiệm đúng với mọi $ x\in \mathbb{R} $.\\
	Vậy hệ phương trình có vô số nghiệm.
	\item Nhân hai vế của phương trình thứ nhất với $ 2 $, ta được
	$ \heva{&2x-6 y =10\\&-2 x+6 y =10} $\\
	Cộng từng vế hai phương trình của hệ, ta được $ 0x=10 $ là phương trình vô nghiệm.\\
	Vậy hệ phương trình vô nghiệm.
	\end{listEX}}
\end{bt}
\begin{bt}
	Giải các hệ phương trình:
	\begin{listEX}[3]
	\item $\heva{
	&x+2y=-2\\ 
	&5x-4y=11.\\}$
	\item $\heva{
	&2x-y=-5\\ 
	&-2x+y=11.\\}$
	\item $\heva{
	&3x+y=2\\ 
	&6x+2y=4.\\}$
	\end{listEX}
	\loigiai{
	\begin{listEX}
	\item Ta có \begin{eqnarray*}
	&&\heva{
	&x+2y=-2\\ 
	&5x-4y=11.\\}\\
	&&\heva{&x=-2y-2\\
	&5(-2y-2)-4y=11}\\
	&&\heva{&x=-2y-2\\
	&-14y=21}\\
	&&\heva{&x=1\\&y=-\dfrac{3}{2}.}
	\end{eqnarray*}
	Vậy hệ phương trình có nghiệm duy nhất $\heva{&x=1\\&y=-\dfrac{3}{2}.}$
	\item Ta có 
	\begin{eqnarray*}
	&&\heva{
	&2x-y=-5\\ 
	&-2x+y=11}\\
	&&\heva{&2x-(11+2x)=-5\\ &y=11+2x}\\
	&&\heva{&2x-11-2x=5\\ &y=11+2x}\\
	&&\heva{&0x=16\\ & y=11+2x.}
	\end{eqnarray*}
	Phương trình $0x=16$ vô nghiệm.\\
	Vậy hệ phương trình vô nghiệm.
	\item Ta có 
	\begin{eqnarray*}
	&&\heva{
	&3x+y=2\\ 
	&6x+2y=4}\\
	&&\heva{&y=2-3x\\ &6x+2(2-3x)=4}\\
	&&\heva{&y=2-3x\\ &6x+4-6x=4}\\
	&&\heva{&y=2-3x\\ &0x=0.}\\
	\end{eqnarray*}
	Phương trình $0x=0$ nghiệm đúng với mọi $x\in\mathbb{R}$.\\
	Vậy hệ phương trình có vô số nghiệm. Các nghiệm của hệ được viết như sau: $\heva{
	&x\in\mathbb{R}\\ 
	&y=2-3x.\\}$
	\end{listEX}
	}
\end{bt}
%%=====Bài 2
\begin{bt}
	Giải các hệ phương trình:
	\begin{listEX}[2]
	\item $\heva{&4x+y=2\\&\dfrac{4}{3}x+\dfrac{1}{3}y=1;}$
	\item $\heva{&x-y\sqrt{2}=0\\&2x+y\sqrt{2}=3;}$
	\item $\heva{&5x\sqrt{3}+y=2\sqrt{2}\\&x\sqrt{6}-y\sqrt{2}=2;}$
	\item $\heva{&2(x+y)+3(x-y)=4\\&(x+y)+2(x-y)=5.}$
	\end{listEX}
	\loigiai{
	\begin{listEX}
	\item $\heva{
	&x+2y=-2\\
	&5x-4y=11}\\
	\heva{&x=-2-2y\\
	&5 \cdot \left(-2-2y\right)-4y=11}\\
	\heva{&x=-2-2y\\
	&-10-10y-4y=11}\\
	\heva{&x=-2-2y\\
	&-14y=21}\\
	\heva{&x=-2-2 \cdot \left( -\dfrac{3}{2}\right) \\
	&y=-\dfrac{3}{2}}\\
	\heva{
	&x=1\\
	&y=-\dfrac{3}{2}.}\\
	$\\
	Vậy hệ phương trình có nghiệm duy nhất là $\left(1;-\dfrac{3}{2}\right)$.
	\item $\heva{
	&x-y\sqrt{2}=0\\
	&2x+y\sqrt{2}=3}\\
	\heva{
	&x=y\sqrt{2}\\
	&2\cdot \left(y\sqrt{2}\right)+y\sqrt{2}=3}\\
	\heva{
	&x=y\sqrt{2}\\
	&3\cdot y\sqrt{2}=3}\\
	\heva{
	&x=\left(\dfrac{\sqrt{2}}{2}\right) \cdot \sqrt{2}\\
	&y=\dfrac{\sqrt{2}}{2}}\\
	\heva{
	&x=1\\
	&y=\dfrac{\sqrt{2}}{2}.}\\
	$\\
	Vậy hệ phương trình có nghiệm duy nhất là $\left(1;\dfrac{\sqrt{2}}{2}\right)$.
	\item $\heva{
	&5x\sqrt{3}+y=2\sqrt{2}\\
	&x\sqrt{6}-y\sqrt{2}=2
	}\\
	\heva{
	&y=2\sqrt{2}-5x\sqrt{3}\\
	&x\sqrt{6}- \left(2\sqrt{2}-5x\sqrt{3} \right)\cdot \sqrt{2}=2
	}\\
	\heva{
	&y=2\sqrt{2}-5x\sqrt{3}\\
	&x\sqrt{6}- 4+5x\sqrt{6} =2
	}\\
	\heva{
	&y=2\sqrt{2}-5x\sqrt{3}\\
	&6x\sqrt{6} =6
	}\\
	\heva{
	&y=2\sqrt{2}-5 \left( \dfrac{\sqrt{6}}{6} \right)\sqrt{3}\\
	&x = \dfrac{\sqrt{6}}{6}
	}\\
	\heva{
	&y=\dfrac{-\sqrt{2}}{2}\\
	&x = \dfrac{\sqrt{6}}{6}
	.}\\
	$\\
	Vậy hệ phương trình có nghiệm duy nhất là $\left(\dfrac{\sqrt{6}}{6};-\dfrac{\sqrt{2}}{2}\right)$.
	\item $\heva{
	&2(x+y)+3(x-y)=4\\
	&(x+y)+2(x-y)=5
	}$\\
	Nhân hai vế của phương trình thứ hai với $ (-1) $, ta được\\
	$$\heva{
	&2(x+y)+3(x-y)=4\\
	&-(x+y)-2(x-y)=(-5).
	}$$\\
	Cộng từng vế của phương trình của hệ, ta được 
	$ (x+y)+(x-y)=(-1) $ hay $ 2x=(-1) $. Suy ra $ x=-\dfrac{1}{2} $;\\
	Thay kết quả vừa tìm được vào phương trình thứ hai ta được: $ \left( -\dfrac{1}{2}+y\right) +2\cdot \left( -\dfrac{1}{2}-y\right) =5 $. Suy ra $ y=-\dfrac{13}{2} $.\\
	Vậy hệ phương trình có nghiệm duy nhất là $\left(-\dfrac{1}{2};-\dfrac{13}{2} \right)$.	
	\end{listEX}
	}
\end{bt}
\begin{bt}
	Giải các hệ phương trình bằng phương pháp thế	
	\begin{listEX}[3]
	\item $\heva{&\sqrt{2}x - \sqrt{3}y = 1\\ &x + \sqrt{3} y = \sqrt{2}}$
	\item $\heva{&x - 2\sqrt{2}y = \sqrt{5}\\ &\sqrt{2}x + y = 1 - \sqrt{10}}$
	\item $\heva{&\left(\sqrt{2} - 1\right)x - y = \sqrt{2}\\ &x + \left(\sqrt{2} + 1\right)y = 1}$
	\end{listEX}
	\loigiai{
	\begin{enumerate}
	\item Giải hệ phương trình $\heva{&\sqrt{2}x - \sqrt{3}y = 1\quad(1)\\ &x + \sqrt{3} y = \sqrt{2}\quad(2)}$.\\
	Từ phương trình $(2)$ suy ra $x = \sqrt{2} - \sqrt{3}y$. Thay vào phương trình $(1)$ ta có 
	$$\sqrt{2}\left( \sqrt{2} - \sqrt{3}y\right) - \sqrt{3}y = 1\Rightarrow y\left(\sqrt{6} + \sqrt{3}\right) = 1\Rightarrow y = \dfrac{\sqrt{6} - \sqrt{3}}{3}$$
	Khi $y = \dfrac{\sqrt{6}- \sqrt{3}}{3}$ thay vào $(2)$ ta có $$x + \sqrt{3}\cdot \left(\dfrac{\sqrt{6}- \sqrt{3}}{3}\right) = \sqrt{2}\Rightarrow x + \sqrt{2} - 1 = \sqrt{2}\Rightarrow x = 1.$$
	Vậy hệ phương trình có nghiệm duy nhất là $\left(1; \dfrac{\sqrt{6} - \sqrt{2}}{3}\right)$.
	\item Giải hệ phương trình $\heva{&x - 2\sqrt{2}y = \sqrt{5}\quad (1)\\ &\sqrt{2}x + y = 1 - \sqrt{10}\quad(2)}$\\
	Từ phương trình $(1)$ suy ra $x = \sqrt{5} - 2\sqrt{2}y$. Thay vào phương trình $(2)$ ta có
	$$\sqrt{2}\left(\sqrt{5} + 2\sqrt{2}y\right) + y = 1 -\sqrt{10}\Rightarrow 5y = 1- 2\sqrt{10}\Rightarrow y = \dfrac{1 -2\sqrt{10}}{5}$$
	Khi $y = \dfrac{1 -2\sqrt{10}}{5}$ thay vào $(1)$ ta có 
	$$x = \sqrt{5} + 2\sqrt{2}\cdot \left(\dfrac{1 -2\sqrt{10}}{5}\right)\Rightarrow x = \dfrac{2\sqrt{2} - 3\sqrt{5}}{5}$$
	Vậy hệ phương trình có nghiệm duy nhất là $\left(\dfrac{2\sqrt{2} - 3\sqrt{5}}{5}; \dfrac{1 -2\sqrt{10}}{5}\right)$.
	\item $\heva{&\left(\sqrt{2} - 1\right)x - y = \sqrt{2}\quad (1)\\ &x + \left(\sqrt{2} + 1\right)y = 1\quad (2)}$\\
	Từ phương trình $(1)$ ta suy ra $y = \left(\sqrt{2} - 1\right)x - \sqrt{2}$. Thay vào phương trình $(2)$ ta có 
	$$x + \left(\sqrt{2} + 1\right)\cdot\left[\left(\sqrt{2} - 1\right)x - \sqrt{2}\right] = 1\Rightarrow 2x - 2 - \sqrt{2} = 1\Rightarrow x = \dfrac{3 + \sqrt{2}}{2}.$$
	Khi $x = \dfrac{3 + \sqrt{2}}{2}$ thay vào $(2)$ ta có
	$$\dfrac{3 + \sqrt{2}}{2} + \left(\sqrt{2} + 1\right)y = 1\Rightarrow \left(\sqrt{2} + 1\right)y = - \dfrac{1 + \sqrt{2}}{2}\Rightarrow y = - \dfrac{1}{2}$$
	Vậy hệ phương trình có nghiệm duy nhất $\left(\dfrac{3 + \sqrt{2}}{2}; -\dfrac{1}{2}\right)$.
	\end{enumerate}
	}
\end{bt}
\begin{bt}
	Giải các hệ phương trình bằng phương pháp thế
	\begin{listEX}[2]
	\item $\heva{&x + \sqrt{5} y = 0 \quad (1)\\ &\sqrt{5}x + 3y = 1 - \sqrt{5}\quad (2)}$
	\item $\heva{&\left(2 - \sqrt{3}\right)x - 3y = 2 + 5\sqrt{3}\quad (1)\\ &4x + y = 4 - 2\sqrt{3}\quad (2)}$
	\end{listEX}
	\loigiai{
	\begin{enumerate}
	\item Giải hệ phương trình $\heva{&x + \sqrt{5} y = 0 \quad (1)\\ &\sqrt{5}x + 3y = 1 - \sqrt{5}\quad (2)}$\\
	Xét phương trình $(1)$ suy ra $x = - \sqrt{5}y$. Thay vào phương trình $(2)$ ta có
	$$\sqrt{5}\cdot\left(-\sqrt{5}y\right) + 3y = 1 - \sqrt{5}\Rightarrow -2y = 1 -\sqrt{5}\Rightarrow y = \dfrac{\sqrt{5} - 1}{2}$$
	Khi $y = \dfrac{\sqrt{5} - 1}{2}$ suy ra $x = \left(- \sqrt{5}\right)\cdot\dfrac{\sqrt{5} - 1}{2}\Rightarrow x = \dfrac{\sqrt{5}- 5}{2}$.\\
	Vậy hệ phương trình có nghiệm duy nhất là $\left(\dfrac{\sqrt{5}- 5}{2}; \dfrac{\sqrt{5}- 1}{2}\right)$
	\item Giải hệ phương trình $\heva{&\left(2 - \sqrt{3}\right)x - 3y = 2 + 5\sqrt{3}\quad (1)\\ &4x + y = 4 -2\sqrt{3}\quad (2)}$\\
	Từ phương trình $(2)$ ta suy ra $y = 4 -2\sqrt{3} - 4x$. Thay vào phương trình $(1)$ ta có 
	$$\left(2 - \sqrt{3}\right)x - 3\left(4 -2\sqrt{3} - 4x\right) = 2 + 5\sqrt{3}\Rightarrow \left(14 - \sqrt{3}\right)x = 14 - \sqrt{3}\Rightarrow x = 1$$
	Khi $x = 1$ suy ra $y = - 2\sqrt{3}$.\\
	Vậy hệ phương trình có nghiệm duy nhất $\left(1; - 2\sqrt{3}\right)$.
	\end{enumerate}
	}
\end{bt}
\begin{bt}
	Dùng MTCT thích hợp để tìm nghiệm của các hệ phương trình sau
	\begin{listEX}[2]
	\item $\heva{& 12 x-5 y+24=0 \\ & -5 x-3 y-10=0;}$
	\item $\heva{& \frac{1}{3} x-y=\frac{2}{3} \\ & x-3 y=2;}$
	\item $\heva{& 3 x-2 y=1 \\ & -x+\frac{2}{3} y=0;}$
	\item $\heva{& \frac{4}{9} x-\frac{3}{5} y=11 \\ & \frac{2}{9} x+\frac{1}{5} y=-2.}$
	\end{listEX}
	\loigiai
	{
	\begin{listEX}[1]
	\item $x=-2$; $y=0$.
	\item Hệ phương trình có vô số nghiệm.
	\item Hệ phương trình vô nghiệm.
	\item $x=\dfrac{9}{2}$; $y=-15$.
	\end{listEX}
	}
\end{bt}
\begin{bt}
	Tìm nghiệm của các hệ phương trình sau bằng máy tính cầm tay:
	\begin{listEX}[2]
	\item $\heva{
	&2x-y=4\\ 
	&3x+5y=-19.\\}$
	\item $\heva{
	&-3x+5y=12\\ 
	&2x+y=5.\\}$
	\end{listEX}
	\loigiai{
	Sử dụng máy tính cầm tay, ta giải được nghiệm của các hệ phương trình sau là
	\begin{listEX}
	\item Hệ phương trình có nghiệm duy nhất là $\left(\dfrac{1}{13};-\dfrac{50}{13}\right)$.
	\item Hệ phương trình có nghiệm duy nhất là $\left(1;3\right)$.
	\end{listEX}
	}
\end{bt}
\begin{bt}
	Cho hệ phương trình $\heva{& 2x-y=-3 \\ & -2m^2x+9y=3(m+3)}$, trong đó $m$ là số đã cho. Giải hệ phương trình trong mỗi trường hợp sau
	\begin{listEX}[3]
	\item $m=-2$;
	\item $m=-3$;
	\item $m=3$.
	\end{listEX}
	\loigiai
	{
	\begin{listEX}[1]
	\item Thay $m=-2$, ta có hệ $\heva{&2x-y=-3 \\ & -8x+9y=3}.$\\
	Từ phương trình thứ nhất của hệ ta có $y=2x+3$. Thế vào phương trình thứ hai của hệ, ta được $-8x+9(2x+3)=3$ hay $10x+27=3$, suy ra $x=-\dfrac{12}{5}$.\\
	Từ đó $y=2\cdot\left(-\dfrac{12}{5}\right)+3=-\dfrac{9}{5}$. Vậy hệ phương trình có nghiệm là $\left(-\dfrac{12}{5};-\dfrac{9}{5}\right) $.
	\item Thay $m=-3$, ta có hệ $\heva{& 2x-y=-3 \\ & -18x+9y=0.}$\\
	Từ phương trình thứ nhất của hệ ta có $y=2x+3$. Thế vào phương trình thứ hai của hệ, ta được $-18x+9(2x+3)=0$ hay $0x+27=0$\tagEX{1}\noindent
	Do không có giá trị nào của $x$ thoả mãn hệ thức $(1)$ nên hệ phương trình đã cho vô nghiệm.
	\item Thay $m=3$, ta có hệ $\heva{&2x-y=-3 \\ & -18x+9y=18}.$\\
	Từ phương trình thứ nhất của hệ ta có $y=2x+3$. Thế vào phương trình thứ hai của hệ, ta được $-18x+9(2x+3)=18$ hay $0x=0$.\tagEX{1}\noindent
	Ta thấy mọi giá trị của $x$ đều thoả mãn $(1)$.\\
	Với giá trị tuỳ ý của $x$ thì giá trị tương ứng của $y$ được tính bởi phương trình $y=2x+3$.\\
	Vậy hệ phương trình đã cho có nghiệm là $(x,2x+3)$ với $x\in\mathbb{R}$ tuỳ ý.
	\end{listEX}
	}
\end{bt}
%%=====Bài 3
\begin{bt}
	Xác định $a$, $b$ để đồ thị hàm số $y=ax+b$ đi qua hai điểm $A$ và $B$ trong mỗi trường hợp sau:
	\begin{listEX}[2]
	\item $A(2;1)$ và $B(4;-2)$;
	\item $A(1;2)$ và $B(3;8)$.	
	\end{listEX}
	\loigiai{
	\begin{listEX}[2]
	\item Đồ thị đi qua $A(2;1)$ nên ta có $1=2a+b$.\\
	Đồ thị đi qua $B(4;-2)$ nên ta có $-2=4a+b$.\\
	Ta có hệ phương trình $\heva{&2a+b=1\\&4a+b=-2.}$\\
	Giải hệ phương trình ta được $a=-\dfrac{3}{2}$ và $b=4$.\\
	Vậy đồ thị hàm số đó là $y=-\dfrac{3}{2}x+4$.
	\item Đồ thị đi qua $A(1;2)$ nên ta có $2=a+b$.\\
	Đồ thị đi qua $B(3;8)$ nên ta có $8=3a+b$.\\
	Ta có hệ phương trình $\heva{&a+b=2\\&3a+b=8.}$\\
	Giải hệ phương trình ta được $a=3$ và $b=-1$.\\
	Vậy đồ thị hàm số đó là $y=3x-1$.
	\end{listEX}
	}
\end{bt}
\begin{bt}
	Xác định $a, b$ để đồ thị của hàm số $y=a x+b$ đi qua hai điểm $A, B$ trong mỗi trường hợp sau
	\begin{listEX}[2]
	\item $A(1 ;-2)$ và $B(-2 ;-11)$;
	\item $A(2 ; 8)$ và $B(-4 ; 5)$.
	\end{listEX}
	\loigiai{
	\begin{listEX}[2]
	\item Gọi đồ thị của hàm số $y=a x+b$ là $ d $.\\
	Ta có $ \heva{A \in d\\B\in d} $ nên $ \heva{&-2=a+b\\&-11=-2a+b} $\\
	$ \heva{&b=-2-a\\&-11=-2a-2-a} $\\
	$ \heva{&b=-2-a\\&-3a=-9} $\\
	$ \heva{&a=3\\&b-5.} $
	\item Gọi đồ thị của hàm số $y=a x+b$ là $ d $.\\
	Ta có	$ \heva{A \in d\\B\in d} $ nên $ \heva{&8=2a+b\\&5=-4a+b} $\\
	$ \heva{&b=8-2a\\&5=-4a+8-2a} $\\
	$ \heva{&b=8-2a\\&-6a=-3} $\\
	$ \heva{&a=\dfrac{1}{2}\\&b=7.} $
	\end{listEX}}
\end{bt}
%------------------
\begin{bt}%[9D3B2]
	Giải các hệ phương trình sau
	\begin{enumEX}{2}
	\item $\begin{cases}
	{6(x + y) = 8 + 2x - 3y}\\
	{5(y - x) = 5 + 3x + 2y;}
	\end{cases}$
	\item $\begin{cases}
	{(x - 2)(y + 1) = x y}\\
	{(x + 8)\cdot(y - 2) = x y.}
	\end{cases}$
	\end{enumEX}
	\loigiai{
	\begin{enumerate}
	\item Hệ phương trình được biến đổi thành $\begin{cases}
	{4x + 9y = 8}\\
	{8x - 3y = - 5.}
	\end{cases}$\\
	Giải hệ phương trình ta được nghiệm $\left( - \dfrac{1}{4}; 1\right)$.
	\item Hệ phương trình được biến đổi thành $\begin{cases}
	{x - 2y = 2}\\
	{x - 4y = - 8.}
	\end{cases}$\\
	Giải hệ phương trình ta được nghiệm $\left(12; 5\right)$.
	\end{enumerate}
	}
\end{bt}
\begin{bt}%[9D3K2]
	Giải các hệ phương trình sau
	\begin{enumEX}{2}
	\item $\begin{cases}
	{\dfrac{7}{x + 2} + \dfrac{3}{y}= 2}\\
	{\dfrac{4}{x + 2} - \dfrac{1}{y}=\dfrac{5}{2};}
	\end{cases}$
	\item $\begin{cases}
	{\sqrt{3x - 1} - \sqrt{2y + 1}= 1}\\
	{2\sqrt{3x - 1} + 3\sqrt{2y + 1}= 12.}
	\end{cases}$
	\end{enumEX}	
	\loigiai{
	\begin{enumerate}
	\item Đặt $\dfrac{1}{x + 2}= u ;\dfrac{1}{y}= v$. Hệ phương trình có dạng $\begin{cases}
	{7u + 3v = 2}\\
	{4u - v =\dfrac{5}{2}.}
	\end{cases}$\\
	Giải hệ ta được $\begin{cases}
	{u =\dfrac{1}{2}}\\
	{v = - \dfrac{1}{2}}
	\end{cases}$, suy ra $\begin{cases}
	{x + 2 = 2}\\
	{y = - 2}
	\end{cases}\Rightarrow\begin{cases}
	{x = 0}\\
	{y = - 2.}
	\end{cases}$\\
	Vậy nghiệm của hệ phương trình là $(0;-2)$.
	\item Đặt $\sqrt{3x - 1}= u\geq 0 ;\sqrt{2y + 1}= v\geq 0$. Hệ phương trình trở thành
	\[\begin{cases}
	{u - v = 1}\\
	{2u + 3v = 12}
	\end{cases}\Rightarrow\begin{cases}
	{u = 3}\\
	{v = 2.}
	\end{cases}\]
	Do đó $\begin{cases}
	{\sqrt{3x - 1}= 3}\\
	{\sqrt{2y + 1}= 2}
	\end{cases}\Rightarrow\begin{cases}
	{x =\dfrac{10}{3}}\\
	{y =\dfrac{3}{2}.}
	\end{cases}$\\
	Vậy nghiệm của hệ là $\left(\dfrac{10}{3};\dfrac{3}{2}\right)$.
	\end{enumerate}
	}
\end{bt}
\begin{bt}%[9D3K4]
	Giải các hệ phương trình sau
	\begin{enumEX}{2}
	\item $\left\{\begin{aligned}&(x-1)(y-2)=(x+1)(y-3)\\&(x-5)(y+4)=(x-4)(y+1)\end{aligned} \right.$ 
	\item $\left\{\begin{aligned}&\dfrac{1}{2}(x+2)(y+3)=\dfrac{1}{2} x y+50\\&\dfrac{1}{2}(x-2)(y-2)=\dfrac{1}{2} x y-32\end{aligned} \right.$ 
	\end{enumEX}
	\loigiai{
	\begin{enumerate}
	\item Hệ phương trình được biến đổi thành $\left\{\begin{aligned}&x-2 y=-5\\&3 x-y=16.\end{aligned} \right.$ \\
	Giải ra ta được nghiệm của hệ là $\left(7 \dfrac{2}{5}; 6 \dfrac{1}{5} \right)$.
	\item Hệ phương trình được biến đổi thành $\left\{\begin{aligned}&3 x+2 y=94\\&x+y=34.\end{aligned} \right.$ \\
	Giải hệ phương trình ta được nghiệm là $\left\{\begin{aligned}&x=26\\&y=8.\end{aligned} \right.$ 
	\end{enumerate}	
	}
\end{bt}
\begin{bt}%[9D3G4]
	Giải các hệ phương trình sau
	\begin{enumEX}{3}
	\item $\left\{\begin{aligned}&\sqrt{x-2}+\sqrt{y-3}=3\\&2 \sqrt{x-2}-3 \sqrt{y-3}=-4\end{aligned} \right.$ 
	\item $\left\{\begin{aligned}&\dfrac{3 x}{x+1}+\dfrac{2}{y+4}=4\\&\dfrac{2 x}{x+1}-\dfrac{5}{y+4}=9\end{aligned} \right.$ 
	\item $\left\{\begin{aligned}&2 \left(x^2-2 x \right)+\sqrt{y+1}=0\\&3 \left(x^2-2 x \right)-2 \sqrt{y+1}+7=0\end{aligned} \right.$ 
	\end{enumEX}
	\loigiai{
	\begin{enumerate}
	\item Đặt $\sqrt{x-2}=u \geq 0$; $\sqrt{y-3}=v \geq 0$.\\
	Hệ phương trình có dạng $\left\{\begin{aligned}&u+v=3\\&2 u-3 v=-4\end{aligned} \right. \Rightarrow \left\{\begin{aligned}&u=1\\&v=2\end{aligned} \right.$ thỏa mãn.\\
	Suy ra $\left\{\begin{aligned}&\sqrt{x-2}=1\\&\sqrt{y-3}=2\end{aligned} \right. \Rightarrow \left\{\begin{aligned}&x=3\\&y=7\end{aligned} \right.$.\\
	Vậy nghiệm của hệ là $(3;7)$.
	\item Đặt $\dfrac{x}{x+1}=u$; $\dfrac{1}{y+4}=v$. \\
	Hệ phương trình có dạng $\left\{\begin{aligned}&3 u+2 v=4\\&2 u-5 v=9\end{aligned} \right. \Rightarrow \left\{\begin{aligned}&u=2\\&v=-1.\end{aligned} \right.$ \\
	Suy ra $\left\{\begin{aligned}&\dfrac{x}{x+1}=2\\&\dfrac{1}{y+4}=-1\end{aligned} \right. \Rightarrow \left\{\begin{aligned}&x=-2\\&y=-5.\end{aligned} \right.$ \\
	Vậy nghiệm của hệ là $(-2;-5)$.
	\item Đặt $(x-1)^2=u \geq 0$; $\sqrt{y+1}=v \geq 0$.\\
	Hệ phương trình có dạng $\left\{\begin{aligned}&2 u+v=2\\&3 u-2 v=-4\end{aligned} \right. \Rightarrow \left\{\begin{aligned}&u=0\\&v=2.\end{aligned} \right.$ \\
	Suy ra $\left\{\begin{aligned}&(x-1)^2=0\\&\sqrt{y+1}=2\end{aligned} \right. \Rightarrow \left\{\begin{aligned}&x=1\\&y=3.\end{aligned} \right.$ \\
	Vậy nghiệm của hệ là $(1;3)$.
	\end{enumerate}	
	}
\end{bt}
\begin{bt}%[9D3K2]
	Xác định tọa độ giao điểm của hai đường thẳng
	\begin{enumerate}
	\item $(d): y = x - 2$ và $\left(d^{\prime}\right) : y = 2x + 1$;
	\item $(d): x + y + 1 = 0$ và $\left(d^{\prime}\right) : x - 2y + 4 = 0$;
	\item $(d): x - 3y + 5 = 0$ và $\left(d^{\prime}\right) : 2x + y - 18 = 0$.
	\end{enumerate}	
	\loigiai{
	Tọa độ giao điểm $M$ của $(d)$ và $(d')$ là
	\begin{enumEX}{3}
	\item $M(-3;-5)$;
	\item $M(-2;1)$;
	\item $M(7;4)$.
	\end{enumEX}	
	}
\end{bt}
\begin{bt}%[9D3K2]
	Cho ba đường thẳng
	\[\begin{array}{l}
	{\left(d_1\right) : x - 2y = - 3;}\\
	{\left(d_2\right) :\sqrt{2}\cdot x + y =\sqrt{2} + 2;}\\
	{\left(d_m\right) : m x - (1 - 2m) y = 5 - m.}
	\end{array}\]
	Xác định tham số $m$ để ba đường thẳng đồng quy.
	\loigiai{
	Tọa độ giao điểm của $(d_1)$ và $(d_2)$ là nghiệm của hệ phương trình
	\[\begin{cases}
	{x - 2y = - 3}\\
	{\sqrt{2}\cdot x + y =\sqrt{2} + 2}
	\end{cases}\Rightarrow\begin{cases}
	{x = 1}\\
	{y = 2.}
	\end{cases}\]
	Vậy tọa độ giao điểm của $(d_1)$ và $(d_2)$ là $M(1;2)$.\\
	Để ba đường thẳng đồng quy thì $M\in (d_m)$ hay
	\[m - (1 - 2m)\cdot 2 = 5 - m\Rightarrow m =\dfrac{7}{6}.\]
	Vậy với $m =\dfrac{7}{6}$ thì ba đường thẳng đồng quy.
	}
\end{bt}
\begin{bt}%[9D3G2]
	Cho hệ phương trình $\begin{cases}
	{m x + 4y = 10 - m}\\
	{x + m y = 4.}
	\end{cases}$
	\begin{enumerate}
	\item Xác định các giá trị nguyên của $m$ để hệ có nghiệm duy nhất $(x;y)$ sao cho $x>0$, $y>0$.
	\item Tìm giá trị nguyên của $m$ để hệ có nghiệm $(x;y)$ với $x$, $y$ là số nguyên dương.
	\end{enumerate}
	\loigiai{
	\begin{enumerate}
	\item Điều kiện để hệ có nghiệm duy nhất là $m\ne \pm 2$. Khi đó hệ có nghiệm $\left(\dfrac{8 - m}{2 + m};\dfrac{5}{2 + m}\right)$. Mặt khác, theo điều kiện $x>0$, $y>0\Rightarrow\begin{cases}
	{\dfrac{8 - m}{2 + m}> 0}\\
	{\dfrac{5}{2 + m}> 0}
	\end{cases}\Rightarrow - 2 < m < 8.$\\
	Với $m\in \mathbb{Z}\Rightarrow m\in \{ - 1 ; 0 ; 1 ; 2 ;\dots ; 7\}$.
	\item $m=\left\lbrace -1;3\right\rbrace $.
	\end{enumerate}
	}
\end{bt}
\begin{bt}%[9D3K4]
	Cho hệ phương trình $\heva{&mx-y=3\\&2x+my=9.}$
	\begin{enumerate}
	\item Giải hệ phương trình khi $m=1$;
	\item Tìm giá trị nguyên của $m$ để hệ có nghiệm duy nhất $(x;y)$ sao cho biểu thức $A=3x-y$ nhận giá trị nguyên.
	\end{enumerate}
	\loigiai{
	\begin{enumerate}
	\item Khi $m=1$ hệ phương trình có dạng $\heva{&x-y=3\\&2x+y=9.}$ \\
	Giải hệ phương trình ta được nghiệm là $(4;1)$.
	\item $\bullet$ Với mọi $m$ hệ luôn có nghiệm $\heva{&x=\dfrac{3m+9}{m^2+2}\\&y=\dfrac{9m-6}{m^2+2}.}$ \\
	$\bullet$ Xét $A = 3x-y = \dfrac{33}{m^2+2}$. \\
	Để $A \in \mathbb{Z} \Rightarrow m^2 + 2 \in \text{Ư}(33)$ mà $m^2 + 2 \geq 2$; $m \in \mathbb{Z}$. \\
	Suy ra $m \in \{1;-1;3;-3 \}$.
	\end{enumerate}	
	}
\end{bt}
\begin{bt}%[9D3G3]
	Với giá trị nào của $m$ thì hai phương trình sau có nghiệm chung
	$$mx^2 + x + 1 = 0\quad \text{và}\quad x^2 + mx + 1 = 0$$
	\loigiai{
	Để thỏa mãn bài toán khi hệ phương trình $\heva{&mx^2 + x + 1 = 0\\ &x^2 + mx + 1 = 0}$ có nghiệm.\\
	Mà
	$$\heva{&mx^2 + x + 1 = 0\\ &x^2 + mx + 1 = 0}\Rightarrow \heva{&mx^2 + x + 1 = 0\\ &mx^2 + x - x^2 - mx = 0}\Rightarrow \heva{&mx^2 + x + 1 = 0\\ &x^2\left(m - 1\right) + x\left(1- m\right) = 0}$$
	$$\Rightarrow \heva{&mx^2 + x + 1 = 0\\ &\left(m - 1\right)\left(x^2 - x\right) = 0}\Rightarrow 	 \heva{&mx^2 + x + 1 = 0\quad (*)\\ &\hoac{& m - 1 = 0 \\ & x^2 - x = 0}}$$
	- Khi $m - 1 = 0\Rightarrow m = 1$ thay vào $(*)$ suy ra $x^2 + x + 1 = 0$. Dễ thấy phương trình vô nghiệm.\\
	- Khi $x^2 - x = 0\Rightarrow\hoac{& x = 0 \\ & x = 1}$.\\
	+ Khi $x = 0$ thay vào $(*)$ không thỏa mãn.\\
	+ Khi $x = 1$ thay vào $(*)$ suy ra $m + 2 = 0\Rightarrow m = - 2$.
	\shortsol{Giá trị của $m = -2$
	}
	}
\end{bt}
%==========
\begin{bt}%[9D3K5]
	Cho một số gồm hai chữ số. Nếu đổi chỗ hai chữ số của nó ta được số mới hơn số cũ là $45$. Tổng của số đã cho và một số mới tạo thành là $77$. Tìm số đã cho.
	\loigiai{
	Gọi chữ số hàng chục là $x$, chữ số hàng đơn vị là $y$.\\
	Điều kiện: $0<x,\, y \leq 9$; $x,\, y \in \mathbb{N}$.\\
	Theo đề bài ta có hệ phương trình $\heva{&10y+x=10x+y+45 \\&10x+y+10y+x=77} \Rightarrow \heva{&x=1 \\&y=6} \text{ (thỏa mãn điều kiện)}$. \\
	Vậy số đã cho là $16$.
	}
\end{bt}
\begin{bt}%[9D3K5]
	Tìm hai số tự nhiên biết tổng của chúng là $100$ và số lớn hơn số bé là $20$.
	\loigiai{
	Gọi số lớn là $x$, số bé là $y$ ($x;\,y \in \mathbb{N}$). \\
	Theo đề bài ta có hệ phương trình $\heva{&x+y=100 \\&x-y=20} \Rightarrow \heva{&x=60 \\&y=40} \text{ (thỏa mãn điều kiện)}$.\\
	Vậy hai số tự nhiên là $60$, $40$.
	}
\end{bt}
\begin{bt}
	Tìm số tự nhiên ${N}$ có hai chữ số, biết rằng tổng của hai chữ số đó bằng $ 12 $, và nếu viết hai chữ số đó theo thứ tự ngược lại thì được một số lớn hơn ${N}$ là $ 36 $ đơn vị.
	\loigiai{Gọi số tự nhiên cần tìm có dạng là $ \overline{ab} $.\\
	Tổng hai chữ số đó bằng $ 12 $ nên $ a+b=12 $.\\
	Viết hai chữ số đó theo thứ tự ngược lại thì được một số lớn hơn ${N}$ là $ 36 $ đơn vị nên $ 10b+a-10a-b=36 $ hay $ -9a+9b=36 $.\\
	Ta có hệ phương trình $ \heva{&a+b=12\\&-9a+9b=36} $. Suy ra $ \heva{&a=4\\&b=8.} $\\
	Vậy số tự nhiên cần tìm là $ 48 $.
	}
\end{bt}
%--------------------
\begin{bt}%[9D3K5]
	Hai người thợ cùng làm công việc trong $16$ giờ thì xong. Nếu người thứ nhất làm một mình trong $15$ giờ rồi người thứ hai làm tiếp $6$ giờ thì hoàn thành được $75$\% công việc. Hỏi mỗi người làm công việc đó một mình hoàn thành trong bao lâu?
	\loigiai{
	Gọi thời gian người thứ nhất làm một mình hoàn thành công việc là $x$ (giờ; $x>0$); người thứ hai làm một mình hoàn thành công việc là $y$ (giờ; $y>0$). \\
	Theo đề bài ta có hệ phương trình $\heva{&\dfrac{1}{x}+\dfrac{1}{y}=\dfrac{1}{16} \\&\dfrac{15}{x}+\dfrac{6}{y}=75\%} \Rightarrow \heva{&x=24 \\&y=48} \text{ (thỏa mãn điều kiện)}$.	\\
	Vậy nếu làm riêng thì người thứ nhất hoàn thành công việc trong $24$ giờ; người thứ hai hoàn thành công việc trong $48$ giờ.
	}
\end{bt}
\begin{bt}%[9D3K5]
	Theo kế hoạch hai tổ sản xuất $600$ sản phẩm trong một thời gian nhất định. Do áp dụng kĩ thuật mới nên tổ I đã vượt mức $18$\% và tổ II đã vượt mứt $21$\%. Vì vậy trong thời gian quy định họ đã hoàn thành vượt mức $120$ sản phẩm. Hỏi số sản phẩm được giao của mỗi tổ theo kế hoạch?
	\loigiai{
	Gọi số sản phẩm tổ I được giao là $x$ (sản phẩm; $x\in \mathbb{N}$) và 
	số sản phẩm tổ II được giao là $y$ (sản phẩm; $y\in \mathbb{N}$).\\
	Theo đề bài ta có hệ phương trình $\heva{&x+y=600\\&\dfrac{180x}{100}+\dfrac{21y}{100}=120} \Rightarrow \heva{&x=200 \\&y=400} \text{ (thỏa mãn điều kiện)}.$\\
	Vậy sản phẩm tổ I được giao là $200$ sản phẩm; tổ II được giao là $400$ sản phẩm.
	}
\end{bt}
\begin{bt}%[9D3K5]
	Để hoàn thành một công việc, hai tổ phải làm chung trong $6$ giờ. Sau $2$ giờ làm chung thì tổ hai được điều đi làm việc khác, tổ một đã hoàn thành công việc còn lại trong $10$ giờ. Hỏi nếu mỗi tổ làm riêng thì sau bao lâu sẽ làm xong công việc đó?
	\loigiai{
	Gọi thời gian tổ I làm một mình hoàn thành công việc là $x$ (giờ; $x>0$) và 
	tổ II làm một mình hoàn thành công việc là $y$ (giờ; $y>0$). \\
	Theo đề bài ta có hệ phương trình: $\heva{&\dfrac{1}{x}+\dfrac{1}{y}=\dfrac{1}{6}\\&2\cdot \left(\dfrac{1}{x}+\dfrac{1}{y} \right)+\dfrac{10}{x}=1} \Rightarrow \heva{&x=15 \\&y=10} \text{ (thỏa mãn điều kiện)}$.
	}
\end{bt}
%------------- s v t
\begin{bt}
	Một ca nô đi xuôi dòng một quãng đường $42 \mathrm{~km}$ hết $1$ giờ $30$ phút và ngược dòng quãng đường đó hết $2$ giờ $6$ phút. Tính tốc độ của ca nô khi nước yên lặng và tốc độ của dòng nước. Biết rằng tốc độ của ca nô khi nước yên lặng không đổi trên suốt quãng đường và tốc độ của dòng nước cũng không đổi khi ca nô chuyển động.
	\loigiai{
	Gọi $ a $ (km/h) là tốc độ dòng nước và $ b $ (km/h) là tốc độ của ca nô khi nước yên lặng ($ x,y>0 $)\\
	Khi đó, tốc đi xuôi dòng là $ a+b $ và tốc độ đi ngược dòng là $ b-a $.\\
	Thời gian đi xuôi dòng là $1$ giờ $30$ phút, tức $ 1{,}5 $ giờ nên
	\[1{,}5\cdot(a+b)=42\ \text{hay}\ a+b= 28.\tag{1} \]
	Thời gian đi ngược dòng là $2$ giờ $6$ phút, tức $ 2{,}1 $ giờ nên
	\[2{,}1\cdot(b-a)=42\ \text{hay}\ -a+b =20.\tag{2} \]
	Từ (1) và (2), ta có hệ phương trình $ \heva{&a+b= 28\\&-a+b =20.} $\\
	Giải hệ phương trình, ta được $ \heva{&x=4\\&y=24.} $
	\\ Vậy tốc độ của dòng nước là $ 4 $ km/h và tốc độ của ca nô khi nước yên lặng là $ 24 $ km/h.
	}
\end{bt}
\begin{bt}%[9D3G5]
	Một ca nô xuôi dòng $81$ km và ngược dòng $42$ km mất $5$ giờ. Một lần khác, ca nô xuôi dòng $9$ km và ngược dòng $7$ km thì mất $40$ phút. Tính vận tốc riêng của ca nô và vận tốc dòng nước. (Biết vận tốc riêng của ca nô; vận tốc của dòng nước không đổi).
	\loigiai{
	Gọi vận tốc riêng của ca nô là $x$ (km/h; $x>0$) và 
	vận tốc dòng nước là $y$ (km/h; $x>y>0$). \\
	Theo đề bài ta có hệ phương trình $\heva{&\dfrac{81}{x+y}+\dfrac{42}{x-y}=5 \\& \dfrac{9}{x+y}+\dfrac{7}{x-y}=\dfrac{2}{3}.}$ \\
	Đặt $u=\dfrac{1}{x+y}$; $v=\dfrac{1}{x-y}$, ta có $\heva{&81u+42v=5\\&9u+7v=\dfrac{2}{3}.}$ \\
	Giải hệ ta được $\heva{&u=\dfrac{1}{27}\\&v=\dfrac{1}{21}} \Rightarrow \heva{&x+y=27 \\ &x-y=21} \Rightarrow \heva{&x=24 \\&y=3} \text{ (thỏa mãn điều kiện)}.$
	}
\end{bt}
\begin{bt}%[9D3K5]
	Một ôtô đi từ Hà Nội và dự định đến Huế lúc $20$h $30$ phút. Nếu xe đi với vận tốc $45$ km/h thì sẽ đến Huế chậm hơn so với dự định là $2$ giờ. Nếu xe chạy với vận tốc $60$ km/h thì sẽ đến Huế sớm hơn $2$ giờ so với dự định. Tính độ dài quãng đường Hà Nội - Huế và thời gian xe xuất phát từ Hà Nội.
	\loigiai{
	Gọi quãng đường Hà Nội - Huế là $x$ (km; $x>0$) và thời gian ôtô dự định đi là $y$ (giờ; $y>0$).\\
	Theo đề bài ta có hệ phương trình $\heva{&x=60 \cdot (y-2) \\&x=45 \cdot (y+2)} \Rightarrow \heva{&x=720 \\&y=14} \text{ (thỏa mãn điều kiện)}$. \\
	Vậy quảng đường Hà Nội - Huế là $720$ km
	và thời gian xe xuất phát từ Hà Nội là 
	\[20 \text{ giờ } 30 \text{ phút } - 14 \text{ giờ } = 6 \text{ giờ } 30 \text{ phút.}\]
	}
\end{bt}
%------------------3
\begin{bt}%[9D3K5]
	Một trường tổ chức cho học sinh đi tham quan bằng ôtô. Nếu xếp mỗi xe $40$ học sinh thì còn thừa $5$ học sinh. Nếu xếp mỗi xe $41$ học sinh thì xe cuối cùng còn thiếu $3$ học sinh. Hỏi có bao nhiêu học sinh đi tham quan và có bao nhiêu ôtô?
	\loigiai{
	Gọi số học sinh đi tham quan là $x$ (người, $x \in \mathbb{N}^*$) và số ôtô là $y$ (ôtô, $y \in \mathbb{N}^*$). \\
	Theo đề bài ta có hệ phương trình $\heva{&x=40y+5 \\&x=41y-3} \Rightarrow \heva{&x=325 \\&y=8} \text{ (thỏa mãn điều kiện)}$.\\
	Vậy số học sinh đi tham quan là $325$ em và số ôtô là $8$.
	}
\end{bt}
\begin{bt}%[9D3K5]
	Cho một hình chữ nhật. Nếu tăng độ dài mỗi cạnh của nó lên $1$ cm thì diện tích của hình chữ nhật sẽ tăng thêm $13$ cm$^2$. Nếu giảm chiều dài đi $2$ cm, chiều rộng đi $1$ cm thì diện tích của hình chữ nhật sẽ giảm đi $15$ cm$^2$. Tính chiều dài và chiều rộng của hình chữ nhật.
	\loigiai{
	Gọi chiều dài hình chữ nhật là $x$ (cm; $x>0$) và chiều rộng là $y$ (cm; $y>0$).\\
	Theo đề bài ta có hệ phương trình $\heva{&(x+1)\cdot (y+1)=xy+13\\&(x-2)\cdot (y-1)=xy-15} \Rightarrow \heva{&x=7 \\&y=5} \text{ (thỏa mãn điều kiện)}$. \\
	Vậy chiều dài của hình chữ nhật là $7$ cm; chiều rộng là $5$ cm.
	}
\end{bt}
%%%%%%%%%%%%%%%%%
\begin{bt}
	Điểm số trung bình của một vận động viên bắn súng sau $100$ lần bắn là $8{,}69$ điểm. Kết quả cụ thể được ghi trong bảng sau, trong đó có hai ô bị mờ không đọc được (đánh dấu "?"):\\
	\begin{center}
	\begin{tabular}{|l|c|c|c|c|c|}
	\hline Điểm số của mối lẩn bắn & $10$ & $9$ & $8$ & $7$ & $6$ \\
	\hline Số lần bắn & $25$ & $42$ & $?$ & $15$ & $?$ \\
	\hline
	\end{tabular}
	\end{center}
	\loigiai{Gọi $ x,y $ lần lượt là số lần bắn được $8$ điểm và $6$ điểm.\\
	Vận động viên thực hiện $100$ lần bắn nên $ 25+42+x+15+y=100 $, tức $ x+y=18 $.\\
	Điểm số trung bình sau 100 lần bắn là $ 8,69 $ điểm nên $$ \dfrac{25\cdot 10+42\cdot 9+x\cdot 8+15\cdot 7+y\cdot6}{100}=8{,}69.$$
	Suy ra $ 8x+6y= 136$.\\
	Ta có hệ phương trình $ \heva{&x+y=18\\&8x+6y= 136} $. Suy ra $ \heva{&x=14\\&y=4.} $\\
	Vậy số lần bắn được $8$ điểm là $14$ lần và số lần bắn được $6$ điểm là $4$ lần. } 
\end{bt}
\begin{bt}
	Năm ngoái, hai đơn vị sản xuất nông nghiệp thu hoạch được $ 3600 $ tấn thóc. Năm nay, hai đơn vị thu hoạch được $ 4095 $ tấn thóc. Hỏi năm nay, mỗi đơn vị thu hoạch được bao nhiêu tấn thóc, biết rằng năm nay, đơn vị thứ nhất làm vượt mức $15 \%$, đơn vị thứ hai làm vượt mức $12 \%$ so với năm ngoái?
	Hãy dùng máy tính cầm tay để kiểm tra lại kết quả thu được.
\end{bt}
\loigiai{Gọi $ x,y $ lần lượt là số tấn thóc mỗi đơn vị thu hoạch được trong năm nay. \\
	Năm nay, hai đơn vị thu hoạch được $ 4095 $ tấn thóc nên $ x+y=4095 $.
	\\ Vì năm nay, đơn vị thứ nhất làm vượt mức $15 \%$, đơn vị thứ hai làm vượt mức $12 \%$ so với năm ngoái nên lượng thóc thu hoạch được ở năm ngoái của mỗi đơn vị lần lượt là $ \dfrac{x}{115\%} $ và $ \dfrac{y}{112\%} $. \\Do đó $\dfrac{x}{115\%}+ \dfrac{y}{112\%}=3600 $.\\
	Ta có hệ phương trình $ \heva{x+y=4095\\\dfrac{x}{115\%}+ \dfrac{y}{112\%}=3600} $. Suy ra $ \heva{x=2415\\y=1680.} $\\
	Vậy năm nay đơn vị thứ nhất thu hoạch được 2415 tấn thóc và đơn vị thứ hai thu hoạch được 1680 tấn thóc.
}
\begin{bt}
	Hai người thợ cùng làm một công việc trong $ 16 $ giờ thì xong. Nếu người thứ nhất làm trong $ 3 $ giờ và người thứ hai làm trong $ 6 $ giờ thì chỉ hoàn thành được $25 \%$ công việc. Hỏi nếu làm riêng thì mỗi người hoàn thành công việc trong bao lâu?
	\loigiai{Gọi $ x, y $ (giờ) lần lượt là thời gian mỗi người hoàn thành công việc nếu làm riêng một mình.\\
	Trong một giờ, người thứ nhất làm được $ \dfrac{1}{x} $ (công việc) và người thứ hai làm được $ \dfrac{1}{y} $ (công việc). \\
	Hai người cùng làm thì sau $ 16 $ giờ thì xong công việc nên $ \dfrac{16}{x}+\dfrac{16}{y}=1 $.\\
	Nếu người thứ nhất làm trong $ 3 $ giờ và người thứ hai làm trong $ 6 $ giờ thì chỉ hoàn thành được $25 \%$ công việc nên $ \dfrac{3}{x}+\dfrac{6}{y}=\dfrac{1}{4} $.\\
	Ta có hệ phương trình $ \heva{&\dfrac{16}{x}+\dfrac{16}{y}=1\\&\dfrac{3}{x}+\dfrac{6}{y}=\dfrac{1}{4}.} $\\
	Đặt ẩn phụ $ u=\dfrac{1}{x} $ và $ v=\dfrac{1}{y} $, ta đưa về hệ phương trình $ \heva{&16u+16v=1\\&3u+6v=\dfrac{1}{4}.} $
	\\Giải hệ phương trình ta được $ \heva{&u=\dfrac{1}{24}\\&v=\dfrac{1}{48}} $, tức là $ \heva{&\dfrac{1}{x}=\dfrac{1}{24}\\& \dfrac{1}{y}=\dfrac{1}{48}.} $ Suy ra $ \heva{&x=24\\&y=48.} $
	\\Vậy nếu làm riêng thì người thứ nhất hoàn thành xong công việc trong $ 24 $ giờ và người thứ hai hoàn thành công việc trong $ 48 $ giờ.}
\end{bt}
\begin{bt}
	Bác Phương chia số tiền $800$ triệu đồng của mình cho hai khoản đầu tư. Sau một năm, tổng số tiền lãi thu được là $54$ triệu đồng. Lãi suất cho khoản đầu tư thứ nhất là $6 \%$ năm và khoản đầu tư thứ hai là $8 \%$ năm. 
	Tính số tiền bác Phương đầu tư cho mỗi khoản.
	\loigiai{
	Gọi $x$, $y$ lần lượt là số tiền hai khoản đầu tư của bác Phương ($x,y>0$).\\
	Tổng số tiền của bác Phương ban đầu là $800$ triệu đồng, nên ta có phương trình
	\[x+y=800.\tag{1}\]
	Lãi suất cho khoản đầu tư thứ nhất là $6 \%$ năm và khoản đầu tư thứ hai là $8 \%$ năm, nên ta có phương trình 
	\[0{,}06\cdot x + 0{,}08 \cdot y =54\tag{2}\]
	Từ (1) và (2) ta có hệ phương trình $\heva{&x+y=800 \\ &0{,}06\cdot x + 0{,}08 \cdot y =54.}$\\
	Giải hệ phương trình ta được $\heva{&x=500\\&y=300}$ (thỏa mãn).\\
	Vậy bác Phương đầu tư cho khoản thứ nhất và thứ hai lần lượt là $500$ triệu đồng và $300$ triệu đồng.
	}
\end{bt}	
\begin{bt}
	Nhân dịp ngày Giỗ Tổ Hùng Vương, một siêu thị điện máy đã giảm giá nhiều mặt hàng để kích cầu mua sắm. Giá niêm yết của một chiếc tủ lạnh và một chiếc máy giặt có tổng số tiền là $25{,}4$ triệu đồng. Tuy nhiên, trong dịp này tủ lạnh giảm $40 \%$ giá niêm yết và máy giặt giảm $25 \%$ giá niêm yết. Vì thế, cô Liên đã mua hai mặt hàng trên với tổng số tiền là $16{,}77$ triệu đồng. Hỏi giá niêm yết của mỗi mặt hàng trên là bao nhiêu?
	\loigiai{
	Gọi $x, y$ là giá trị của tủ lạnh và máy giặt khi chưa giảm giá ($x,y>0$).\\
	Tổng số tiền của một chiếc tủ lạnh và một chiếc máy giặt trước khi giảm giá là $25{,}4$, nên ta có phương trình 
	\[x+y=25{,}4.\tag{1}\]
	Tổng số tiền của một chiếc tủ lạnh và một chiếc máy lạnh sau khi được giảm là $16{,}77$, nên ta có phương trình 
	\[0{,}6 \cdot x + 0{,}75 \cdot y =16{,}77.\tag{2}\]
	Từ (1) và (2) ta có hệ phương trình $\heva{&x+y=25{,}4 \\ &0{,}6 \cdot x + 0{,}75 \cdot y =16{,}77.}$\\
	Giải hệ phương trình ta được $\heva{&x=12{,}5\\&y=10{,}2}$ (thỏa mãn).
	}
\end{bt}
%%=====Bài 4
\begin{bt}
	Trong tháng thứ nhất, hai tổ sản xuất được $800$ chi tiết máy. So với tháng thứ nhất, trong tháng thứ hai, tổ một sản xuất vượt $15\%$, tổ hai sản xuất vượt $20\%$ nên trong tháng này, cả hai tổ đã sản xuất được $945$ chi tiết máy. Hỏi trong tháng thứ nhất, mỗi tổ sản xuất được bao nhiêu chi tiết máy?
	\loigiai{
	Gọi số chi tiết máy mà tổ 1 sản xuất được trong tháng đầu là
	$x$ (chi tiết máy).\\
	Số chi tiết máy mà tổ 2 sản xuất được trong tháng đầu là 
	$y$ (chi tiết máy).\\
	Điều kiện: $x;y<800$ và $x;y \in \mathbb{N}^*$.\\
	Trong tháng đầu hai tổ công nhân sản xuất được 800 chi tiết máy nên ta có phương trình $x+y=800$.\\
	Sang tháng thứ hai tổ một vượt mức $15\%$, tổ hai sản xuất vượt mức $20\%$, do đó cuối tháng cả hai tổ sản xuất được $945$ chi tiết máy nên ta có phương trình $115\%x+120\%y=945$ hay $1,15x+1,2y=945$.\\
	Ta có hệ phương trình: 
	$\heva{&x+y=800\\&1,15x+1,2y=945.}$\\
	Giải hệ phương trình ta được $\heva{&x=300\\&y=500}$ (thỏa mãn).\\
	Vậy trong tháng đầu tổ một sản xuất được 300 chi tiết máy, tổ hai sản xuất được 500 chi tiết máy.
	}
\end{bt}
%%=====Bài 5
\begin{bt}
	Hai tổ sản xuất cùng may một loại áo khoác xuất khẩu. Nếu tổ thứ nhất may trong $7$ ngày và tổ thứ hai may trong $5$ ngày thì cả hai tổ may được $1540$ chiếc áo. Biết rằng trong mỗi ngày tổ thứ hai may được nhiều hơn tổ thứ nhất $20$ chiếc áo. Hỏi trong một ngày mỗi tổ may được bao nhiêu chiếc áo? (Năng suất may áo của mỗi tổ trong các ngày là như nhau.)
	\loigiai{Gọi $a$, $b$
	lần lượt là số áo may của tổ thứ nhất và tổ thứ hai ($a$, $b\in \mathbb{N}^*$).\\
	Nếu tổ thứ nhất may trong $7$ ngày và tổ thứ hai may trong $5$ ngày thì cả hai tổ may được $1540$ chiếc áo, ta có phương trình $7a+5b=1540$.\\
	Biết rằng trong mỗi ngày tổ thứ hai may được nhiều hơn tổ thứ nhất $20$ chiếc áo, ta có phương trình $b-a=20$ hay $-a+b=20$.\\
	Ta có hệ phương trình $\heva{&7a+5b=1540\\&-a+b=20.}$\\
	Giải hệ phương trình, ta được $a=120$, $b=140$ (thỏa mãn).\\
	Vậy tổ thứ nhất may được $120$ áo, tổ hai may được $140$ áo.
	}
\end{bt}
%%=====Bài 6
\begin{bt}
	Trên mỗi cánh đồng, người ta cấy $60$ ha lúa giống mới và $40$ ha lúa giống cũ, thu hoạch được tất cả $660$ tấn thóc. Hỏi năng suất lúa giống mới trên $1$ ha bằng bao nhiêu? Biết rằng $3$ ha trồng lúa giống mới thu hoạch được ít hơn $4$ ha trồng lúa giống cũ là $3$ tấn. 
	\loigiai{
	Gọi $ x, \: y $ lần lượt là năng suất lúa giống mới và lúa giống cũ trên 1 ha $ \left(x \in \mathbb{N}^{*}, y \in \mathbb{N}^{*} \right) $.\\
	Người ta cấy 60 ha lúa giống mới và 40 ha lúa giống cũ, thu hoạch được tất cả 660 tấn thóc, nên ta có phương trình
	$$ 60x+40y=660. \: (1) $$
	Người ta thấy 3 ha trồng lúa mới thu hoạch ít hơn 4 $ ha $ trồng giống lúa cũ là 3 tấn, nên ta có phương trình
	$$ 4y-3x=3. \: (2) $$
	Từ $ (1) $ và $ (2) $, ta có hệ phương trình
	$ \heva{
	&60x+40y=660 \\
	& -3x+4y=3.
	} $\\
	Giải hệ phương trình ta được $ \heva{ &x=11\\
	&y=9} $ (thỏa mãn).\\
	Vậy năng suất lúa giống mới là 11 ha và lúa giống cũ là 9 ha. 
	}
\end{bt}
\begin{bt}
	Tìm các hệ số $x, y$ để cân bằng mỗi phương trình phản ứng hoá học sau
	\begin{listEX}[2]
	\item $2 \mathrm{Fe}+y \mathrm{Cl}_2 \rightarrow x \mathrm{FeCl}_3$;
	\item $x \mathrm{FeCl}_3+\mathrm{Fe} \rightarrow y \mathrm{FeCl}_2$.
	\end{listEX}
	\loigiai{
	\begin{enumerate}
	\item Gọi $x, y$ lần lượt là hệ số của $\mathrm{Fe}$ và $\mathrm{Cl_2}$ thỏa mãn cân bằng phương trình hóa học 
	\[2 \mathrm{Fe}+y \mathrm{Cl}_2 \rightarrow x \mathrm{FeCl}_3\]
	Cân bằng số nguyên tử $\mathrm{Fe}$ và số nguyên tử $\mathrm{Cl_2}$ ở hai vế ta được hệ 
	\[\heva{&2=x \\& 2y=3x.}\]
	Giải hệ phương trình này ta được $x=2$ và $y=3$.\\
	Đưa các hệ số tìm được vào phương trình hóa học, ta có 
	\[2 \mathrm{Fe}+ 3\mathrm{Cl}_2 \rightarrow 2\mathrm{FeCl}_3.\]
	\item Gọi $x, y$ lần lượt là hệ số của $\mathrm{Fe}$ và $\mathrm{Cl}$ thỏa mãn cân bằng phương trình hóa học 
	\[x \mathrm{FeCl}_3+\mathrm{Fe} \rightarrow y \mathrm{FeCl}_2\]
	Cân bằng số nguyên tử $\mathrm{Fe}$ và số nguyên tử $\mathrm{Cl}$ ở hai vế ta được hệ 
	\[\heva{&x+1=y \\& 3x=2y.}\]
	Giải hệ phương trình này ta được $x=2$ và $y=3$.\\
	Đưa các hệ số tìm được vào phương trình hóa học, ta có 
	\[2 \mathrm{FeCl}_3+\mathrm{Fe} \rightarrow 3 \mathrm{FeCl}_2.\]
	\end{enumerate}
	}
\end{bt}
%%=====Bài 7
\begin{bt}
	Cân bằng các phương trình hóa học sau bằng phương pháp đại số.
	trong mỗi trường hợp sau:
	\begin{listEX}[2]
	\item $\text{Ag}+\text{Cl}_2 \to \text{AgCl}$;
	\item $\text{CO}_2+\text{C} \to \text{CO}$.
	\end{listEX}
	\loigiai{
	\begin{listEX}
	\item Gọi $x$; $y$ lần lượt là hệ số của Ag và $\text{Cl}_2$ trong phương trình hóa học ($x;y\in \mathbb{R}$).
	$$x\text{Ag}+y\text{Cl}_2 \to \text{AgCl}$$
	Cân bằng số nguyên tử của Ag và số nguyên tử Cl ở hai vế, ta được hệ
	$$\heva{&x=1\\&2y=1.}$$
	Giải hệ phương trình này, ta được $x=1$ và $y=\dfrac{1}{2}$.\\
	Đưa các hệ số của phương trình hóa học, ta có
	$$2\text{Ag}+\dfrac{1}{2}\text{Cl}_2 \to \text{AgCl}$$
	Nhân hai vế của phương trình hóa học với $2$, ta được
	$$2\text{Ag}+\text{Cl}_2 \to 2\text{AgCl}$$
	\item Gọi $x$; $y$ lần lượt là hệ số của $\text{CO}_2$ và C trong phương trình hóa học ($x;y\in \mathbb{R}$).
	$$x\text{CO}_2+y\text{C} \to \text{CO}$$
	Cân bằng số nguyên tử của C và số nguyên tử O ở hai vế, ta được hệ
	$$\heva{&x+y=1\\&2x=1.}$$
	Giải hệ phương trình này, ta được $x=\dfrac{1}{2}$ và $y=\dfrac{1}{2}$.\\
	Đưa các hệ số của phương trình hóa học, ta có
	$$\dfrac{1}{2}\text{CO}_2+\dfrac{1}{2}\text{C} \to \text{CO}$$
	Nhân hai vế của phương trình hóa học với $2$, ta được
	$$\text{CO}_2+\text{C} \to 2\text{CO}$$
	\end{listEX}
	}
\end{bt}