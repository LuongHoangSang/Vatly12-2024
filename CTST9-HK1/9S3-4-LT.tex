\section*{LUYỆN TẬP CHUNG 2}
%%==========Bài 1
\begin{bt}%[9D1B6]
	Đưa thừa số ra ngoài dấu căn:
	\begin{listEX}[2]
	\item $\sqrt{75 a^3}$; 
	\item $\sqrt{98 a^5\left(b^2-6 b+9\right)}$. 
	\end{listEX}
	\loigiai{
	\begin{listEX}
	\item $\sqrt{75 a^3}=5a\sqrt{3a}\ (a\geq 0)$; 
	\item $\sqrt{98 a^5\left(b^2-6 b+9\right)}=\heva{7a^2(b-3)\sqrt{2a}&\ \ \text{nếu } b\geq 3\\ 7a^2(3-b)\sqrt{2a}&\ \ \text{nếu } b<3.}$
	\end{listEX}
	}
\end{bt}
%%==========Bài 2
\begin{bt}%[9D1B6]
	Rút gọn biểu thức 
	\begin{listEX}[2]
	\item $2 \sqrt{125}-5 \sqrt{45}+6 \sqrt{20}$; 
	\item $2 \sqrt{75}-4 \sqrt{27}+\sqrt{12}$. 
	\end{listEX}
	\loigiai{
	\begin{listEX}[2]
	\item $2 \sqrt{125}-5 \sqrt{45}+6 \sqrt{20}=7\sqrt{5}$; 
	\item $2 \sqrt{75}-4 \sqrt{27}+\sqrt{12}=0$.
	\end{listEX}	
	}
\end{bt}
%%==========Bài 3
\begin{bt}%[9D1K6]
	So sánh các số sau
	\begin{listEX}[2]
	\item $3\sqrt{7}$ và $2\sqrt{15}$;
	\item $-4\sqrt{5}$ và $-5\sqrt{3}$.
	\end{listEX}
	\loigiai{
	\begin{listEX}[2]
	\item $3\sqrt{7}>2\sqrt{15}$;
	\item $-4\sqrt{5}<-5\sqrt{3}$.
	\end{listEX}
	}
\end{bt}
%%==========Bài 4
\begin{bt}%[9D1B6]
	Khử mẫu của biểu thức lấy căn
	\begin{listEX}[2]
	\item $\sqrt{\dfrac{3}{80}}$; 
	\item $\sqrt{\dfrac{2}{75}}$. 
	\end{listEX}
	\loigiai{
	\begin{listEX}[2]
	\item $\sqrt{\dfrac{3}{80}}=\dfrac{1}{20}\sqrt{15}$; 
	\item $\sqrt{\dfrac{2}{75}}=\dfrac{1}{15}\sqrt{6}$.
	\end{listEX}	
	}
\end{bt}
%%==========Bài 5
\begin{bt}%[9D1B6]
	Trục căn thức ở mẫu
	\begin{listEX}[3]
	\item $\dfrac{a-2 \sqrt{a}}{\sqrt{a}-2}$; 
	\item $\dfrac{13}{2 \sqrt{3}-5}$;
	\item $\dfrac{-\sqrt{2}}{1-\sqrt{2}+\sqrt{3}}$. 
	\end{listEX}
	\loigiai{
	\begin{listEX}[3]
	\item $\dfrac{a-2 \sqrt{a}}{\sqrt{a}-2}=\sqrt{a}$; 
	\item $\dfrac{13}{2 \sqrt{3}-5}=-\left(2\sqrt{3}+5 \right)$;
	\item $\dfrac{-\sqrt{2}}{1-\sqrt{2}+\sqrt{3}}=\dfrac{1-\sqrt{2}-\sqrt{3}}{2}$. 
	\end{listEX}
	}
\end{bt}
%%==========Bài 6
\begin{bt}%[9D1B6]
	Trục căn thức ở mẫu
	\begin{listEX}[3]
	\item $\dfrac{8}{\sqrt{5}-3}$; 
	\item $\dfrac{1}{5 \sqrt{2}-2 \sqrt{5}}$;
	\item $\dfrac{\sqrt{5}-\sqrt{7}}{\sqrt{5}+\sqrt{7}}$. 
	\end{listEX}
	\loigiai{
	\begin{listEX}[3]
	\item $\dfrac{8}{\sqrt{5}-3}=-2\left(\sqrt{5}+3\right)$; 
	\item $\dfrac{1}{5 \sqrt{2}-2 \sqrt{5}}=\dfrac{5 \sqrt{2}+2 \sqrt{5}}{30}$;
	\item $\dfrac{\sqrt{5}-\sqrt{7}}{\sqrt{5}+\sqrt{7}}=\sqrt{35}-6$.
	\end{listEX}
	}
\end{bt}
%%==========Bài 7
\begin{bt}%[9D1B8]
	Rút gọn các biểu thức sau:
	\begin{listEX}[2]
	\item $\sqrt{6} + 3\sqrt{\dfrac{2}{3}} - 4\sqrt{\dfrac{3}{2}} + 12\sqrt{\dfrac{1}{6}}$;
	\item $6\sqrt{a} + 3\sqrt{25a^3} - 2\sqrt{36a b^2} - 2\sqrt{9a}$ với $a, b>0$.
	\end{listEX}
	\loigiai{
	\begin{listEX}[2]
	\item $2\sqrt{6}$;
	\item $3\left(5a-4b\right)\sqrt{a}$.
	\end{listEX}
	}
\end{bt}
%%==========Bài 8
\begin{bt}
	Rút gọn các biểu thức sau
	\begin{listEX}[2]
	\item $\dfrac{5+3 \sqrt{5}}{\sqrt{5}}-\dfrac{1}{\sqrt{5}-2}$;
	\item $\sqrt{(\sqrt{7}-2)^{2}}-\sqrt{63}+\dfrac{\sqrt{56}}{\sqrt{2}}$;
	\item $\dfrac{\sqrt{(\sqrt{3}+\sqrt{2})^{2}}+\sqrt{(\sqrt{3}-\sqrt{2})^{2}}}{2 \sqrt{12}}$;
	\item $\sqrt{(\sqrt{3}+1)^{2}}-\sqrt{3(\sqrt{3}+1)^{2}}+\sqrt{4}$.
	\end{listEX}
	\loigiai{
	\begin{listEX}
	\item $\dfrac{5+3 \sqrt{5}}{\sqrt{5}}-\dfrac{1}{\sqrt{5}-2}=\dfrac{\sqrt{5}(\sqrt{5}+3)}{\sqrt{5}}-\dfrac{\sqrt{5}+2}{\sqrt{5}^2-2^2}=\sqrt{5}+3-\sqrt{5}-2=1$.
	\item $\sqrt{(\sqrt{7}-2)^{2}}-\sqrt{63}+\dfrac{\sqrt{56}}{\sqrt{2}}=\sqrt{7}-2-3\sqrt{7}+2\sqrt{7}=-2$.
	\item $\dfrac{\sqrt{(\sqrt{3}+\sqrt{2})^{2}}+\sqrt{(\sqrt{3}-\sqrt{2})^{2}}}{2 \sqrt{12}}=\dfrac{\sqrt{3}+\sqrt{2}+\sqrt{3}-\sqrt{2}}{4\sqrt{3}}=\dfrac{2\sqrt{3}}{4\sqrt{3}}=\dfrac{1}{2}$.
	\item 
	$\sqrt{(\sqrt{3}+1)^{2}}-\sqrt{3} \cdot \sqrt{(\sqrt{3}+1)^{2}}+\sqrt{2^{2}}=(\sqrt{3}+1)-\sqrt{3}(\sqrt{3}+1)+2=0$.
	\end{listEX}
	}
\end{bt}
%%==========Bài 9
\begin{bt}
	Tính giá trị của các biểu thúc sau	
	\begin{listEX}[2]
	\item $3 \sqrt{45}+\dfrac{5 \sqrt{15}}{\sqrt{3}}-2 \sqrt{245}$;
	\item $\dfrac{\sqrt{12}-\sqrt{4}}{\sqrt{3}-1}-\dfrac{\sqrt{21}+\sqrt{7}}{\sqrt{3}+1}+\sqrt{7}$;
	\item $\dfrac{3-\sqrt{3}}{1-\sqrt{3}}+\sqrt{3}(2 \sqrt{3}-1)+\sqrt{12}$;
	\item $\dfrac{\sqrt{3}-1}{\sqrt{2}}+\dfrac{\sqrt{2}}{\sqrt{3}-1}-\dfrac{6}{\sqrt{6}}$.
	\end{listEX}
	\loigiai{
	\begin{listEX}
	\item $3 \sqrt{45}+\dfrac{5 \sqrt{15}}{\sqrt{3}}-2 \sqrt{245}=3\sqrt{5\cdot9}+\dfrac{5\sqrt{5}\sqrt{3}}{\sqrt{3}}-2\sqrt{5\cdot49}=9\sqrt{5}+5\sqrt{5}-14\sqrt{5}=0$.
	\item $\dfrac{\sqrt{12}-\sqrt{4}}{\sqrt{3}-1}-\dfrac{\sqrt{21}+\sqrt{7}}{\sqrt{3}+1}+\sqrt{7}=\dfrac{2(\sqrt{3}-1)}{\sqrt{3}-1}-\dfrac{\sqrt{7}(\sqrt{3}+1)}{\sqrt{3}+1}+\sqrt{7}=2-\sqrt{7}+\sqrt{7}=2$.
	\item $\dfrac{3-\sqrt{3}}{1-\sqrt{3}}+\sqrt{3}(2 \sqrt{3}-1)+\sqrt{12}=\dfrac{\sqrt{3}(\sqrt{3}-1)}{1-\sqrt{3}}+6-\sqrt{3}+2\sqrt{3}=-\sqrt{3}+6-\sqrt{3}+2\sqrt{3}=6$.
	\item $\dfrac{\sqrt{3}-1}{\sqrt{2}}+\dfrac{\sqrt{2}}{\sqrt{3}-1}-\dfrac{6}{\sqrt{6}}=\dfrac{\sqrt{6}-\sqrt{2}}{2}+\dfrac{\sqrt{6}+\sqrt{2}}{2}-\sqrt{6}=\dfrac{2\sqrt{6}}{2}-\sqrt{6}=0$.
	\end{listEX}
	}
\end{bt}
%%==========Bài 11
\begin{bt}
	Cho biểu thức $P=\dfrac{\sqrt{x}}{\sqrt{x}+1}+\dfrac{\sqrt{x}}{\sqrt{x}-1}$ với $x>1$.
	\begin{listEX}
	\item Rút gọn $P$.
	\item Sử dụng kết quả câu $a$, tính giá trị của $P$ khi $x=101$.
	\end{listEX}
	\loigiai{
	\begin{listEX}
	\item $P=\dfrac{\sqrt{x}(\sqrt{x}-1)+\sqrt{x}(\sqrt{x}+1)}{(\sqrt{x}+1)(\sqrt{x}-1)}=\dfrac{x-\sqrt{x}+x+\sqrt{x}}{x-1}=\dfrac{2 x}{x-1}$.
	\item Khi $x=101$ (thoả mãn điều kiện) giá trị của $P=\dfrac{2 \cdot 101}{101-1}=\dfrac{202}{100}=2{,}02$.
	\end{listEX}
	}
\end{bt}
%%==========Bài 12
\begin{bt}
	Giả sử lực $F$ của gió khi thổi theo phương vuông góc với bề mặt cánh buồm của một con thuyển tỉ lệ thuận với bình phương tốc độ của gió, hệ số tỉ lệ là $30$. Trong đó, lực $F$ được tính bằng $N$ (Newton) và tốc độ được tính bằng $m/s$.
	\begin{listEX}
	\item Khi tốc độ của gió là $10 \mathrm{~m} / \mathrm{s}$ thì lực $F$ là bao nhiêu Newton?
	\item Nếu cánh buồm chỉ có thể chịu được một áp lực tối đa là $12~000 \mathrm{~N}$ thì con thuyền đó có thể đi được trong gió với tốc độ gió tối đa là bao nhiêu?
	\end{listEX}
	\loigiai{
	\begin{listEX}
	\item Gọi tốc độ gió là $v$ ($m/s$).\\
	Với $v=10$ thì $F=30 \cdot 10^2=3~000$.
	\item Cánh buồm chịu được một áp lực tối đa là $12~000 \mathrm{~N}$ thì con thuyền có thể đi được trong gió với tốc độ gió tối đa là $\sqrt{\dfrac{12~000}{30}}=\sqrt{400}=20$ $(m/s)$.
	\end{listEX}
	}
\end{bt}
%%==========Bài 13
\begin{bt}%[9D1B6]
	Tính 
	\begin{listEX}
	\item $\left(\dfrac{1}{\sqrt{2}-\sqrt{3}}\right)^2$;
	\item $\dfrac{1}{\sqrt{1}+\sqrt{2}}+\dfrac{1}{\sqrt{2}+\sqrt{3}}+\dfrac{1}{\sqrt{3}+\sqrt{4}}+\cdots+\dfrac{1}{\sqrt{99}+\sqrt{100}}$. 
	\end{listEX}
	\loigiai{
	\begin{listEX}
	\item $\left(\dfrac{1}{\sqrt{2}-\sqrt{3}}\right)^2=5+2\sqrt{6}$;
	\item \allowdisplaybreaks $\begin{aligned}[t] 
	\dfrac{1}{\sqrt{1}+\sqrt{2}}+\dfrac{1}{\sqrt{2}+\sqrt{3}}+\dfrac{1}{\sqrt{3}+\sqrt{4}}+\cdots+\dfrac{1}{\sqrt{99}+\sqrt{100}}&=\dfrac{1-\sqrt{2}}{1-2}+\dfrac{\sqrt{2}-\sqrt{3}}{2-3}+\cdots +\dfrac{\sqrt{99}-\sqrt{100}}{99-100} \\ 
	&=\sqrt{100}-1=9.
	\end{aligned}$ 
	\end{listEX}	
	}
\end{bt}
%%==========Bài 14
\begin{bt}%[9D1K6]
	Cho $x=\dfrac{\sqrt{75}+\sqrt{12}}{\sqrt{147}-\sqrt{48}}$. Chứng minh rằng $3x$ là một số nguyên.
	\loigiai{
	Ta có $x=\dfrac{\sqrt{75}+\sqrt{12}}{\sqrt{147}-\sqrt{48}} = \dfrac{7}{3}$, do đó $3x=7 \in \mathbb{Z}$.	
	}
\end{bt}
%%==========Bài 15
\begin{bt}%[9D1G6]
	Biến đổi $\dfrac{26}{10+4 \sqrt{3}}$ về dạng $a+b\sqrt{3}$. Tính tích $a\cdot b$.
	\loigiai{
	$\dfrac{26}{10+4 \sqrt{3}}=\dfrac{13}{5+2 \sqrt{3}}=5-2 \sqrt{3}$.\\
	Vậy $a=5$; $b=-2$. Do đó $a\cdot b=5 \cdot (-2)=-10$. 	
	}
\end{bt}
%%==========Bài 16
\begin{bt}%[9D1G6]
	Tìm các cặp số nguyên dương $(x;y)$ trong đó $x<y$ sao cho $\sqrt{x}+\sqrt{y}=\sqrt{539}$.
	\loigiai{
	$\sqrt{539}=\sqrt{49 \cdot 11} =7\sqrt{11}$.
	Ta có \allowdisplaybreaks $\begin{aligned}[t] 
	7\sqrt{11}&=\sqrt{11}+6 \sqrt{11}=2 \sqrt{11}+5 \sqrt{11}=3 \sqrt{11}+4 \sqrt{11}\\ 
	7\sqrt{11}&=\sqrt{11}+\sqrt{36\cdot 11}=\sqrt{4\cdot 11}+\sqrt{25\cdot 11}=\sqrt{9\cdot 11}+\sqrt{16\cdot 11}\\
	7\sqrt{11}&=\sqrt{11}+\sqrt{396}=\sqrt{44}+\sqrt{275}=\sqrt{99}+\sqrt{176}. 
	\end{aligned}$	\\
	Vậy có ba cặp số thỏa yêu cầu bài: $(11;396)$; $(44;275)$; $(99;176)$.
	}
\end{bt}
%%==========Bài 17
\begin{bt}%[9D1K8]
	Rút gọn rồi tính giá trị của biểu thức $P$ với $x=0{,}36$: 
	$$P =\dfrac{\sqrt{x}}{\sqrt{x} + 3} - \dfrac{3}{3 - \sqrt{x}} - \dfrac{6\sqrt{x}}{x - 9}.$$
	\loigiai{$P=\dfrac{\sqrt{x}-3}{\sqrt{x}+3}$ với điều kiện $0\leq x\neq 9$. Khi $x=0{,}36$, ta có $P=-\dfrac{2}{3}$.}
\end{bt}
%%==========Bài 18
\begin{bt}%[9D1K8]
	Chứng minh đẳng thức sau với $x\geq 0$, $y>0$, $y\neq 1$, $x\neq y$:
	$$\left(\dfrac{\sqrt{x} + \sqrt{y}}{\sqrt{x} - \sqrt{y}} - \dfrac{\sqrt{x} - \sqrt{y}}{\sqrt{x} + \sqrt{y}}\right)\cdot\dfrac{\sqrt{y} - 1}{y - \sqrt{y}}=\dfrac{4\sqrt{x}}{x - y}.$$
	\loigiai{Rút gọn vế trái được $\dfrac{4\sqrt{xy}}{x-y}\cdot\dfrac{1}{\sqrt{y}}=\dfrac{4\sqrt{x}}{x-y}$.}
\end{bt}
%%==========Bài 19
\begin{bt}%[9D1G8]
	Cho biểu thức $P =\left(x + \dfrac{1}{\sqrt{x}}\right)\left(\dfrac{\sqrt{x} - 1}{x - \sqrt{x} + 1} - \dfrac{1}{\sqrt{x} + 1}\right)$.
	\begin{listEX}
	\item Rút gọn $P$.
	\item Tìm các giá trị nguyên của $x$ để $P$ có giá trị nguyên.
	\end{listEX}	
	\loigiai{
	\begin{listEX}[2]
	\item $P=\dfrac{\sqrt{x}-2}{\sqrt{x}}$ với $x>0$;
	\item $x\in\left\lbrace 1; 4\right\rbrace $.
	\end{listEX}
	}
\end{bt}
%%==========Bài 20
\begin{bt}%[9D1G8]
	Cho biểu thức $P =\left(\dfrac{\sqrt{x}}{x - 36} - \dfrac{\sqrt{x} - 6}{x + 6\sqrt{x}}\right)\cdot\dfrac{x\sqrt{x} - 36\sqrt{x}}{2\left(\sqrt{x} - 3\right)\left(x - 2\sqrt{x} + 3\right)}$.	
	\begin{listEX}
	\item Rút gọn $P$.
	\item Với giá trị nào của $x$ thì $P$ có giá trị lớn nhất? Giá trị lớn nhất đó là bao nhiêu?
	\end{listEX}	
	\loigiai{
	\begin{listEX}
	\item $P=\dfrac{6}{x-2\sqrt{x}+3}$ với điều kiện $x>0$, $x\neq 9$, $x\neq 36$.
	\item $P=\dfrac{6}{\left(\sqrt{x}-1\right)^2+2}\leq\dfrac{6}{2}=3$ vì $\left(\sqrt{x}-1\right)^2\geq 0$.\\Suy ra $\max P=3$ đạt được khi $x=1$.
	\end{listEX}
	}
\end{bt}
%%==========Bài 21
\begin{bt}%[9D1G8]
	Cho biểu thức $P =\dfrac{2\sqrt{x} + 3}{\sqrt{x} + 3} + \dfrac{3\sqrt{x} - 2}{\sqrt{x} - 1} - \dfrac{15\sqrt{x} - 11}{x + 2\sqrt{x} - 3}$.	
	\begin{listEX}[2]
	\item Rút gọn $P$.
	\item Tìm giá trị nhỏ nhất của $P$.
	\end{listEX}	
	\loigiai{\begin{listEX}
	\item $P=\dfrac{5\sqrt{x}-2}{\sqrt{x}+3}$ với $x\geq 0; x\neq 1$.
	\item $P=\dfrac{5\sqrt{x}+15-17}{\sqrt{x}+3}=5-\dfrac{17}{\sqrt{x}+}$. Do đó $P\geq 5-\dfrac{17}{3}=-\dfrac{2}{3}$ vì $\sqrt{x}\geq 0$.\\
	Vậy $\min P=-\dfrac{2}{3}$ khi $x=0$.
	\end{listEX}}
\end{bt}
%==============
%==============