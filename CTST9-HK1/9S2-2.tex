\setcounter{section}{1}
\section{BẤT PHƯƠNG TRÌNH BẬC NHẤT MỘT ẨN}
\subsection{Trọng tâm kiến thức}
\begin{tomtat}
\subsubsection{Khái niệm bất phương trình bậc nhất một ẩn}
\paragraph{Bất phương trình bậc nhất một ẩn}
\begin{boxdn}
	Bất phương trình dạng $ax + b < 0$ (hoặc $ax + b > 0$; $ax + b \leq 0$; $ax + b \geq 0$) trong đó $a$, $b$ là hai số đã cho, $a \neq 0$ được gọi là \textbf{bất phương trình bậc nhất một ẩn $x$}.
\end{boxdn}
\paragraph{Nghiệm của bất phương trình}
\begin{boxdn}
	\begin{itemize}
	\item Với bất phương trình bậc nhất bậc nhất có ẩn là $x$, số $x_0$ được gọi là {\it nghiệm} của phất phương trình nếu ta thay $x=x_0$ thì nhận được một khẳng định đúng. 
	\item Giải một bất phương trình là tìm tất cả các nghiệm của bất phương trình đó.
	\end{itemize}
\end{boxdn}
\subsubsection{Cách giải bất phương trình bậc nhất một ẩn}
\begin{boxdn}
	Bất phương trình $ax+b>0$ ($a \neq 0$) được giải như sau:
	\begin{itemize}
	\item Cộng hai vế của bất phương trình với $-b$, ta được bất phương trình:
	\[ax > -b.\]
	\item Nhân hai vế của bất phương trình nhận được với $\dfrac{1}{a}$:
	\begin{itemize}
	\item Nếu $a>0$ thì nhận được nghiệm của bất phương trình đã cho là $x > -\dfrac{b}{a}$.
	\item Nếu $a<0$ thì nhận được nghiệm của bất phương trình đã cho là $x<-\dfrac{b}{a}$.
	\end{itemize}
	\end{itemize}
\end{boxdn}
\begin{luuy}
	\begin{enumerate}[\bf ---]
	\item Các bất phương trình $ax + b > 0$, $ax + b \leq 0$, $ax + b \geq 0$ được giải tương tự.
	\item
	Ta cũng có thể giải được các bất phương trình một ẩn đưa được về dạng $ax + b < 0$, $ax + b > 0$, $ax + b \leq 0$, $ax + b \geq 0$.
	\end{enumerate}
\end{luuy}
\end{tomtat}
%%%%%%%%%%%%%
\subsection{Các dạng bài tập}
\begin{dang}{Nhận biết bất phương trình bậc nhất, nghiệm của bất phương trình}
\end{dang}
%%==========Ví dụ 1
\begin{vd}
	Bất phương trình nào sau đây là bất phương trình bậc nhất một ẩn?
	\begin{listEX}[4]
	\item $0x < 0$;
	\item $3x < 0$;
	\item $x^3 + 1 \geq 0$;
	\item $-x + 1 \leq 0$;
	\item $a+2\,023 > 0$;
	\item $0x - 5 <0$;
	\item $5x - 7 \leq 0$;
	\item $x^2 +1 \leq 0$.
	\end{listEX}
	\loigiai{
	\begin{itemize}
	\item Hai bất phương trình $0x < 0$ và $x^3 + 1 \geq 0$ không phải là bất phương trình bậc nhất một ẩn.
	\item Bất phương trình $3x < 0$ có dạng $ax+b<0$ với $a = 3 \neq 0$ và $b=0$, nên nó là bất phương trình bậc nhất một ẩn.
	\item Bất phương trình $-x + 1 \leq 0$ có dạng $ax+b \leq 0$ với $a = -1 \neq 0$ và $b=1$, nên nó là bất phương trình bậc nhất một ẩn.
	\item Bất phương trình $a+2\,023 > 0$ có dạng $ax+b>0$ với $a = 1 \neq 0$ và $b=2\,023$, nên nó là bất phương trình bậc nhất một ẩn.
	\item Bất phương trình $5x - 7 \leq 0$ có dạng $ax +b \leq 0$ với $a = 5$ và $b=-7$, nên nó là bất phương trình bậc nhất một ẩn.
	\item Hai bất phương trình $0x - 5 <0$ và $x^2 +1 \leq 0$ không phải là bất phương trình bậc nhất một ẩn.
	\end{itemize}
	}
\end{vd}
%%==========Ví dụ 2
\begin{vd}
	Bất phương trình nào sau đây là bất phương trình bậc nhất một ẩn $x$?
	\begin{listEX}[4]
	\item $3x + 16 \leq 0$;
	\item $-5x + 5 > 0$;
	\item $x^2 - 4 > 0$;
	\item $-3x < 0$;
	\item $-3x + 7 \leq 0$;
	\item $4x - \dfrac{3}{2}> 0$;
	\item $x^3 > 0$;
	\item $2x^2-19\leq0$.
	\end{listEX}
	\loigiai{
	\begin{itemize}
	\item a), b), d), e), f) là bất phương trình bậc nhất một ẩn $x$.
	\item c), h) không là bất phương trình bậc nhất một ẩn $x$ vì $x^2 - 4$, $2x^2-19$ là các đa thức bậc hai.
	\item g) không là bất phương trình bậc nhất một ẩn $x$ vì $x^3$ là một đa thức bậc ba.
	\end{itemize}
	}
\end{vd}
%%==========Ví dụ 3
\begin{vd}
	Kiểm tra xem giá trị $x=5$ có phải là nghiệm của mỗi bất phương trình bậc nhất sau hay không?
	\begin{listEX}[3]
	\item $6x-29>0$.
	\item $11x-52>0$.
	\item $x-2\leq0$.
	\end{listEX}
	\loigiai{
	\begin{enumerate}
	\item Thay $x=5$, ta có: $6\cdot5-29>0$ là khẳng định đúng.
	Vậy $x=5$ là nghiệm của bất phương trình $6x-29>0$.
	\item Thay $x=5$, ta có: $11\cdot5-52>0$ là khẳng định đúng.
	Vậy $x=5$ là nghiệm của bất phương trình $11x-52>0.$
	\item Thay $x=5$, ta có: $5-2\leq0$ là khẳng định không đúng.
	Vậy $x=5$ không là nghiệm của bất phương trình $x-2\leq0$. 
	\end{enumerate}
	}
\end{vd}
%%==========Ví dụ 4
\begin{vd}
	Trong hai giá trị $x=1$ và $x=2$, giá trị nào là nghiệm của bất phương trình $3x-4 \leq 0$?
	\loigiai{
	\begin{listEX}[1]
	\item Thay $x=1$ vào bất phương trình, ta được $3 \cdot 1 - 4 \leq 0$ là khẳng định đúng. Vậy $x=1$ là một nghiệm của bất phương trình đã cho.
	\item Thay $x=2$ vào bất phương trình, ta được $3 \cdot 2 - 4 \leq 0$ là khẳng định sai. Vậy $x=2$ không là nghiệm của bất phương trình đã cho.
	\end{listEX}
	}
\end{vd}
%%==========Ví dụ 5
\begin{vd}
	Tìm một số là nghiệm và một số không phải là nghiệm của bất phương trình $4x+5>0$.
	\loigiai{
	\begin{listEX}[1]
	\item Lấy $x=0$ thay vào bất phương trình đã cho, ta thấy $4 \cdot 0 + 5 > 0$ là khẳng định đúng. Vậy $x=0$ là một nghiệm của bất phương trình đã cho.
	\item Lấy $x=-2$ thay vào bất phương trình đã cho, ta được $4 \cdot (-2) + 5 > 0$ là khẳng định sai. Vậy $x=-2$ không phải là nghiệm của bất phương trình đã cho.
	\end{listEX}
	}
\end{vd}
%%==========Ví dụ 6
\begin{vd}
	Nêu hai ví dụ về bất phương trình bậc nhất một ẩn $x$.
	\loigiai{
	Hai bất phương trình bậc nhất một ẩn $x$ là
	\begin{listEX}[2]
	\item $2x+4\leq 0$.
	\item $-x-3>0$.
	\end{listEX}
	}
\end{vd}
%%==========Ví dụ 7
\begin{vd}
	Trong các số $-2$; $0$; $5$, những số nào là nghiệm của bất phương trình $2x - 10 < 0$?
	\loigiai{Chỉ có $-2$ và $0$ là nghiệm của bất phương trình đã cho. }
\end{vd}
%%==========Ví dụ 8
\begin{vd}%[8D4B3]
	Kiểm tra xem $x=-5$ có phải là nghiệm của bất phương trình $2x+7<1-3x$ không?
	\loigiai{
	Thay $x=-5$ vào hai vế của bất phương trình đã cho, ta được
	$$-2\cdot (-5)+7<1-3\cdot(-5)$$
	hay $-3<16$ (bất đẳng thức đúng).\\
	Vậy $x=-5$ là một nghiệm của bất phương trình đã cho.
	}
\end{vd}
%---------------
\begin{dang}{Giải bất phương trình bậc nhất một ẩn}
\end{dang}
%%==========Ví dụ 9
\begin{vd}
	Giải các bất phương trình sau:
	\begin{listEX}[3]
	\item $-2x - 4 > 0$;
	\item $2x + 1 >0$;
	\item $0{,}5x - 6 \leq 0$;
	\item $-2x + 3 \leq 0$;
	\item $5x-3<0$;
	\item $-6x-2 \geq 0$.
	\end{listEX}
	\loigiai{
	\begin{enumerate}
	\item Ta có: $\begin{aligned}[t]
	2x+1&>0 & &\\
	2x &>-1 & &\text{ (cộng hai vế với }-1) \\
	(2x) \cdot \dfrac{1}{2}&>(-1) \cdot \dfrac{1}{2}& &\text{ (nhân hai vế với }\dfrac{1}{2})\\
	x&>-\dfrac{1}{2}.& &
	\end{aligned}
	$ \\
	Vậy nghiệm của bất phương trinh là $x>-\dfrac{1}{2}$.
	\item Ta có: $\begin{aligned}[t]
	0{,}5 x-6 &\leq 0& &\\ 
	0{,}5 x & \leq 6 & & \text { (cộng hai vế với } 6) \\
	(0{,}5 x )\cdot 2 & \leq 6\cdot 2 & & \text { (nhân hai vế với } 2) \\
	x & \leq 12 . & &
	\end{aligned}$ \\
	Vậy nghiệm của bất phương trình là $x \leq 12$.
	\item Ta có: $\begin{aligned}[t]
	-2x+3 &\leq 0& &\\ 
	-2x & \leq -3 & & \text { (cộng hai vế với } -3) \\
	(-2x )\cdot \left(-\dfrac{1}{2}\right) & \geq (-3) \cdot \left(-\dfrac{1}{2}\right) & & \text { (nhân hai vế với } -\dfrac{1}{2}) \\
	x & \geq \dfrac{3}{2}. & &
	\end{aligned}$ \\
	Vậy nghiệm của bất phương trình là $x \leq 12$.
	\item 
	Ta có: $\begin{aligned}[t]
	-2x - 4 &> 0 &&\\
	-2x &> 0 + 4 &&\text {(cộng hai vế của bất phương trình với } 4)\\
	x &< 4 \cdot\left(-\dfrac{1}{2}\right) && \text {(nhân hai vế với số âm } -\dfrac{1}{2} \text {(và đổi chiều bất đẳng thức)}\\
	x &< -2. &&
	\end{aligned}$\\
	Vậy nghiệm của bất phương trình là $x < -2$.
	\item Ta có: $\begin{aligned}[t]
	5x-3 	& < 0& &\\ 
	5x 	& < 3 & & \text { (cộng hai vế với } 3) \\
	5x \cdot \dfrac{1}{5} & < 3 \cdot \dfrac{1}{5} & & \text { (nhân hai vế với } \dfrac{1}{5}) \\
	x & < \dfrac{3}{5}. & &
	\end{aligned}$ \\
	Vậy nghiệm của bất phương trình là $x < \dfrac{3}{5}$.
	\item Ta có: $\begin{aligned}[t]
	-6x-2 &\geq 0& &\\ 
	-6x & \geq 2 & & \text { (cộng hai vế với } 2) \\
	(-6x )\cdot \left(-\dfrac{1}{6}\right) & \leq 2 \cdot \left(-\dfrac{1}{6}\right) & & \text { (nhân hai vế với } -\dfrac{1}{6}) \\
	x & \leq \dfrac{1}{3}. & &
	\end{aligned}$ \\
	Vậy nghiệm của bất phương trình là $x \leq \dfrac{1}{3}$.
	\end{enumerate}
	}
\end{vd}
%%==========Ví dụ 10
\begin{vd}
	Giải các bất phương trình:
	\begin{listEX}[4]
	\item $6x + 5 < 0$;
	\item $-2x - 7 > 0$;
	\item $2x + 5 < 3x - 4$;
	\item $-3x + 5 \geq -4x + 3$;
	\item $5x + 7 > 8x - 5$;
	\item $-4x + 3 \leq 3x - 1$;
	\item $-0{,}3x+12>0$.
	\item $\dfrac{3}{4}x-6\leq 0$.
	\end{listEX}
	\loigiai{
	\begin{listEX}[2]
	\item 
	Ta có $\begin{aligned}[t]
	& 6x + 5 < 0 \\
	& 6x < -5 \\
	& x < -\dfrac{5}{6}.
	\end{aligned}$\\
	Vậy nghiệm của bất phương trình là $x < -\dfrac{5}{6}$.
	\item 
	Ta có $\begin{aligned}[t]
	& -2x - 7 > 0 \\
	& -2x > 7 \\
	& x < -\dfrac{7}{2}.
	\end{aligned}$\\
	Vậy nghiệm của bất phương trình là $x < -\dfrac{7}{2}$.
	\item 
	Ta có 
	$\begin{aligned}[t]
	&2x + 5 < 3x - 4 \\
	& 2x - 3x < -4 -5 \\
	& -x < -9 \\
	& x > 9.
	\end{aligned}$\\
	Vậy nghiệm của bất phương trình là $x > 9$.
	\item 
	Ta có
	$\begin{aligned}[t]
	& -3x + 5 \geq -4x + 3 \\
	& -3x + 4x \geq 3 - 5 \\
	& x \geq -2.
	\end{aligned}$\\
	Vậy nghiệm của bất phương trình là $x \geq -2$.
	\item 
	Ta có $\begin{aligned}[t] & 5x + 7 > 8x - 5\\ & 5x-8x > -5-7\\&-3x>-12\\&x<4.\end{aligned}$\\
	Vậy nghiệm của bất phương trình là $x<4$.
	\item 
	Ta có $\begin{aligned}[t] & -4x + 3 \leq 3x - 1\\ & -4x-3x \leq -1-3\\& -7x \leq -4\\& x \geq \dfrac{4}{7}.\end{aligned}$\\
	Vậy nghiệm của bất phương trình là $x \geq \dfrac{4}{7}$.
	\item 
	Ta có $\begin{aligned}[t]
	-0{,}3x+12&>0\\
	-0{,}3x&>-12\\
	x&<\dfrac{-12}{-0{,}3}\\
	x&<40.
	\end{aligned}$\\
	Vậy nghiệm của bất phương trình là $x<40$.\\
	\item 
	Ta có $\begin{aligned}[t]
	\dfrac{3}{4}x-6&\leq0\\
	\dfrac{3}{4}x&\leq6\\
	x&\leq6\cdot\dfrac{4}{3}\\
	x&\leq8.
	\end{aligned}$\\
	Vậy nghiệm của bất phương trình là $x\leq8.$
	\end{listEX}
	}
\end{vd}
%%==========Ví dụ 11
\begin{vd}
	Giải các bất phương trình:
	\begin{listEX}[2]
	\item $-8x-27<0$;
	\item $\dfrac{5}{4}x+20\geq 0$;
	\item $2x-5 \leq 4x+3$;
	\item $5+7x>4x-7$;
	\item $3x-\left(6+2x\right)\leq3\left(x+4\right)$;
	\item $2\left(x-0{,}5\right)-1{,}4\geq1{,}5-\left(x+1{,}2\right).$
	\end{listEX}
	\loigiai{	
	\begin{listEX}[2]
	\item 
	Ta có:
	$\begin{aligned}[t]
	-8x-27&<0\\
	-8x&<27\\
	x&>\dfrac{-27}{8}.
	\end{aligned}$\\
	Vậy nghiệm của bất phương trình là $x>\dfrac{-27}{8}$.
	\item 
	Ta có:
	$\begin{aligned}[t]
	\dfrac{5}{4}x+20&\geq0\\
	\dfrac{5}{4}x&\geq-20\\
	x&\geq-20\cdot\dfrac{4}{5}\\
	x&\geq-16.
	\end{aligned}$\\
	Vậy nghiệm của bất phương trình là $x\geq-16.$
	\item
	Ta có:
	$\begin{aligned}[t]
	2x -5 	& \leq 4x+3 \\
	-5 - 3 	& \leq 4x - 2x \\
	-8	& \leq 2x \\
	-4	& \leq x.
	\end{aligned}$\\
	Vậy nghiệm của bất phương trình là $x \geq -4$
	\item
	Ta có: 
	$\begin{aligned}[t]
	5+7x 	& > 4x-7 \\
	7x-4x	& > -7-5 \\
	3x	& > -12 \\
	x	& > -4.
	\end{aligned}$\\
	Vậy nghiệm của bất phương trình là $x > -4$.
	\item
	Ta có: 
	$\begin{aligned}[t]
	3x-\left(6+2x\right)&\leq3\left(x+4\right)\\
	3x-6-2x&\leq3x+12\\
	x-6&\leq3x+12\\
	-6-12&\leq3x-x\\
	-18&\leq2x\\
	x&\geq-9.
	\end{aligned}$\\
	Vậy nghiệm của bất phương trình là $x\geq-9$.
	\item
	Ta có:
	$\begin{aligned}[t]
	2\left(x-0{,}5\right)-1{,}4&\geq 1{,}5-\left(x+1{,}2\right)\\
	2x-1-1{,}4&\geq 1{,}5-x-1{,}2\\
	2x-1{,}5&\geq 0{,}3-x\\
	2x+x&\geq 0{,}3+1{,}5\\
	3x&\geq 1{,}8\\
	x&\geq 0{,}6.
	\end{aligned}$\\
	Vậy nghiệm của bất phương trình là $x\geq0.6$
	\end{listEX}
	}
\end{vd}
%%==========Ví dụ 12
\begin{vd}%[8D4K4]
	Giải các bất phương trình sau:
	\begin{enumEX}{2}
	\item $\dfrac{2x-5}{18}<\dfrac{4x+3}{10}$;
	\item $\dfrac{4x-1}{9}<\dfrac{5-3x}{6}$.
	\end{enumEX}
	\loigiai{
	\begin{enumEX}{2}
	\item Ta có
	$\begin{aligned}[t]
	\dfrac{2x-5}{18}&<\dfrac{4x+3}{10}\\
	\dfrac{5(2x-5)}{90}&<\dfrac{9(4x+3)}{90}\\
	10x-25&<36x+27 \\
	10x-36x&<27+25 \\
	-26x&<52\\
	x&>\dfrac{52}{-26}\\
	x&>-2.
	\end{aligned}$\\
	Vậy nghiệm của bất phương trình là $x>-2$.
	\item Ta có
	$\begin{aligned}[t]
	\dfrac{4x-1}{9}&<\dfrac{5-3x}{6}\\
	\dfrac{2(4x-1)}{18}&<\dfrac{3(5-3x)}{18}\\
	8x-2&<15-9x\\
	8x+9x&<15+2\\
	17x&<17\\
	x&<1.
	\end{aligned}$\\
	Vậy nghiệm của bất phương trình là $x<1$.
	\end{enumEX}
	}
\end{vd}
%%==========Ví dụ 13
\begin{vd}%[8D4K4]
	Giải các bất phương trình sau
	\begin{enumEX}{2}
	\item $\dfrac{5x+2}{5}<\dfrac{4x-3}{4}$;
	\item $\dfrac{3(2x+1)}{20}+1<\dfrac{3x+13}{10}$.
	\end{enumEX}
	\loigiai{
	\begin{enumEX}{2}
	\item Ta có 
	$\begin{aligned}[t]
	\dfrac{5x+2}{5}&<\dfrac{4x-3}{4}\\
	\dfrac{4(5x+2)}{20}&<\dfrac{5(4x-3)}{20}\\
	20x+8&<20x-15\\
	20x-20x&<-8-15\\
	0x&<-23.
	\end{aligned}$\\
	Bất phương trình này vô nghiệm.
	\item Ta có
	$\begin{aligned}[t]
	\dfrac{3(2x+1)}{20}+1&<\dfrac{3x+13}{10}\\
	\dfrac{3(2x+1)}{20}+\dfrac{20}{20}&<\dfrac{2(3x+13)}{20}\\
	6x+3+20&<6x+26 \\
	6x-6x&<26-3-20\\
	0x&<3.
	\end{aligned}$\\
	Bất phương trình này có nghiệm bất kì.
	\end{enumEX}
	}
\end{vd}
%%==========Ví dụ 14
\begin{vd}%[8D4K4]
	Tìm nghiệm chung của hai bất phương trình:
	$$\dfrac{3x+17}{10}>\dfrac{5x+22}{15} \;(1) \text{ và } \dfrac{x-4}{30}-1>\dfrac{2x-7}{24} \;(2)$$
	\loigiai{
	\begin{listEX}[2]
	\item[]
	Ta có
	$\begin{aligned}[t]
	\dfrac{3x+17}{10}&>\dfrac{5x+22}{15}\\
	\dfrac{3(3x+17)}{30}&>\dfrac{2(5x+22)}{30} \\
	9x+51&>10x+44\\
	9x-10x&>44-51\\
	-x&>-7\\
	x&<7.\;(*)
	\end{aligned}$
	\item[]
	Ta có
	$\begin{aligned}[t]
	\dfrac{x-4}{30}-1&>\dfrac{2x-7}{24}\\
	\dfrac{4(x-4)}{120}-\dfrac{120}{120}&>\dfrac{5(2x-7)}{120}\\
	4x-16-120&>10x-35 \\
	4x-10x&>16+120-35 \\
	-6x&>101\\
	x&<-\dfrac{101}{6}.\;(**)
	\end{aligned}$\\
	Từ (*) và (**) suy ra nghiệm chung của hai bất phương trình là $x<-\dfrac{101}{6}$.
	\end{listEX}
	}
\end{vd}
%%==========Ví dụ 15
\begin{vd}%[8D4K4]
	Tìm nghiệm nguyên âm của bất phương trình
	$$\dfrac{2x+4}{3}-\dfrac{4x-7}{18}>\dfrac{2x-5}{9}-\dfrac{2x-1}{15}.$$
	\loigiai{
	Ta có 
	$$\begin{aligned}[t]
	\dfrac{2x+4}{3}-\dfrac{4x-7}{18}&>\dfrac{2x-5}{9}-\dfrac{2x-1}{15}\\
	\dfrac{30(2x+4)}{90}-\dfrac{5(4x-7)}{90}&>\dfrac{10(2x-5)}{90}-\dfrac{6(2x-1)}{90} \\
	60x+120-20x+35&>20x-50-12x+6\\
	60x-20x-20x+12x&>-120-35-50+6\\
	32x&>-199\\
	x&>\dfrac{-199}{32}.
	\end{aligned}$$
	Vì $x$ là số nguyên âm nên $x\in \lbrace -6; -5; -4; -3; -2; -1\rbrace$.
	}
\end{vd}
%=====================
\begin{dang}{Giải bài toán bằng cách lập bất phương trình}
\end{dang}
%%==========Ví dụ 16
\begin{vd}
	Bạn Thanh có $100$ nghìn đồng. Bạn muốn mua một cái bút giá $18$ nghìn đồng và một số quyển vở, mỗi quyển vở giá $7$ nghìn đồng. Hỏi bạn Thanh mua được nhiều nhất bao nhiêu quyển vở?
	\loigiai{
	Gọi $x$ (quyển) là số vở mà Thanh có thể mua. Theo bài ra, ta có bất phương trình:
	\[\begin{aligned}
	7x + 18 &\leq 100 \\
	7x &\leq 100-18 \\
	7x &\leq 82 \\
	x &\leq \dfrac{82}{7}.
	\end{aligned}\]
	Vì số vở là số tự nhiên nên Thanh có thể mua nhiều nhất $11$ quyển vở.
	}
\end{vd}
%%==========Ví dụ 17
\begin{vd}
	Để hưởng ứng phong trào ``Trồng cây gây rừng'', lớp $9$A có kế hoạch trồng ít nhất $1\,000$ cây xanh. Lớp $9$A đã trồng được $540$ cây. Để đạt được kế hoạch đề ra, lớp $9$A cần trồng thêm ít nhất bao nhiêu cây xanh nữa?
	\loigiai{
	Gọi $x$ là số cây xanh cần trồng thêm của lớp $9$A. \\
	Theo đề bài, để lớp $9$A đạt được kế hoạch đề ra, ta phải có 
	$$\begin{aligned}
	x+540 	& \geq 1\,000 \\
	x & \geq 1\,000 - 540\\
	x & \geq 460.
	\end{aligned}$$
	Vậy để đạt được kế hoạch đề ra, lớp $9$A phải trồng thêm ít nhất $460$ cây xanh nữa.
	}
\end{vd}
%%==========Ví dụ 18
\begin{vd}
	Trong một kì thi gồm ba môn Toán, Ngữ văn và Tiếng Anh, điểm số môn Toán và Ngữ văn tính theo hệ số $2$, điểm số môn Tiếng Anh tính theo hệ số $1$. Để trúng tuyển, điểm số trung bình của ba môn ít nhất phải bằng $8$. Bạn Na đã đạt $9{,}1$ điểm môn Toán và $6{,}9$ điểm môn Ngữ văn. Hãy lập và giải bất phương trình để tìm điểm số Tiếng Anh tối thiểu mà bạn Na phải đạt để trúng tuyển.
	\loigiai{
	Gọi $x$ là điểm số môn Tiếng Anh của bạn Na.\\
	Theo đề bài, để bạn Na trúng tuyển, ta phải có
	$$\begin{aligned}
	\dfrac{2 \cdot 9{,}1 + 2 \cdot 6{,}9 + x}{5} 	& \geq 8 \\
	2 \cdot 9{,}1 + 2 \cdot 6{,}9 + x 	& \geq 40 \\
	18{,}2 + 13{,}8 + x 	& \geq 40 \\
	x 	& \geq 8.
	\end{aligned}$$
	Vậy để trúng tuyển, bạn Na phải đạt ít nhất $8$ điểm môn Tiếng Anh.
	}
\end{vd}
%%==========Ví dụ 19
\begin{vd}
	Một ngân hàng đang áp dụng lãi suất gửi tiết kiệm kì hạn $12$ tháng là $7{,}4\%$/năm. Bà Mai dự kiến gửi một khoản tiền vào ngân hàng này và cần số tiền lãi hằng năm ít nhất là $60$ triệu để chi tiêu. Hỏi số tiền bà Mai cần gửi tiết kiệm ít nhất là bao nhiêu (làm tròn đến triệu đồng)?
	\loigiai{
	Gọi $x$ (triệu đồng) là số tiền bà Mai cần gửi tiết kiệm.\\
	Ta có số tiền lại gửi tiết kiệm $x$ (triệu đồng) trong một năm là $0{,}074 \cdot x$ (triệu đồng).\\
	Để có số tiền lãi ít nhất là 60 triệu đồng/năm thì ta phải có:
	\[\begin{aligned}[t]
	0{,}074x &\geq 60 \\
	x &\geq 60: 0{,}074 \\
	x &\geq 810{,}81.
	\end{aligned}\]
	Vậy bà Mai cần gửi ngân hàng ít nhất $811$ triệu đồng.
	}
\end{vd}
%%==========Ví dụ 20
\begin{vd}
	Trong một cuộc thi tuyển dụng việc làm, ban tổ chức quy định mỗi người ứng tuyển phải trả lời $25$ câu hỏi ở vòng sơ tuyển. Mỗi câu hỏi này có sẵn bốn đáp án, trong đó chỉ có một đáp án đúng. Người ứng tuyển chọn đáp án đúng sẽ được cộng thêm $2$ điểm, chọn đáp án sai bị trừ đi $1$ điểm. Ở vòng sơ tuyển, ban tổ chức tặng cho mỗi người dự thi $5$ điểm và theo quy định người ứng tuyển phải trả lời hết $25$ câu hỏi; người nào có số điểm từ $25$ trở lên mới được dự thi vòng tiếp theo. Hỏi người ứng tuyển phải trả lời chính xác ít nhất bao nhiêu câu hỏi ở vòng sơ tuyển thì mới được vào vòng tiếp theo?
	\loigiai{
	Gọi $x$, $25-x$ ($x \in \mathbb{N}$, $x \leq 25$) lần lượt là số câu trả lời đúng và sai của người ứng tuyển.\\
	Số điểm của người ứng tuyển sau $25$ câu hỏi là $5+2x-(25-x)=3x-20$ điểm.\\
	Để vượt qua vòng sơ tuyển cần ít nhất $25$ điểm nên ta có bất phương trình:\\
	\[\begin{aligned}[t] 
	& 3x-20 \geq 25\\ 
	& 3x \geq 45\\
	& x \geq 15.\\
	\end{aligned}\]
	Vậy người ứng tuyển phải trả lời chính xác ít nhất $15$ câu hỏi.
	}
\end{vd}
%%==========Ví dụ 21
\begin{vd}
	Bác Ngọc gửi tiền tiết kiệm kì hạn 12 tháng ở một ngân hàng với lãi suất $7{,}2$\%/năm. Bác Ngọc dự định tổng số tiền nhận được sau khi gửi $12$ tháng ít nhất là $21440000$ đồng. Hỏi bác Ngọc phải gửi số tiền tiết kiệm ít nhất là bao nhiêu để đạt được dự định đó?
	\loigiai{
	Giả sử bác Ngọc gửi $x$ (đồng) tiền tiết kiệm kì hạn $12$ tháng $(x>0)$ ). Khi đó, tổng số tiền bác Ngọc nhận được sau khi gửi $12$ tháng là
	$$
	x+7{,}2 \% \cdot x=\left(1+\frac{7{,}2}{100}\right) x=\frac{1072}{1000} x=\frac{134}{125} x \text { (đồng). }
	$$
	Theo giả thiết, ta có $\dfrac{134}{125} x \geq 21440000$.\\
	Giải bất phương trình trên, ta có
	$$
	\begin{aligned}
	\frac{134}{125} x &\geq 21440000 \\
	x &\geq 21440000 \cdot \frac{125}{134}\\
	x &\geq 20000000
	\end{aligned}
	$$
	Vậy bác Ngọc phải gửi số tiền tiết kiệm ít nhất là $20$ triệu đồng để đạt được dự định.
	}
\end{vd}
%%==========Ví dụ 22
\begin{vd}
	Tổng chi phí của một doanh nghiệp sản xuất áo sơ mi là $410$ triệu đồng/tháng. Giá bán của mỗi chiếc áo sơ mi là $350$ nghìn đồng. Hỏi trung bình mỗi tháng doanh nghiệp phải bán được ít nhất bao nhiêu chiếc áo sơ mi để thu được lợi nhuận ít nhất là $1{,}38$ tỉ đồng sau $1$ năm?
	\loigiai{
	Giả sử trung bình mỗi tháng doanh nghiệp bán được $x$ chiếc áo sơ mi $\left(x \in \mathbb{N}^{*}\right)$.\\	
	Lợi nhuận của doanh nghiệp sau $12$ tháng là	
	\[12 (350000x-410000 000)\ (\text{đồng}). \]	
	Do đó, để doanh nghiệp thu được lợi nhuận ít nhất là $1{,}38$ tỉ đồng thì	
	\[12 (350000 x-410000000) \geq 1380000000. \]	
	Giải bất phương trình trên, ta có
	$$
	\begin{aligned}
	12 (350000 x-410000000) & \geq 1380000000 \\
	350000 x-410000000 & \geq 115000000 \\
	350000 x & \geq 115000000+410000000 \\
	350000 x & \geq 525000000 \\
	x & \geq \frac{525000000}{350000} \\
	x & \geq 1500.
	\end{aligned}
	$$
	Vậy trung bình mỗi tháng doanh nghiệp phải bán được ít nhất $1500$ chiếc áo sơ mi để doanh nghiệp thu được lợi nhuận ít nhất là $1{,}38$ tỉ đồng sau $1$ năm.
	}
\end{vd}
%%%%%%%%%%%%%%%%%%%
\subsection{Bài tập vận dụng}
%%==========Bài 1
\begin{bt}
	Bất phương trình nào sau đây là bất phương trình bậc nhất một ẩn?
	\begin{listEX}[4]
	\item $2 x-5>0$;
	\item $3 y+1 \geq 0$;
	\item $0 x-3<0$;
	\item $x^{2}>0$.
	\end{listEX}
	\loigiai{
	\begin{listEX}[1]
	\item $2 x-5>0$ là bất phương trình bậc nhất ẩn $x$.
	\item $3 y+1 \geq 0$ là bất phương trình bậc nhất ẩn $y$.
	\item $0 x-3<0$, vì $a=0$ nên không phải là bất phương trình bậc nhất một ẩn.
	\item $x^{2}>0$, vì có chứa $x^2$ nên không phải là bất phương trình bậc nhất một ẩn.
	\end{listEX}
	}
\end{bt}
%%==========Bài 2
\begin{bt}
	Bất phương trình nào sau đây là bất phương trình bậc nhất một ẩn?
	\begin{listEX}[3]
	\item $3x-6>0$.
	\item $-13x+20<0$.
	\item $7y\geq0$.
	\end{listEX}
	\loigiai{
	Bất phương trình ở các câu $a,b,c$ là bất phương trình bậc nhất một ẩn. 
	}
\end{bt}
%%==========Bài 3
\begin{bt}
	Kiểm tra xem số nào là nghiệm của mỗi bất phương trình tương ứng sau đây.
	\begin{enumEX}{2}
	\item $x^{2}-3 x+2>0$ với $x=-3 ; x=1{,}5$.
	\item $2-2 x<3 x+1$ với $x=\dfrac{2}{5} ; x=\dfrac{1}{5}$.
	\end{enumEX}
	\loigiai{
	\begin{enumEX}{1}
	\item Thay $x=-3$, ta có $(-3)^{2}-3 (-3)+2>0$ là khẳng định đúng. \\
	Vậy $x=-3$ là nghiệm của bất phương trình. \\
	Thay $x=-1{,}5$, ta có $(-1{,}5)^{2}-3 (-1{,}5)+2>0$ là khẳng định không đúng. \\
	Vậy $x=-1{,}5$ không là nghiệm của bất phương trình. 
	\item Thay $x=\dfrac{2}{5} $, ta có $2-2 \cdot \dfrac{2}{5} <3 \cdot \dfrac{2}{5} +1$ là khẳng định đúng. \\
	Vậy $x=\dfrac{2}{5} $ là nghiệm của bất phương trình. \\
	Thay $x=\dfrac{1}{5}$, ta có $2-2 \cdot \dfrac{1}{5} <3 \cdot \dfrac{1}{5} +1$ là khẳng định không đúng. \\
	Vậy $x=\dfrac{1}{5}$ không là nghiệm của bất phương trình. 
	\end{enumEX}
	}
\end{bt}
%%==========Bài 4
\begin{bt}
	Tìm $x$ sao cho:
	\begin{listEX}[2]
	\item Giá trị của biểu thức $2 x+1$ là số dương;
	\item Giá trị của biểu thức $3 x-5$ là số âm.
	\end{listEX}
	\loigiai{
	\begin{listEX}[1]
	\item Giá trị của biểu thức $2 x+1$ là số dương nên:
	$$\begin{aligned}[t]
	2x + 1 & > 0	\\
	2x & > -1	&&(\text{cộng } -1 \text{ vào cả hai vế}) \\
	x & > \dfrac{-1}{2}	&&(\text{nhân } \dfrac{1}{2} \text{ vào cả hai vế})
	\end{aligned}$$
	Vậy $x > \dfrac{-1}{2}$.
	\item Giá trị của biểu thức $3 x-5$ là số âm nên:
	$\begin{aligned}[t]
	3x-5&<0	\\
	3x &<5	&&(\text{cộng } 5 \text{ vào cả hai vế}) \\
	x &<\dfrac{5}{3} &&(\text{nhân } \dfrac{1}{3} \text{ vào cả hai vế}) 
	\end{aligned}$\\
	Vậy $x <\dfrac{5}{3}$.
	\end{listEX}
	}
\end{bt}
%%==========Bài 5
\begin{bt}
	Giải các bất phương trình sau:
	\begin{listEX}[4]
	\item $x - 5 \geq 0$;
	\item $x + 5\leq 0$;
	\item $-2x - 6 > 0$;
	\item $4x - 12 < 0$.
	\end{listEX}
	\loigiai{
	\begin{listEX}[4]
	\item $\begin{aligned}[t] x - 5&\geq 0\\ x &\geq 5.\end{aligned}$
	\item $\begin{aligned}[t] x + 5&\leq 0\\ x &\leq -5.\end{aligned}$
	\item $\begin{aligned}[t] -2x - 6&\geq 0\\ -2x &> 6 \\ x&<-3.\end{aligned}$
	\item $\begin{aligned}[t] 4x - 12&< 0\\ 4x &< 12 \\x&<3. \end{aligned}$
	\end{listEX}
	}
\end{bt}
%%==========Bài 6
\begin{bt}
	Giải các bất phương trình sau:
	\begin{listEX}[4]
	\item $6<x-3$;
	\item $\dfrac {1}{2}\cdot x>5$;
	\item $-8 x+1 \geq 5$;
	\item $7<2 x+1$.
	\end{listEX}
	\loigiai{
	\begin{listEX}[1]
	\item Gải bất phương trình:
	$\begin{aligned}[t]
	6&<x-3	\\
	6+3&<x	&&(\text{cộng } 3 \text{ vào cả hai vế}) \\
	9&<x.
	\end{aligned}$\\
	Vậy nghiệm bất phương trình là $x>9$.
	\item Gải bất phương trình:
	$\begin{aligned}[t]
	\dfrac {1}{2}\cdot x	&>5	 \\
	x&>10 	&&(\text{nhân } 2 \text{ vào cả hai vế})
	\end{aligned}$\\
	Vậy nghiệm của bất phương trình là $x>10$.
	\item Gải bất phương trình:
	$\begin{aligned}[t]
	-8x+1 & \geq 5	\\
	-8x & \geq 4 	&&(\text{cộng } -1 \text{ vào cả hai vế}) \\
	x &\leq \dfrac{4}{-8}	&&(\text{nhân } \dfrac{-1}{8} \text{ vào cả hai vế}) \\
	x & \leq \dfrac{-1}{2} &&(\text{nhân số âm nên đảo chiều } 
	\end{aligned}$\\
	Vậy nghiệm bất phương trình là $x\leq \dfrac{-1}{2}$.
	\item Gải bất phương trình:
	$\begin{aligned}[t]
	7&<2x + 1	\\	
	6 &<2x	&&(\text{cộng } -1 \text{ vào cả hai vế}) \\
	\dfrac{6}{2}&<x	&&(\text{nhân } \dfrac{1}{2} \text{ vào cả hai vế}) \\
	3&<x.
	\end{aligned}$\\
	Vậy nghiệm bất phương trình là $x>3$.
	\end{listEX}
	}
\end{bt}
%%==========Bài 7
\begin{bt}
	Giải các bất phương trình sau:
	\begin{listEX}[4]
	\item $x-7<2-x$;
	\item $x+2 \leq 2+3 x$
	\item $4+x>5-3 x$;
	\item $-x+7 \geq x-3$.
	\end{listEX}
	\loigiai{
	\begin{listEX}[1]
	\item Giải các bất phương trình:
	$\begin{aligned}[t]
	x - 7 &< 2 - x \\
	x &< 9 - x &&\text{(cộng } 7 \text{ vào cả hai vế)} \\
	2x &< 9 &&\text{(cộng } x \text{ vào cả hai vế)} \\
	x &< \dfrac{9}{2} & &\text{(nhân } \dfrac{1}{2} \text{ vào cả hai vế)}
	\end{aligned}$\\
	Nghiệm của bất phương trình là $x < \dfrac{9}{2}$.
	\item Giải các bất phương trình:
	$\begin{aligned}[t]
	x + 2 &\leq 2 + 3x \\
	x &\leq 3x &&\text{(cộng } -2 \text{ vào cả hai vế)}\\
	0 &\leq 2x &&\text{(cộng } -x \text{ vào cả hai vế)} \\
	0 &\leq x &&\text{(nhân } \dfrac{1}{2} \text{ vào cả hai vế)}
	\end{aligned}$\\
	Nghiệm của bất phương trình là $x \geq 0$.
	\item Giải các bất phương trình:
	$\begin{aligned}[t]
	4 + x &> 5 - 3x \\
	4x &> 1 && \text{(cộng } 3x \text{ và } -4 \text{ từ cả hai bên)} \\
	x &> \dfrac{1}{4} && \text{(nhân cho hai vế cho } \dfrac{1}{4})
	\end{aligned}$\\
	Nghiệm của bất phương trình là $x > \dfrac{1}{4}$.
	\item Giải các bất phương trình:
	$\begin{aligned}[t]
	-x + 7 &\geq x - 3 \\
	7 &\geq 2x - 3 &&(\text{cộng } x \text{ vào cả hai vế}) \\
	10 &\geq 2x &&(\text{cộng } 3 \text{ vào cả hai vế}) \\
	5 &\geq x 	&&(\text{nhân cho } \dfrac{1}{2} \text{ vào cả hai vế})
	\end{aligned}$\\
	Vậy nghiệm của bất phương trình là $x \leq 5$.
	\end{listEX}
	}
\end{bt}
%%==========Bài 8
\begin{bt}
	Giải các bất phương trình sau:
	\begin{listEX}[4]
	\item $3x + 2 > 2x + 3$;
	\item $5x + 4 < -3x - 2$;
	\item $\dfrac {2}{3}(2 x+3)<7-4 x$;
	\item $\dfrac {1}{4}(x-3) \leq 3-2 x$.
	\end{listEX}
	\loigiai{
	\begin{listEX}[1]
	\item Giải bất phương trình:
	$\begin{aligned}[t] 
	3x + 2 &> 2x + 3 \\	
	3x-2x &> 3-2 &&(\text{cộng } -2x \text{ vào cả hai vế})\\ 
	x&>1.
	\end{aligned}$\\
	Vậy nghiệm của bất phương trình là $x>1$.
	\item Giải bất phương trình:
	$\begin{aligned}[t] 
	5x + 4 &< -3x - 2\\	
	5x+3x &< -2-4 &&(\text{cộng } 3x \text{ vào cả hai vế})\\
	8x&<-6 \\ 
	x&<-\dfrac{3}{4} &&(\text{nhân } \dfrac{1}{8} \text{ vào cả hai vế}).
	\end{aligned}$\\
	Vậy nghiệm của bất phương trình là $x<-\dfrac{3}{4}$.
	\item Giải bất phương trình:
	$\begin{aligned}[t]
	\dfrac{2}{3}(2x+3) &< 7-4x \\
	2x + 3 &< \dfrac{21}{2} - 6x &&(\text{nhân } \dfrac{3}{2} \text{ vào cả hai vế}) \\
	8x + 3 &< \dfrac{21}{2} &&(\text{cộng } 6x \text{ vào cả hai vế}) \\
	8x &< \dfrac{21}{2} -3 &&(\text{cộng } -3 \text{ vào cả hai vế}) \\
	8x &< \dfrac{15}{2}	&&(\text{cộng } 3 \text{ vào cả hai vế}) \\
	x &< \dfrac{15}{16} 	&&(\text{nhân } \dfrac{1}{8} \text{ vào cả hai vế})
	\end{aligned}$\\
	Vậy nghiệm của bất phương trình là $x < \dfrac{15}{16}$.
	\item Giải bất phương trình:
	$\begin{aligned}[t]
	\dfrac{1}{4}(x-3)& \leq 3-2x \\
	(x-3) &\leq 4(3 - 2x)	&&(\text{nhân } 4 \text{ vào cả hai vế}) \\
	x-3 &\leq 12 - 8x	&&(\text{nhân phân phối } 4 \text{ và mở ngoặc}) \\
	9x - 3&\leq 12	&&(\text{cộng } 8x \text{ vào cả hai vế}) \\
	9x &\leq 15	&&(\text{cộng } 3 \text{ vào cả hai vế}) \\
	x &\leq \dfrac{5}{3}	&&(\text{nhân } \dfrac{1}{9} \text{ vào cả hai vế}) 
	\end{aligned}$\\
	Vậy, nghiệm của bất phương trình là $x \leq \dfrac{5}{3}$.
	\end{listEX}
	}
\end{bt}
%%==========Bài 9
\begin{bt}
	Giải các bất phương trình
	\begin{enumEX}{3}
	\item $2 x+6>1$;
	\item $0{,}6 x+2>6 x+9$;
	\item $1{,}7 x+4 \geq 2+1{,}5 x$;
	\item $\dfrac{8-3 x}{2}-x<5$;
	\item $3-2 x-\dfrac{6+4 x}{3}>0$;
	\item $0{,}7 x+\dfrac{2 x-4}{3}-\dfrac{x}{6}>1$.
	\end{enumEX}
	\loigiai{
	\begin{enumEX}{2}
	\item $\begin{aligned}[t]
	2x+6 & \geq 1\\
	2x & \geq 1-6 \\
	2x & \geq -5\\
	x & \geq -\dfrac{5}{2}.	
	\end{aligned}$\\
	Vậy nghiệm của bất phương trình là $x \geq -\dfrac{5}{2}$.
	\item $\begin{aligned}[t]
	0{,}6 x+2 & >6 x+9\\
	0{,}6 x -6x & >9-2\\
	-5{,}4x & >7\\
	x & < 7: (-5{,}4)\\
	x & <- \dfrac{35}{27}.
	\end{aligned}$\\
	Vậy nghiệm của bất phương trình là $x < -\dfrac{35}{27}$.	
	\item $\begin{aligned}[t]
	1{,}7 x+4 & \geq 2+1{,}5 x\\
	1{,}7 x-1{,}5 x & \geq 2-4\\
	0{,}2 x & \geq -2\\
	x & \geq -2:0{,}2\\
	x & \geq -10
	\end{aligned}$\\
	Vậy nghiệm của bất phương trình là $x\geq -10$.
	\item $\begin{aligned}[t]
	\dfrac{8-3 x}{2}-x& <5\\
	8-3x-2x & < 10 \\
	-5x & < 10-8 \\
	x > 2:(-5) \\
	x> -0{,}4.
	\end{aligned}$\\
	Vậy nghiệm của bất phương trình là $x> -0{,}4$.
	\item $\begin{aligned}[t]
	3-2 x-\dfrac{6+4 x}{3}& >0\\
	9-6x-(6+4x) & > 0 \\
	9-6x-6-4x & >0 \\
	-10x & > -3 \\
	x &< (-3):(-10) \\
	x&< -0{,}3.
	\end{aligned}$\\
	Vậy nghiệm của bất phương trình là $x<-0{,}3$.
	\item $\begin{aligned}[t]
	0{,}7 x+\dfrac{2 x-4}{3}-\dfrac{x}{6}& >1\\
	4{,}2 x+2(2x-4)-x& >6 \\
	4{,}2 x+4x-8-x& >6 \\
	4{,}2 x+4x-x& >6+8 \\
	7{,}2 x& >14 \\
	x& >14: 7{,}2 \\
	x& >\dfrac{35}{18}. 
	\end{aligned}$\\
	Vậy nghiệm của bất phương trình là $x>\dfrac{35}{18}$.
	\end{enumEX}
	}
\end{bt}
%%==========Bài 10
\begin{bt}%[8D4B3]
	Tìm số nguyên lớn nhất thỏa mãn mỗi bất phương trình sau
	\begin{enumEX}{2}
	\item $9-5x>1{,}5$;
	\item $\dfrac{3x-17}{20}>\dfrac{5x+1}{15}$.
	\end{enumEX}
	\loigiai{\begin{enumerate}
	\item 
	Ta có 
	$\begin{aligned}[t]
	9-5x&>1{,}5\\
	-5x&>-7{5}\\
	x&<\dfrac{7{,}5}{5}
	\end{aligned}$\\
	Do đó nguyên lớn nhất thỏa mãn bất phương trình trên là $ x=1$.
	\item 
	Ta có
	$\begin{aligned}[t]
	\dfrac{3x-17}{20}&>\dfrac{5x+1}{15}\\
	\dfrac{15(3x-17)}{300}&>\dfrac{20(5x+1)}{300} \\
	 45x-255&>100x+20\\
	 -55x&>275 \\
	 x&<-\dfrac{275}{55} \\
	 x&<-5.
	\end{aligned}$\\
	Do đó nguyên lớn nhất thỏa mãn bất phương trình trên là $x=-6$.
	\end{enumerate}
	}
\end{bt}
%%==========Bài 11
\begin{bt}%[8D4B4]
	Tìm nghiệm nguyên chung của hai bất phương trình
	\begin{enumEX}{2}
	\item $15x-4>8$ và $7-6x>-20$;
	\item $\dfrac{2}{3}x+5>9$ và $\dfrac{x-18}{7}>1$.
	\end{enumEX}
	\loigiai{
	\begin{enumerate}
	\item Ta có 
	$15x-4>8$
	hay
	$15x>12$, suy ra
	$x>\dfrac{12}{15}$\\
	Ta có
	$7-6x>-20$
	hay $-6x>-27$, suy ra
	$x<\dfrac{27}{6}.$\\
	Vậy nguyên chung của hai bất phương trình trên là $x\in \lbrace 1;2;3;4\rbrace$.
	\item 
	Ta có $\dfrac{2}{3}x+5>9$ hay $\dfrac{2}{3}x>4$, suy ra $x>6$;\\
	Ta có $\dfrac{x-18}{7}>1$ hay $x-18>7$, suy ra $x>25$.\\
	Vậy nguyên chung của hai bất phương trình trên là $x>25$.
	\end{enumerate}
	}
\end{bt}
%%==========Bài 12
\begin{bt}%[8D4B4]
	Tìm tập hợp các giá trị của $x$ để biểu thức $\dfrac{3-2x}{5}$ lớn hơn giá trị của biểu thức $\dfrac{x-14}{10}$.
	\loigiai{
	Giải bất phương trình 
	$\begin{aligned}[t]
	\dfrac{3-2x}{5}&>\dfrac{x-14}{10}\\
	 \dfrac{2(3-2x)}{10}&>\dfrac{x-14}{10}\\
	 6-4x&>x-14\\
	 20&>5x\\
	 x&<\dfrac{20}{5}\\
	 x&<4.
	\end{aligned}$\\
	Vậy $x<4$ là giá trị cần tìm.
	}
\end{bt}
%%==========Bài 13
\begin{bt}%[8D4B4]
	Cho phương trình $5x-4=3m+2$ (1) trong đó $x$ là ẩn số, $m$ là một số cho trước. Tìm giá trị của $m$ để phương trình (1) có nghiệm dương.
	\loigiai{
	Giải phương trình (1) theo $m$, ta được $x=\dfrac{3m+6}{5}$.
	Ta có
	$x>0$ khi
	$\dfrac{3m+6}{5}>0$, hay
	$3m>-6$, suy ra
	$m>-2$.\\
	Vậy $m>-2$ là giá trị cần tìm.
	}
\end{bt}
%%==========Bài 14
\begin{bt}%[8D4B4]
	Giải các bất phương trình sau
	\begin{enumEX}{2}
	\item $\dfrac{3(2x+1)}{20}+1>\dfrac{3x+52}{10}$;
	\item $\dfrac{4x-1}{2}+\dfrac{6x-19}{6}\leq \dfrac{9x-11}{3}$.
	\end{enumEX}
	\loigiai{
	\begin{listEX}[2]
	\item Ta có
	$\begin{aligned}[t]
	\dfrac{3(2x+1)}{20}+1&>\dfrac{3x+52}{10}\\
	 \dfrac{3(2x+1)}{20}+\dfrac{20}{20}&>\dfrac{2(3x+52)}{20}\\
	 6x+3+20&>6x+104\\
	 0x&>81.
	\end{aligned}$\\
	Vậy bất phương trình vô nghiệm.
	\item Ta có
	$\begin{aligned}[t]
	\dfrac{4x-1}{2}+\dfrac{6x-19}{6}&\leq \dfrac{9x-11}{3}\\
	 \dfrac{3(4x-1)}{6}+\dfrac{6x-19}{6}&\leq \dfrac{2(9x-11)}{6}\\
	 12x-3+6x-19&\leq 18x-22\\
	 0x&\leq 0.
	\end{aligned}$\\
	Bất phương trình có nghiệm bất kì.
	\end{listEX}
	}
\end{bt}
%%==========Bài 15
\begin{bt}
	Tìm $x>0$ sao cho ở hình vẽ bên dưới chu vi của hình tam giác luôn lớn hơn chu vi của hình chữ nhật
	\begin{center}
	\begin{tikzpicture}[line join=round, line cap=round, >=stealth,font=\footnotesize, scale=0.8]%Hinh nón
	\tikzset{every node/.style={scale=0.7}}% thu nhỏ phóng tỏ tex trong hình
	%\clip(-2,-1) rectangle (7,5);
	% set up coordinates for an easy use
	\coordinate (C) at (0,0);
	\coordinate (B) at (5,0);
	\coordinate (A) at (3,3);
	\coordinate (m) at ($(A)!0.5!(B)$);	
	\coordinate (n) at ($(A)!0.5!(C)$);
	\coordinate (p) at ($(B)!0.5!(C)$);
	\draw (A)--(B)--(C)--(A);	
	\draw[fill=black] 
	(A) circle (0.05) 
	(B) circle (0.05)
	(C) circle (0.05) 
	(m) node[right] {$x+2$}
	(n) node[left] {$x+4$}
	(p) node[below] {$x+5$}
	;	
	\end{tikzpicture} \hspace*{1cm} 
	\begin{tikzpicture}[line join=round, line cap=round, >=stealth,font=\footnotesize, scale=0.8]%Hinh nón
	\tikzset{every node/.style={scale=0.7}}% thu nhỏ phóng tỏ tex trong hình
	%\clip(-2,-1) rectangle (7,5);
	% set up coordinates for an easy use
	\coordinate (A) at (0,0);
	\coordinate (B) at (5,0);
	\coordinate (C) at (5,3);
	\coordinate (D) at (0,3);
	\coordinate (m) at ($(A)!0.5!(B)$);	
	\coordinate (n) at ($(B)!0.5!(C)$);
	\draw (A)--(B)--(C)--(D)--(A);	
	\draw[fill=black] 
	(A) circle (0.05) 
	(B) circle (0.05)
	(C) circle (0.05) 
	(D) circle (0.05) 
	(m) node[below] {$x+3$}
	(n) node[right] {$x+1$}	
	;	
	\end{tikzpicture}
	\end{center}
	\loigiai{
	Chu vi tam giác là $x+2+x+4+x+5= 3x+11$.\\
	Chu vi hình chữ nhật là $2(x+3+x+1)=4x+8$.\\
	Theo bài ra ta có 
	$$
	\begin{aligned}
	3x+11&> 4x+8\\
	3x-4x&> 8-11\\
	-x & > -3 \\
	x & < 3
	\end{aligned}
	$$
	Vậy $0<x<3$.
	}
\end{bt}
%%==========Bài 16
\begin{bt}
	Một ngân hàng đang áp dụng lãi suất gửi tiết kiệm kì hạn $1$ tháng là $0{,}4\%$. Hỏi nếu muốn có số tiền lãi hằng tháng ít nhất là $3$ triệu đồng thì số tiền gửi tiết kiệm ít nhất là bao nhiêu (làm tròn đến triệu đồng)?
	\loigiai{
	Gọi $x$ (triệu đồng) là số tiền cần gửi. Theo đề bài ta có bất phương trình:
	\[\begin{aligned}[t] 0{,}4\%x &\geq 3 \\ 
	\dfrac{0{,}4}{100} \cdot x &\geq 3 \\
	x &\geq \dfrac{3\cdot 1000}{4}\\x &\geq 750.
	\end{aligned}\]
	Vậy để có lãi ít nhất $3$ triệu/tháng thì số tiền gửi tiết kiệm ít nhất là $750$ triệu.
	}
\end{bt}
%%==========Bài 17
\begin{bt}
	Một hãng taxi có giá mở cửa là $15$ nghìn đồng và giá $12$ nghìn đồng cho mỗi kilômét tiếp theo. Hỏi với $200$ nghìn đồng thì hành khách có thể di chuyển được tối đa bao nhiêu kilômét (làm tròn đến hàng đơn vị)?
	\loigiai{
	Gọi $x$ km ($x>0$) là số km mà hành khách có thể đi, ta có bất phương trình:
	\[\begin{aligned}[t] 15+12x-12&\leq 200\\ 12x &\leq 200-3 \\x &\leq \dfrac{197}{12} = 16{,}416\ldots\end{aligned}\]
	Vậy số km tối đa khách hàng có thể đi là $16$ km.
	}
\end{bt}
%%==========Bài 18
\begin{bt}
	Người ta dùng một loại xe tải để chở bia cho một nhà máy. Mỗi thùng bia $24$ lon nặng trung bình $6{,}7$ kg. Theo khuyến nghị, trọng tải của xe (tức là tổng khối lượng tối đa cho phép mà xe có thể chở) là $5{,}25$ tấn. Hỏi xe có thể chở được tối đa bao nhiêu thùng bia, biết bác lái xe nặng $65$ kg?
	\loigiai{
	Gọi $x$ thùng ($x \in \mathbb{N}^*$) là số thùng bia mà xe có thể chở, ta có bất phương trình:
	\[\begin{aligned}[t] 65+6{,}7x &\leq 5{,}25 \cdot 1000 \\6{,}7x &\leq 5185 \\x &\leq \dfrac{5185}{6{,}7} = 773{,}880\ldots\end{aligned}\]
	Vậy số thùng bia tối đa mà xe có thể chở là $773$ thùng.
	}
\end{bt}
%%==========Bài 19
\begin{bt}
	Một kho chứa $100$ tấn xi măng, mỗi ngày đều xuất đi $20$ tấn xi măng. Gọi $x$ là số ngày xuất xi măng của kho đó. Tìm $x$ sao cho khối lượng xi măng còn lại trong kho ít nhất là $10$ tấn sau $x$ ngày xuất hàng.
	\loigiai{
	Sau $x$ ngày khối lượng xi măng được xuất đi là $20x$.\\
	Khối lượng xi măng còn lại trong kho sau $x$ ngày là $100-20x$.\\
	Theo bài ra ta có $100-20x \geq x$.
	$$
	\begin{aligned}
	100-20x& \geq x\\
	-20x -x &\geq -100 \\
	-21x & \geq -100 \\
	x & \leq (-100):(-21) \\
	x & \leq \dfrac{100}{21}
	\end{aligned}
	$$
	Vậy $0x \in \{1;2;3;4\}$.
	}
\end{bt}
%%==========Bài 20
\begin{bt}
	Một kì thi Tiếng Anh gồm bốn kĩ năng: nghe, nói, đọc và viết. Kết quả của bài thi là điểm số trung bình của bốn kĩ nảng này. Bạn Hà đã đạt được điểm số của ba kĩ năng nghe, đọc, viết lần lượt là $6,5;\, 6,5;\, 5,5$. Hỏi bạn Hà cần đạt bao nhiêu điểm trong kĩ năng nói để kết quả đạt được của bài thi ít nhất là $6,25$?
	\loigiai{
	\begin{itemize}
	\item Để tính điểm số trung bình cần đạt, chúng ta có thể sử dụng công thức sau:
	\[\text{Điểm trung bình}=\dfrac{\text{Tổng điểm các kĩ năng}}{\text{Số lượng kĩ năng}}\]
	\item Trong trường hợp này, chúng ta muốn biết điểm số cần đạt trong kĩ năng "nói" để đạt được điểm trung bình ít nhất là $6{,}25$. Điểm số trung bình mong muốn là trung bình của $6{,}5; 6{,}5; 5{,}5$ và một số $x$ (điểm số trong kĩ năng "nói"). Ta có bất phương trình:
	\begin{center}
	$\begin{aligned}[t]
	\dfrac{6{,}5 + 6{,}5 + 5{,}5 + x}{4} &\geq 6{,}25\\
	18{,}5 + x& \geq 25\\
	x &\geq 25 - 18{,}5\\
	x &\geq 6{,}5.
	\end{aligned}$
	\end{center}
	\item Vậy, để đạt được điểm trung bình ít nhất là $6{,}25$, bạn Hà cần đạt ít nhất $6{,}5$ điểm trong kĩ năng "nói".
	\end{itemize}
	}
\end{bt}