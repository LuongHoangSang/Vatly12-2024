\section*{ÔN TẬP CHƯƠNG I}
\subsection{Bài tập tự luận}
%%==========Bài 1
\begin{bt}
	Giải các phương trình sau:
	\begin{listEX}[2]
	\item $2(x+1)=(5 x-1)(x+1)$;
	\item $(-4 x+3) x=(2 x+5) x$.
	\end{listEX}
	\loigiai{%
	\begin{enumerate}
	\item \begin{eqnarray*}
	2(x+1)&=&(5 x-1)(x+1)\\
	2(x+1)-(5x-1)(x+1)&=&0\\
	(x+1)(2-5x+1)&=&0\\
	(x+1)(-5x+3)&=&0\\
	x+1=0~\text{hoặc}~-5x+3&=&0\\
	x=-1~\text{hoặc}~x&=&\dfrac{3}{5}.
	\end{eqnarray*}
	\item \begin{eqnarray*}
	(-4 x+3) x&=&(2 x+5) x\\
	(-4x+3)x-(2x+5)x&=&0\\
	x(-4x+3-2x-5)&=&0\\
	x(-6x-2)&=&0\\
	x=0~\text{hoặc}~-6x-2&=&0\\
	x=0~\text{hoặc}~x&=&-\dfrac{1}{3}.
	\end{eqnarray*}
	\end{enumerate}	
	}
\end{bt}
%%==========Bài 2
\begin{bt}
	Giải các phương trình sau
	\begin{listEX}[2]
	\item $(3x - 1)^2 - (x + 2)^2 = 0$.
	\item $x(x + 1) = 2(x^2 - 1)$.
	\end{listEX}
	\loigiai{
	\begin{listEX}[1]
	\item Ta có $$\begin{aligned}[t]
	(3x - 1)^2 - (x + 2)^2 &= 0\\
	\left[(3x-1) - (x+2)\right]\left[(3x-1) + (x+2)\right] &=0\\
	(2x-3)(4x+1) &= 0\\
	2x-3 = 0 \textrm{ hoặc } 4x + 1 &= 0.
	\end{aligned}$$
	Khi đó
	\begin{itemize}
	\item $2x-3 = 0 \Rightarrow 2x = 3 \Rightarrow x = \dfrac{3}{2}$.
	\item $4x + 1 = 0 \Rightarrow 4x = -1 \Rightarrow x = -\dfrac{1}{4}$.
	\end{itemize}
	Vậy phương trình đã cho có nghiệm là $x = \dfrac{3}{2}$, $x = -\dfrac{1}{4}$.
	\item Ta có 
	$$\begin{aligned}[t]
	x(x + 1) &= 2(x^2 - 1)\\
	x(x+1) &= 2(x-1)(x+1)\\
	(x+1)\left[x - 2(x-1)\right] &= 0\\
	(x+1)\left(-x + 2\right) &= 0\\
	x + 1 = 0\,\textrm{ hoặc } -x + 2 &= 0.
	\end{aligned}$$
	Khi đó
	\begin{itemize}
	\item $x + 1 = 0 \Rightarrow x = -1$.
	\item $-x + 2 = 0 \Rightarrow x = 2$.
	\end{itemize}
	Vậy phương trình đã cho có nghiệm là $x = -1$, $x = 2$.
	\end{listEX}
	}
\end{bt}
%%==========Bài 3
\begin{bt}
	Giải các phương trình:
	\begin{listEX}[2]
	\item $\left(5x+2\right)\left(2x-7\right)=0$.
	\item $\left(\dfrac{1}{2}x+5\right)\left(-\dfrac{2}{3}x-\dfrac{4}{3}\right)=0$.
	\item $y^2-5y+2\left(y-5\right)=0$.
	\item $9x^2-1=\left(3x-1\right)\left(2x+7\right)$.
	\end{listEX}
	\loigiai{
	\begin{enumerate}
	\item 
	Ta có
	$\left(5x+2\right)\left(2x-7\right)=0$
	nên
	$2x-7=0$ hoặc $5x+2=0$.
	\begin{itemize}
	\item $2x-7=0$ hay $2x=7$, suy ra $x=\dfrac{7}{2}$.
	\item $5x+2=0$ hay $5x=-2$, suy ra $x=-\dfrac{2}{5}$.
	\end{itemize}
	Vậy phương trình đã cho có hai nghiệm là $x=\dfrac{7}{2};x=-\dfrac{2}{5}$.
	\item
	Ta có $\left(\dfrac{1}{2}x+5\right)\left(-\dfrac{2}{3}x-\dfrac{4}{3}\right)=0$
	nên
	$\dfrac{1}{2}x+5=0$ hoặc $-\dfrac{2}{3}x-\dfrac{4}{3}=0$
	\begin{itemize}
	\item $\dfrac{1}{2}x+5=0$ hay $\dfrac{1}{2}x=-5$, suy ra $x=-10$.
	\item $-\dfrac{2}{3}x-\dfrac{4}{3}=0$ hay $-\dfrac{2}{3}x=\dfrac{4}{3}$, suy ra $x=-2$.
	\end{itemize}
	Vậy phương trình đã cho có hai nghiệm là $x=-2; x=-10$.
	\item Ta có
	\begin{align*}
	y^2-5y+2\left(y-5\right)&=0 \\
	y^2-3y-10&=0\\
	y^2+2y-5y-10&=0\\
	y(y+2)-5(y+2)&=0\\
	(y+2)(y-5)&=0\\
	y+2=0\,\text{hoặc }y-5&=0.
	\end{align*}
	\begin{itemize}
	\item $y+2=0$ suy ra $y=-2$.
	\item $y-5-0$ suy ra $y=5$.
	\end{itemize}
	Vậy phương trình đã cho có hai nghiệm là $x=-2; x=5$.
	\item Ta có
	\begin{align*}
	9x^2-1&=\left(3x-1\right)\left(2x+7\right)\\
	9x^2-1&=6x^2+19x-7 \\
	3x^2-19x+6&=0\\
	3x^2-18x-x+6&=0\\
	3x(x-6)-(x-6)&=0\\
	(3x-1)(x-6)&=0\\
	3x-1=0\,\text{hoặc }x-6&=0.
	\end{align*}
	\begin{itemize}
	\item $3x-1=0$ hay $3x=1$, suy ra $x=\dfrac{1}{3}$.
	\item $x-6-0$ suy ra $x=6$.
	\end{itemize}
	Vậy phương trình đã cho có hai nghiệm là $x=6; x=\dfrac{1}{3}$.
	\end{enumerate}
	}
\end{bt}
%%==========Bài 4
\begin{bt}
	Giải các phương trình:
	\begin{listEX}[2]
	\item $(3 x+7)(4 x-9)=0$;
	\item $(5 x-0,2)(0,3 x+6)=0$;
	\item $x(2 x-1)+5(2 x-1)=0$;
	\item $x^2-9-(x+3)(3 x+1)=0$;
	\item $x^2-10 x+25=5(5-x)$;
	\item $4 x^2=(x-12)^2$.
	\end{listEX}
	\loigiai{
	\begin{listEX}
	\item 
	Ta có $(3 x+7)(4 x-9)=0$ nên $(3 x+7)=0$ hoặc $(4 x-9)=0$.
	\begin{itemize}
	\item $3 x+7=0$ hay $3 x=-7$, suy ra $x=-\dfrac{7}{3}$.
	\item $4 x-9=0$ hay	$4 x=9$, suy ra $x=\dfrac{9}{4}$.
	\end{itemize}
	Vậy phương trình đã cho có hai nghiệm $x=-\dfrac{7}{3}$ và $x=\dfrac{9}{4}$.
	\item Ta có $(5 x-0{,}2)(0{,}3 x+6)=0$ nên $5 x-0{,}2=0$ hoặc $0{,}3 x+6=0$
	\begin{itemize}
	\item $5 x-0{,}2=0$ hay $5 x=0{,}2$, suy ra $x=0{,}04$. 
	\item $0{,}3 x+6=0$ hay $0{,}3 x=-6$, suy ra $x=-20$.
	\end{itemize}
	Vậy phương trình đã cho có hai nghiệm $x=0,04$ và $x=-20$.
	\item Ta có
	$$\begin{aligned}[t]
	x(2 x-1)+5(2 x-1)&=0\\
	(2x-1)(x+5)&=0
	\end{aligned}$$
	\begin{itemize}
	\item $2x-1=0$ hay $2x=1$, suy ra $x=\dfrac{1}{2}$.
	\item $x+5=0$ suy ra $x=-5$.
	\end{itemize}
	Vậy phương trình đã cho có hai nghiệm $x=\dfrac{1}{2}$ và $x=-5$.
	\item Ta có
	$$\begin{aligned}[t]
	x^2-9-(x+3)(3 x+1)&=0\\
	(x-3)(x+3)-(x+3)(3 x+1)&=0 \\ 
	(x+3)(x-3-3 x-1)&=0 \\ 
	(x+3)(-2 x-4)&=0
	\end{aligned}$$	
	\begin{itemize}
	\item $x+3=0$ suy ra $x=-3$.
	\item $-2x-4=0$ hay $-2x=4$, suy ra $x=-2$.
	\end{itemize}	
	Vậy phương trình đã cho có hai nghiệm $x=-3$ và $x=-2$.
	\item Ta có
	$$\begin{aligned} 
	x^2-10 x+25&=5(5-x)\\
	(x-5)^2+5(x-5)&=0 \\ 
	(x-5)(x-5+5)&=0 \\
	x(x-5)&=0	\\
	x=0 &\text { hoặc } x=5.
	\end{aligned}$$
	Vậy phương trình đã cho có hai nghiệm $x=0$ và $x=5$.	
	\item Ta có 
	$$\begin{aligned}
	4 x^2&=(x-12)^2\\
	(2 x)^2-(x-12)^2&=0 \\ 
	(2 x-x+12)(2 x+x+12)&=0 \\ 
	(x+12)(3 x+12)&=0
	\end{aligned}$$
	\begin{itemize}
	\item $x+12=0$ suy ra $x=-12$.
	\item $3x+12=0$ hay $3x=-12$, suy ra $x=-4$.
	\end{itemize}
	Vậy phương trình đã cho có hai nghiệm $x=-12$ và $x=-4$.
	\end{listEX}	}
\end{bt}
%%==========Bài 5
\begin{bt}%[8D3K2]
	Giải phương trình
	\begin{eqnarray*}
	(x+3)^3-(x+1)^3=56
	\end{eqnarray*}
	\loigiai{
	\begin{align*}
	&(x+3)^3-(x+1)^3=56\\
	\Rightarrow\, & x^3+ 9x^2+27x+27-x^3-3x^2-3x-1=56\\
	\Rightarrow\, & 6x^2+24x+26=56\\
	\Rightarrow\, & 6(x^2+4x-5)=0\\
	\Rightarrow\, & 6(x^2-x+5x-5)=0\\
	\Rightarrow\, & x(x-1)+5(x-1)=0\\
	\Rightarrow\, & (x-1)(x+5)=0.
	\end{align*}
	Kết luận $S=\{1;-5\}$.
	}
\end{bt}
%%==========Bài 6
\begin{bt}%[8D3K1]
	Giải phương trình
	\begin{eqnarray*}
	x^3+(x-1)^3=(2x-1)^3.	\qquad\qquad (1)
	\end{eqnarray*}
	\loigiai{
	Ta thấy $x+(x-1)=2x-1$. Đặt $x-1=y$ thì $(1)$ có dạng
	\begin{align*}
	& x^3+y^3=(x+y)^3\\
	\Rightarrow\, & x^3 + y^3 = x^3+y^3 +3xy(x+y)\\
	\Rightarrow\, & xy(x+y=0)\\
	\Rightarrow\, &\hoac{&x=0\\&y=0\\&x+y=0}\\
	\Rightarrow\,& \hoac{&x=0\\&x-1=0\\&2x-1=0} \\
	\Rightarrow\, &\hoac{&x=0\\&x=1\\&x=\dfrac{1}{2}}.
	\end{align*}
	Kết luận $S=\left\lbrace 0;\dfrac{1}{1};1\right\rbrace $
	}
\end{bt}
%%==========Bài 7
\begin{bt}%[8D3K1]
	Giải phương trình
	\begin{eqnarray*}
	(x+1)^2(x+2)+(x-1)^2(x-2)=12.
	\end{eqnarray*}
	\loigiai{
	Rút gọn vế trái của phương trình, ta được
	\begin{align*}
	2x^3+10x=12& \Rightarrow\, x^3+5x-6=0\\
	& \Rightarrow\, (x^3-1)+5(x-1)=0\\
	& \Rightarrow\, (x-1)(x^2+x+6)=0
	\end{align*}
	Dễ dàng chứng minh được $x^2+x+6\ne 0$. Do đó $S=\{1\}$.
	}
\end{bt}
%%==========Bài 8
\begin{bt}%[8D3K1]
	Giải các phương trình sau $x^5=x^4+x^3+x^2+x+2.$
	\loigiai{
	\begin{align*}
	& x^5=x^4+x^3+x^2+x+2\\
	\Rightarrow\, & (x^5-1)-(x^4+x^3+x^2+x+1)\\
	\Rightarrow\, & (x-2)(x^4+x^3+x^2+x+1)=0\\
	\Rightarrow\, & x-2=0 
	\end{align*}
	Vì phương trình $x^4+x^3+x^2+x+1=0$ vô nghiệm.\\
	Vậy nghiệm $x=2$.
	}
\end{bt}
%%==========Bài 9
\begin{bt}%[8D3K1]
	Giải các phương trình sau
	\begin{enumEX}{2}
	\item $x^3+2x^2+x+2=0$
	\item $x^3+2x^2-x-2.$
	\item $x^3-x^2-21x+45.$
	\item $x^3+3x^2+4x+2=0.$
	\item $x^4+x^2+6x-8=0.$
	\item $(x^2+1)^2=4(2x-1).$
	\item $(x-1)^3+(2x+3)^3=27x^3+8.$
	\item $6x^4-x^3-7x^2+x+1=0.$
	\end{enumEX}
	\loigiai{
	\begin{enumEX}{1}
	\item $x^3+2x^2+x+2=0\Rightarrow\, (x+2)(x^2+1)=0$. Nghiệm $x=-2$.
	\item $x^3+2x^2-x-2\Rightarrow\, (x+2)(x^2-1)$. Nghiệm $x=-2;-1;1$.
	\item $x^3-x^2-21x+45\Rightarrow\, (x-3)^2(x+5)=0$. Nghiệm $x=-5;3$.
	\item $x^3+3x^2+4x+2=0\Rightarrow\, (x=1)(x^2+2x+2)=0$. Nghiệm $x=-1$.
	\item $x^4+x^2+6x-8=0\Rightarrow\, (x-1)(x+2)(x^2-x+4)=0$. Nghiệm $x=-2;1$.
	\item $(x^2+1)^2=4(2x-1)\Rightarrow\, (x-1)^2(x^2+2x+5)=0$. Nghiệm $x=1$.
	\item $(x-1)^3+(2x+3)^3=27x^3+8.$ Nghiệm $x=-\dfrac{2}{3};-\dfrac{1}{2};3$.
	\item $6x^4-x^3-7x^2+x+1=0\Rightarrow\, (x^2-1)(2x-1)(3x+1)=0$. Nghiệm $x=-1;-\dfrac{1}{3};\dfrac{1}{2};1$.
	\end{enumEX}
	}
\end{bt}
%%==========Bài 10
\begin{bt}
	Giải các phương trình sau
	\begin{listEX}[2]
	\item $\dfrac{x}{x - 5} - \dfrac{2}{x + 5} = \dfrac{x^2}{x^2 - 25}$.
	\item $\dfrac{1}{x + 1} - \dfrac{x}{x^2 - x + 1} = \dfrac{3}{x^3 + 1}$.
	\end{listEX}
	\loigiai{
	\begin{listEX}[1]
	\item Điều kiện: $\heva{&x - 5 \neq 0\\&x + 5 \neq 0} \Leftrightarrow \heva{&x \neq 5\\&x \neq -5.}$\\
	Ta có
	$$\begin{aligned}[t]
	\dfrac{x}{x - 5} - \dfrac{2}{x + 5} &= \dfrac{x^2}{x^2 - 25}\\
	\dfrac{x(x+5)}{x^2 - 25} - \dfrac{2(x-5)}{x^2 - 25} &= \dfrac{x^2}{x^2 - 25}\\
	\dfrac{x(x+5) - 2(x-5)}{x^2 - 25} &= \dfrac{x^2}{x^2 - 25}\\
	\dfrac{x^2 + 3x + 10}{x^2 - 25} &= \dfrac{x^2}{x^2 - 25}\\
	x^2 + 3x + 10 &= x^2\\
	3x + 10 &= 0\\
	x &= -\dfrac{10}{3} \textrm{ (nhận).}
	\end{aligned}$$
	Vậy phương trình đã cho có nghiệm $x = -\dfrac{10}{3}$.
	\item Điều kiện: $\heva{&x + 1 \neq 0\\&x^2 - x + 1 \neq 0} \Rightarrow x \neq -1$.\\
	Ta có
	$$\begin{aligned}[t]
	\dfrac{1}{x + 1} - \dfrac{x}{x^2 - x + 1} &= \dfrac{3}{x^3 + 1}\\
	\dfrac{x^2 - x + 1}{x^3 + 1} - \dfrac{x(x+1)}{x^3 + 1} &= \dfrac{3}{x^3 + 1}\\
	\dfrac{x^2 - x + 1 - x(x+1)}{x^3 + 1} &= \dfrac{3}{x^3 + 1}\\
	\dfrac{-2x + 1}{x^3 + 1} &= \dfrac{3}{x^3 + 1}\\
	-2x + 1 &= 3\\
	x &= 1 \textrm{ (nhận).}
	\end{aligned}$$
	Vậy phương trình đã cho có nghiệm $x = 1$.
	\end{listEX}
	}
\end{bt}
%%==========Bài 11
\begin{bt}
	Giải các phương trình:
	\begin{listEX}[2]
	\item $\dfrac{5}{x+2}+\dfrac{3}{x-1}=\dfrac{3x+4}{\left(x+2\right)\left(x-1\right)}$.
	\item $\dfrac{4}{2x-3}+\dfrac{3}{x\left(2x-3\right)}=\dfrac{5}{x}$.
	\item $\dfrac{2}{x-3}+\dfrac{3}{x+3}=\dfrac{3x-5}{x^2-9}$.
	\item $\dfrac{x-1}{x+1}-\dfrac{x+1}{x-1}=\dfrac{8}{x^2-1}$.
	\end{listEX}
	\loigiai{
	\begin{enumerate}
	\item 
	Với điều kiện $\heva{&x\ne -2\\&x\ne 1}$, ta có:
	\begin{align*}
	\dfrac{5}{x+2}+\dfrac{3}{x-1}&=\dfrac{3x+4}{\left(x+2\right)\left(x-1\right)}\\
	\dfrac{5x-5}{\left(x+2\right)\left(x-1\right)}+\dfrac{3x+6}{\left(x+2\right)\left(x-1\right)}&=\dfrac{3x+4}{\left(x+2\right)\left(x-1\right)}\\
	5x-5+3x+6-3x-4&=0\\
	5x-3&=0\\
	x&=\dfrac{3}{5}\quad \left( \text{thỏa mãn}\right).
	\end{align*}
	Vậy nghiệm của phương trình là $x=\dfrac{3}{5}$.
	\item 
	Với điều kiện: $\heva{&x\ne 0\\&x\ne \dfrac{3}{2}}$, ta có:
	\begin{align*}
	\dfrac{4}{2x-3}-\dfrac{3}{x\left(2x-3\right)}&=\dfrac{5}{x}\\
	\dfrac{4x}{x\left(2x-3\right)}-\dfrac{3}{x\left(2x-3\right)}&=\dfrac{10x-15}{x\left(2x-3\right)}\\
	4x-3-10x+15&=0\\
	-6x+12&=0\\
	x&=2 \quad \left( \text{thỏa mãn}\right).
	\end{align*}
	Vậy nghiệm của phương trình là $x=2$.
	\item 
	Với điều kiện $\heva{&x\ne 3\\&x\ne -3}$, ta có:
	\begin{align*}
	\dfrac{2}{x-3}+\dfrac{3}{x+3}&=\dfrac{3x-5}{x^2-9}\\
	2x+6+3x-9&=3x-5\\
	2x+2&=0\\
	x&=-1\quad \left( \text{thỏa mãn}\right).
	\end{align*}
	Vậy nghiệm của phương trình là $x=-1$.
	\item
	Với điều kiện $\heva{&x\ne 1\\&x\ne -1}$, ta có:
	\begin{align*}
	\dfrac{x-1}{x+1}-\dfrac{x+1}{x-1}&=\dfrac{8}{x^2-1}\\
	\left(x-1\right)^2-\left(x+1\right)^2&=8\\
	-4x&=8\\
	x&=-2 \quad \left( \text{thỏa mãn}\right).
	\end{align*}
	Vậy nghiệm của phương trình là $x=-2$.
	\end{enumerate}	
	}
\end{bt}
%%==========Bài 12
\begin{bt}
	Giải các phương trình:
	\begin{listEX}[3]
	\item $\dfrac{-6}{x+3}=\dfrac{2}{3}$;
	\item $\dfrac{x-2}{2}+\dfrac{1}{2 x}=0$;
	\item $\dfrac{8}{3 x-4}=\dfrac{1}{x+2}$;
	\item $\dfrac{x}{x-2}+\dfrac{2}{(x-2)^2}=1$;
	\item $\dfrac{3 x-2}{x+1}=4-\dfrac{x+2}{x-1}$;
	\item $\dfrac{x^2}{(x-1)(x-2)}=1-\dfrac{1}{x-1}$. 
	\end{listEX}
	\loigiai{
	\begin{listEX}[2]
	\item $\dfrac{-6}{x+3}=\dfrac{2}{3}$.\\
	Điều kiện xác định: $x \ne -3$.\\
	$\begin{aligned} 
	\dfrac{-6}{x+3} & =\dfrac{2}{3} \\ 
	\dfrac{-18}{3(x+3)} & =\dfrac{2(x+3)}{3(x+3)} \\ 
	-18 & =2 x+6 \\ 
	2 x & =-24 \\ 
	x & =-12.
	\end{aligned}$
	\\ Ta thấy $x=-12$ thỏa mãn điều kiện xác định.\\
	Vậy phương trình đã cho có nghiệm $x=-12$.
	\item $\dfrac{x-2}{2}+\dfrac{1}{2 x}=0$.\\
	Điều kiện xác định: $x \ne 0$.\\
	$\begin{aligned} 
	\dfrac{x(x-2)}{2 x}+\dfrac{1}{2 x}&=0 \\ 
	x^2-2 x+1&=0 \\ 
	(x-1)^2&=0 \\ 
	x-1&=0 \\ 
	x&=1.
	\end{aligned}$
	\\ Ta thấy $x=1$ thỏa mãn điều kiện xác định.\\
	Vậy phương trình đã cho có nghiệm $x=1$.
	\item $\dfrac{8}{3 x-4}=\dfrac{1}{x+2}$.\\
	Điều kiện xác định: $x \neq-2 $; $x \neq \dfrac{4}{3}$.\\
	$\begin{aligned} 
	\dfrac{8}{3 x-4}&=\dfrac{1}{x+2}\\
	8(x+2) & =3 x-4 \\ 
	8 x+16 & =3 x-4 \\ 
	5 x & =-20 \\ 
	x & =-4.
	\end{aligned}$
	\\ Ta thấy $x=-4$ thỏa mãn điều kiện xác định.\\
	Vậy phương trình đã cho có nghiệm $x=-4$.
	\item $\dfrac{x}{x-2}+\dfrac{2}{(x-2)^2}=1$.\\
	Điều kiện xác định: $x \neq 2 $.\\
	$\begin{aligned} 
	\dfrac{x}{x-2}+\dfrac{2}{(x-2)^2}&=1\\
	x(x-2)+2 & =(x-2)^2 \\ 
	x^2-2 x+2 & =x^2-4 x+4 \\ 
	2 x & =2 \\ 
	x & =1.
	\end{aligned}$
	\\ Ta thấy $x=1$ thỏa mãn điều kiện xác định.\\
	Vậy phương trình đã cho có nghiệm $x=1$.
	\item $\dfrac{3 x-2}{x+1}=4-\dfrac{x+2}{x-1}$.\\
	Điều kiện xác định: $x \neq -1 $ và $x \neq 1$.\\
	$\begin{aligned}
	\dfrac{3 x-2}{x+1}&=4-\dfrac{x+2}{x-1}\\
	(3 x-2)(x-1) & =4
	\left(x^2-1\right)-(x+2)(x+1) \\ 
	3 x^2-3 x-2 x+2 & =4 x^2-4-x^2-3 x-2 \\ 
	-2 x & =-8 \\ 
	x & =4.
	\end{aligned}$
	\\ Ta thấy $x=4$ thỏa mãn điều kiện xác định.\\
	Vậy phương trình đã cho có nghiệm $x=4$.	
	\item $\dfrac{x^2}{(x-1)(x-2)}=1-\dfrac{1}{x-1}$. \\
	Điều kiện xác định: $x \neq 1 $ và $x \neq 2 $.\\
	$\begin{aligned} 
	\dfrac{x^2}{(x-1)(x-2)}&=1-\dfrac{1}{x-1}\\
	x^2 & =(x-1)(x-2)-(x-2) \\ 
	x^2 & =x^2-3 x+2-x+2 \\ 
	4 x & =4 \\ 
	x & =1.
	\end{aligned}$
	\\ Ta thấy $x=1$ không thỏa mãn điều kiện xác định.\\
	Vậy phương trình đã cho vô nghiệm.
	\end{listEX}
	}
\end{bt}
%%==========Bài 13
\begin{bt}
	Giải phương trình 
	$\dfrac{1}{x-1}+\dfrac{2}{x^{2}+x+1}=\dfrac{x^{2}+x}{x^{3}-1}$.
	\loigiai{%
	ĐKXĐ: $x \neq 1$. Quy đồng mẫu hai vế của phương trình:
	\begin{eqnarray*} 
	\dfrac{\left(x^{2}+x+1\right)+2(x-1)}{(x-1)\left(x^{2}+x+1\right)}&=&\dfrac{x^{2}+x}{x^{3}-1}
	\\
	\dfrac{x^{2}+3 x-1}{x^{3}-1}&=&\dfrac{x^{2}+x}{x^{3}-1}
	\end{eqnarray*}
	Suy ra $x^{2}+3 x-1=x^{2}+x$ hay $2 x-1=0$.\\
	Giải phương trình: $2 x-1=0$
	\begin{eqnarray*}
	2 x&=&1 \\
	x&=&\dfrac{1}{2}(\text { thỏa mãn ĐKXĐ). }
	\end{eqnarray*}
	Vậy phương trình có nghiệm là $x=\dfrac{1}{2}$.
	}
\end{bt}
%%==========Bài 14
\begin{bt}
	Giải phương trình $\dfrac{x}{x+3}-\dfrac{2}{x-3}=\dfrac{-2 x-6}{x^{2}-9}$.
	\loigiai{%
	ĐKXĐ: $x \neq 3$ và $x \neq-3$.\\
	Quy đồng mẫu hai vế của phương trình:
	\begin{eqnarray*}
	\dfrac{x(x-3)-2(x+3)}{(x+3)(x-3)}&=&\dfrac{-2 x-6}{x^{2}-9} \\
	\dfrac{x^{2}-5 x-6}{x^{2}-9}&=&\dfrac{-2 x-6}{x^{2}-9} .
	\end{eqnarray*}
	Suy ra $x^{2}-5 x-6=-2 x-6$ hay $x^{2}-3 x=0$.\\
	Giải phương trình $x^{2}-3 x=0$ :
	\begin{eqnarray*}
	\begin{aligned}
	& x(x-3)=0 \\
	& x=0 \text { hoặc } x-3=0 \\
	& x=0 \text { (thỏa mãn ĐKXĐ) hoặc } x=3 \text { (không thỏa mãn ĐKXĐ). 
	}
	\end{aligned}
	\end{eqnarray*}
	Vậy phương trình có nghiệm là $x=0$.
	}
\end{bt}
%%==========Bài 15
\begin{bt}
	Giải các phương trình sau:
	\begin{listEX}[2]
	\item $\dfrac{1}{x+2}-\dfrac{2}{x^{2}-2 x+4}=\dfrac{x-4}{x^{3}+8}$
	\item $\dfrac{2 x}{x-4}+\dfrac{3}{x+4}=\dfrac{x-12}{x^{2}-16}$.
	\end{listEX}
	\loigiai{%
	\begin{enumerate}
	\item $\dfrac{1}{x+2}-\dfrac{2}{x^{2}-2 x+4}=\dfrac{x-4}{x^{3}+8}$\\
	ĐKXĐ: $x \neq -2$.\\
	Quy đồng mẫu hai vế của phương trình:
	\begin{eqnarray*}
	\dfrac{x^2-2x+4-2(x+2)}{(x+2)(x^2-2x+4)}&=&\dfrac{x-4}{x^{3}+8}\\
	\dfrac{x^2-2x+4-2x-4}{x^3+8}&=&\dfrac{x-4}{x^3+8}\\
	\dfrac{x^2-4x}{x^3+8}&=&\dfrac{x-4}{x^3+8}
	\end{eqnarray*}
	Suy ra 
	\begin{eqnarray*}
	x^2-4x&=&x-4\\
	x(x-4)&=&x-4\\
	x(x-4)-(x-4)&=&0\\
	(x-4)(x-1)&=&0\\
	x-4=0~\text{hoặc}~x-1&=&0\\
	x=4~\text{hoặc}~x=1
	\end{eqnarray*}
	Vậy phương trình có nghiệm là $x=4$ hoặc $x=1$.
	\item $\dfrac{2 x}{x-4}+\dfrac{3}{x+4}=\dfrac{x-12}{x^{2}-16}$.\\
	ĐKXĐ: $x \neq -4$ và $x \neq 4$.\\
	Quy đồng mẫu hai vế của phương trình:
	\begin{eqnarray*}
	\dfrac{2x(x+4)+3(x-4)}{(x-4)(x+4)}&=&\dfrac{x-12}{x^2-16}\\
	\dfrac{2x^2+8x+3x-12}{x^2-16}&=&\dfrac{x-12}{x^2-16}\\
	\dfrac{2x^2+11x-12}{x^2-16}&=&\dfrac{x-12}{x^2-16}
	\end{eqnarray*}
	Suy ra 
	\begin{eqnarray*}
	2x^2+11x-12&=&x-12\\
	2x^2+11x-x&=&0\\
	2x^2+10x&=&0\\
	2x(x+5)&=&0\\
	2x=0~\text{hoặc}~x+5&=&0\\
	x=0~\text{hoặc}~x=-5
	\end{eqnarray*}
	Vậy phương trình có nghiệm là $x=0$ hoặc $x=-5$.
	\end{enumerate}
	}
\end{bt}
%%==========Bài 16
\begin{bt}%[8D3Y5]
	Giải phương trình:
	$$ \dfrac{x-1}{x-2}+\dfrac{x+3}{x-4}=\dfrac{2}{(x-2)(4-x)}. \qquad (1)$$
	\loigiai{
	ĐKXĐ của phương trình là $x \neq 2$, $x \neq 4$.\\
	Biến đổi phương trình (1):
	$$(x-1)(x-4)+(x+3)(x-2)=-2.$$
	Thu gọn phương trình, ta được 
	\[2x(x-2)=0. \tag{2}\]
	Nghiệm của (2) là $x_1 =0$, $x_2 =2$. Trong đó, $x_1=0$ thỏa mãn ĐKXĐ, $x_2=2$ không thỏa mãn ĐKXĐ.\\
	\textit{Kết luận:} $S=\{0\}$. 
	}
\end{bt}
%%==========Bài 17
\begin{bt}%[8D3Y5]
	% bai 1 
	Giải các phương trình:
	\begin{listEX}[2]
	\item $\dfrac{x+2}{x+1}+\dfrac{3}{x-2}=\dfrac{3}{x^2-x-2}+1$;
	\item $\dfrac{x+6}{x-5}+\dfrac{x-5}{x+6}=\dfrac{2x^2+23x+61}{x^2+x-30}$;
	\item $\dfrac{6}{x-5}+\dfrac{x+2}{x-8}=\dfrac{18}{(x-5)(8-x)}-1$;
	\item $\dfrac{x-4}{x-1}+\dfrac{x+4}{x+1}=2$;
	\item $\dfrac{3}{x+1}-\dfrac{1}{x-2}=\dfrac{9}{(x+1)(2-x)}$;
	\item $\dfrac{x^2-x}{x+3}-\dfrac{x^2}{x-3}=\dfrac{7x^2-3x}{9-x^2}$.
	\end{listEX}
	\loigiai{
	\begin{enumerate}
	\item ĐKXĐ của phương trình là: $x \neq -1$, $x \neq 2$.\\
	Biến đổi phương trình, ta được
	\begin{align*}
	& \ (x+2)(x-2)+3(x+1)=3+(x+1)(x-2)\\
	\Rightarrow\, & \ 4x=2 \Rightarrow\, x=\dfrac{1}{2}.
	\end{align*}
	Vậy $S = \left\{ \dfrac{1}{2} \right\}$.
	\item ĐKXĐ của phương trình là: $x \neq -6$, $x \neq 5$.\\
	Biến đổi phương trình, ta được
	\begin{align*}
	& \ (x+6)^2 +(x-5)^2=2x^2+23x+61\\
	\Rightarrow\, & \ 2x=23x\\
	\Rightarrow\, & \ x=0.
	\end{align*}
	Vậy $S=\left\{ 0 \right\} $.
	\item ĐKXĐ của phương trình là: $x \neq 8$, $x \neq 5$.\\
	Biến đổi phương trình, ta được
	\begin{align*}
	& \ 6(x-8)+(x+2)(x-5)=-18-(x-5)(x-8)\\
	\Rightarrow\, & \ 2x^2-10x=0\\
	\Rightarrow\, & \ \hoac {&x=0 \\ &x=5} .
	\end{align*}
	Vì $x=5$ bị loại nên $S=\{0\}$.
	\item ĐKXĐ của phương trình là: $x \neq \pm 1$.\\
	Biến đổi phương trình, ta được
	\begin{align*}
	& \ (x-4)(x+1)+(x+4)(x-1)=2x^2-2\\
	\Rightarrow\, & \ -2=0.
	\end{align*}
	Vậy $S= \emptyset$.
	\item ĐKXĐ của phương trình là: $x \neq - 1$, $x \neq 2$.\\
	Biến đổi phương trình, ta được
	\begin{align*}
	& \ 3(x-2)-(x+1)=-9\\
	\Rightarrow\, & \ 2x=-2\\
	\Rightarrow\, & \ x=-1.
	\end{align*}
	Vậy $S= \emptyset$.
	\item ĐKXĐ của phương trình là: $x \neq \pm 3$.
	Biến đổi phương trình, ta được
	\begin{align*}
	& \ (x^2-x)(x-3)-x^2(x+3)=-7x^2+3x\\
	\Rightarrow\, & \ 0=0.
	\end{align*}
	Vậy phương trình có vô số nghiệm: $x$ bất kỳ khác $ \pm 3$.
	\end{enumerate}
	}
\end{bt}
%%==========Bài 18
\begin{bt}%[8D3B5]
	%bai 2
	Giải các phương trình sau:
	\begin{listEX}[2]
	\item $\dfrac{x+1}{x^2+x+1}-\dfrac{x-1}{x^2-x+1}=\dfrac{3}{x(x^4+x^2+1)}$;
	\item $\dfrac{x+2}{x^2+2x+4}-\dfrac{x-2}{x^2-2x+4}=\dfrac{6}{x(x^4+4x^2+16)}$.
	\end{listEX}
	\loigiai{
	\begin{enumerate}
	\item ĐKXĐ của phương trình là $x \neq 0$.\\
	Biến đổi phương trình, ta được
	\begin{align*}
	& \ (x+1)x(x^2-x+1)-(x-1)x(x^2+x+1)=3\\
	\Rightarrow\, & \ 2x=3\\
	\Rightarrow\, & \ x=\dfrac{3}{2}.
	\end{align*}
	Vậy $S=\left\{\dfrac{3}{2}\right\}$.
	\item ĐKXĐ của phương trình: $x \neq 0$.\\
	Biến đổi phương trình, ta được
	\begin{align*}
	& \ x(x+2)(x^2-2x+4)-x(x-2)(x^2+2x+4)=6\\
	\Rightarrow\, & \ 16x =6\\
	\Rightarrow\, & \ x= \dfrac{3}{8}.
	\end{align*}
	Vậy $S=\left\{ \dfrac{3}{8}\right\}$.
	\end{enumerate}
	}
\end{bt}
%%==========Bài 19
\begin{bt}%Bài 2
	Để loại bỏ $x$ một loại tảo độc khỏi một hồ nước, người ta ước tính chi phí cần bỏ ra là
	\begin{eqnarray*}
	C(x)=\dfrac{50 x}{100-x} \text { (triệu đồng), với } 0 \leq x<100.
	\end{eqnarray*}
	Nếu bỏ ra $450$ triệu đồng, người ta có thể loại bỏ được bao nhiêu phần trăm 
	loại tảo độc đó?
	\loigiai{%
	Thế $C(x)=450$ ta có
	\begin{eqnarray*}
	450&=&\dfrac{50 x}{100-x}
	\end{eqnarray*}
	ĐKXĐ: $x \neq 100$\\
	Quy đồng mẫu hai vế của phương trình:
	\begin{eqnarray*}
	\dfrac{450(100-x)}{100-x}&=&\dfrac{50 x}{100-x}\\
	\dfrac{45000-450x)}{100-x}&=&\dfrac{50 x}{100-x}
	\end{eqnarray*}
	Suy ra phương trình
	\begin{eqnarray*}
	450(100-x)&=&50x\\
	45\,000-450x&=&50x\\
	45\,000&=&50x+450x\\
	45\,000&=&500x\\
	x&=&90
	\end{eqnarray*}
	Vậy nếu bỏ ra $450$ triệu đồng thì loại bỏ được $90\%$ loại tảo độc đó.
	}
\end{bt}
%%==========Bài 20
\begin{bt}
	Một hãng viễn thông nước ngoài có hai gói cước như sau
	\begin{center}
	\begin{tabular}{|l|l|}
	\hline
	\qquad \qquad \qquad Gói cước A & \qquad \qquad \qquad Gói cước B\\
	\hline
	Cước thuê bao hàng tháng $32$ USD & Cước thuê bao hàng tháng là $44$ USD \\
	\hline
	$45$ phút miễn phí & Không có phút miễn phí\\
	\hline
	$0{,}4$ USD cho mỗi phú thêm & $0{,}25$ USD/phút\\
	\hline
	\end{tabular}
	\end{center}
	\begin{enumerate}
	\item Hãy viết một phương trình xác định thời gian gọi (phút) mà phí phải trả trong cùng một tháng của hai gói cước là như nhau và giải phương trình đó.
	\item Nếu khách hàng chỉ gọi tối đa là $180$ phút trong $1$ tháng thì nên dùng gói cước nào? Nếu khách hàng gọi $500$ phút trong $1$ tháng thì nên dùng gói cước nào?
	\end{enumerate}
	\loigiai{
	\begin{enumerate}
	\item Vì cước thuê bao hàng tháng của gói cước A và gói cước B lần lượt là $32$ USD và $44$ USD nên để phí phải trả trong cùng một tháng của hai gói cước là như nhau thì thời gian gọi phải trên $45$ phút.\\
	Gọi $x$ (phút) là thời gian gọi của một khách hàng $(x > 45)$.\\
	Phí phải trả trong một tháng khi sử dụng gói cước A là $0{,}4\cdot(x-45) + 32$ (USD).\\
	Phí phải trả trong một tháng khi sử dụng gói cước B là $0{,}25x + 44$ (USD).\\
	Để phí phải trả trong cùng một tháng của hai gói cước là như nhau thì
	\begin{eqnarray*}
	0{,}4\cdot(x-45) + 32 &=& 0{,}25x + 44\\
	\dfrac{2}{5}(x-45) + 32 &=& \dfrac{1}{4}x + 44\\
	\dfrac{3}{20}x &=& 30\\
	x &=& 200 \textrm{ (nhận).}
	\end{eqnarray*}
	Thời gian gọi thỏa mãn yêu cầu đề bài là $200$ phút.
	\item Nếu khách hàng chỉ gọi tối đa là $180$ phút trong $1$ tháng thì
	\begin{itemize}
	\item phí tối đa sử dụng theo gói cước A là $0{,}4\cdot(180-45) + 32 = 86$ USD.
	\item phí tối đa sử dụng theo gói cước B là $0{,}25\cdot 180 + 44 = 89$ USD.
	\end{itemize}
	Vì $86 < 89$ do đó khách hàng nên sử dụng gói cước A.\\
	Nếu khách hàng gọi $500$ phút trong $1$ tháng thì
	\begin{itemize}
	\item phí tối đa sử dụng theo gói cước A là $0{,}4\cdot(500-45) + 32 = 214$ USD.
	\item phí tối đa sử dụng theo gói cước B là $0{,}25\cdot 500 + 44 = 169$ USD.
	\end{itemize}
	Vì $214 > 169$ do đó khách hàng nên sử dụng gói cước B.
	\end{enumerate}
	}
\end{bt}
%%==========Bài 21
\begin{bt}
	Cho hai phương trình
	\begin{center}
	$-2x+5y=7;\qquad (1)$\\
	$4x-3y=7.\qquad (2)$
	\end{center}
	Trong các cặp số $(2;0)$, $(1;-1)$, $(-1;1)$, $(-1;6)$, $(4;3)$ và $(-2;-5)$, cặp số nào là
	\begin{listEX}[1]
	\item Nghiệm của phương trình $(1)$?
	\item Nghiệm của phương trình $(2)$?
	\item Nghiệm của hệ gồm phương trình $(1)$ và phương trình $(2)$?
	\end{listEX}
	\loigiai{
	\begin{listEX}[1]
	\item Nghiệm của phương trình $(1)$ là $(-1;1)$ và $(4;3)$ vì $-2\cdot (-1)+5\cdot 1=7$ và $-2\cdot 4+5\cdot 3=7$.
	\item Nghiệm của phương trình $(2)$ là $(1;-1)$ và $(4;3)$, $(-2;-5)$ vì $4\cdot 1-3\cdot (-1)=7$; $4\cdot 4-3\cdot 3=7$; $4\cdot (-2)-3\cdot (-5)=7$.
	\item Nghiệm của hệ gồm phương trình $(1)$ và phương trình $(2)$ là $(4;3)$.
	\end{listEX}	
	}
\end{bt}
%%==========Bài 22
\begin{bt}
	Giải các hệ phương trình:
	\begin{enumEX}{3}
	\item $\heva{& 2 x+5 y=10 \\ & \dfrac{2}{5} x+y=1.}$
	\item $\heva{& 0{,}2 x+0{,}1 y=0{,}3 \\ & 3 x+y=5.}$
	\item $\heva{& \dfrac{3}{2} x-y=\dfrac{1}{2} \\ & 6 x-4 y=2.}$	
	\end{enumEX}
\loigiai{
\begin{enumerate}
	\item Ta có $\heva{& 2 x+5 y=10 \\ & \dfrac{2}{5} x+y=1.}$\\
	Từ phương trình thứ hai của hệ ta có $y=1-\dfrac{2}{5}x$. Thế vào phương trình thứ nhất của hệ ta được $2x+5\cdot (1-\dfrac{2}{5}x)=10\Rightarrow 2x+5-2x=10\Rightarrow 0x=5$.\hfill $(1)$\\
	Do không có giá trị nào của $x$ thỏa mãn hệ thức $(1)$ nên hệ phương trình đã cho vô nghiệm.
	\item Ta có $\heva{& 0{,}2 x+0{,}1 y=0{,}3 \\ & 3 x+y=5}\Rightarrow \heva{&-2x-y=-3 \\ & 3 x+y=5.}$\\
	Cộng hai vế phương trình của hệ mới, ta được $x=2$.\\
	Thay $x=2$ vào phương trình thứ hai của hệ, ta được $3\cdot 2+y=5\Rightarrow y=-1$.\\
	Vậy nghiệm của hệ phương trình là $(2;-1)$.
	\item Ta có $\heva{& \dfrac{3}{2} x-y=\dfrac{1}{2} \\ & 6 x-4 y=2.}$\\
	Từ phương trình thứ nhất của hệ ta có $y=\dfrac{3}{2}x-\dfrac{1}{2}$. Thế vào phương trình thứ hai của hệ ta được $6x-6x+2=2\Rightarrow 0=0$.\hfill $(2)$\\
	Ta thấy rằng với mọi giá trị của $x$ thì hệ thức $(2)$ luôn đúng.\\
	Do đó hệ đã cho có vô số nghiệm.
\end{enumerate}
}
\end{bt}
%%==========Bài 23
\begin{bt}
	Giải các hệ phương trình:
	\begin{listEX}[4]
	\item $\heva{&3x+2y=7\\ &x-7y=-13.}$
	\item $\heva{&4x+y=2\\&8x+3y=5.}$
	\item $\heva{&5x-4y=3\\ &2x+y=4.}$
	\item $\heva{&3x-2y=10\\ &x-\dfrac{2}{3}y=3\dfrac{1}{3}}.$
	\end{listEX}
	\loigiai{
	\begin{enumerate}
	\item Ta có $\heva{&3x+2y=7\\ &x-7y=-13}\Rightarrow \heva{&3x+2y=7\\ &3x-21y=-39}\Rightarrow \heva{&23y=46\\ &3x+2y=7} \Rightarrow \heva{&y=2\\ &x=1.}$\\
	Vậy $S=\left\{(1;2)\right\}$.
	\item Ta có $\heva{&4x+y=2\\&8x+3y=5}\Rightarrow \heva{&8x+2y=4\\ &8x+3y=5}\Rightarrow \heva{&-y=-1\\ &4x+y=2}\Rightarrow \heva{&x=\dfrac{1}{4}\\ &y=1.}$\\
	Vậy $S=\left\{\left(\dfrac{1}{4};1\right)\right\}$.
	\item Ta có $\heva{&5x-4y=3\\ &2x+y=4}\Rightarrow \heva{&5x-4y=3\\ &8x+4y=16}\Rightarrow \heva{&13x=19\\ &2x+y=4}\Rightarrow \heva{&x=\dfrac{19}{13}\\ &y=\dfrac{14}{13}.}$\\
	Vậy $S=\left\{\left(\dfrac{19}{13};\dfrac{14}{13}\right)\right\}$.
	\item Ta có $\heva{&3x-2y=10\\ &x-\dfrac{2}{3}y=3\dfrac{1}{3}}\Rightarrow \heva{&3x-2y=10\\ &3x-2y=10}\Rightarrow 3x-2y=10.$\\
	Do đó hệ phương trình có vô số nghiệm.
	\end{enumerate}
	}
\end{bt}
%%==========Bài 24
\begin{bt}
	Giải các hệ phương trình:
	\begin{listEX}[3]
	\item $\heva{&x+3 y=-2 \\ &5 x+8 y=11;}$
	\item $\heva{&2 x+3 y=-2 \\ &3 x-2 y=-3;}$
	\item $\heva{&2 x-4 y=-1 \\ &-3 x+6 y=2.}$
	\end{listEX}	
	\loigiai{
	\begin{listEX}
	\item Ta có
	$\begin{aligned}[t] 
	\heva{&x+3 y=-2 \\ &5 x+8 y=11}
	\Rightarrow	\heva{&-5 x-15 y=10 \\ &5 x+8 y=11}
	\Rightarrow	\heva{&-7 y=21 \\ &5 x+8 y=11} 
	\Rightarrow	\heva{&y=-3 \\ &5 x+8 \cdot(-3)=11} 
%	\Rightarrow	\heva{&y=-3 \\ &5 x=35}
	\Rightarrow	\heva{&y=-3 \\ &x=7.}
	\end{aligned}$
	\\ Vậy hệ phương trình đã cho có nghiệm $(x;y)=(7;-3)$.
	\item Ta có
	$\begin{aligned}[t] 
	\heva{&2 x+3 y=-2 \\ &3 x-2 y=-3}
	\Rightarrow	\heva{&4 x+6 y=-4 \\ &9 x-6 y=-9} 
	\Rightarrow	\heva{&13 x=-13 \\ &9 x-6 y=-9}
	\Rightarrow	\heva{&x=-1 \\ &9\cdot (-1)-6 y=-9}
	\Rightarrow	\heva{&x=-1 \\ &y=0.}
	\end{aligned}$
	\\ Vậy hệ phương trình đã cho có nghiệm $(x;y)=(-1;0)$.
	\item Ta có
	$\begin{aligned}[t]
	\heva{&2 x-4 y=-1 \\ &-3 x+6 y=2}
	\Rightarrow	\heva{&6 x-12 y=-3 \\&-6 x+12 y=4} 
	\Rightarrow	\heva{&0 x+0 y=1 \text{ (phuơng trình vô nghiệm)}\\&-6 x+12 y=0.}
	\end{aligned}$
	\\ Vậy hệ phương trình đã cho vô nghiệm.
	\end{listEX}
	}
\end{bt}
%%==========Bài 25
\begin{bt}
	Giải các hệ phương trình:
	\begin{enumEX}{3}
	\item $\heva{& 0{,}5 x+2 y=-2{,}5 \\ & 0{,}7 x-3 y=8{,}1.}$
	\item $\heva{& 5 x-3 y=-2 \\ & 14 x+8 y=19.}$
	\item $\heva{& 2(x-2)+3(1+y)=-2 \\ & 3(x-2)-2(1+y)=-3.}$	
	\end{enumEX}
	\loigiai{
	\begin{enumerate}
	\item Ta có $\heva{& 0{,}5 x+2 y=-2{,}5 \\ & 0{,}7 x-3 y=8{,}1}\Rightarrow \heva{& 1{,}5 x+6y=-7{,}5 \\ & 1{,}4 x-6y=16{,}2}$\\
	Cộng hai vế của hệ mới, ta được $2{,}9x=8{,}7\Rightarrow x=3$.\\
	Thay $x=3$ vào phương trình thứ nhất của hệ ta được $0{,}5\cdot 3+2y=-2{,}5\Rightarrow y=-2$.\\
	Vậy hệ phương trình đã cho có nghiệm là $(3;-2)$.	
	\item Ta có $\heva{& 5 x-3 y=-2 \\ & 14 x+8 y=19}\Rightarrow \heva{& 40x-24 y=-16 \\ & 42 x+24 y=57.}$\\
	Cộng hai vế của hệ mới, ta được $82x=41\Rightarrow x=\dfrac{1}{2}$.\\
	Thay $x=\dfrac{1}{2}$ vào phương trình thứ nhất ta được $5\cdot \dfrac{1}{2}-3y=-2\Rightarrow y=\dfrac{3}{2}$.\\
	Vậy hệ phương trình đã cho có nghiệm là $\left(\dfrac{1}{2};\dfrac{3}{2}\right)$.
	\item Ta có $\heva{& 2(x-2)+3(1+y)=-2 \\ & 3(x-2)-2(1+y)=-3}\Rightarrow \heva{&2x+3y=-1\\&3x-2y=5.}\Rightarrow \heva{&4x+6y=-2\\&9x-6y=15.}$\\
	Cộng hai vế của hệ mới, ta được $13x=15\Rightarrow x=1$.\\
	Thay $x=1$ vào phương trình thứ nhất của hệ mới ta được $4\cdot 1+6y=-2\Rightarrow y=-1$.\\
	Vậy hệ phương trình đã cho có nghiệm là $(1; -1)$.	
	\end{enumerate}
	}
\end{bt}
%%==========Bài 26
\begin{bt}
	Giải hệ phương trình $\heva{&0{,}5x+0{,}6y=0{,}4\\&0{,}4x-0{,}9y=1{,}7.}$
	\loigiai{
	Nhân hai vế của phương trình với $10$, ta được $\heva{&5x+6y=4\\&4x-9y=17}$. \qquad $(1)$\\
	Ta giải hệ $(1)$. Nhân hai vế của phương trình thứ nhất với $3$ và nhân hai vế của phương trình thứ hai với $2$, ta được hệ
	\begin{center}
	$\heva{&15x+18y=12\\&8x-18y=34}$. \qquad $(2)$
	\end{center}
	Cộng từng vế của hai phương trình của hệ $(2)$ ta được $23x=46$, suy ra $x=2$.\\
	Thế $x=2$ vào phương trình thứ nhất của $(1)$, ta được $5\cdot 2+6y=4$ hay $6y=-6$, suy ra $y=-1$.\\
	Vậy hệ phương trình đã cho có nghiệm là $(2;-1)$.
	}
\end{bt}
%%==========Bài 27
\begin{bt}
	Giải các hệ phương trình sau bằng phương pháp thế
	\begin{listEX}[3]
	\item $\heva{&2x-y=1\\&x-2y=-1}$;
	\item $\heva{&0{,}5x-0{,}5y=0{,}5\\&1{,}2x-1{,}2y=1{,}2}$;
	\item $\heva{&x+3y=-2\\&5x-4y=28}$.
	\end{listEX}
	\loigiai{
	\begin{listEX}[1]
	\item Từ phương trình thứ nhất ta có $y=2x-1$. Thế vào phương trình thứ hai, ta được $x-2(2x-1)=-1$ hay $-3x+2=-1$, suy ra $x=1$.\\
	Từ đó $y=2\cdot 1-1=1$. Vậy hệ phương trình đã cho có nghiệm là $(2;1)$.
	\item Từ phương trình thứ nhất ta có $x=1+y$. \qquad $(1)$\\
	Thế vào phương trình thứ hai, ta được $1{,}2(1+y)-1{,}2y=1{,}2$ hay $0y=0$.\qquad $(2)$\\
	Ta thấy mọi giá trị của $y$ đều thỏa mãn $(2)$.\\
	Với giá trị tùy ý của $y$, giá trị tương ứng của $x$ được tính bởi $(1)$.\\
	Vậy hệ phương trình đã cho có nghiệm là $(1+y;y)$ với $y\in \mathbb{R}$ tùy ý.
	\item Từ phương trình thứ nhất ta có $x=-2-3y$. Thế vào phương trình thứ hai, ta được $5(-2-3y)-4y=28$ hay $-10-19y=28$, suy ra $y=-2$.\\
	Từ đó $x=-2-3\cdot (-2)=4$. Vậy hệ phương trình đã cho có nghiệm là $(4;-2)$.
	\end{listEX}
	}
\end{bt}
%%==========Bài 28
\begin{bt}
	Giải các hệ phương trình sau bằng phương pháp cộng đại số
	\begin{listEX}[3]
	\item $\heva{&5x+7y=-1\\&3x+2y=-5}$;
	\item $\heva{&2x-y=11\\&-0{,}8x+1{,}2y=1}$;
	\item $\heva{&4x-3y=6\\&0{,}4x+0{,}2y=0{,}8}$.
	\end{listEX}
	\loigiai{
	\begin{listEX}[1]
	\item 
	Nhân hai vế của phương trình thứ nhất với $2$ và nhân hai vế của phương trình thứ hai với $-7$, ta được hệ
	\begin{center}
	$\heva{&10x+14y=-2\\&-21x-14y=35.}$ \qquad $(1)$
	\end{center}
	Cộng từng vế của hai phương trình của hệ $(1)$ ta được $-11x=33$, suy ra $x=-3$.\\
	Thế $x=-3$ vào phương trình thứ nhất của $(1)$, ta được $5\cdot (-3)+7y=-1$ hay $7y=14$, suy ra $y=2$.\\
	Vậy hệ phương trình đã cho có nghiệm là $(-3;2)$.
	\item 
	Nhân hai vế của phương trình thứ nhất với $4$ và nhân hai vế của phương trình thứ hai với $10$, ta được hệ
	\begin{center}
	$\heva{&8x-4y=44\\&-8x+12y=10.}$ \qquad $(1)$
	\end{center}
	Cộng từng vế của hai phương trình của hệ $(1)$ ta được $8y=54$, suy ra $y=6{,}75$.\\
	Thế $y=6{,}75$ vào phương trình thứ nhất của $(1)$, ta được $2x-6{,}75=11$ hay $2x=17{,}75$, suy ra $x=8{,}875$.\\
	Vậy hệ phương trình đã cho có nghiệm là $(8{,}875;6{,}75)$.
	\item 
	Nhân hai vế của phương trình thứ nhất với $2$ và nhân hai vế của phương trình thứ hai với $30$, ta được hệ
	\begin{center}
	$\heva{&8x-6y=12\\&12x+6y=24}$. \qquad $(1)$
	\end{center}
	Cộng từng vế của hai phương trình của hệ $(1)$ ta được $20x=36$, suy ra $x=1{,}8$.\\
	Thế $x=-3$ vào phương trình thứ nhất của $(1)$, ta được $4\cdot 1{,}8-3y=6$ hay $-3y=-1{,}2$, suy ra $y=0{,}4$.\\
	Vậy hệ phương trình đã cho có nghiệm là $(1{,}8;0{,}4)$.
	\end{listEX}
	}
\end{bt}
%%==========Bài 29
\begin{bt}%[9D3B7]
	Giải các hệ phương trình sau
	\begin{enumEX}{2}
	\item $\heva{& \dfrac{7}{\sqrt{x - 7}} - \dfrac{4}{\sqrt{y + 6}}=\dfrac{5}{3} \\ & \dfrac{5}{\sqrt{x - 7}} + \dfrac{3}{\sqrt{y + 6}}= 2\dfrac{1}{6}}$
	\item $\heva{& \dfrac{4}{x + y - 1} + \dfrac{5}{y - 2x - 3}=\dfrac{5}{2} \\ & \dfrac{3}{x + y - 1} - \dfrac{1}{y - 2x - 3}=\dfrac{7}{5}}$
	\end{enumEX}
	\loigiai
	{
	\begin{enumerate}
	\item Điều kiện $x > 7$; $y > -6$.\\
	Đặt $\dfrac{1}{\sqrt{x - 7}} = u > 0$; $\dfrac{1}{\sqrt{y + 6}}= v > 0$.\\
	Hệ phương trình có dạng 
	$\begin{aligned}[t]
	\heva{& 7u - 4v =\dfrac{5}{3} \\ & 5u + 3v = 2 \dfrac{1}{6}.}
	\end{aligned}$\\
	Giải hệ phương trình ta được $\heva{& u = \dfrac{1}{3} \\ & v = \dfrac{1}{6}}$ (thỏa mãn điều kiện).\\
	Suy ra $\heva{& \sqrt{x - 7} = 3 \\ & \sqrt{y + 6} = 6} \Rightarrow \heva{& x = 16 \\ & y = 30}$ (thỏa mãn điều kiện).\\
	Vậy hệ phương trình có nghiệm $(16 ; 30)$.
	\item Đặt $\dfrac{1}{x + y - 1}= u$; $\dfrac{1}{y - 2x - 3} = v$. (Điều kiện $x + y - 1 \neq 0$; $y - 2x - 3 \neq 0$).\\
	Hệ phương trình có dạng $\heva{& 4u + 5v =\dfrac{5}{2} \\ & 3u - v =\dfrac{7}{5}.}$\\
	Giải hệ phương trình ta được $\heva{& u = \dfrac{1}{2} \\ & v = \dfrac{1}{10}.}$\\
	Suy ra $\heva{& x + y - 1 = 2 \\ & y - 2x - 3 = 10} \Rightarrow \heva{& x + y = 3 \\ & -2x + y = 13} \Rightarrow \heva{& x = - \dfrac{10}{3} \\ & y = 6 \dfrac{1}{3}}$ (thỏa mãn điều kiện).\\
	Vậy hệ phương trình có nghiệm $\left( - \dfrac{10}{3};6 \dfrac{1}{3} \right) $.
	\end{enumerate}
	}
\end{bt}
%%==========Bài 30
\begin{bt}%[9D3B7]
	Giải các hệ phương trình sau	
	\begin{enumEX}{2}
	\item $\heva{& x + y + z = 8 \\ & 3x - 2y + z = 1 \\ & 4x + y + 2z = 19}$
	\item $\heva{& 2x + y + z = 23 \\ & x + 2y + z = 20 \\ & x + y + 2z = 17}$
	\end{enumEX}
	\loigiai
	{
	\begin{enumerate}
	\item $\heva{& x + y + z = 8 &(1) \\ & 3x - 2y + z = 1 &(2) \\ & 4x + y + 2z = 19 &(3)}$\\
	Từ phương trình ($1$) ta có $z = 8 - x - y$ thay vào phương trình ($2$), ($3$) ta có
	\[\heva{& 3x - 2y + 8 - x - y = 1 \\ & 4x + y + 2 \cdot (8 - x - y) = 19} \Rightarrow \heva{& 2x - 3y = -7 \\ & 2x - y = 3} \Rightarrow \heva{& x = 4 \\ & y = 5.}\]
	Suy ra $z = 8 - 4 - 5 = -1$.\\
	Vậy hệ có nghiệm là $(4; 5; -1)$.
	\item $\heva{& 2x + y + z = 23 &(1)\\ & x + 2y + z = 20 &(2)\\ & x + y + 2z = 17 &(3)}$\\
	Từ phương trình ($1$), ($2$), ($3$) cộng vế với vế ta có
	\[4x + 4y + 4z = 60 \Rightarrow x + y + z = 15. \tag{4}\]
	\begin{itemize}
	\item Từ (1) và (4) $\Rightarrow x + 15 = 23 \Rightarrow x = 8$.
	\item Từ (2) và (4) $\Rightarrow y + 15 = 20 \Rightarrow x = 5$.
	\item Từ (3) và (4) $\Rightarrow z + 15 = 17 \Rightarrow z = 2$.
	\end{itemize}
	Vậy hệ có nghiệm là $(8; 5; 2)$.
	\end{enumerate}
	}
\end{bt}
%%==========Bài 31
\begin{bt}%[9D3B7]
	Với giá trị $m \neq 0$ nào thì hệ $\heva{& mx - y = 2 \\ & 3x + my = 5}$ có nghiệm $(x;y)$ thỏa mãn \[x + y = 1 - \dfrac{m^2}{m^2 + 3}.\]
	\loigiai
	{
	\[\heva{& mx - y = 2 &(1)\\ & 3x + my = 5 &(2)}\]
	\begin{itemize}
	\item Từ phương trình $(1) \Rightarrow y = mx - 2$, thay vào phương trình ($2$) ta có
	\[3x + m(m x - 2) = 5\Rightarrow x =\dfrac{2m + 5}{m^2 + 3}.\]
	Suy ra $y =\dfrac{5m - 6}{m^2 + 3}$.
	\item Ta có $x + y = 1 - \dfrac{m^2}{m^2 + 3}\Rightarrow\dfrac{2m + 5}{m^2 + 3} + \dfrac{5m - 6}{m^2 + 3}= 1 - \dfrac{m^2}{m^2 + 3}\Rightarrow m = - \dfrac{4}{7}$.
	\end{itemize}
	}
\end{bt}
%%==========Bài 32
\begin{bt}%[9D3B7]
	Xác định $a$ để hệ $\heva{& 2x + y = a + 2 \\ & x - y = a}$ có nghiệm $(x;y)$ thỏa mãn $x < y$.
	\loigiai
	{
	\[\heva{& 2x + y = a + 2 \\ & x - y = a} \Rightarrow \heva{& 3x = 2a + 2 \\ & x - y = a} \Rightarrow \heva{& x = \dfrac{2a + 2}{3} \\ & y = \dfrac{2 - a}{3}.}\]
	Ta có $x < y \Rightarrow \dfrac{2a + 2}{3} < \dfrac{2 - a}{3} \Rightarrow a < 0$.\\
	Vậy với $a < 0$ thì hệ có nghiệm $(x;y)$ thỏa mãn $x < y$.
	}
\end{bt}
%%==========Bài 33
\begin{bt}
	Tìm các hệ số $x$, $y$ trong phản ứng hóa học đã được cân bằng sau:
	\begin{center}
	$3\mathrm{Fe}+x\mathrm{O}_2\to y\mathrm{Fe}_3\mathrm{O}_4$.
	\end{center}
	\loigiai{
	Vì số nguyên tử của $\mathrm{Fe}$ và $\mathrm{O}$ ở cả hai vế của phương trình phản ứng phải bằng nhau nên ta có hệ phương trình
	$\heva{&3=3y\\&2x=4y}$ hay $\heva{&y=1\\&x=2y.}$\\
	Giải hệ này ta được $x=2$, $y=1$.
	}
\end{bt}
%%==========Bài 34
\begin{bt}
	Tìm hai số $a$ và $b$ để đường thẳng $y=ax+b$ đi qua hai điểm $A(-2;-1)$ và $B(2;3)$.
	\loigiai{
	Đường thẳng $y=ax+b$ đi qua điểm $A(-2;-1)$ nên $a(-2)+b=-1$ hay $-2a+b=-1$.\\
	Tương tự, đường thẳng $y=ax+b$ đi qua điểm $B(2;3)$ nên $a\cdot 2+b=3$ hay $2a+b=3$.\\
	Từ đó, ta có hệ phương trình với hai ẩn là $a$ và $b$
	\begin{center}
	$\heva{&-2a+b=-1\\&2a+b=3}$.
	\end{center} 
	Cộng từng vế hai phương trình của hệ, ta được $2b=2$, suy ra $b=1$.\\
	Thay $b=1$ vào phương trình thứ nhất, ta có $-2a+1=-1$, suy ra $a=1$.\\
	Vậy ta có đường thẳng $y=x+1$.
	}
\end{bt}
%%==========Bài 35
\begin{bt}
	Tìm các hệ số $x$, $y$ trong phản ứng hóa học đã được cân bằng sau:
	\begin{center}
	$4\mathrm{Al}+x\mathrm{O}_2\to y\mathrm{Al}_2\mathrm{O}_3$.
	\end{center}
	\loigiai{
	Vì số nguyên tử của $\mathrm{Al}$ và $\mathrm{O}$ ở cả hai vế của phương trình phản ứng phải bằng nhau nên ta có hệ phương trình
	$\heva{&4=2y\\&2x=3y}$ hay $\heva{&y=2\\&2x=3y}$.\\
	Giải hệ này ta được $x=3$, $y=2$.
	}
\end{bt}
%%==========Bài 36
\begin{bt}
	Tìm $a$ và $b$ sao cho hệ phương trình $\heva{&ax+by=1\\&ax+(b-2)y=3}$ có nghiệm là $(1;-2)$.
	\loigiai{
	Thay $(1;-2)$ vào hệ phương trình ta có
	$\heva{&a-2b=1\\&a+(b-2)(-2)=3}\Leftrightarrow \heva{&a-2b=1\\&a-2b=-1.}$\\
	Hệ trên vô nghiệm nên không tồn tại giá trị $a$, $b$ thỏa đề bài.
	}
\end{bt}
%%==========Bài 37
\begin{bt}%[9D3B7]
	Giải các hệ phương trình sau
	\begin{enumEX}{2}
	\item $\heva{& \dfrac{5x}{x + 4} + \dfrac{2y}{2y - 3}= 27 \\ & \dfrac{2x}{x + 4} - \dfrac{6y}{2y - 3}= 4}$
	\item $\heva{& \dfrac{3}{x + 2} - \dfrac{y}{y + 1}= - 1 \\ & \dfrac{x}{x + 2} + \dfrac{2}{y + 1}= - \dfrac{5}{3}}$
	\end{enumEX}
	\loigiai
	{
	\begin{enumerate}
	\item Đặt $\dfrac{x}{x + 4}= u$; $\dfrac{y}{2y - 3}= v$, điều kiện $x \neq -4$; $y \neq \dfrac{3}{2}$.\\
	Hệ phương trình có dạng $\heva{& 5u + 2v = 27 \\ & 2u - 6v = 4} \Rightarrow \heva{& u = 5 \\ & v = 1.}$\\
	Từ đó suy ra nghiệm của hệ là $\left( -5; 3\right) $.
	\item Biến đổi hệ phương trình dưới dạng $\heva{& \dfrac{3}{x + 2} + \dfrac{1}{y + 1}= 0 \\ & - \dfrac{2}{x + 2} + \dfrac{2}{y + 1}= - \dfrac{8}{3}.}$\\
	Đặt $\dfrac{1}{x + 2}= u$; $\dfrac{1}{y + 1}= v$, điều kiện $x \neq -2$; $y \neq -1$.\\
	Hệ phương trình có dạng $\heva{& 3u + v = 0 \\ & -2u + 2v = -\dfrac{8}{3}} \Rightarrow \heva{& u = \dfrac{1}{3} \\ & v = -1.}$\\
	Từ đó suy ra nghiệm của hệ là $\left( 1; -2\right)$. 
	\end{enumerate}
	}
\end{bt}
%%==========Bài 38
\begin{bt}%[9D3K7]
	Giải các hệ phương trình sau
	\begin{enumEX}{2}
	\item $\heva{& x + y + z = 6 \\ & 2x + 2y - 3z = -8 \\ & 3x - y + 2z = 46}$
	\item $\heva{& x + y - z = 2 \\ & y + z - x = 6 \\& z + x - y = 4}$
	\end{enumEX}	
	\loigiai
	{
	\begin{enumerate}
	\item Từ phương trình đầu suy ra $z = 6 - x - y$ thay vào hai phương trình sau và thu gọn, ta được $$\heva{& x + y = 2 \\ & x - 3y = 34} \Rightarrow \heva{& x = 10 \\ & y = -8}. \text{ Do đó }z = 6 - 10 + 8 = 4.$$
	Vậy hệ phương trình có nghiệm là $(x; y; z) = (10; -8; 4)$.
	\item Cộng vế với vế của ba phương trình ta được $x + y + z = 12$, sau đó kết hợp với từng phương trình, ta tìm được nghiệm của hệ là $(x; y; z) = (3; 4; 5)$. 
	\end{enumerate}
	}
\end{bt}
%%==========Bài 39
\begin{bt}
	Tìm hai số nguyên dương biết tổng của chúng bằng $1006$ , nếu lấy số lớn chia cho số bé được thương là $2$ và số dư là $124$.
	\loigiai{
	Gọi $x$, $y$ lần lượt là số lớn và số bé với $x,y \in \mathbb{Z}^+$.\\
	Tổng của chúng bằng $1006$, nên ta có phương trình $x+y=1006$.\\
	Số lớn chia cho số bé được thương là $2$ và số dư là $124$, nên ta có phương trình $x=2y+124$.\\
	Ta có hệ phương trình $$\heva{&x+y=1006\\&x=2y+124}\Rightarrow \heva{&x=712\\&y=294.}$$ 
	Vậy hai số đó là $712$ và $294$.
	}
\end{bt}
%%==========Bài 40
\begin{bt}
	Tìm số tự nhiên $N$ có hai chữ số, biết rằng nếu viết thêm chữ số $3$ vào giữa hai chữ số của số $N$ thì được một số lớn hơn số $2N$ là $585$ đơn vị, và nếu viết hai chữ số của số $N$ theo thứ tự ngược lại thì được một số nhỏ hơn số $N$ là $18$ đơn vị.
	\loigiai{
	Gọi số có hai chữ số cần tìm là $\overline{ab}$, với $a,b\in \mathbb{N}^{*}$.\\
	Từ giả thiết bài toán ta có hệ \[\heva{&100a+30+b-2(10a+b)=585\\&10a+b-(10b+a)=18}\Rightarrow \heva{&80a-b=555\\&9a-9b=18}\Rightarrow \heva{&80a-b=555\quad(1)\\&-a+b=-2\quad(2).}\]
	Cộng hai vế của phương trình $(1)$ và $(2)$ của hệ ta được $79a=553\Rightarrow a=7$.\\
	Thay $a=7$ vào phương trình thử $(2)$ của hệ ta được $-7+b=-2\Rightarrow b=5$.\\
	Các giá trị $a=7$ và $b=5$ đều thỏa mãn điều kiện bài toán.\\
	Vậy số $N$ cần tìm là $75$.
	}
\end{bt}
%%==========Bài 41
\begin{bt}
	Trong một đợt khuyến mãi, siêu thị giảm giá cho mặt hàng A là $20 \%$ và mặt hàng B là $15 \%$ so vối giá niêm yết. Một khách hàng mua 2 món hàng A và 1 món hàng B thì phải trả số tiền là $362\,000$ đồng. Nhưng nếu mua trong khung giờ vàng thì mặt hàng A được giảm giá $30 \%$ và mặt hàng B được giảm giá $25 \%$ so vối giá niêm yết. Một khách hàng mua 3 món hàng A và 2 món hàng B trong khung giờ vàng nên phải trả số tiền là $552\,000$ đồng. Tính giá niêm yết của mỗi mặt hàng A và B. 
	\loigiai{
	Gọi $x$, $y$ (đồng) lần lượt là giá niêm yết của mỗi mặt hàng A và B ($x>0$, $y>0$).\\
	Một khách hàng mua 2 món hàng A và 1 món hàng B thì phải trả số tiền là $362\,000$ đồng nên ta có
	$$80\%x\cdot2+85\%y=362\,000\text{ hay } 1{,}6x+0{,}85y=362\,000.\qquad (1)$$
	Trong giờ vàng, khách hàng mua 3 món hàng A và 2 món hàng B phải trả số tiền là $552\,000$ đồng nên ta có
	$$70\%x\cdot3+75\%y\cdot2=552\,000\text{ hay } 2{,}1x+1{,}5y=552\,000.\qquad (2)$$
	Từ (1) và (2), ta có hệ phương trình $\heva{&1{,}6x+0{,}85y=362\,000\\&2{,}1x+1{,}5y=552\,000.}$\\
	Giải hệ phương trình trên, ta được $ \heva{&x=120\,000 \\&y=200\,000.}$\\
	Vậy giá niêm yết của mặt hàng A là $120\,000$ đồng, mặt hàng B là $200\,000$ đồng.
	}
\end{bt}
%%==========Bài 42
\begin{bt}
	Một nhóm công nhân cần phải cắt cỏ ở một số mặt sân cỏ. Nếu nhóm công nhân đó sử dụng 3 máy cắt cỏ ngồi lái và 2 máy cắt cỏ đẩy tay trong 10 phút thì cắt được $2990 \mathrm{~m}^2$ cỏ. Nếu nhóm công nhân đó sử dụng 4 máy cắt cỏ ngồi lái và 3 máy cắt cỏ đẩy tay trong 10 phút thì cắt được $4060 \mathrm{~m}^2$ cỏ. Hỏi trong 10 phút, mỗi loại máy trên sẽ cắt được bao nhiêu mét vuông cỏ?
	\loigiai{
	Gọi $x$, $y$ lần lượt là số mét vuông cỏ cắt được trong 10 phút của máy cắt cỏ ngồi lái và máy cắt cỏ đẩy tay ($x>0$, $y>0$).\\
	Vì khi sử dụng 3 máy cắt cỏ ngồi lái và 2 máy cắt cỏ đẩy tay trong 10 phút thì cắt được $2990 \mathrm{~m}^2$ cỏ nên ta có phương trình
	$$3x+2y=2990.\qquad (1)$$
	Vì khi sử dụng 4 máy cắt cỏ ngồi lái và 3 máy cắt cỏ đẩy tay trong 10 phút thì cắt được $4060 \mathrm{~m}^2$ cỏ nên ta có phương trình
	$$4x+3y=4060.\qquad (2)$$
	Từ (1) và (2), ta có hệ phương trình
	$\heva{&3x+2y=2990\\& 4x+3y=4060.}$\\
	Giải hệ trên ta được $ \heva{&x=850\\& y=220.}$\\
	Vậy trong 10 phút, máy cắt cỏ ngồi lái cắt được $850 \mathrm{~m}^2$ và máy cắt cỏ đẩy tay cắt được $220 \mathrm{~m}^2$.
	}
\end{bt} 
%%==========Bài 43
\begin{bt}
	Tại một buổi biểu diễn nhằm gây quỹ từ thiện, ban tổ chức đã bán được $500$ vé. Trong đó có hai loại vé: vé loại I giá $100\,000$ đồng; vé loại II giá $75\,000$ đồng. Tổng số tiền thu được từ bán vé là $44\,500\,000$ đồng. Tính số vé bán ra của mổi loại.
	\loigiai{
	Gọi $x$, $y$ lần lượt là số vé loại I, II mà ban tổ chức bán ra ($x \in \mathbb{N}$, $y \in \mathbb{N}$).\\
	Vì ban tổ chức bán được $500$ vé nên ta có phương trình $$x+y=500.\qquad (1)$$
	Vì tổng số tiền thu được từ bán vé là $44\,500\,000$ đồng nên ta có phương trình
	$$100\,000x+ 75\,000y=44\,500\,000.\qquad (2)$$
	Từ (1) và (2), ta có hệ phương trình
	$\heva{&x+y=500\\& 100\,000x+ 75\,000y=44\,500\,000.} $\\
	Giải hệ phương trình trên, ta được $ \heva{&x=280\\& y=220.}$\\
	Vậy số vé loại I bán được là $280$ vé, số vé loại II bán được là $220$ vé.
	}
\end{bt}
%%==========Bài 44
\begin{bt}
	Nhà máy luyện thép hiện có sẵn loại thép chứa $10 \%$ carbon và loại thép chứa $20 \%$ carbon. Giả sử trong quá trình luyện thép các nguyên liệu không bị hao hụt. Tính khối lượng thép mỗi loại cần dùng để luyện được $1000$ tấn thép chứa $16 \%$ carbon từ hai loại thép trên.
	\loigiai{
	Gọi khối lượng thép chứa $10 \%$ carbon là $x$ tấn và khối lượng thép chứa $20 \%$ carbon là $y$ tấn với $x,y >0$.\\
	Khối lượng carbon có trong $x$ tấn thép $10 \%$ carbon là $10 \% x=0{,}1x$.\\
	Khối lượng carbon có trong $y$ tấn thép $20 \%$ carbon là $20 \% x=0{,}2y$.\\
	$1000$ tấn thép chứa $16 \%$ carbon, nên ta có phương trình $0{,}1x+0{,}2y=1000\cdot 16 \%$.\\
	Khối lượng thép mỗi loại cần dùng để luyện được $1000$ tấn, nên ta có phương trình $x+y=1000$.\\
	Ta có hệ phương trình $$\heva{&0{,}1x+0{,}2y=1000\cdot 16 \%\\&x+y=1000}\Rightarrow \heva{&x+2y=1600\\&x+y=1000}\Rightarrow \heva{&x=400\\&y=600.}$$ 
	Vậy cần $400$ tấn thép loại $10 \%$ carbon và $600$ tấn thép loại $20 \%$ carbon.
	}
\end{bt}
%%==========Bài 45
\begin{bt}
	Trong một xí nghiệp, hai tổ công nhân $A$ và $B$ lắp ráp cùng một loại bộ linh kiện điện tử. Nếu tổ $A$ lắp ráp trong $5$ ngày, tổ $B$ lắp ráp trong $4$ ngày thì xong $1900$ bộ linh kiện. Biết rằng mỗi ngày tổ $A$ lắp ráp nhiều hơn tổ $B$ là $20$ bộ linh kiện. Hỏi trong một ngày mỗi tổ ráp được bao nhiêu bộ linh kiện điện tử? (Năng suất lắp ráp của mỗi tổ trong các ngày là như nhau).
	\loigiai{
	Gọi $x$, $y$ lần lượt là bộ linh kiện điện tử của tổ $A$ và $B$ trong một ngày với $x,y \in \mathbb{N}^*$.\\
	Mỗi ngày tổ $A$ lắp ráp nhiều hơn tổ $B$ là $20$ bộ linh kiện, nên ta có phương trình $x-y=20$.\\
	Tổ $A$ lắp ráp trong $5$ ngày, tổ $B$ lắp ráp trong $4$ ngày thì xong $1900$ bộ linh kiện, nên ta có phương trình $5x+4y=1900$.\\
	Ta có hệ phương trình $$\heva{&x-y=20\\&5x+4y=1900}\Rightarrow \heva{&x=220\\&y=200.}$$ 
	Vậy mỗi ngày tổ $A$ ráp được $220$ bộ linh kiện điện tử, tổ $B$ ráp được $200$ bộ linh kiện điện tử.
	}
\end{bt}
%%==========Bài 46
\begin{bt}
	Giải bài toán cố sau:
	\begin{center}
	Quýt, cam mười bảy quả tươi\\
	Đem chia cho một trăm người cùng vui\\
	Chia ba mỗi quá quýt rồi\\
	Còn cam mỗi quả chia mười vừa xinh\\
	Trăm người, trăm miếng ngọt lành\\
	Quýt, cam mỗi loại tính rành là bao?
	\end{center}
	\loigiai{
	Gọi $x$, $y$ lần lượt là số quả quýt và số quả cam với $x,y \in \mathbb{N}^*$.\\
	Quýt, cam mười bảy quả tươi, nên ta có phương trình $x+y=17$.\\
	Trăm người, trăm miếng ngọt lành, nên ta có phương trình $3x+10y=100$.\\
	Ta có hệ phương trình $$\heva{&x+y=17\\&3x+10y=100}\Rightarrow \heva{&x=10\\&y=7.}$$ 
	Vậy có $10$ quả cam và $7$ quả quýt.
	}
\end{bt}
%%==========Bài 47
\begin{bt}
	Nhân kỉ niệm ngày Quốc khánh $2 / 9$, một nhà sách giảm giá mỗi cây bút bi là $20 \%$ và mỗi quyển vở là $10 \%$ so với giá niêm yết. Bạn Thanh vào nhà sách mua $20$ quyển vở và $10$ cây bút bi. Khi tính tiền, bạn Thanh đưa $175000$ đồng và được trả lại $3000$ đồng. Tính giá niêm yết của mỗi quyển vở và mỗi cây bút bi, biết rằng tổng số tiền phải trả nếu không được giảm giá là $195000$ đồng.
	\loigiai{
	Gọi $x$, $y$ lần lượt là giá tiền niêm yết của mỗi quyển vở và mỗi cây bút bi với $x,y >0$.\\
	Số tiền mua $20$ quyển vở và $10$ cây bút bi khi giảm giá là $$20\cdot 90\%x+10\cdot 80\%y=175000-3000\Rightarrow 18x+8y=172000.$$
	Tổng số tiền phải trả nếu không được giảm giá là $195000$ đồng, nên ta có phương trình $$20x+10y=195000.$$
	Ta có hệ phương trình $$\heva{&18x+8y=172000\\&20x+10y=195000}\Rightarrow \heva{&x=8000\\&y=3500.}$$ 
	Vậy giá niêm yết của mỗi quyển vở là $8000$ đồng, mỗi cây bút bi và $3500$ đồng.
	}
\end{bt}
%%==========Bài 48
\begin{bt}
	Ở giải bóng đá Ngoại hạng Anh mùa giải $2003 - 2004$, đội Arsenal đã thi đấu $38$ trận mà không thua trận nào và giành chức vô địch với $90$ điểm. Biết rằng với mỗi trận đấu, đội thắng được $3$ điểm, đội thua không có điểm và nếu hai đội hoà nhau thì mỗi đội được $1$ điểm. Mùa giải đó đội Arsenal đã giành được bao nhiêu trận thắng?
	\loigiai{
	Gọi $x$, $y$ lần lượt là số trận thắng và số trận hòa với $x,y \in \mathbb{N}^*$.\\
	Đội Arsenal đã thi đấu $38$ trận mà không thua trận, nên ta có phương trình $x+y=38$.\\
	Mỗi trận đấu, đội thắng được $3$ điểm, đội thua không có điểm và nếu hai đội hoà nhau thì mỗi đội được $1$ điểm, nên ta có phương trình $3x+y=90$.\\
	Ta có hệ phương trình $$\heva{&x+y=38\\&3x+y=90}\Rightarrow \heva{&x=26\\&y=12.}$$ 
	Vậy có $26$ trận thắng và $12$ trận hòa.
	}
\end{bt}
%%==========Bài 49
\begin{bt}
	Trên cánh đồng có diện tích $160$ ha của một đơn vị sản xuất, người ta dành $60$ ha để cấy thí điểm giống lúa mới, còn lại vẫn cấy giống lúa cũ. Khi thu hoạch, đấu tiên người ta gặt $8$ ha giống lúa cũ và $7$ ha giống lúa mới để đối chứng. Kết quả là $7$ ha giống lúa mới cho thu hoạch nhiều hơn $8$ ha giống lúa cũ là $2$ tấn thóc. Biết rằng tổng số thóc (cả hai giống) thu hoạch cả vụ trên $160$ ha là $860$ tấn. Hỏi năng suất của mỗi giống lúa trên $1$ ha là bao nhiêu tấn thóc?
	\loigiai{
	Gọi năng suất lúa trên $1$ ha của giống lúa mới là $x$ (tấn), của giống lúa cũ là $y$ (tấn). Điều kiện $x>0$ và $y>0$.\\
	Theo giả thiết ta có hệ phương trình
	\[\heva{&7x-8y=2\\&60x+100y=860}\Rightarrow \heva{&y=\dfrac{7x-2}{8}\quad(1)\\&6x+10y=86.\quad(2)}\]
	Thay $y=\dfrac{7x-2}{8}$ và phương trình $(2)$ ta được $6x+10\cdot \dfrac{7x-2}{8}=86\Rightarrow x=6$.\\
	Thay $x=6$ vào phương trình $(1)$ ta được $y=\dfrac{7\cdot 6-2}{8}=5$.\\
	Các giá trị $x=6$, $y=5$ thỏa mãn điều kiện bài toán.\\
	Vậy năng suất trên $1$ ha của giống lúa mới và lúa cũ lần lượt là $6$ tấn và $5$ tấn.	
	}
\end{bt}
%%==========Bài 50
\begin{bt}%[9D3B7]
	Một hợp tác vận tải có $15$ xe ô tô nhỏ và $10$ xe ô tô lớn thì vận chuyển được $690$ khách. Nếu hợp tác vận tải rút bớt $10$ xe ô tô nhỏ và tăng thêm $4$ xe ô tô lớn thì số khác chuyển được tăng thêm $20$ người. Hỏi mỗi loại xe chở được bao nhiêu người?
	\loigiai
	{
	Gọi số khách mà một ô tô nhỏ chở được là $x$ (khách, $x \in \mathbb{N}$);\\
	Số khách mà một ô tô lớn chở được là $y$ (khách, $y 
	\in \mathbb{N}$).\\
	Theo đề bài ta có hệ phương trình $\heva{& 15x + 10y = 690 \\ & 5x + 14y = 710} \Rightarrow \heva{& x = 16 \\ & y = 45}$ (thỏa mãn điều kiện).\\
	Vậy loại xe ô tô nhỏ chở được $16$ khách, loại xe ô tô lớn chở được $45$ khách.
	}
\end{bt}
%%==========Bài 51
\begin{bt}%[9D3B7]
	Một hình chữ nhật có chu vi $26$ m. Nếu tăng chiều dài thêm $2$ m và tăng chiều rộng thêm $3$m thì diện tích tăng thêm $40$ m$^2$. Tính kích thước của hình chữ nhật.
	\loigiai
	{
	Gọi chiều dài hình chữ nhật là $x$ (m, $x > 0$); chiều rộng hình chữ nhật là $y$ (m, $y > 0$).\\
	Chu vi hình chữ nhật là $26$ m, ta có phương trình $(x + y) \cdot 2 = 26$. \hfill(1)\\
	Nếu tăng chiều dài thêm $2$ m và tăng chiều rộng them $3$m thì diện tích tăng thêm $40 \mbox{m}^2$, ta có phương trình $(x + 2)\cdot (y + 3) = xy + 40$. \hfill(2)\\
	Từ ($1$) và ($2$) ta có hệ phương trình $\heva{& (x + y) \cdot 2 = 26 \\ & (x + 2) \cdot (y + 3) = xy + 40} \Rightarrow \heva{& x = 8 \\ & y = 5}$ thỏa mãn điều kiện.\\
	Vậy kích thước hình chữ nhật là $8$ m; $5$ m.
	}
\end{bt}
%%==========Bài 52
\begin{bt}%[9D3B7]
	Hai trường THCS có tất cả $300$ học sinh dự thi vào lớp $10$ THPT. Biết rằng trường thứ nhất có $75\%$ số học sinh đỗ, trường thứ hai có $60\%$ số học sinh đỗ nên cả hai trường có $207$ học sinh đỗ vào lớp $10$. Hỏi mỗi trường có bao nhiêu học sinh dự thi.	
	\loigiai
	{
	Gọi số học sinh của trường thứ nhất dự thi là $x$ (em; $x \in \mathbb{N}$).\\
	Số học sinh của trường thứ hai dự thi là $y$ (em; $y \in \mathbb{N}$).\\
	Theo đề bài ta có hệ phương trình $\heva{& x + y = 300 \\ & \dfrac{75x}{100} + \dfrac{60y}{100} = 207} \Rightarrow \heva{& x = 180 \\ & y = 120}$ (thỏa mãn điều kiện).\\
	Vậy số học sinh của trường thứ nhất là $180$ em; số học sinh của trường thứ hai là $120$ em.
	}
\end{bt}
%%==========Bài 53
\begin{bt}%[9D3B7]
	Hai tổ sản xuất, tổ I làm trong $6$ ngày, tổ II làm trong $8$ ngày được tất cả $620$ sản phẩm. Biết rằng số sản phẩm tổ I làm trong $4$ ngày đúng bằng số sản phẩm tổ II làm trong $5$ ngày. Hỏi mỗi ngày, mỗi tổ làm được bao nhiêu sản phẩm?	
	\loigiai
	{
	Gọi số sản phẩm tổ I làm trong một ngày là $x$ (sản phẩm; $x \in \mathbb{N}$).\\
	Số sản phẩm tổ II làm trong một ngày là $y$ (sản phẩm; $y \in \mathbb{N}$).\\
	Theo đề bài ta có hệ phương trình $\heva{& 6x + 8y = 620 \\ & 4x = 5y} \Rightarrow \heva{& x = 50 \\ & y = 40}$ (thỏa mãn điều kiện).\\
	Vậy số sản phẩm làm trong một ngày của tổ I là $50$ sản phẩm; tổ II là $40$ sản phẩm.
	}
\end{bt}
%%==========Bài 54
\begin{bt}%[9D3B7]
	Một xe tải lớn chở $10$ chuyến hàng và một xe nhỏ chở $5$ chuyến hàng thì được $60$ tấn. Biết rằng $3$ chuyến của xe lớn chở nhiều hơn $7$ chuyến xe nhỏ là $1$ tấn. Hỏi mỗi xe chở được bao nhiêu tấn hàng một chuyến?
	\loigiai
	{
	Gọi lượng hàng xe lớn chở trong một chuyến là $x$ (tấn; $x \in \mathbb{N}$).\\
	Lượng hàng xe nhỏ chở trong một chuyến là $y$ (tấn; $y \in \mathbb{N}$).\\
	Theo đề bài ta có hệ phương trình $\heva{& 10x + 5y = 60 \\ & 3x - 7y = 1} \Rightarrow \heva{& x = 5 \\ & y = 2}$ (thỏa mãn điều kiện).\\
	Vậy xe lớn một chuyến chở được $5$ tấn, xe nhỏ chở được $2$ tấn.
	}
\end{bt}
%%==========Bài 55
\begin{bt}%[9D3K7]
	Hai phân xưởng của nhà máy theo kế hoạch phải làm $300$ sản phẩm. Nhưng phân xưởng I đã thực hiện $110\%$ kế hoạch, phân xưởng II đã thực hiện $120\%$ kế hoạch, do đó đã sản xuất được $340$ sản phẩm. Tính số sản phẩm mỗi phân xưởng phải làm theo kế hoạch.	
	\loigiai
	{
	Gọi số sản phẩm phân xưởng I làm theo kế hoạch là $x$ (sản phẩm, $x > 0$).\\
	Số sản phẩm phân xưởng II làm theo kế hoạch là $y$ (sản phẩm, $y > 0$).\\
	Theo đề bài ta có hệ phương trình $\heva{& x + y = 300 \\ & \dfrac{110x}{100} + \dfrac{120y}{100}= 340} \Rightarrow \heva{& x = 200 \\ & y = 100}$ (thỏa mãn điều kiện).\\
	Vậy phân xưởng I làm theo kế hoạch là $200$ sản phẩm, phân xưởng II làm theo kế hoạch là $100$ sản phẩm. 
	}
\end{bt}
%%==========Bài 56
\begin{bt}
	Một người mua hai loại hàng và phải trả tổng cộng là $21{,}7$ triệu đồng, kể cả thuế giá trị gia tăng (VAT) với mức $10\%$ đối với loại hàng thứ nhất và $8 \%$ đối với loại hàng thứ hai. Nếu thuế VAT là $9 \%$ đối với cả hai loại hàng thì người đó phải trả tổng cộng $21{,}8$ triệu đồng. Hỏi nếu không kể thuế VAT thì người đó phải trả bao nhiêu tiền cho mỗi loại hàng?
	\loigiai{
	Giả sử giá của loại hàng thứ nhất và thứ hai không tính VAT lần lượt là $x$ triệu đồng và $y$ triệu đồng. Điều kiện $0<x<2{,}17$, $0<x<2{,}17$.\\
	Nếu áp dụng mức thuế VAT $10\%$ đối với loại hàng thứ nhất và $8\%$ đối với loại hàng thứ hai thì
	\begin{itemize}
	\item Giá mặt hàng thứ nhất tính cả thuế VAT là $x+10 \% \cdot x=x+0{,}1 x=1{,}1 x$.
	\item Giá mặt hàng thứ hai tính cả thuế VAT là $y+8 \% \cdot y=y+0{,}08 y=1{,}08 y$.	
	\end{itemize}
	Số tiền người đó phải trả là $2{,}17$ triệu đồng nên ta có phương trình: \[1{,}1 x + 1{,}08y =2{,}17\Rightarrow 110 x + 108y =217.\]
	Nếu áp dụng mức thuế VAT $9\%$ đối với cả hai loại hàng thì:
	\begin{itemize}
	\item Giá mặt hàng thứ nhất tính cả thuế VAT là $x+9 \% \cdot x=x+0{,}09 x=1{,}09x$.
	\item Giá mặt hàng thứ hai tính cả thuế VAT là $y + 9\% y=y+0{,}09 y=1{,}09 y$.	
	\end{itemize}
	Số tiền người đó phải trả là $2{,}18$ triệu đồng nên ta có phương trình
	\[1{,}09 x+1{,}09 y=2{,}18 \Rightarrow x+y=2.\]
	Khi đó ta có hệ phương trình
	\[\heva{&110 x + 108y =217\\&x+y=2}\Rightarrow \heva{&110 x + 108y =217&\quad(1)\\&y=2-2x.&\quad(2)}\]
	Thay $(1)$ vào $(2)$ ta được $110x+108(2-2x)=217\Rightarrow x=0{,}5$.\\
	Thay $x=0{,}5$ vào phương trình $(2)$, ta được $0{,}5+y=2\Rightarrow y=1{,}5$.\\
	Vậy nếu không kể thuế VAT thì người đó phải trả: $0{,}5$ triệu cho loại hàng thứ nhất và $1{,}5$ triệu cho loại hàng thứ hai.
	}
\end{bt}
%%==========Bài 57
\begin{bt}
	Một ca nô đi xuôi dòng từ địa điểm A đến địa điểm B và lại ngược dòng từ địa điểm B về địa điểm A mất 9 giờ, tốc độ của ca nô khi nước yên lặng không đổi trên suốt quãng đường đó và tốc độ của dòng nước cũng không đổi khi ca nô chuyển động. Biết thời gian ca nô đi xuôi dòng $5 \mathrm{~km}$ bằng thời gian ca nô đi ngược dòng $4 \mathrm{~km}$ và quãng đường AB là $160 \mathrm{~km}$. Tính tốc độ của ca nô khi nước yên lặng và tốc độ của dòng nước. 
	\loigiai{
	Gọi vận tốc xuôi dòng của ca nô là $x$ (km/h), thời gian đi xuôi dòng của ca nô là $y$ (giờ) ($x>0$, $y>0$).\\
	Vì quãng đường AB là $160 \mathrm{~km}$ nên ta có phương trình $xy=160$.\qquad (1)\\
	Vì ca nô đi xuôi dòng $5 \mathrm{~km}$ bằng thời gian ca nô đi ngược dòng $4 \mathrm{~km}$ nên vận tốc ngược dòng là $\dfrac{4}{5}x$.\\
	Vì ca nô đi xuôi dòng từ địa điểm A đến địa điểm B và lại ngược dòng từ địa điểm B về địa điểm A mất 9 giờ nên thời gian đi ngược dòng là $9-y$.\\
	Khi đó ta có phương trình $\dfrac{4}{5}x\cdot(9-y)=160$.\qquad(2).\\
	Từ (1) và (2), ta có hệ phương trình
	$$\heva{&xy=160\\&\dfrac{4}{5}x\cdot(9-y)=160}\text{ hay } \heva{&xy=160 \\&\dfrac{36}{5}x-\dfrac{4}{5}xy=160}$$
	Thay $xy=160$ vào phương trình $\dfrac{36}{5}x-\dfrac{4}{5}xy=160$, ta được $\dfrac{36}{5}x-\dfrac{4}{5}\cdot 160=160$.\\
	Giải phương trình trên, ta được $x=40$ (km/h). Do đó $x=\dfrac{160}{40}=4$ (giờ).\\
	Do đó vận tốc xuôi dòng là $40\mathrm{~km/h}$, vận tốc ngược dòng $\dfrac{4}{5}\cdot40=32\mathrm{~km/h}$.\\
	Vận tốc của dòng nước là $(40-32):2=4 \mathrm{~km/h}$.\\
	Vận tốc của ca nô khi nước yên lặng là $40-4=36 \mathrm{~km/h}$.
	}
\end{bt}
%%==========Bài 58
\begin{bt}
	Hai vật chuyển động đều trên một đường tròn đường kính $20$ cm, xuất phát cùng một lúc, từ cùng một điểm. Nếu chuyển động ngược chiều thì cứ sau $4$ giây chúng lại gặp nhau. Nếu chuyển động cùng chiều thì cứ $20$ giây chúng lại gặp nhau. Tính vận tốc (cm/s) của mỗi vật.
	\loigiai{
	Gọi vận tốc của hai vật lần lượt là $x$ (cm/s) và $y$ (cm/s). Điều kiện $x$, $y>0$.\\
	Chu vi vòng tròn là $20\cdot\pi$ (cm).\\
	Khi chuyển động cùng chiều, cứ $20$ giây chúng lại gặp nhau, nghĩa là quãng đường $2$ vật đi được trong $20$ giây chênh lệch nhau đúng bằng $1$ vòng tròn.Do đó, ta có phương trình: $20x-20y=20\pi$.\\
	Khi chuyển động ngược chiều, cứ $4$ giây chúng lại gặp nhau, nghĩa là tổng quãng đường hai vật đi được trong $4$ giây là đúng $1$ vòng tròn.
	Khi đó, ta có phương trình: $4x+4y=20\pi$.\\
	Ta có hệ phương trình:
	\[\heva{&20x-20y=20\pi\\&4x+4y=20\pi}\Rightarrow \heva{&x-y=\pi&\quad(1) \\
	&x+y=5 \pi. &\quad(2)}\]
	Cộng hai vế của phương trình $(1)$ và $(2)$ ta được $2x=6\pi \Rightarrow x=3\pi$ (thỏa mãn).\\
	Thay $x=3\pi$ vào phương trình $(2)$ ta được $3\pi+y=5\pi \Rightarrow y=2\pi$ (thỏa mãn).\\
	Vậy vận tốc của hai vật là $3\pi$ cm/s, $2\pi$ cm/s.
	}
\end{bt}
%%==========Bài 59
\begin{bt}%[9D3B7]
	Hai vòi nước cùng chảy thì sau $5$h $50$ phút sẽ đầy bể. Nếu để hai vòi cùng chảy trong $5$ giờ rồi khóa vòi thứ nhất lại thì vòi thứ hai phải chảy trong $2$ giờ nữa mới đầy bể. Tính xem nếu để mỗi vòi chảy một mình thì trong bao lâu sẽ đầy bể?
	\loigiai
	{
	Đổi $5$h $50$ phút $= 5 \dfrac{5}{6}$ giờ $= \dfrac{35}{6}$ giờ.\\
	Gọi thời gian vòi một chảy một mình đẩy bể hết $x$ (giờ, $x > 0$);\\
	Thời gian vòi hai chảy một mình đầy bể hết $y$ (giờ, $y > 0$).\\
	Hai vòi chảy thì sau $5$h $50$ phút $\left( \dfrac{35}{6} \mbox{ giờ}\right)$ sẽ đầy bể, ta có phương trình 
	\[ \dfrac{1}{x} + \dfrac{1}{y}=\dfrac{6}{35}. \tag{1}\] 	
	Nếu hai vòi cùng chảy trong $5$ giờ rồi khóa một vòi lại thì vòi hai chảy tiếp $2$ giờ mới đầy bể, ta có phương trình
	\[ 5\cdot\left(\dfrac{1}{x} + \dfrac{1}{y}\right) + \dfrac{2}{y}= 1. \tag{2}\]
	Từ ($1$) và ($2$) ta có hệ phương trình $\heva{& \dfrac{1}{x} + \dfrac{1}{y}=\dfrac{6}{35} \\ & 5\cdot\left(\dfrac{1}{x} + \dfrac{1}{y}\right) + \dfrac{2}{y}= 1} \Rightarrow \heva{& x = 10 \\ & y = 14}$ (thỏa mãn điều kiện).\\
	Vậy vòi một chảy một mình đầy bể trong $10$ giờ, vòi hai chảy đầy bể một mình trong $14$ giờ.
	}
\end{bt}
%%==========Bài 60
\begin{bt}%[9D3K7]
	Hai người thợ cùng làm chung một công việc trong $7$ giờ $12$ phút thì xong. Nếu người thứ nhất làm trong $5$ giờ và người thứ hai làm trong $6$ giờ thì cả hai chỉ làm được $\dfrac{3}{4}$ công việc. Hỏi mỗi người làm một mình trong thời gian bao lâu hoàn thành công việc đó?	
	\loigiai
	{
	Ta có $7$ giờ $12$ phút $= \dfrac{36}{5}$ giờ.\\
	Gọi thời gian người thứ nhất làm một mình hoàn thành công việc hết $x$ (giờ, $x \in \mathbb{N}$).\\
	Người thứ hai làm một mình hoàn thành công việc hết $y$ (giờ, $y \in \mathbb{N}$).\\
	Theo đề bài ta có hệ phương trình $\heva{& \dfrac{1}{x} + \dfrac{1}{y}=\dfrac{5}{36} \\ & \dfrac{5}{x} + \dfrac{6}{y}=\dfrac{3}{4}} \Rightarrow \heva{& x = 12 \\ & y = 18}$ (thỏa mãn điều kiện).\\
	Vậy người thứ nhất làm một mình hoàn thành công việc hết $12$ giờ; người thứ hai làm một mình hoàn thành công việc hết $18$ giờ.
	}
\end{bt}
%%==========Bài 61
\begin{bt}
	Cân bằng các phương trình hoá học sau bằng phương pháp đại số.
	\begin{listEX}[3]
	\item $Fe+Cl_2 \rightarrow FeCl_3$.
	\item $SO_2+O_2 \underset{V_2 O_5}{\stackrel{t^{\circ}}{\longrightarrow}} SO_3$.
	\item $Al+O_2 \rightarrow Al_2 O_3$.
	\end{listEX}
	\loigiai{
	\begin{enumerate}
	\item Gọi $x$, $y$ lần lượt là hệ số của $Fe$ và $Cl_2$ thỏa mãn cân bằng phương trình hóa học $x,y >0$.
	$$xFe+yCl_2 \rightarrow FeCl_3.$$
	Cân bằng số nguyên tử $Fe$ và số nguyên tử $Cl$ ở hai vế ta có hệ
	$$\heva{&x=1\\&2y=3}\Rightarrow \heva{&x=1\\&y=\dfrac{3}{2}.}$$
	Khi đó ta có $$Fe+\dfrac{3}{2}Cl_2 \rightarrow FeCl_3.$$
	Do hệ số của phương trình hóa học phải là số nguyên nên nhân hai về phương trình hóa học trên với $2$, ta được
	$$2Fe+3Cl_2 \rightarrow 2FeCl_3.$$
	\item Gọi $x$, $y$ lần lượt là hệ số của $SO_2$ và $O_2$ thỏa mãn cân bằng phương trình hóa học $x,y >0$.
	$$xSO_2+yO_2 \underset{V_2 O_5}{\stackrel{t^{\circ}}{\longrightarrow}} SO_3.$$
	Cân bằng số nguyên tử $S$ và số nguyên tử $O$ ở hai vế ta có hệ
	$$\heva{&x=1\\&2x+2y=3}\Rightarrow \heva{&x=1\\&y=\dfrac{1}{2}.}$$
	Khi đó ta có $$SO_2+\dfrac{1}{2}O_2 \underset{V_2 O_5}{\stackrel{t^{\circ}}{\longrightarrow}} SO_3.$$
	Do hệ số của phương trình hóa học phải là số nguyên nên nhân hai về phương trình hóa học trên với $2$, ta được
	$$2SO_2+O_2 \underset{V_2 O_5}{\stackrel{t^{\circ}}{\longrightarrow}} 2SO_3.$$
	\item Gọi $x$, $y$ lần lượt là hệ số của $Al$ và $O_2$ thỏa mãn cân bằng phương trình hóa học $x,y >0$.
	$$xAl+yO_2 \rightarrow Al_2 O_3.$$
	Cân bằng số nguyên tử $Al$ và số nguyên tử $O$ ở hai vế ta có hệ
	$$\heva{&x=2\\&2y=3}\Rightarrow \heva{&x=2\\&y=\dfrac{3}{2}.}$$
	Khi đó ta có $$2Al+\dfrac{3}{2}O_2 \rightarrow Al_2 O_3.$$
	Do hệ số của phương trình hóa học phải là số nguyên nên nhân hai về phương trình hóa học trên với $2$, ta được
	$$4Al+3O_2 \rightarrow 2Al_2 O_3.$$
	\end{enumerate}
	}
\end{bt}
%%==========Bài 62
\begin{bt}
	Trong phòng thí nghiệm, cô Linh muốn tạo ra $500 \mathrm{~g}$ dung dịch $\mathrm{HCl} $ $19 \%$ từ hai loại dung dịch $\mathrm{HCl} $ $10 \%$ và $\mathrm{HCl}$ $ 25 \%$. Hỏi cô Linh cẩn dùng bao nhiêu gam cho mỗi loại dung dịch đó?
	\loigiai{
	Gọi $x$, $y$ lần lượt là số gam dung dịch $\mathrm{HCl} $ $10 \%$ và $\mathrm{HCl}$ $ 25 \%$ cô Linh cần dùng ($x>0$, $y>0$).\\
	Vì dung dịch tạo thành có $500 \mathrm{~g}$ nên ta có phương trình
	$$x+y=500.\qquad (1)$$
	Vì nồng độ dung dịch sau khi tạo là $\mathrm{HCl} $ $19 \%$ nên ta có
	$$\dfrac{10\%x+25\%y}{500}=19\%.\text{ hay }0{,}1x+0{,}25y=95.\qquad(2)$$
	Từ (1) và (2), ta có hệ phương trình $\heva{&x+y=500\\&0{,}1x+0{,}25y=95.}$\\
	Giải hệ phương trình trên, ta được $ \heva{&x=200 \\&y=300.}$\\
	Vậy cô Linh đã dùng $200 \mathrm{~gam}$ dung dịch $\mathrm{HCl} $ $10 \%$, và $300 \mathrm{~gam}$ dung dịch $\mathrm{HCl} $ $25 \%$.
	}
\end{bt}
%%%%%%%%%%%%%%
\subsection{Bài tập trắc nghiệm}
\Opensolutionfile{ans}[ans/ans-9T1-OTC]
%%==========Câu 1
\begin{ex}
	Tất cả các nghiệm của phương trình $\left(x+3\right)\left(2x-6\right)=0$ là
	\choice
	{$x=-3$}
	{$x=3$}
	{\True $x=3$ hay $x=-3$}
	{$x=2$}
	\loigiai{Ta có:
	$\left(x+3\right)\left(2x-6\right)=0$ nên $x+3=0$ hoặc $2x-6=0$
	\begin{itemize}
	\item $x+3=0$ suy ra $x=-3$.
	\item $2x-6=0$ hay $2x=6$, suy ra $x=3$.
	\end{itemize}
	Phương trình có hai nghiệm $x=-3$, $x=-3$.
	}
\end{ex}
%%==========Câu 2
\begin{ex}
	Điều kiện xác định của phương trình $\dfrac{2x+3}{x-4}+2=\dfrac{1}{x-3}$
	\choice
	{ $x\ne 4$}
	{$x \ne 3$}
	{\True $ x \ne 4$ và $ x \ne 3$}
	{$x=4$ và $x=3$}
	\loigiai{
	Để phương trình $\dfrac{2x+3}{x-4}+2=\dfrac{1}{x-3}$ xác định thì $\heva{&x-4\ne0\\&x-3\ne0}$, suy ra $\heva{&x\ne 4\\&x\ne 3.}$
	}
\end{ex}
%%==========Câu 3
\begin{ex} 
	Nghiệm của phương trình $\dfrac{1}{x}-\dfrac{3}{2 x}=\dfrac{1}{6}$ là
	\choice{$x=3$}
	{\True $x=-3$}
	{$x=6$}
	{$x=-6$} 
	\loigiai{
	Với điều kiện xác định $x \ne 0$, ta có:
	$$\begin{aligned} \dfrac{1}{x}-\dfrac{3}{2 x}& =\dfrac{1}{6} \\ \dfrac{6}{6 x}-\dfrac{9}{6 x}& =\dfrac{x}{6 x} \\ 6-9& =x \\ x& =-3.\end{aligned}$$	
	Ta thấy $x=-3$ thỏa mãn điều kiện xác định.\\
	Vậy phương trình đã cho có nghiệm $x=-3$.
	}
\end{ex} 
%%==========Câu 4
\begin{ex}
	Nghiệm của phương trình $\dfrac{x+2}{x-4}-1=\dfrac{30}{\left(x+3\right)\left(x-4\right)}$ là 
	\choice
	{\True $x=2$}
	{$x=-3$}
	{$x=4$}
	{$x=-2$}
	\loigiai{
	Với điều kiện $\heva{&x\ne 4\\&x\ne -3}$, ta có:
	\begin{align*}
	\dfrac{x+2}{x-4}-1&=\dfrac{30}{\left(x+3\right)\left(x-4\right)}\\
	\dfrac{\left(x+2\right)\left(x+3\right)}{\left(x+3\right)\left(x-4\right)}-\dfrac{(x-4)(x+3)}{\left(x+3\right)\left(x-4\right)}&=\dfrac{30}{\left(x+3\right)\left(x-4\right)}\\
	\dfrac{x^2+5x+6}{\left(x+3\right)\left(x-4\right)}-\dfrac{x^2-x-12}{\left(x+3\right)\left(x-4\right)}&=\dfrac{30}{\left(x+3\right)\left(x-4\right)}\\
	x^2+5x+6-x^2+x+12-30&=0\\
	6x-12&=0\\
	x&=2 \quad \left( \text{thỏa mãn}\right)
	\end{align*}
	}
\end{ex}
%%==========Câu 5
\begin{ex}
	Điều kiện xác định của phương trình $\dfrac{x}{2x + 1} + \dfrac{3}{x - 5} = \dfrac{x}{(2x+1)(x-5)}$ là
	\choice
	{$x \neq -\dfrac{1}{2}$}
	{$x \neq -\dfrac{1}{2}$ và $x \neq -5$}
	{$x \neq 5$}
	{\True $x \neq -\dfrac{1}{2}$ và $x \neq 5$}
	\loigiai{
	Điều kiện xác định của phương trình đã cho là
	$$ \heva{&2x + 1 \neq 0\\&x - 5 \neq 0} \Rightarrow \heva{&x \neq -\dfrac{1}{2}\\&x \neq 5.} $$
	Vậy điều kiện xác định đã cho là $x \neq -\dfrac{1}{2}$ và $x \neq 5$.
	}
\end{ex}
%%==========Câu 6
\begin{ex}
	Phương trình $x - 1 = m + 4$ có nghiệm lớn hơn $1$ với
	\choice
	{$m \geq -4$}
	{$m \leq 4$}
	{\True $m > -4$}
	{$m < -4$}
	\loigiai{
	Ta có $\begin{aligned}[t]
	x - 1 &= m + 4\\
	x &= m + 
	\end{aligned}$\\
	Để phương trình $x - 1 = m + 4$ có nghiệm lớn hơn $1$ thì
	\begin{eqnarray*}
	m + 5 &>& 1\\
	m &>& -4.
	\end{eqnarray*}
	}
\end{ex}
%%==========Câu 7
\begin{ex}
	Phương trình nào sau đây \textbf{không} phải là phương trình bậc nhất hai ẩn?
	\choice
	{$5x-y=3$}
	{$\sqrt{5}x+0y=0$}
	{$0x-4y=\sqrt{6}$}
	{\True $0x+0y=12$}
	\loigiai{
	Phương trình \textbf{không} phải là phương trình bậc nhất hai ẩn $0x+0y=12$.
	}
\end{ex}
%%==========Câu 8
\begin{ex}
	Đường thẳng biểu diễn tất cả các nghiệm của phương trình $3x-y=2$
	\choice
	{vuông góc với trục tung}
	{vuông góc với trục hoành}
	{đi qua gốc tọa độ}
	{\True đi qua điểm $A(1;1)$}
	\loigiai{
	Vì $3\cdot 1-1=2$ nên đường thẳng biểu diễn tất cả các nghiệm của phương trình $3x-y=2$ đi qua điểm $A(1;1)$.
	}
\end{ex}
%%==========Câu 9
\begin{ex}
	Trên mặt phẳng toạ độ $O x y$, cho các điểm $A(1; 2), B(5; 6), C(2; 3), D(-1;-1)$. Đường thẳng $4 x-3 y=-1$ đi qua hai điểm nào trong các điểm đã cho?
	\choice
	{$A$ và $B$}
	{$B$ và $C$}
	{\True $C$ và $D$}
	{$D$ và $A$}
	\loigiai{
	\begin{itemize}
	\item Thay tọa độ điểm $A(1;2)$ và đường thẳng $4x-3y=-1$, ta được $4\cdot 1-3\cdot 2=-2\ne -1$.\\
	\item Thay tọa độ điểm $B(5;6)$ và đường thẳng $4x-3y=-1$, ta được $4\cdot 5-3\cdot 6=2\neq -1$.\\
	\item Thay tọa độ điểm $C(2;3)$ và đường thẳng $4x-3y=-1$, ta được $4\cdot 2-3\cdot 3=-1$.\\
	\item Thay tọa độ điểm $D(-1;-1)$ và đường thẳng $4x-3y=-1$, ta được $4\cdot (-1)-3\cdot (-1)=-1$.
	\end{itemize}
	Vậy đường thẳng $4x-3y=-1$ đi qua hai điểm $C$ và $D$.
	}
\end{ex}
%%==========Câu 10
\begin{ex}
	Cặp số $(-2;-3)$ là nghiệm của phương trình nào sau đây?
	\choice
	{$\heva{&x-2y=3\\&2x+y=4}$}
	{$\heva{&2x-y=-1\\&x-3y=8}$}
	{\True $\heva{&2x-y=-1\\&x-3y=7}$}
	{$\heva{&4x-2y=0\\&x-3y=5}$}
	\loigiai{Ta có:
	$\heva{&2x-y=-1\\&x-3y=7} \Rightarrow \heva{&x=-2\\&y=-3.}$\\
	Do đó cặp số $(-2;-3)$ là nghiệm của phương trình $\heva{&2x-y=-1\\&x-3y=7.}$
	}
\end{ex}
%%==========Câu 11
\begin{ex}
	Cặp số nào sau đây là nghiệm của hệ phương trình $\heva{& 5 x+7 y=-1 \\ & 3 x+2 y=-5}$?
	\choice
	{$(-1; 1)$}
	{\True $(-3; 2)$}
	{$(2;-3)$}
	{$(5; 5)$}
	\loigiai{
	\allowdisplaybreaks
	\begin{eqnarray*}
	\heva{& 5 x+7 y=-1 \\ & 3 x+2 y=-5}\Rightarrow 	\heva{&-15 x-21 y=3 \\ & 15 x+10 y=-25}
	\end{eqnarray*}
	Cộng từng vế hai phương trình của hệ mới, ta được $-11y=-22\Rightarrow y=2$.\\
	Thế giá trị $y=2$ vào phương trình thứ nhất của hệ phương trình, ta có \[5x+7\cdot 2=-1\Rightarrow x=-3.\]
	Vậy hệ phương trình có nghiệm là $(-3;2)$.
	}
\end{ex}
%%==========Câu 12
\begin{ex}
	Hệ phương trình $\heva{&1{,}5 x-0{,}6 y=0{,}3 \\ &-2 x+y=-2.}$
	\choice
	{có nghiệm là $(0;-0{,}5)$}
	{có nghiệm là $(1; 0)$}
	{\True có nghiệm là $(-3;-8)$}
	{vô nghiệm}
	\loigiai{
	\[\heva{&1{,}5 x-0{,}6 y=0{,}3 \\ &-2 x+y=-2}\Rightarrow \heva{&5 x-2 y=1 \\ &-x+2y=-4.}\]
	Cộng từng vế hai phương trình của hệ mới, ta được $x=-3$.\\
	Thế giá trị $x=-3$ vào phương trình thứ hai của hệ phương trình, ta có $-2\cdot (-3)+y=-2\Rightarrow y=-8$.\\
	Vậy hệ phương trình có nghiệm là $(-3;-8)$.
	}
\end{ex}
%%==========Câu 13
\begin{ex}
	Hệ phương trình $\heva{& 0{,}6 x+0{,}3 y=1{,}8 \\ & 2 x+y=-6.}$
	\choice
	{có một nghiệm}
	{\True vô nghiệm}
	{có vô số nghiệm}
	{có hai nghiệm}
	\loigiai{
	\[\heva{& 0{,}6 x+0{,}3 y=1{,}8 \\ & 2 x+y=-6}\Rightarrow \heva{& 2 x+ y=3 \\ & 2 x+y=-6.}\]
	Từ phương trình thứ nhất của hệ mới, ta có $y=3-2x$. Thế vào phương trình thứ hai của hệ ta được $2x+3-2x=-6$ hay $0x=-9$.\hfill $(1)$\\
	Do không có giá trị nào của $x$ thỏa mãn hệ thức $(1)$ nên hệ phương trình đã cho vô nghiệm.
	}
\end{ex}
%%==========Câu 14
\begin{ex}
	Nghiệm của hệ phương trình $\heva{&x+y=9 \\& x-y=-1}$ là
	\choice
	{\True $(x; y)=(4; 5)$}
	{$(x; y)=(5; 4)$}
	{$(x; y)=(-5;-4)$}
	{$(x; y)=(-4;-5)$}
	\loigiai{
	Ta có
	$\begin{aligned}
	\heva{&x+y=9 \\ &x-y=-1}
	\heva{&2 x=8 \\ &x-y=-1}
	\heva{&x=4 \\ &y=4+1}
	\heva{&x=4 \\ &y=5.}
	\end{aligned}$\\
	Vậy hệ phương trình đã cho có nghiệm $(x;y)=(4;5)$.
	}
\end{ex} 
\Closesolutionfile{ans}