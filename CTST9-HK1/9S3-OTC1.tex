\section*{ÔN TẬP CHƯƠNG III}
%%%%%%%%%%%%%%%%%%%
\subsection{Bài tập tự luận}
%%==========Bài 1
\begin{bt}
	Áp dụng quy tắc khai phương một thương, hãy tính
	\begin{listEX}[4]
	\item $\sqrt{\dfrac{121}{9}}$;
	\item $\sqrt{\dfrac{144}{169}}$;
	\item $\sqrt{3\dfrac{6}{25}}$;
	\item $\sqrt{4\dfrac{21}{25}}$.
	\end{listEX}
	\loigiai{
	\begin{listEX}[2]
	\item $\sqrt{\dfrac{121}{9}}=\dfrac{\sqrt{121}}{\sqrt{9}}=\dfrac{11}{3}$.
	\item $\sqrt{\dfrac{144}{169}}=\dfrac{\sqrt{144}}{\sqrt{169}}=\dfrac{12}{13}$.
	\item $\sqrt{3\dfrac{6}{25}}=\sqrt{\dfrac{81}{25}}=\dfrac{\sqrt{81}}{\sqrt{25}}=\dfrac{9}{5}$.
	\item $\sqrt{4\dfrac{21}{25}}=\sqrt{\dfrac{121}{25}}=\dfrac{\sqrt{121}}{\sqrt{25}}=\dfrac{11}{5}$.
	\end{listEX}
	}
\end{bt}
%%==========Bài 2
\begin{bt}
	Áp dụng quy tắc chia hai căn bậc hai, hãy tính
	\begin{listEX}[4]
	\item $\dfrac{\sqrt{999}}{\sqrt{444}}$;
	\item $\dfrac{\sqrt{160}}{\sqrt{0{,}4}}$;
	\item $\dfrac{\sqrt{9+6\sqrt{2}}}{\sqrt{3}}$;
	\item $\sqrt{2+\sqrt{3}}:\sqrt{\dfrac{1}{2}}$.
	\end{listEX}
	\loigiai{
	\begin{listEX}
	\item $\dfrac{\sqrt{999}}{\sqrt{444}}=\sqrt{\dfrac{999}{444}}=\sqrt{\dfrac{9}{4}}=\dfrac{\sqrt{9}}{\sqrt{4}}=\dfrac{3}{2}$.
	\item $\dfrac{\sqrt{160}}{\sqrt{0{,}4}}=\sqrt{\dfrac{160}{0{,}4}}=\sqrt{\dfrac{1600}{4}}=\dfrac{\sqrt{1600}}{\sqrt{4}}=\dfrac{40}{2}=20$.
	\item $\dfrac{\sqrt{9+6\sqrt{2}}}{\sqrt{3}}=\sqrt{\dfrac{9+6\sqrt{2}}{3}}=\sqrt{3+2\sqrt{2}}=\sqrt{\left(\sqrt{2}+1\right)^2}=\sqrt{2}+1$;
	\item $\sqrt{2+\sqrt{3}}:\sqrt{\dfrac{1}{2}}=\sqrt{\left(2+\sqrt{3}\right):\dfrac{1}{2}}=\sqrt{4+2\sqrt{3}}=\sqrt{\left(\sqrt{3}+1\right)^2}=\sqrt{3}+1$.
	\end{listEX}
	}
\end{bt}
%%==========Bài 3
\begin{bt}
Không sử dụng MTCT, tính giá trị của biểu thức 
$$A=\sqrt{\left(\sqrt{3}-2\right)^2}+\sqrt{4\left(2+\sqrt{3}\right)^2}-\dfrac{1}{2-\sqrt{3}}.$$
\loigiai{
	$$\begin{aligned}[t]
	A&=\sqrt{\left(\sqrt{3}-2\right)^2}+\sqrt{4\left(2+\sqrt{3}\right)^2}-\dfrac{1}{2-\sqrt{3}}\\
	&=|\sqrt{3}-2|+2|2+\sqrt{3}|-\dfrac{2+\sqrt{3}}{2^2-\left(\sqrt{3}\right)^2}\\
	&=2-\sqrt{3}+4+2\sqrt{3}-2-\sqrt{3}\\
	&=4.
	\end{aligned}$$
}
\end{bt}
%%==========Bài 4
\begin{bt}
	Tìm $x$, biết:
	\begin{listEX}[4]
	\item $x^2=10$;
	\item $\sqrt{ x }=8$;
	\item $x ^3=-0,027$;
	\item $\sqrt[3]{ x }=-\dfrac{2}{3}$.
	\end{listEX}
	\loigiai{
	\begin{listEX}
	\item 
	Vì $x^2=10$
	nên $x=\sqrt{10}\text{ hoặc } x=-\sqrt{10}.$
	\item 
	Ta có $\sqrt{x}=8$
	suy ra $x=8^2=64.$
	\item 
	Ta có $x^3=-0{,}027$
	suy ra $x=\sqrt[3]{-0{,}027}=-0{,}3$
	\item 
	Ta có $\sqrt[3]{x}=-\dfrac{2}{3}$
	suy ra $x=\left(-\dfrac{2}{3}\right)^3
	=-\dfrac{8}{27}.$
	\end{listEX}
	}
\end{bt}
%%==========Bài 5
\begin{bt}
	Biết rằng $1<a<5$, rút gọn biểu thức $A =\sqrt{(a-1)^2}+\sqrt{(a-5)^2}$.
	\loigiai{
	Ta có 
	$A=\sqrt{(a-1)^2}+\sqrt{(a-5)^2}
	=\left|a-1\right|+\left|a-5\right|
	=a-1+5-a=4.$
	}
\end{bt}
%%==========Bài 6
\begin{bt}
	Rút gọn các biểu thức sau
	\begin{listEX}[3]
	\item $\sqrt{1\dfrac{9}{16}\cdot 5\dfrac{4}{9}\cdot 0{,}01}$;
	\item $\sqrt{\dfrac{165^2-124^2}{164}}$;
	\item $\sqrt{\dfrac{149^2-76^2}{457^2-384^2}}$.
	\end{listEX}
	\loigiai{
	\begin{listEX}
	\item $\sqrt{1\dfrac{9}{16}\cdot 5\dfrac{4}{9}\cdot 0{,}01}=\sqrt{\dfrac{25}{16}\cdot\dfrac{49}{9}\cdot\dfrac{1}{100}}=\sqrt{\dfrac{25}{16}}\cdot\sqrt{\dfrac{49}{9}}\cdot\sqrt{\dfrac{1}{100}}=\dfrac{5}{4}\cdot\dfrac{7}{3}\cdot\dfrac{1}{10}=\dfrac{7}{24}$.
	\item $\sqrt{\dfrac{165^2-124^2}{164}}=\sqrt{\dfrac{(165-124)(165+124)}{164}}=\sqrt{\dfrac{41\cdot 289}{164}}=\sqrt{\dfrac{289}{4}}=\dfrac{\sqrt{289}}{\sqrt{4}}=\dfrac{17}{2}$.
	\item $\sqrt{\dfrac{149^2-76^2}{457^2-384^2}}=\sqrt{\dfrac{(149-76)(149+76)}{(457-384)(457+384)}}=\sqrt{\dfrac{73\cdot 225}{73\cdot 841}}=\sqrt{\dfrac{225}{841}}=\dfrac{\sqrt{225}}{\sqrt{841}}=\dfrac{15}{29}$.
	\end{listEX}
	}
\end{bt}
%%==========Bài 7
\begin{bt}
	Rút gọn các biểu thức sau
	\begin{listEX}[3]
	\item $\dfrac{\sqrt{96x^5}}{\sqrt{24x}}$ ($x>0$);
	\item $\dfrac{\sqrt{18(x+1)^3}}{\sqrt{2}x+\sqrt{2}}$ ($x>-1$);
	\item $\dfrac{\sqrt{3x^4y^4}}{\sqrt{27x^2y^4}}$ ($x<0$, $y>0$).
	\end{listEX}
	\loigiai{
	\begin{listEX}
	\item $\dfrac{\sqrt{96x^5}}{\sqrt{24x}}=\sqrt{\dfrac{96x^5}{42x}}=\sqrt{4x^4}=2x^2$.
	\item $\dfrac{\sqrt{18(x+1)^3}}{\sqrt{2}x+\sqrt{2}}=\dfrac{\sqrt{18(x+1)^3}}{\sqrt{2(x+1)^2}}=\sqrt{\dfrac{18(x+1)^3}{2(x+1)^2}}=\sqrt{9(x+1)}=3\sqrt{x+1}$ (với $x>-1$).
	\item $\dfrac{\sqrt{3x^4y^4}}{\sqrt{27x^2y^4}}=\sqrt{\dfrac{3x^4y^4}{27x^2y^4}}=\sqrt{\dfrac{x^2}{9}}=\dfrac{\sqrt{x^2}}{\sqrt{9}}=\dfrac{-x}{3}$ (với $x<0$, $y>0$).
	\end{listEX}
	}
\end{bt}
%%==========Bài 8
\begin{bt}
	Rút gọn các biểu thức sau
	\begin{listEX}[2]
	\item $(2x-2)\sqrt{\dfrac{(x-2)^4}{(x-1)^2}}$ ($x<1$);
	\item $\sqrt{\dfrac{4+12a+9a^2}{b^2}}$ ($a\le -\dfrac{2}{3}$).
	\end{listEX}
	\loigiai{
	\begin{listEX}
	\item $(2x-2)\sqrt{\dfrac{(x-2)^4}{(x-1)^2}}=2(x-1)\cdot\dfrac{\sqrt{(x-2)^4}}{\sqrt{(x-1)^2}}=2(x-1)\cdot\dfrac{(x-2)^2}{-(x-1)}=-2(x-2)^2$ (với $x<1$).
	\item $\sqrt{\dfrac{4+12a+9a^2}{b^2}}=\sqrt{\dfrac{(2+3a)^2}{b^2}}=\dfrac{\sqrt{(2+3a)^2}}{\sqrt{b^2}}=\dfrac{-(2+3a)}{|b|}$ (với $a\le -\dfrac{2}{3}$).
	\end{listEX}
	}
\end{bt}
%%==========Bài 9
\begin{bt}
	Trục căn thức ở mẫu các biểu thức sau
	\begin{listEX}[3]
	\item $\dfrac{4-2 \sqrt{6}}{\sqrt{48}}$;
	\item $\dfrac{3-\sqrt{5}}{3+\sqrt{5}}$
	\item $\dfrac{a}{a-\sqrt{a}}$ với $a >0,a\neq 1$.
	\end{listEX}
	\loigiai{
	\begin{listEX}
	\item $\dfrac{4-2 \sqrt{6}}{\sqrt{48}}=\dfrac{(4-2\sqrt{6})\sqrt{48}}{48}=\dfrac{2\cdot (2-\sqrt{6})\cdot 4\sqrt{3}}{48}=\dfrac{(2-\sqrt{6})\cdot \sqrt{3}}{6}=\dfrac{2\sqrt{3}-3\sqrt{2}}{6}.$
	\item
	$	\dfrac{3-\sqrt{5}}{3+\sqrt{5}}=\dfrac{(3-\sqrt{5})^2}{3^2-(\sqrt{5})^2}=\dfrac{9-6\sqrt{5}+5}{9-5}=\dfrac{14-6\sqrt{5}}{4}=\dfrac{7-3\sqrt{5}}{2}
	$
	\item
	$
	\dfrac{a}{a-\sqrt{a}}=\dfrac{(\sqrt{a})^2}{\sqrt{a}(\sqrt{a}-1)}=\dfrac{\sqrt{a}}{\sqrt{a}-1}=\dfrac{\sqrt{a}(\sqrt{a}+1)}{a-1}=\dfrac{a+\sqrt{a}}{a-1}
	$
	\end{listEX}
	}
\end{bt}
%%==========Bài 10
\begin{bt}
	Biết rằng $a >0, b >0$ và $ab =16$. Tính giá tri của biểu thức $A=a\sqrt{\dfrac{12b}{a}}+b\sqrt{\dfrac{3a}{b}}$.
	\loigiai{
	\begin{eqnarray*}
	\allowdisplaybreaks
	A&=&a\sqrt{\dfrac{12b}{a}}+b\sqrt{\dfrac{3a}{b}}\\
	&=&\sqrt{12ab}+\sqrt{3ab}\\
	&=&\sqrt{12\cdot 16}+\sqrt{3\cdot 16}\\
	&=&\sqrt{192}+\sqrt{48}\\
	&=&8\sqrt{3}+4\sqrt{3}=12\sqrt{3}.
	\end{eqnarray*}
	}
\end{bt}
%%==========Bài 11
\begin{bt}
	Tính $\dfrac{\sqrt{3}+\sqrt{2}}{\sqrt{3}-\sqrt{2}}-\dfrac{\sqrt{3}-\sqrt{2}}{\sqrt{3}+\sqrt{2}}$.
	\loigiai{
	\begin{eqnarray*}
	&&\dfrac{\sqrt{3}+\sqrt{2}}{\sqrt{3}-\sqrt{2}}-\dfrac{\sqrt{3}-\sqrt{2}}{\sqrt{3}+\sqrt{2}}\\
	&=&\dfrac{(\sqrt{3}+\sqrt{2})^2+(\sqrt{3}-\sqrt{2})^2}{(\sqrt{3})^2-(\sqrt{2})^2}\\
	&=&\dfrac{3+2\sqrt{6}+2+3-2\sqrt{6}+2}{3-2}\\
	&=&\dfrac{10}{1}=10.
	\end{eqnarray*}
	}
\end{bt}
%%==========Bài 12
\begin{bt}%[MaT-SGK9-Moi]%[Phạm Hoàng Điệp- Phạm Hoài]
	Rút gọn biểu thức
	\begin{listEX}[2]
	\item $A=\sqrt{40^2-24^2}$;
	\item $B=(\sqrt{12}+2\sqrt{3}-\sqrt{27})\cdot \sqrt{3}$;
	\item $C=\dfrac{\sqrt{63^3+1}}{\sqrt{63^2-62}}$;
	\item $D=\sqrt{60}-5\sqrt{\dfrac{3}{5}}-3\sqrt{\dfrac{5}{3}}$.
	\end{listEX}
	\loigiai{ 
	\begin{listEX}[1]
	\item $A=\sqrt{40^2-24^2} = \sqrt{(40+24)(40-24)}=\sqrt{64\cdot 16} = \sqrt{64}\cdot \sqrt{16}=8 \cdot 4 =32$;
	\item $\begin{aligned}[t]
	B&=(\sqrt{12}+2\sqrt{3}-\sqrt{27})\cdot \sqrt{3}=(\sqrt{2^2\cdot 3}+2\sqrt{3}-\sqrt{3^3})\cdot \sqrt{3} =(2\sqrt{3}+2\sqrt{3}-3\sqrt{3})\cdot \sqrt{3} \\
	&= \sqrt{3}\cdot \sqrt{3} = 3;
	\end{aligned}$
	\item $C=\dfrac{\sqrt{63^3+1}}{\sqrt{63^2-62}}=\sqrt{\dfrac{(63+1)(63^2-63+1)}{63^2-62}} = \sqrt{\dfrac{64\cdot(63^2-62)}{63^2-62}} =8;$
	\item $D=\sqrt{60}-5\sqrt{\dfrac{3}{5}}-3\sqrt{\dfrac{5}{3}} = \sqrt{4\cdot 15} - \sqrt{5^2\cdot \dfrac{3}{5}} -\sqrt{3^2\cdot \dfrac{5}{3}} = 2\sqrt{15} -\sqrt{15} - \sqrt{15}=0$.
	\end{listEX}
	}
\end{bt}
%%==========Bài 13
\begin{bt}%[MaT-SGK9-Moi]%[Phạm Hoàng Điệp- Phạm Hoài]
	Trục căn thức ở mẫu
	\begin{listEX}[2]
	\item $\dfrac{x^2+x}{\sqrt{x+1}}$ với $x>-1$;
	\item $\dfrac{3}{\sqrt{x}-2}$ với $x>0$, $x \ne 4$;
	\item $\dfrac{\sqrt{3}-\sqrt{5}}{\sqrt{3}+\sqrt{5}}$;
	\item $\dfrac{x^2-90}{\sqrt{x}-\sqrt{3}}$ với $x>0$, $x \ne 3$.
	\end{listEX}
	\loigiai{
	\begin{listEX}[1]
	\item 
	Với $x>-1$. Ta có $\dfrac{x^2+x}{\sqrt{x+1}} = \dfrac{x(x+1)}{\sqrt{x+1}}=\dfrac{x\cdot \left(\sqrt{x+1}\right)^2}{\sqrt{x+1}} = x\sqrt{x+1}$;
	\item Với $x>0$, $x \ne 4$. Ta có $\dfrac{3}{\sqrt{x}-2} =\dfrac{3(\sqrt{x}+2)}{(\sqrt{x}-2)(\sqrt{x}+2)} = \dfrac{3(\sqrt{x}+2)}{x-4}$;
	\item $\dfrac{\sqrt{3}-\sqrt{5}}{\sqrt{3}+\sqrt{5}}=\dfrac{(\sqrt{3}-\sqrt{5})(\sqrt{3}-\sqrt{5})}{(\sqrt{3}+\sqrt{5})(\sqrt{3}-\sqrt{5})}=\dfrac{(\sqrt{3}-\sqrt{5})^2}{3-5}=\dfrac{8-2\sqrt{15}}{-2}=\sqrt{15}-4$;
	\item Với $x>0$, $x \ne 3$. \\Ta có $\dfrac{x^2-9}{\sqrt{x}-\sqrt{3}} = \dfrac{(x-3)(x+3)}{\sqrt{x}-\sqrt{3}}=\dfrac{(\sqrt{x}-\sqrt{3})(\sqrt{x}+\sqrt{3})(x+3)}{\sqrt{x}-\sqrt{3}} =(\sqrt{x}+\sqrt{3})(x+3)$.
	\end{listEX}
	}
\end{bt}
%%==========Bài 14
\begin{bt}%[MaT-SGK9-Moi]%[Phạm Hoàng Điệp- Phạm Hoài]
	So sánh
	\begin{listEX}[3]
	\item $2\sqrt{3}$ và $3\sqrt{2}$;
	\item $7\sqrt{\dfrac{3}{7}}$ và $\sqrt{2}\cdot \sqrt{11}$;
	\item $\dfrac{2}{\sqrt{5}}$ và $\dfrac{6}{\sqrt{10}}$.
	\end{listEX}
	\loigiai{
	\begin{listEX}[1]
	\item Ta có $2\sqrt{3} = \sqrt{2^2\cdot 3} = \sqrt{12}$; $3\sqrt{2} = \sqrt{3^2\cdot 2} = \sqrt{18}$.\\
	Mà $\sqrt{12}<\sqrt{18}$ nên $2\sqrt{3} < 3\sqrt{2} $.
	\item Ta có $7\sqrt{\dfrac{3}{7}}=\sqrt{7^2\cdot\dfrac{3}{7}} =\sqrt{21}$; $\sqrt{2}\cdot \sqrt{11}=\sqrt{2\cdot 11} = \sqrt{22}$.\\
	Mà $\sqrt{21}<\sqrt{22}$ nên $7\sqrt{\dfrac{3}{7}}<\sqrt{2}\cdot \sqrt{11}$.
	\item Ta có $\dfrac{2}{\sqrt{5}} = \dfrac{2\sqrt{5}}{5}=\dfrac{\sqrt{20}}{5}$; $\dfrac{6\sqrt{10}}{10}=\dfrac{3\sqrt{10}}{5} = \dfrac{\sqrt{90}}{5}$.\\
	Mà $\dfrac{\sqrt{20}}{5}<\dfrac{\sqrt{90}}{5}$ nên $\dfrac{2}{\sqrt{5}}<\dfrac{6}{\sqrt{10}}$.
	\end{listEX}
	}
\end{bt}
%%==========Bài 15
\begin{bt}%[MaT-SGK9-Moi]%[Phạm Hoàng Điệp- Phạm Hoài]
	Cho biểu thức $M=\dfrac{a\sqrt{a}+b\sqrt{b}}{\sqrt{a}+\sqrt{b}}$ với $a>0$, $b>0$.
	\begin{listEX}[2]
	\item Rút gọn biểu thức $M$.
	\item Tính giá trị của biểu thức tại $a=2$, $b=8$.
	\end{listEX}
	\loigiai{
	\begin{listEX}
	\item 	Với $a>0$, $b>0$. Ta có\\
	$M=\dfrac{a\sqrt{a}+b\sqrt{b}}{\sqrt{a}+\sqrt{b}}=\dfrac{\sqrt{a^3}+\sqrt{b^3}}{\sqrt{a}+\sqrt{b}}=\dfrac{(\sqrt{a}+\sqrt{b})(\sqrt{a^2}-\sqrt{ab}+\sqrt{b^2})}{\sqrt{a}+\sqrt{b}} = a+b-\sqrt{ab}$.
	\item Với $a=2$, $b=8$, ta có $M=2+8-\sqrt{2\cdot 8} = 6$.
	\end{listEX}
	}
\end{bt}
%%==========Bài 16
\begin{bt}%[MaT-SGK9-Moi]%[Phạm Hoài - Phạm Hoàng Điệp]
	Cho biểu thức $N=\dfrac{x \sqrt{x}+8}{x-4}-\dfrac{x+4}{\sqrt{x}-2}$ với $x \geq 0$ và $x \neq 4$.
	\begin{listEX}
	\item Rút gọn biểu thức $N$.
	\item Tính giá trị của biểu thức tại $x=9$.
	\end{listEX}
	\loigiai{\begin{listEX}
	\item \begin{eqnarray*}
	N&=&\dfrac{x \sqrt{x}+8}{x-4}-\dfrac{x+4}{\sqrt{x}-2}\\
	&=&\dfrac{x \sqrt{x}+8}{\left(\sqrt{x}-2\right)\left(\sqrt{x}+2\right)}-\dfrac{x+4}{\sqrt{x}-2}\\
	&=&\dfrac{x \sqrt{x}+8}{\left(\sqrt{x}-2\right)\left(\sqrt{x}+2\right)}-\dfrac{\left(x+4\right)\left(\sqrt{x}+2\right)}{\left(\sqrt{x}-2\right)\left(\sqrt{x}+2\right)}\\
	&=&\dfrac{x \sqrt{x}+8-\left(x+4\right)\left(\sqrt{x}+2\right)}{\left(\sqrt{x}-2\right)\left(\sqrt{x}+2\right)}\\
	&=&\dfrac{\left(\sqrt{x}+2\right)\left(x-2\sqrt{x}+4\right)-\left(x+4\right)\left(\sqrt{x}+2\right)}{\left(\sqrt{x}-2\right)\left(\sqrt{x}+2\right)}\\
	&=&\dfrac{\left(\sqrt{x}+2\right)\left[\left(x-2\sqrt{x}+4\right)-\left(x+4\right)\right]}{\left(\sqrt{x}-2\right)\left(\sqrt{x}+2\right)}\\
	&=&\dfrac{\left(x-2\sqrt{x}+4\right)-\left(x+4\right)}{\sqrt{x}-2}\\
	&=&\dfrac{x-2\sqrt{x}+4-x-4}{\sqrt{x}-2}\\
	&=&\dfrac{-2\sqrt{x}}{\sqrt{x}-2}.
	\end{eqnarray*}
	Do đó $N=\dfrac{-2\sqrt{x}}{\sqrt{x}-2}$.
	\item Với $x=9\Rightarrow N=\dfrac{-2\sqrt{9}}{\sqrt{9}-2}=\dfrac{-6}{1}=-6$.
	\end{listEX}
	}
\end{bt}
%%==========Bài 17
\begin{bt}
	Cho biểu thức $A=\dfrac{\sqrt{x}+2}{\sqrt{x}-2}-\dfrac{4}{\sqrt{x}+2}$ ($x>0$, $x\ne 4$).\\
	\begin{listEX}[2]
	\item Rút gọn biểu thức $A$.
	\item Tính giá trị của $A$ tại $x=14$.
	\end{listEX}
	\loigiai{
	\begin{listEX}
	\item Với $x>0$, $x\ne 4$ ta có $\begin{aligned}[t]
	A&=\dfrac{\sqrt{x}+2}{\sqrt{x}-2}-\dfrac{4}{\sqrt{x}+2}\\
	&=\dfrac{\left(\sqrt{x}+2\right)^2-4\left(\sqrt{x}-2\right)}{\left(\sqrt{x}-2\right)\left(\sqrt{x}+2\right)}\\
	&=\dfrac{x+4\sqrt{x}+4-4\sqrt{x}+8}{\left(\sqrt{x}-2\right)\left(\sqrt{x}+2\right)}\\
	&=\dfrac{x+12}{x-4}.
	\end{aligned}$
	\item Tại $x=14$ (thỏa mãn $x>0$, $x\ne 4$), giá trị của biểu thức $A$ là $\dfrac{14+12}{14-4}=\dfrac{13}{5}$.
	\end{listEX}
	}
\end{bt}
%%==========Bài 18
\begin{bt}
	Rút gọn các biểu thức sau
	\begin{listEX}[2]
	\item $\left(a \sqrt{\dfrac{3}{a}}+3 \sqrt{\dfrac{a}{3}}+\sqrt{12 a^3}\right): \sqrt{3 a}$ với $a>0$;
	\item $\dfrac{1-a}{1+\sqrt{a}}+\dfrac{1-a\sqrt{a}}{1-\sqrt{a}}$ với $a \geq 0,a\neq 1$.
	\end{listEX}
	\loigiai{
	\begin{listEX}
	\item Với $a>0$, ta có
	\begin{eqnarray*}
	\allowdisplaybreaks
	&&\left(a \sqrt{\dfrac{3}{a}}+3 \sqrt{\dfrac{a}{3}}+\sqrt{12 a^3}\right): \sqrt{3 a}\\
	&=&(\sqrt{3a}+\sqrt{3a}+2a\sqrt{3a}):\sqrt{3a}\\
	&=&\dfrac{\sqrt{3a}+\sqrt{3a}+2a\sqrt{3a}}{\sqrt{3a}}\\
	&=&\dfrac{2\sqrt{3a}+2a\sqrt{3a}}{\sqrt{3a}}\\
	&=&\dfrac{\sqrt{3a}(2+2a)}{\sqrt{3a}}\\
	&=&2+2a.
	\end{eqnarray*}
	\item Với $a\geq 0$, $a\neq1$, ta có
	\begin{eqnarray*}
	&&\dfrac{1-a}{1+\sqrt{a}}+\dfrac{1-a\sqrt{a}}{1-\sqrt{a}}\\
	&=&\dfrac{1-(\sqrt{a})^2}{1+\sqrt{a}}+\dfrac{1-(\sqrt{a})^3}{1-\sqrt{a}}\\
	&=&\dfrac{(1-\sqrt{a})(1+\sqrt{a})}{1+\sqrt{a}}+\dfrac{(1-\sqrt{a})(1+\sqrt{a}+a)}{1-\sqrt{a}}\\
	&=&1-\sqrt{a}+1+\sqrt{a}+a\\
	&=&2+a.
	\end{eqnarray*}
	\end{listEX}
	}
\end{bt}
%%==========Bài 19
\begin{bt}
	Cho biểu thức $P =\left(\dfrac{1}{a+\sqrt{a}}-\dfrac{1}{\sqrt{a}+1}\right): \dfrac{\sqrt{a}-1}{a+2 \sqrt{a}+1}$ với $a >0$ và $a \neq 1$.
	\begin{listEX}[2]
	\item Rút gọn biểu thức $P$.
	\item Tính giá tri của $P$ khi $a =0{,}25$.
	\end{listEX}
	\loigiai{
	\begin{listEX}
	\item Với $a>0$ và $a\neq 1$, ta có
	\begin{eqnarray*}
	\allowdisplaybreaks
	P&=&\left(\dfrac{1}{a+\sqrt{a}}-\dfrac{1}{\sqrt{a}+1}\right):\dfrac{\sqrt{a}-1}{a+2 \sqrt{a}+1}\\
	&=&\left[\dfrac{1}{\sqrt{a}(\sqrt{a}+1)}-\dfrac{\sqrt{a}}{\sqrt{a}(\sqrt{a}+1)}\right]:\dfrac{\sqrt{a}-1}{(\sqrt{a}+1)^2}\\
	&=&\dfrac{1-\sqrt{a}}{\sqrt{a}(\sqrt{a}+1)}\cdot \dfrac{(\sqrt{a}+1)^2}{\sqrt{a}-1}\\
	&=&-\dfrac{\sqrt{a}+1}{\sqrt{a}}\\
	&=&-\dfrac{a+\sqrt{a}}{a}.
	\end{eqnarray*}
	\item Với $a=0{,}25$ thỏa $a>0$ và $a\neq 1$, ta có \[P=-\dfrac{0{,}25+\sqrt{0{,}25}}{0{,}25}=-\dfrac{0{,}25+0{,}5}{0{,}25}=-\dfrac{0{,}75}{0{,}25}=-3.\]
	\end{listEX}
	}
\end{bt}
%%==========Bài 20
\begin{bt}
	\immini{Cho hình hộp chữ nhật có chiều dài $\sqrt{12}\mathrm{~cm}$, chiều rộng $\sqrt{8}\mathrm{~cm}$, chiều cao $\sqrt{6}\mathrm{~cm}$ như hình bên.
	\begin{listEX}
	\item Tính thể tích của hình hộp chữ nhật đó.
	\item Tính diện tích xung quanh của hình hộp chữ nhật đó.
	\end{listEX}}
	{
	\begin{tikzpicture}[scale=0.7]
	\path
	(0,0) coordinate (A)
	(4,0) coordinate (B)
	(35:2) coordinate (D)
	($(B)+(D)-(A)$) coordinate (C)
	;
	\foreach \x in {A,B,C,D}
	{
	\path ($(\x)+(0,3)$) coordinate (\x');
	}
	\draw (A)--(B)--(C) (A)--(A') (B)--(B') (C)--(C') (A')--(B')--(C')--(D')--cycle;
	\draw[dashed] (A)--(D)--(C) (D)--(D');
	\path (A)--(B)node[below,midway]{$\sqrt{12}\mathrm{~cm}$};
	\path (C)--(B)node[right,midway]{$\sqrt{8}\mathrm{~cm}$};
	\path (C)--(C')node[midway,right]{$\sqrt{6}\mathrm{~cm}$};
	\end{tikzpicture}
	}
	\loigiai{
	\begin{listEX}
	\item Thể tích của hình hộp chữ nhật trên là 
	\[\sqrt{12}\cdot \sqrt{8}\cdot\sqrt{6}=14\mathrm{~(cm^3)}.\]
	\item Diện tích xung quanh của hình hộp chữ nhật là
	\[2\cdot (\sqrt{12}+\sqrt{8})\cdot \sqrt{6}=8\sqrt{3}+12\sqrt{2}\mathrm{~(cm^2)}.\]
	\end{listEX}
	}
\end{bt}
%%==========Bài 21
\begin{bt}
	Một trục số được vẽ trên lưới ô vuông như hình dưới.
	\begin{center}
	\begin{tikzpicture}[>=stealth,line cap=round,line join=round,scale=0.75]
	\pgfmathsetmacro{\a}{sqrt(10)}
	\pgfmathsetmacro{\b}{sqrt(2)}
	\draw[thin,blue!25] (-4,0) grid (8,4);
	\draw [thick,->] (-4,0)--(8.5,0);
	\draw[<->,shift={(-0.1,0)}] (-4,3)--(-4,4)node[left,midway]{$1$};
	\draw[<->,shift={(0,0.1)}] (-3,4)--(-4,4)node[above,midway]{$1$};
	\foreach \x in {-1,0,1,6}{\path (\x,0)node[below]{$\x$};}
	\path
	(0,0) coordinate (O)
	(-1,3) coordinate (A)
	(6,0) coordinate (B)
	(7,1) coordinate (C)
	(-1,0) coordinate (E)
	(7,0) coordinate (F)
	(180:\a) coordinate (M)
	(0:\a) coordinate (N)
	(6,0)+(180:\b) coordinate (Q)
	(6,0)+(0:\b) coordinate (P);
	\draw[thick,red] (-5:\a) arc(-5:185:\a);
	\draw[thick,red] ($(6,0)+(-5:\b)$) arc(-5:185:\b);
	\filldraw[gray!25,draw=black] (A)--(O)--(E)--cycle;
	\filldraw[gray!25,draw=black] (B)--(C)--(F)--cycle;
	\foreach \t/\g in {O/45,M/-135,N/-45,P/-45,Q/-135,A/135,B/135,C/45}{
	\draw[fill=black] (\t) circle (1pt) node[shift={(\g:7pt)},font=\scriptsize]{$ \t $};
	}
	\path (current bounding box.south)node[below]{\itshape Hình 1};
	\end{tikzpicture}
	\end{center}
	\begin{listEX}
	\item Đường tròn tâm $O$ bán kính $OA$ cắt trục số tại hai điểm $M$ và $N$. Hai điểm $M$ và $N$ biễu diễn hai số thực nào?
	\item Đường tròn tâm $B$ bán kính $BC$ cắt trục số tại hai điểm $P$ và $Q$. Hai điểm $P$ và $Q$ biểu diễn hai số thực nào?
	\end{listEX}
	\loigiai{
	\begin{listEX}
	\item Ta có $OA=\sqrt{3^2+1^2}=\sqrt{10}$ (Định lí Pythagoras). Suy ra $OM=ON=\sqrt{10}$.\\
	Vậy điểm $M$ biểu diễn số $-\sqrt{10}$; điểm $N$ biểu diễn số $\sqrt{10}$.
	\item Ta có $BC=\sqrt{1^2+1^2}=\sqrt{2}$ (Định lí Pythagoras). Suy ra $OQ=6-\sqrt{2}$; $OP=6+\sqrt{2}$.\\
	Vậy điểm $P$ biểu diễn số $6+\sqrt{2}$; điểm $Q$ biểu diễn số $6-\sqrt{2}$.
	\end{listEX}
	}
\end{bt}
%%==========Bài 22
\begin{bt}
	Biết rằng nhiệt lượng tỏa ra trên dây dẫn được tính bởi công thức $Q=I^2Rt$, trong đó $Q$ là nhiệt lượng tính bằng đơn vị Joule ($J$), $R$ là điện trở tính bằng đơn vị Ohm ($\Omega$), $I$ là cường độ dòng điện tính bằng đơn vị Ampe ($A$), $t$ là thời gian tính bằng giây ($s$). Dòng điện chạy qua một dây dẫn có $R=10\ \Omega$ trong thời gian $5$ giây.
	\begin{listEX}
	\item Thay dấu \lq\lq?\rq\rq\ trong bảng sau bằng các giá trị thích hợp.
	\begin{center}
	\begin{tabular}{|c|c|c|c|}
	\hline
	$I$ ($A$) & $1$ & $1{,}5$ & $2$\\
	\hline
	$Q$ ($J$) & ? & ? & ? \\
	\hline
	\end{tabular}
	\end{center}
	\item Cường độ dòng điện là bao nhiêu Ampe để nhiệt lượng tỏa trên dây dẫn đạt $800$ $J$?
	\end{listEX}
	\loigiai{
	\begin{listEX}
	\item
	\begin{tabular}{|c|c|c|c|}
	\hline
	$I$ ($A$) & $1$ & $1{,}5$ & $2$\\
	\hline
	$Q$ ($J$) & $50$ & $112{,}5$ & $200$ \\
	\hline
	\end{tabular}
	\item Từ công thức $Q=I^2Rt$ suy ra $I=\sqrt{\dfrac{Q}{Rt}}=\sqrt{\dfrac{800}{10\cdot5}}=4$ ($A$).
	\end{listEX}
	Vậy khi nhiệt lượng tỏa ra trên dây dẫn đạt $800$ $J$ thì cường độ dòng điện là $4$ ($A$).
	}
\end{bt}
%%==========Bài 23
\begin{bt}%[MaT-SGK9-Moi]%[Phạm Hoài - Phạm Hoàng Điệp]	
	Ngày $28 / 9 / 2018$, sau trận động đất $7{,}5$ độ Richter, cơn sóng thần (tiếng Anh là Tsunami) cao hơn $6$ m đã tràn vào đảo Sulawesicuar (Indonesia) và tàn phá thành phố Palu gây thiệt hại vô cùng to lớn. Tốc độ cơn sóng thần $v$ (m/s) và chiều sâu đại dương $d$ (m) của nơi bắt đầu sóng thần liên hệ bởi công thức $v=\sqrt{d g}$, trong đó $g=9{,}81$ m/s$^2$.
	\begin{listEX}
	\item Hãy tính tốc độ cơn sóng thần xuất phát từ Thái Bình Dương, ở độ sâu trung bình $400$ m (làm tròn kết quả đến hàng phần trăm của mét trên giây).
	\item Theo tính toán của các nhà khoa học địa chất, tốc độ cơn sóng thần ngày $28/9/2018$ là $800$ km/h, hãy tính chiều sâu đại dương của nơi tâm chấn động đất gây ra sóng thần (làm tròn kết quả đến hàng đơn vị của mét).
	\end{listEX}
	\loigiai{\begin{listEX}
	\item $v=\sqrt{d g}=\sqrt{400\cdot 9{,}81}\approx 62{,}64$ (m/s).
	\item Đổi đơn vị $800$ (km/h)$= \dfrac{2\, 000}{9}$ (m/s).\\
	Ta có $v=\sqrt{ dg}\Rightarrow v^2=dg\Rightarrow d=\dfrac{v^2}{g}= \dfrac{\left(\dfrac{2\, 000}{9}\right)^2}{9{,}81}\approx 5\, 033{,}9$ (m).
	\end{listEX}}
\end{bt}
%%==========Bài 24
\begin{bt}%[MaT-SGK9-Moi]%[Phạm Hoài - Phạm Hoàng Điệp]
	Khi bay vào không gian, trọng lượng $P$ (N) của một phi hành gia ở vị trí cách mặt đất một độ cao $h$ (m) được tính theo công thức
	$$
	P=\dfrac{28\, 014 \cdot 10^{12}}{\left(64 \cdot 10^5+h\right)^2}
	$$
	(\textit{Nguồn: Chuyên đề Vật lí $11$, NXB Đại học Sư phạm, năm $2023$})
	\begin{listEX}
	\item Trọng lượng của phi hành gia là bao nhiêu Newton khi cách mặt đất $10\, 000 $ m (làm tròn kết quả đến hàng phần mười)?
	\item Ở độ cao bao nhiêu mét thì trọng lượng của phi hành gia là $619$ N (làm tròn kết quả đến hàng phần mười)?
	\end{listEX}
	\loigiai{
	\begin{listEX}
	\item $P=\dfrac{28\, 014 \cdot 10^{12}}{\left(64 \cdot 10^5+h\right)^2}=\dfrac{28\, 014 \cdot 10^{12}}{\left(64 \cdot 10^5+10\, 000\right)^2}\approx 681{,}8$ (N).\\
	Vậy khi cách mặt đất $10\, 000$ m thì trọng lượng của phi hành gia là $681{,}8$ N.
	\item \begin{eqnarray*}
	619&=&\dfrac{28\, 014 \cdot 10^{12}}{\left(64 \cdot 10^5+h\right)^2}\\
	28\, 014 \cdot 10^{12}&=&619\cdot \left(64 \cdot 10^5+h\right)^2\\
	\dfrac{28\, 014 \cdot 10^{12}}{619}&=&\left(64 \cdot 10^5+h\right)^2\\
	\sqrt{\dfrac{28\, 014 \cdot 10^{12}}{619}}&=&64 \cdot 10^5+h\\
	h&=& 	\sqrt{\dfrac{28\, 014 \cdot 10^{12}}{619}}-64 \cdot 10^5\\
	h&=& 327\, 322{,}34.
	\end{eqnarray*}
	Vậy ở độ cao $327\, 322{,}34$ thì trọng lượng của phi hành gia là $619$ N.
	\end{listEX}
	}
\end{bt}
%%==========Bài 25
\begin{bt}%[MaT-SGK9-Moi]%[Phạm Hoài - Phạm Hoàng Điệp]
	Áp suất $P$ (lb/in$^2$) cần thiết để ép nước qua một ống dài $L$ (ft) và đường kính $d$ (in) với tốc độ $v$ (ft/s) được cho bởi công thức $P=0{,}00161 \cdot \dfrac{v^2 L}{d}$\\
	(\textit{Nguồn: John W. Cell, Engineering Problems Illustrating Mathematics, McGraw-Hill Book Company, Inc. New York and London, năm $1943$}).
	\begin{listEX}
	\item Hãy tính $v$ theo $P$ $L$ và $d$.
	\item Cho $P=198{,}5 $; $L=11\, 560 $; $d=6$. Hãy tính tốc độ $v$ (làm tròn kết quả đến hàng đơn vị của feet trên giây).
	Biết rằng $1$ in$=2{,}54$ cm; $1$ ft(feet) $=0{,}3048$ m; $1$ lb(pound) $=0{,}45359237 $ kg; $1$ lb/in$^2=6\, 894{,}75729$ Pa(Pascal).
	\end{listEX}
	\loigiai{
	\begin{listEX}
	\item \begin{eqnarray*}
	P&=&0{,}00161 \cdot \dfrac{v^2 L}{d}\\
	Pd&=&0{,}00161 \cdot v^2 \cdot L\\
	v^2&=&\dfrac{Pd}{0{,}00161 \cdot L}\\
	v&=&\sqrt{\dfrac{Pd}{0{,}00161 \cdot L}}.
	\end{eqnarray*}
	\item $v=\sqrt{\dfrac{Pd}{0{,}00161 \cdot L}}=\sqrt{\dfrac{198{,}5 \cdot 6}{0{,}00161 \cdot 11\, 560}}\approx 8{,}0$.
	\end{listEX}
	}
\end{bt}
%%%%%%%%%
%%==========Bài 26
\begin{bt}
	Rút gọn các biểu thức sau:
	\begin{listEX}[2]
	\item $\sqrt{11-6\sqrt{2}}-\sqrt{11+6\sqrt{2}} $.
	\item $\sqrt{(2-\sqrt{5})^2}+\sqrt{14-6\sqrt{5}} $.
	\item $(2+\sqrt{7})\sqrt{11-4\sqrt{7}} $.
	\item $\sqrt{(3+\sqrt{2})^2}+\sqrt{6-4\sqrt{2}} $.
	\item $\sqrt{9-3\sqrt{8}}-\dfrac{\sqrt{3}-1}{\sqrt{2}}+\sqrt{5-2\sqrt{6}}-\sqrt{2-\sqrt{3}}$.
	\item $\dfrac{2-\sqrt{3}}{\sqrt{2}+\sqrt{2+\sqrt{3}}}+\dfrac{2+\sqrt{3}}{\sqrt{2}-\sqrt{2-\sqrt{3}}}$.
	\item $\sqrt{\left(1-\sqrt{2}\right)^2}+\sqrt{11-6\sqrt{2}}$.
	\item $\dfrac{2}{\sqrt{5}+\sqrt{3}}-\sqrt{\dfrac{2}{4-\sqrt{15}}}+6\sqrt{\dfrac{1}{3}}$.
	\item $\dfrac{1}{2}\sqrt{12-8\sqrt{2}}+\sqrt{17-12\sqrt{2}}-4\sqrt{2}$.
	\item $\sqrt{19-8\sqrt{3}}+\sqrt{4-2\sqrt{3}}$.
	\item $\sqrt{12+3\sqrt{3}+\sqrt{4+2\sqrt{3}}}-2\sqrt{3}$.
	\end{listEX}
	\loigiai{
	\begin{listEX}
	\item $\sqrt{9+2 -2\cdot 3\cdot \sqrt{2}} -\sqrt{9+2+2\cdot 3\cdot \sqrt{2}}\\
	=\sqrt{(3-\sqrt{2})^2} -\sqrt{(3+\sqrt{2})^2} \\
	=3-\sqrt{2} -3-\sqrt{2} \\
	=-2\sqrt{2}$.
	\item $\sqrt{(\sqrt{5} -2)^2} +\sqrt{9+5-2.3\sqrt{5}} \\
	=\sqrt{5} -2 +\sqrt{(3-\sqrt{5})^2} \\
	=\sqrt{5} -2 +3-\sqrt{5} \\
	=1$.
	\item $ (2+\sqrt{7}) \sqrt{7+4 -2\cdot 2\sqrt{7}} \\
	=(2+\sqrt{7}) \sqrt{{(\sqrt{7} -2)}^2} \\
	=(\sqrt{7} +2) (\sqrt{7} -2) \\
	=7-4\\
	=3$.
	\item $3+\sqrt{2} +\sqrt{4+2 -2\cdot 2\sqrt{2}} \\
	=3+\sqrt{2} +\sqrt{{(2-\sqrt{2})}^2} \\
	=3+\sqrt{2} +2-\sqrt{2} \\
	=5$.
	\item $\sqrt{9-3\sqrt{8}}-\dfrac{\sqrt{3}-1}{\sqrt{2}}+\sqrt{5-2\sqrt{6}}-\sqrt{2-\sqrt{3}}$\\
	$=\sqrt{6+3-2\sqrt{6}\sqrt{3}}-\dfrac{\sqrt{6}-\sqrt{2}}{2}+\sqrt{3+2-2\sqrt{3}\sqrt{2}}-\dfrac{\sqrt{8-4\sqrt{3}}}{2}$\\
	$=\sqrt{(\sqrt{6}-\sqrt{3})^2}-\dfrac{\sqrt{6}-\sqrt{2}}{2}+\sqrt{(\sqrt{3}-\sqrt{2})^2}-\dfrac{\sqrt{(\sqrt{6}-\sqrt{2})^2}}{2}$\\
	$=\sqrt{6}-\sqrt{3}-\dfrac{\sqrt{6}-\sqrt{2}}{2}+\sqrt{3}-\sqrt{2}-\dfrac{\sqrt{6}-\sqrt{2}}{2}$\\
	$=0$.
	\item $\dfrac{2-\sqrt{3}}{\sqrt{2}+\sqrt{2+\sqrt{3}}}+\dfrac{2+\sqrt{3}}{\sqrt{2}-\sqrt{2-\sqrt{3}}}$\\
	$=\dfrac{2\sqrt{2}-\sqrt{6}}{2+\sqrt{4+2\sqrt{3}}}+\dfrac{2\sqrt{2}+\sqrt{6}}{2-\sqrt{4-2\sqrt{3}}}$\\
	$=\dfrac{2\sqrt{2}-\sqrt{6}}{3+\sqrt{3}}+\dfrac{2\sqrt{2}+\sqrt{6}}{3-\sqrt{3}}$\\
	$=\dfrac{(2\sqrt{2}-\sqrt{6})(3-\sqrt{3})+(2\sqrt{2}+\sqrt{6})(3+\sqrt{3})}{(3+\sqrt{3})(3-\sqrt{3})}$\\
	$=\dfrac{6\sqrt{2}-2\sqrt{6}-3\sqrt{6}+3\sqrt{2}+
	6\sqrt{2}+2\sqrt{6}+3\sqrt{6}+3\sqrt{2}}{(3+\sqrt{3})(3-\sqrt{3})}$\\
	$=3\sqrt{2}$.
	\item $\sqrt{\left(1-\sqrt{2}\right)^2}+\sqrt{11-6\sqrt{2}}=\sqrt{2}-1+\sqrt{\left(3-\sqrt{2}\right)^2}=\sqrt{2}-1+3-\sqrt{2}=2$. 
	\item $\dfrac{2}{\sqrt{5}+\sqrt{3}}-\sqrt{\dfrac{2}{4-\sqrt{15}}}+6\sqrt{\dfrac{1}{3}}\\
	=\dfrac{2(\sqrt{5}-\sqrt{3})}{2}-\sqrt{2\left(4+\sqrt{15}\right)}+\dfrac{6\sqrt{3}}{3}\\
	=\sqrt{5}-\sqrt{3}-\sqrt{8+2\sqrt{15}}+2\sqrt{3}\\
	=\sqrt{5}+\sqrt{3}-\sqrt{\left(\sqrt{5}+\sqrt{3}\right)^2}\\
	=\sqrt{5}+\sqrt{3}-\left(\sqrt{5}+\sqrt{3}\right)=0$.	
	\item $\dfrac{1}{2}\sqrt{12-8\sqrt{2}}+\sqrt{17-12\sqrt{2}}-4\sqrt{2}$\\
	$=\dfrac{1}{2}\sqrt{\left(2\sqrt{2}-2\right)^2}+\sqrt{\left(3-2\sqrt{2}\right)^2}-4\sqrt{2}$.\\
	$=\dfrac{1}{2}\left|2\sqrt{2}-2\right|+\left|3-2\sqrt{2}\right|-4\sqrt{2}$\\
	$=\dfrac{1}{2}\left(2\sqrt{2}-2\right)+3-2\sqrt{2}-4\sqrt{2}$\\
	$=-5\sqrt{2}+2$.
	\item $\sqrt{19-8\sqrt{3}}+\sqrt{4-2\sqrt{3}}=\sqrt{(4-\sqrt{3})^2}+\sqrt{(\sqrt{3}-1)^2}=4-\sqrt{3}+\sqrt{3}-1=3$.
	\item $\sqrt{12+3\sqrt{3}+\sqrt{4+2\sqrt{3}}}-2\sqrt{3}=\sqrt{12+3\sqrt{3}+\sqrt{(\sqrt{3}+1)^2}}-2\sqrt{3}$.
	\\ $=\sqrt{12+3\sqrt{3}+1+\sqrt{3}}-2\sqrt{3}=\sqrt{13+4\sqrt{3}}-2\sqrt{3}$.
	\\ $=\sqrt{(2\sqrt{3}+1)^2}-2\sqrt{3}=2\sqrt{3}+1-2\sqrt{3}=1$.
	\end{listEX}}
\end{bt}
%%==========Bài 27
\begin{bt}
	Cho các biểu thức 
	\begin{center}
	$A= \sqrt{20a + 92 + \sqrt{a^4 + 16a^2 + 64}}$.
	$B= a^4 + 20a^3+100a^2$.
	\end{center}
	\begin{listEX}
	\item Rút gọn $A$.
	\item Tìm $a$ để $A+B=0$.
	\end{listEX}
	\loigiai{
	\begin{listEX}
	\item $A= \sqrt{20a +92 + \sqrt{(a^2+8)^2}}= \sqrt{(a+10)^2}= \vert a+10 \vert$.
	\item $A+B= \vert a+10 \vert + a^2\cdot (a+10)^2 =0 \Rightarrow a=-10$.
	\end{listEX}
	}
\end{bt} 
%%==========Bài 28
\begin{bt}
	Rút gọn các biểu thức sau
	\begin{listEX}[2]
	\item $\sqrt{48}-2\sqrt{75}+\sqrt{108}-\dfrac{1}{7}\sqrt{147}$;
	\item $\left(\sqrt{44}+\sqrt{11}\right)\sqrt{11}$;
	\item $\sqrt{24}-6\sqrt{\dfrac{1}{6}}-\dfrac{3\sqrt{2}}{\sqrt{3}}$;
	\item $\sqrt{11-6\sqrt{2}}-\sqrt{11+6\sqrt{2}}$;
	\item $\sqrt{\left(2-\sqrt{5}\right)^2}+\sqrt{14-6\sqrt{5}} $;
	\item $\left(2+\sqrt{7}\right)\sqrt{11-4\sqrt{7}} $;
	\item $\sqrt{\left(3+\sqrt{2}\right)^2}+\sqrt{6-4\sqrt{2}} $;
	\item $\sqrt{9-3\sqrt{8}}-\dfrac{\sqrt{3}-1}{\sqrt{2}}+\sqrt{5-2\sqrt{6}}-\sqrt{2-\sqrt{3}}$.
	\end{listEX}
	\loigiai{
	\begin{listEX}	
	\item $\begin{aligned}[t]
	\sqrt{48}-2\sqrt{75}+\sqrt{108}-\dfrac{1}{7}\sqrt{147} &= 4\sqrt{3} -2 \cdot 5\sqrt{3} +6\sqrt{3} -\dfrac{1}{7} \cdot 7\sqrt{3}\\ 
	&=\sqrt{3} \left( 4-2\cdot 5+6-\dfrac{1}{7} \cdot 7 \right) =-\sqrt{3}.
	\end{aligned}$	
	\item $\left(\sqrt{44}+\sqrt{11}\right)\sqrt{11}=\left(2\sqrt{11} +\sqrt{11}\right) \sqrt{11} =3\sqrt{11}\cdot \sqrt{11} =3 \cdot 11=33$.
	\item $\sqrt{24}-6\sqrt{\dfrac{1}{6}}-\dfrac{3\sqrt{2}}{\sqrt{3}}=2 \sqrt{6} -\sqrt{6} -\sqrt{6} =0$.
	\item $\begin{aligned}[t]
	\sqrt{11-6\sqrt{2}}-\sqrt{11+6\sqrt{2}} &=\sqrt{9+2 -2 \cdot 3 \cdot \sqrt{2}} -\sqrt{9+2+2 \cdot 3 \cdot \sqrt{2}}\\	
	&=\sqrt{{(3-\sqrt{2})}^2} -\sqrt{{(3+\sqrt{2})}^2}\\
	&=3-\sqrt{2}-\left(3+\sqrt{2}\right)=-2\sqrt{2}.
	\end{aligned}$
	\item $\begin{aligned}[t]
	\sqrt{\left(2-\sqrt{5}\right)^2}+\sqrt{14-6\sqrt{5}} &=\sqrt{\left(\sqrt{5} -2\right)^2} +\sqrt{9+5-2\cdot 3\sqrt{5}} \\
	&=\sqrt{5}-2 +\sqrt{\left(3-\sqrt{5}\right)^2} \\
	&=\sqrt{5}-2 +3-\sqrt{5}=1.
	\end{aligned}$
	\item $\begin{aligned}[t]
	\left(2+\sqrt{7}\right)\sqrt{11-4\sqrt{7}} &=(2+\sqrt{7}) \sqrt{7+4 -2\cdot 2\sqrt{7}} \\
	&=\left(2+\sqrt{7}\right)\sqrt{\left(\sqrt{7}-2\right)^2} \\
	&=\left(\sqrt{7}+2\right)\left(\sqrt{7}-2\right)=7-4=3.
	\end{aligned}$
	\item $\begin{aligned}[t]
	\sqrt{\left(3+\sqrt{2}\right)^2}+\sqrt{6-4\sqrt{2}} &=3+\sqrt{2} +\sqrt{4+2-2\cdot 2\sqrt{2}}\\
	&=3+\sqrt{2}+\sqrt{\left(2-\sqrt{2}\right)^2} \\
	&=3+\sqrt{2}+2-\sqrt{2}=5.
	\end{aligned}$
	\item $\begin{aligned}[t]
	&\sqrt{9-3\sqrt{8}}-\dfrac{\sqrt{3}-1}{\sqrt{2}}+\sqrt{5-2\sqrt{6}}-\sqrt{2-\sqrt{3}}\\
	=&\sqrt{6+3-2\sqrt{6}\cdot \sqrt{3}}-\dfrac{\sqrt{6}-\sqrt{2}}{2}+\sqrt{3+2-2\sqrt{3}\cdot \sqrt{2}}-\dfrac{\sqrt{8-4\sqrt{3}}}{2}\\ =&\sqrt{\left(\sqrt{6}-\sqrt{3}\right)^2}-\dfrac{\sqrt{6}-\sqrt{2}}{2}+\sqrt{\left(\sqrt{3}-\sqrt{2}\right)^2}-\dfrac{\sqrt{\left(\sqrt{6}-\sqrt{2}\right)^2}}{2}\\
	=&\sqrt{6}-\sqrt{3}-\dfrac{\sqrt{6}-\sqrt{2}}{2}+\sqrt{3}-\sqrt{2}-\dfrac{\sqrt{6}-\sqrt{2}}{2}=0.
	\end{aligned}$	
	\end{listEX}}
\end{bt} 
%%==========Bài 29
\begin{bt}
	Rút gọn các biểu thức sau
	\begin{listEX}
	\begin{multicols}{2}
	\item $A=x+3 +\sqrt{x^2-6x+9}, \ \ (x \le 3)$.
	\item $B=\left|x-2\right|+\dfrac{\sqrt {x^2-4x+4} }{x-2}, \ \ (x < 2)$.
	\item $C=\sqrt {x^2+4x+4}-\sqrt {x^2}, \ \ (-2\le x \le 0)$.
	\item $D=2x-1-\dfrac{\sqrt {x^2-10x+25}}{x-5}$.
	\end{multicols}
	\end{listEX}
	\loigiai{
	\begin{listEX}	
	\item	
	$A=x+3 +\sqrt{x^2-6x+9}
	=x+3+\left| x-3 \right|
	=x+3+(3-x)=6.$
	\item 
	$B=\left|x-2\right|+\dfrac{\sqrt {x^2-4x+4} }{x-2}
	=-(x-2)+\dfrac{\left|x-2 \right|}{x-2}
	= -x+2+\dfrac{2-x}{x-2}=-x+1.$
	\item 
	$C=\sqrt{x^2+4x+4}-\sqrt{x^2}
	=\left| x+2\right|+\left| x \right|
	=x+2-x=2.$
	\item $D=2x-1-\dfrac{\sqrt {x^2-10x+25}}{x-5}
	=2x-1-\dfrac{\left|x-5 \right|}{x-5}$
	\begin{itemize}
	\item Với $x\geq 5$ thì $D=2x-1-\dfrac{x-5}{x-5}=2x-1-1=2x-2$.
	\item Với $x<5$ thì $D=2x-1-\dfrac{-(x-5)}{x-5}=2x-1+1=2x$.
	\end{itemize}
	\end{listEX}
	}
\end{bt}
%%==========Bài 30
\begin{bt}
	Phân tích biểu thức sau thành nhân tử
	\begin{listEX}\begin{multicols}{2}
	\item $A=\sqrt{x^2-4}-\sqrt{x^2+2x},(x>2)$;
	\item $B=\sqrt{x^3+1}+\sqrt{x(x+2)+1},(x>-1)$;
	\item $C=x-2\sqrt{x-1},(x>1)$;
	\item $D=x+\sqrt{x}-2,(x>0)$. \end{multicols}	
	\end{listEX}
	\loigiai{
	\begin{listEX}\begin{multicols}{2}
	\item $\begin{aligned}[t]
	A&=\sqrt{x^2-4}-\sqrt{x^2+2x}\\
	&=\sqrt{(x-2)(x+2)}-\sqrt{x(x+2)}\\
	&=\sqrt{x+2}(\sqrt{x-2}-\sqrt{x}).
	\end{aligned}$
	\item $\begin{aligned}[t]
	B&=\sqrt{x^3+1}+\sqrt{x(x+2)+1}\\
	&=\sqrt{(x+1)(x^2-x+1)}+\sqrt{(x+1)^2}\\
	&=\sqrt{x+1}(\sqrt{x^2-x+1}+\sqrt{x+1}).
	\end{aligned}$
	\item 
	$\begin{aligned}[t]
	C&=x-2\sqrt{x-1}\\
	&=x-1-2\sqrt{x-1}+1\\
	&=(\sqrt{x-1}+1)^2.
	\end{aligned}$
	\item 
	$\begin{aligned}[t]
	D&=x+\sqrt{x}-2\\
	&=x-\sqrt{x}+2\sqrt{x}-2\\
	&=\sqrt{x}(\sqrt{x}-1)+2(\sqrt{x}-1)\\
	&=(\sqrt{x}-1)(\sqrt{x}+2).
	\end{aligned}$ \end{multicols}	
	\end{listEX}
	}
\end{bt}
%%%%%%%%%%%%%%%%%%
\subsection{Bài tập trắc nghiệm}
\Opensolutionfile{ans}[ans-9T3-OTC]
%%==========Câu 1
\begin{ex}
	Căn bậc hai của $4$ là
	\choice
	{$2$}
	{$-2$}
	{\True $2$ và $-2$}
	{$\sqrt{2}$ và $-\sqrt{2}$}
	\loigiai{
	Căn bậc hai của $4$ là $2$ và $-2$.
	}
\end{ex}
%%==========Câu 2
\begin{ex}%[MaT-SGK9-Moi]%[Phạm Hoàng Điệp- Phạm Hoài]
	Căn bậc hai của $16$ là
	\choice
	{$4$}
	{\True $4$ và $-4$}
	{$256$}
	{$256$ và $-256$}
	\loigiai{
	Căn bậc hai của $16$ là $4$ và $-4$.
	}
\end{ex}
%%==========Câu 3
\begin{ex}
	Căn bậc hai số học của $49$ là
	\choice
	{\True $7$}
	{$-7$}
	{$7$ và $-7$}
	{$\sqrt{7}$ và $-\sqrt{7}$}
	\loigiai{
	Căn bậc hai số học của $49$ là $7$.
	}
\end{ex}
%%==========Câu 4
\begin{ex}%[MaT-SGK9-Moi]%[Phạm Hoàng Điệp- Phạm Hoài]
	Nếu $\sqrt{x} = 9$ thì $x$ bằng
	\choice 
	{$3$}
	{$3$ hoặc $-3$}
	{$81$}
	{$81$ hoặc $-81$}
	\loigiai{
	Ta có 	$\sqrt{81} = 9$ nên $x=81$.
	}
\end{ex}
%%==========Câu 5
\begin{ex}
	Rút gọn biểu thức $\sqrt[3]{\left(4-\sqrt{17}\right)^3}$ ta được	
	\choice
	{$4+\sqrt{17}$}
	{\True $4-\sqrt{17}$}
	{$\sqrt{17}-4$}
	{$-4-\sqrt{17}$}
	\loigiai{
	Ta có $\sqrt[3]{\left(4-\sqrt{17}\right)^3}=4-\sqrt{17}$.
	}
\end{ex}
%%==========Câu 6
\begin{ex}
	Độ dài đường kính (mét) của hình tròn có diện tích $4$ m$^2$ sau khi làm tròn đến chữ số thập phân thứ hai bằng
	\choice
	{\True $2{,}66$}
	{$2{,}50$}
	{$1{,}13$}
	{$1{,}12$}
	\loigiai{
	Ta có diện tích hình tròn $S=\dfrac{\pi d^2}{4}$ suy ra $d=\sqrt{\dfrac{4S}{\pi}}=\sqrt{\dfrac{4\cdot 4}{\pi}}\approx 2{,}26$.
	}
\end{ex}
%%==========Câu 7
\begin{ex}
	Một vật rơi tự do từ độ cao $396{,}9$ m. Biết quãng đường chuyển động $S$ (mét) của vật phụ thuộc vào thời gian $t$ (giây) bởi công thức $S=4{,}9t^2$. Vật chạm đất sau
	\choice
	{$8$ giây}
	{$5$ giây}
	{$11$ giây}
	{\True $9$ giây}
	\loigiai{
	Từ công thức $S=4{,}9t^2$ suy ra $t=\sqrt{\dfrac{S}{4{,}9}}=\sqrt{\dfrac{396{,}9}{4}}=9$.
	}
\end{ex}
%%==========Câu 8
\begin{ex}
	Biểu thức nào sau đây có giá trị khác với các biểu thức còn lại?
	\choice
	{$(-\sqrt{5})^2$}
	{$\sqrt{5^2}$}
	{$\sqrt{(-5)^2}$}
	{\True $-(\sqrt{5})^2$}
	\loigiai{
	Ta có 
	\begin{itemize}
	\item $(-\sqrt{5})^2=\sqrt{5^2}=\sqrt{(-5)^2}=5$;
	\item $-(\sqrt{5})^2=-5$.
	\end{itemize}
	Biểu thức $-(\sqrt{5})^2$ có giá trị khác với các biểu thức còn lại.
	}
\end{ex}
%%==========Câu 9
\begin{ex}
	Có bao nhiêu số tự nhiên $x$ để $\sqrt{16-x}$ là số nguyên?
	\choice
	{$2$}
	{$3$}
	{$4$}
	{\True $5$}
	\loigiai{
	Điều kiện $x\leq 16$.\\
	Để $\sqrt{16-x}$ là số nguyên thì $16-x$ phải là số chính phương.\\
	Suy ra $16-x\in\left\{0;1;4;9;16;25;\ldots\right\}$.\\
	Suy ra $x\in\left\{16;15;11;7;0;-9;\ldots\right\}$.\\
	Mà $x\in\mathbb{N}$ nên $x\in\left\{0;7;11;15;16\right\}$. Vậy có $5$ giá trị.
	}
\end{ex}
%%==========Câu 10
\begin{ex}
	Giá trị của biểu thức $\sqrt{16}+\sqrt[3]{-64}$ bằng
	\choice
	{\True $0$}
	{$-2$}
	{$8$}
	{$-4$}
	\loigiai{
	Ta có $\sqrt{16}+\sqrt[3]{-64}=\sqrt{4^2}+\sqrt[3]{(-4)^3}=4+(-4)=0$.
	}
\end{ex}
%%==========Câu 11
\begin{ex}
	Đẳng thức nào sau đây không đúng?
	\choice
	{$\sqrt{16}+\sqrt{144}=16$}
	{$\sqrt{0{,}64} \cdot \sqrt{9}=2{,}4$}
	{$\sqrt{(-18)^2}:\sqrt{6^2}=3$}
	{\True $\sqrt{(-3)^2}-\sqrt{7^2}=-10$}
	\loigiai{
	Ta có
	\begin{itemize}
	\item $\sqrt{16}+\sqrt{144}=4+12=16$;
	\item $\sqrt{0{,}64} \cdot \sqrt{9}=0{,}8\cdot 3=2{,}4$;
	\item $\sqrt{(-18)^2}:\sqrt{6^2}=18:6=3$;
	\item $\sqrt{(-3)^2}-\sqrt{7^2}=3-7=-4$.
	\end{itemize}
	}
\end{ex}
%%==========Câu 12
\begin{ex}
	Biết rằng $2{,}6^2=6{,}76$. Giá trị của biểu thức $\sqrt{0{,}0676}$ bằng
	\choice
	{$0{,}0026$}
	{$0{,}026$}
	{$0{,}26$}
	{$2{,}6$}
	\loigiai{
	Ta có $2{,}6^2=6{,}76\Rightarrow 0{,}0676=\dfrac{2{,}6^2}{10^2}=\left(\dfrac{2{,}6}{10}\right)^2=0{,}26^2$.\\
	Vậy $\sqrt{0{,}0676}=0{,}26$.
	}
\end{ex}
%%==========Câu 13
\begin{ex}
	Rút gọn biểu thức $\sqrt{9 a}-\sqrt{16 a}+\sqrt{64 a}$ với $a \geq 0$, ta có kết quả
	\choice
	{$15 \sqrt{a}$}
	{$15 a$}
	{\True $7 \sqrt{a}$}
	{$7 a$}
	\loigiai{
	Ta có $\sqrt{9 a}-\sqrt{16 a}+\sqrt{64 a}=3\sqrt{a}-4\sqrt{a}+8\sqrt{a}=7\sqrt{a}$.
	}
\end{ex}
%%==========Câu 14
\begin{ex}
	Cho $a =2 \sqrt{3}+\sqrt{2}$, $b =3 \sqrt{2}-2 \sqrt{3}$. Rút gọn biểu thức $\sqrt{3}a-\sqrt{2} b$, ta có kết quả
	\choice
	{\True $3 \sqrt{6}$}
	{$-\sqrt{6}$}
	{$6 \sqrt{3}$}
	{$12-\sqrt{6}$}
	\loigiai{
	Ta có $\sqrt{3}a-\sqrt{2}b=\sqrt{3}\cdot (2 \sqrt{3}+\sqrt{2})-\sqrt{2}\cdot (3 \sqrt{2}-2 \sqrt{3})=6+\sqrt{6}-6+2\sqrt{6}=3\sqrt{6}$.
	}
\end{ex}
%%==========Câu 15
\begin{ex}
	Trục căn thức ở mẫu biểu thức $\dfrac{\sqrt{6}-\sqrt{3}}{\sqrt{3 a}}$ với $a >0$, ta có kết quả
	\choice
	{$\dfrac{\sqrt{2}-1}{\sqrt{a}}$}
	{$\dfrac{(\sqrt{6}-\sqrt{3}) \sqrt{a}}{3a}$}
	{\True $\dfrac{(\sqrt{2}-1) \sqrt{a}}{a}$}
	{$\sqrt{2a}-\sqrt{a}$}
	\loigiai{
	Ta có $\dfrac{\sqrt{6}-\sqrt{3}}{\sqrt{3 a}}=\dfrac{(\sqrt{6}-\sqrt{3})\sqrt{3a}}{3a}=\dfrac{(3\sqrt{2}-3)\sqrt{a}}{3a}=
	\dfrac{(\sqrt{2}-1)\sqrt{a}}{a}$.
	}
\end{ex}
%%==========Câu 16
\begin{ex}
	Kết quả của phép tính $\sqrt{27}: \sqrt{6} \cdot 2 \sqrt{18}$ là
	\choice
	{$12$}
	{\True $18$}
	{$72$}
	{$144$}
	\loigiai{
	Ta có $\sqrt{27}: \sqrt{6} \cdot 2 \sqrt{18}=\dfrac{3\sqrt{3}}{\sqrt{3}\cdot \sqrt{2}}\cdot 2\cdot 3\sqrt{2}=\dfrac{3}{\sqrt{2}}\cdot 6\sqrt{2}=18$.
	}
\end{ex}
%%==========Câu 17
\begin{ex}
	Rút gọn biểu thức $\dfrac{1}{2 \sqrt{a}+\sqrt{2}}-\dfrac{1}{2 \sqrt{a}-\sqrt{2}}$ với $a \geq 0,a\neq \dfrac{1}{2}$, ta có kết quả
	\choice
	{\True $\dfrac{\sqrt{2}}{1-2 a}$}
	{$\dfrac{\sqrt{2}}{2 a-1}$}
	{$\dfrac{\sqrt{a}}{2 a-1}$}
	{$\dfrac{\sqrt{2}}{1-a}$}
	\loigiai{
	Ta có
	\begin{eqnarray*}
	\dfrac{1}{2 \sqrt{a}+\sqrt{2}}-\dfrac{1}{2 \sqrt{a}-\sqrt{2}}&=&\dfrac{(2\sqrt{a}-\sqrt{2})-(2\sqrt{a}+\sqrt{2})}{(2\sqrt{a}+\sqrt{2})(2\sqrt{a}-\sqrt{2})}=\dfrac{2\sqrt{a}-\sqrt{2}-2\sqrt{a}-\sqrt{2}}{(2\sqrt{2})-(\sqrt{2})^2}\\
	&=&\dfrac{-2\sqrt{2}}{4a-2}=\dfrac{\sqrt{2}}{1-2a}
	\end{eqnarray*}
	}
\end{ex}
\Closesolutionfile{ans/9T3-OTC}	