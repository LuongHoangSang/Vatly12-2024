\section*{LUYỆN TẬP CHUNG}
%%==========Bài 1
\begin{bt}
	\immini
	{ Cho hình bên. Tính $\sin C$ và $\tan B$.
	}
	{\begin{tikzpicture}[line join = round, line cap = round,>=stealth,font=\footnotesize,scale=.7] 
	%
	\tkzDefPoints{0/0/B} 
	\coordinate (C) at ($(B)+(6,0)$); 
	\tkzDefTriangle[two angles = 55 and 35](B,C) \tkzGetPoint{A}
	\coordinate (H) at ($(B)!(A)!(C)$);
	\tkzDrawSegments(A,H)
	\pgfresetboundingbox 
	\tkzMarkRightAngles[size=.2](B,A,C A,H,C) 
	\tkzDrawPolygon(A,B,C)
	\tkzDrawSegments(A,H)
	\tkzLabelSegments[color=black,below](B,H){$1$}
	\tkzLabelSegments[color=black,below](H,C){$2$}
	\tkzDrawPoints[fill=black](A,B,C,H) 
	\tkzLabelPoints[above](A) 
	\tkzLabelPoints[left](B) 
	\tkzLabelPoints[right](C)
	\tkzLabelPoints[below](H) 
	\end{tikzpicture}
	}
	\loigiai
	{Ta có $AB^2 = BC\cdot BH = 3\cdot 1$ suy ra $AB = \sqrt{3}$.\\
	Tương tự $AH^2 = BH\cdot CH = 1\cdot 2$ suy ra $AH = \sqrt{2}$.\\
	Do đó $\sin C = \dfrac{AB}{BC} = \dfrac{\sqrt{3}}{3}$ và $\tan B = \dfrac{AH}{BH} = \dfrac{\sqrt{2}}{1} = \sqrt{2}$.
	}
\end{bt}
%%==========Bài 2
\begin{bt}%Vi du 1. 
	Cho tam giác $A B C$ vuông tại $A, A B=5 \mathrm{~cm}$, $A C=12 \mathrm{~cm}$.
	\begin{enumEX}{1}
	\item Tính các tỉ số lượng giác của góc $B$.
	\item Từ kết quả câu a) suy ra các tỉ số lượng giác của góc $C$.
	\end{enumEX}
	\loigiai{
	\begin{enumEX}{1}
	\item Xét tam giác $A B C$ vuông tại $A$. Theo định lí Pythagore, ta có 
	\immini{
	$B C^{2}=A B^{2}+A C^{2}=5^{2}+12^{2}=169$, suy ra $B C=13 \mathrm{~cm}$.\\
	Ta có $\sin B=\dfrac{A C}{B C}=\dfrac{12}{13}, \cos B=\dfrac{A B}{B C}=\dfrac{5}{13},$\\
	$\tan B=\dfrac{A C}{A B}=\dfrac{12}{5}, \cot B=\dfrac{A B}{A C}=\dfrac{5}{12}$.
	\item Do $\widehat{B}+\widehat{C}=90^{\circ}$ nên \\
	$\sin C=\cos B=\dfrac{5}{13}, \cos C=\sin B=\dfrac{12}{13},$\\
	$\tan C=\cot B=\dfrac{5}{12}$, $\cot C=\tan B=\dfrac{12}{5}$.
	}{
	\begin{tikzpicture}[line join=round, line cap=round, >=stealth,font=\footnotesize, scale=0.75]
	\tikzset{every node/.style={scale=1}}% thu nhỏ phóng tỏ tex trong hình
	\coordinate (A) at (0,0);
	\coordinate (B) at (0,3);
	\coordinate (C) at (5.5,0);
	\coordinate (m) at ($(A)!0.5!(C)$);
	\coordinate (n) at ($(B)!0.5!(A)$);
	\draw 
	(A)--(B)--(C)--(A) 
	;	
	\draw[fill=black] 
	(A) circle (0.05) node[below] {$A$}
	(B) circle (0.05) node[above] {$B$}
	(C) circle (0.05) node[below] {$C$}
	(m) node[below] {$12$ cm}
	(n) node[right] {$5$ cm}
	;
	\draw pic[draw=black,angle radius=5pt] {right angle = B--A--C};	
	\end{tikzpicture}
}
	\end{enumEX}
	}
\end{bt}
%%==========Bài 3
\begin{bt} %4.15. 
	Cho tam giác $A B C$ có chân đường cao $A H$ nằm giữa $B$ và $C$. Biết $H B=3 \mathrm{~cm}$, $H C=6 \mathrm{~cm}, \widehat{H A C}=60^{\circ}$. Hãy tính độ dài các cạnh (làm tròn đến $\mathrm{cm}$), số đo các góc của tam giác $A B C$ (làm tròn đến độ).	
	\loigiai{
	\immini{
	Xét tam giác $AHC$ vuông tại $H$,
	ta có $AC = \dfrac{HC}{\sin 60^\circ} = \dfrac{6}{\frac{\sqrt{3}}{2}} \approx 7$ (cm).\\
	$\tan \widehat{HAC} = \dfrac{HC}{HA}$ nên
	$HA = \dfrac{HC}{\tan \widehat{HAC}} = \dfrac{6}{\tan 60^\circ} = \dfrac{6}{\sqrt{3}} \approx 3$ (cm). \\
	Xét tam giác vuông $ABH$ vuông tại $H$,
	theo định lý Pythagore ta có\\
	$AB^2=HB^2+HC^2 $ \\
	$\Rightarrow AB = \sqrt{HB^2+HA^2} = \sqrt{3^2+3^3} \approx 4$ (cm).\\
	$BC=HB+HC=3+6=9$ (cm).\\
	Vậy $AB= 4$ cm, $AC = 7$ cm và $BC = 9$ cm.
	}{
	\begin{tikzpicture}[line join=round, line cap=round, >=stealth,font=\footnotesize, scale=0.8]%Hinh nón
	\tikzset{every node/.style={scale=1}}% thu nhỏ phóng tỏ tex trong hình
	\coordinate (B) at (0,0); 
	\coordinate (C) at (7,0); 
	\coordinate (A) at (2,3.5); 
	\coordinate (H) at ($(B)!(A)!(C)$);
	\coordinate (m) at ($(H)!0.5!(B)$);
	\coordinate (n) at ($(H)!0.5!(C)$);
	\draw 
	(A)--(B)--(C)--(A) (A)--(H)
	;
	\draw[fill=black] 
	(A) circle (0.05) node[above] {$A$}
	(B) circle (0.05) node[below] {$B$}
	(C) circle (0.05) node[below] {$C$}
	(H) circle (0.05) node[below] {$H$}
	($(A)+(0.45,-0.6)$) node {\tiny$60^\circ$}
	(m) node[below] {$3$ cm}
	(n) node[below] {$6$ cm}
	;
	\draw pic[draw,angle radius=4mm] {angle = H--A--C};
	\draw pic[draw,angle radius=4mm] {right angle = A--H--C};
	\end{tikzpicture}
	}	
	}
\end{bt}
%%==========Bài 4
\begin{bt} %4.17. 
	Tính các số liệu còn thiếu (dấu \lq\lq?\rq\rq) ở hình sau với góc làm tròn đến độ, với độ dài làm tròn đến chữ số thập phân thứ nhất.
	\begin{center}
	\begin{tikzpicture}[scale=.95, line join=round, line cap=round, >=stealth]	
	\tikzset{every node/.style={scale=1}}% thu nhỏ phóng tỏ tex 
	\definecolor{maungoai}{RGB}{143,227,255}
	\definecolor{mautrong}{RGB}{176,240,255}
	\definecolor{lua}{RGB}{103,201,25}	
	\coordinate (A) at (0,0);
	\coordinate (B) at (3,0);
	\coordinate (C) at (3,2.5);
	\coordinate (m) at ($(A)!0.5!(B)$);
	\coordinate (n) at ($(C)!0.5!(B)$);
	\draw (A)--(B)--(C)--(A)
	;
	\draw[fill=black] (A) node[below]{$A$} circle (1pt) 
	(B) circle (1pt) node[below]{$B$} 
	(C) circle (1pt) node[above]{$C$} 	
	(m) node[below]{$3$} 
	(n) node[right]{?} 
	($(A)+(1,0.3)$) node {$40^\circ$}
	;	
	\draw pic[draw,angle radius=2mm] {right angle = A--B--C};
	\draw pic[draw,angle radius=4mm] {angle = B--A--C};
	\path (current bounding box.south) node[below=2mm]{a)};
	\end{tikzpicture}
	\begin{tikzpicture}[scale=.95, line join=round, line cap=round, >=stealth]	
	\tikzset{every node/.style={scale=1}}% thu nhỏ phóng tỏ tex 
	\definecolor{maungoai}{RGB}{143,227,255}
	\definecolor{mautrong}{RGB}{176,240,255}
	\definecolor{lua}{RGB}{103,201,25}	
	\coordinate (A) at (0,0);
	\coordinate (B) at (2.7,0);
	\coordinate (C) at (2.7,2.5);
	\coordinate (D) at (0,2.5);
	\coordinate (m) at ($(B)!0.5!(C)$);
	\coordinate (n) at ($(D)!0.5!(B)$);
	\draw (A)--(B)--(C)--(D)--(A) (D)--(B)
	;
	\draw[fill=black] (A) node[below]{A} circle (1pt) 
	(B) circle (1pt) node[below]{$B$} 
	(C) circle (1pt) node[above]{$C$} 	
	(D) circle (1pt) node[above]{$D$} 
	(m) node[right]{$7$} 
	(n) node[right]{$10$} 
	($(B)+(-1,0.3)$) node {?}
	;	
	\draw pic[draw,angle radius=2mm] {right angle = D--A--B};
	\draw pic[draw,angle radius=2mm] {right angle = B--C--D};
	\draw pic[draw,angle radius=2mm] {right angle = C--D--A};
	\draw pic[draw,angle radius=4mm] {angle = D--B--A};
	\path (current bounding box.south) node[below=2mm]{b)};
	\end{tikzpicture}
	\begin{tikzpicture}[scale=.75, line join=round, line cap=round, >=stealth]	
	\tikzset{every node/.style={scale=1}}% thu nhỏ phóng tỏ tex 
	\definecolor{maungoai}{RGB}{143,227,255}
	\definecolor{mautrong}{RGB}{176,240,255}
	\definecolor{lua}{RGB}{103,201,25}	
	\coordinate (A) at (0,0);
	\coordinate (x) at (7,0);
	\coordinate (C) at (2,3);
	\coordinate (m) at ($(C)!1!90:(A)$); % quay -120^0 tâm O biến C thành B 
	\coordinate (n) at ($(C)!1.5!(m)$);
	\coordinate (a) at ($(A)!.5!(C)$);
	\coordinate (b) at ($(n)!.5!(C)$);
	\draw (A)--(n)--(C)--(A) 
	;
	\draw[fill=black] (A) node[below]{A} circle (1pt) 
	(C) circle (1pt) node[above]{$C$} 	
	(n) circle (1pt) node[below]{$B$} 
	(a) node[left]{$5$}
	(b) node[right]{$7$}
	($(A)+(1,0.4)$) node {?}
	;	
	\draw pic[draw,angle radius=2mm] {right angle = A--C--n};
	\draw pic[draw,angle radius=4mm] {angle = n--A--C};
	\path (current bounding box.south) node[below=2mm]{c)};
	\end{tikzpicture}
	\begin{tikzpicture}[scale=.65, line join=round, line cap=round, >=stealth]	
	\tikzset{every node/.style={scale=1}}% thu nhỏ phóng tỏ tex 
	\definecolor{maungoai}{RGB}{143,227,255}
	\definecolor{mautrong}{RGB}{176,240,255}
	\definecolor{lua}{RGB}{103,201,25}	
	\coordinate (A) at (0,0);
	\coordinate (x) at (6,0);
	\coordinate (y) at (6,5);
	\coordinate (B) at ($(A)!0.4!(y)$);
	\coordinate (C) at ($(A)!0.7!(y)$);
	\coordinate (H) at ($(A)!(B)!(x)$);
	\coordinate (K) at ($(A)!(C)!(x)$);
	\coordinate (a) at ($(A)!0.5!(B)$);
	\coordinate (b) at ($(B)!.5!(C)$);
	\coordinate (c) at ($(B)!.5!(H)$);
	\coordinate (d) at ($(C)!.5!(K)$);
	\draw (x)--(A)--(y) (B)--(H) (C)--(K) 
	;
	\draw[fill=black] (A) node[below]{$A$} circle (1pt) 
	(B) circle (1pt) node[above]{$B$} 
	(C) circle (1pt) node[above]{$C$} 	
	(H) circle (1pt) node[below]{$H$} 
	(K) circle (1pt) node[below]{$K$} 
	(a) node[above]{$3$}
	(b) node[above]{$2$}
	(c) node[right]{?}
	(d) node[right]{?}
	($(A)+(1.15,0.3)$) node {$35^\circ$}
	;	
	\draw pic[draw,angle radius=2mm] {right angle = B--H--A};
	\draw pic[draw,angle radius=2mm] {right angle = C--K--A};
	\draw pic[draw,angle radius=4mm] {angle = x--A--y};
	\path (current bounding box.south) node[below=2mm]{d)};
	\end{tikzpicture}
	\end{center}
	\loigiai{
	\begin{enumEX}{1}
	\item 
	Từ hình vẽ ta có $\tan 40^\circ = \dfrac{BC}{AB} $ 
	$ \Rightarrow BC = AB \cdot \tan 40^\circ = 3\cdot \tan 40^\circ \approx 2{,}5$.
	\item 
	Từ hình vẽ ta có $\sin \widehat{ABD} = \dfrac{AD}{DB} = \dfrac{BC}{DB} =\dfrac{7}{10} $ 
	$ \Rightarrow \widehat{ABD} \approx 44^\circ 25'$.
	\item
	Từ hình vẽ ta có $\tan \widehat{CAB} = \dfrac{CB}{CA} = \dfrac{7}{5} $ 
	$ \Rightarrow \widehat{CAB} \approx 54^\circ 27'$.
	\item
	Từ hình vẽ, xét tam giác $ABH$ vuông tại $H$, ta có \\
	$HB=AB \cdot \sin \widehat{35^\circ} = 3\cdot \sin 35^\circ \approx 1{,}7$.\\
	$CK=AC \cdot \sin \widehat{35^\circ} = (3+2)\cdot \sin 35^\circ \approx 2{,}9$.
	\end{enumEX}
	}
\end{bt}
%%==========Bài 5
\begin{bt}
	Giải tam giác $ABC$ vuông tại $A$, biết
	\begin{listEX}[2]
	\item $AB = 2{,}7$ và $AC = 4{,}5$;
	\item $AC = 4{,}0$ và $BC = 4{,}8$.
	\end{listEX}
	\loigiai
	{\begin{enumerate}
	\item Xét $\triangle ABC$ vuông ở $A$, ta có $\tan B = \dfrac{AC}{AB} = \dfrac{4{,}5}{2{,}7}\approx \tan 59^{\circ}04'$
	\immini
	{Suy ra $\widehat{B}\approx 59^{\circ}04'$ mà $\widehat{B} + \widehat{C} = 90^{\circ}$ nên
	$$\widehat{C} = 90^{\circ} - \widehat{B} = 90^{\circ} - 59^{\circ}04' = 30^{\circ}56'.$$
	Mặt khác, theo định lí Py-ta-go ta có 	
	$$BC = \sqrt{AB^2 + AC^2} = \sqrt{2{,}7^2 + 4{,}5^2}\approx 5{,}25.$$
	}
	{\begin{tikzpicture}[line join = round, line cap = round,>=stealth,font=\footnotesize,scale=0.8]
	\tkzDefPoints{0/0/B} 
	\coordinate (C) at ($(B)+(6,0)$); 
	\tkzDefTriangle[two angles = 55 and 35 ](B,C) \tkzGetPoint{A} 
	\pgfresetboundingbox 
	\tkzDrawPolygon(A,B,C)
	\tkzLabelSegments[color=black,left](A,B){$2{,}7$}
	\tkzLabelSegments[color=black,right](A,C){$4{,}5$}
	\tkzDrawPoints[fill=black](A,B,C) 
	\tkzLabelPoints[above](A) 
	\tkzLabelPoints[left](B) 
	\tkzLabelPoints[right](C) 
	\tkzMarkRightAngles(B,A,C)
	\end{tikzpicture}
	}
	\item Xét $\triangle ABC$ vuông ở $A$, ta có $\sin B = \dfrac{AC}{BC} = \dfrac{4{,}0}{4{,}8}\approx \sin 56^{\circ}44'$
	\immini
	{Suy ra $\widehat{B}\approx 56^{\circ}44'$ mà $\widehat{B} + \widehat{C} = 90^{\circ}$ nên
	$$\widehat{C} = 90^{\circ} - \widehat{B} = 90^{\circ} - 56^{\circ}44' = 33^{\circ}16'.$$
	Mặt khác, theo định lí Py-ta-go ta có 	
	$$AB = \sqrt{BC^2 - AC^2} = \sqrt{4{,}8^2 - 4{,}0^2}\approx 2{,}65.$$
	}
	{\begin{tikzpicture}[line join = round, line cap = round,>=stealth,font=\footnotesize,scale=0.8]
	\tkzDefPoints{0/0/B} 
	\coordinate (C) at ($(B)+(6,0)$); 
	\tkzDefTriangle[two angles = 56 and 34 ](B,C) \tkzGetPoint{A} 
	\pgfresetboundingbox 
	\tkzDrawPolygon(A,B,C)
	\tkzLabelSegments[color=black,right](A,C){$4{,}0$}
	\tkzLabelSegments[color=black,below](B,C){$4{,}8$}
	\tkzDrawPoints[fill=black](A,B,C) 
	\tkzLabelPoints[above](A) 
	\tkzLabelPoints[left](B) 
	\tkzLabelPoints[right](C) 
	\tkzMarkRightAngles(B,A,C)
	\end{tikzpicture}
	}	
	\end{enumerate}
	}
\end{bt}
%%==========Bài 6
\begin{bt}
	Giải tam giác $ABC$ vuông tại $A$, biết
	\begin{listEX}[2]
	\item $BC = 4{,}5$ và $\widehat{C} = 35^{\circ}$;
	\item $AB = 3{,}1$ và $\widehat{B} = 65^{\circ}$.
	\end{listEX}
	\loigiai
	{\begin{enumerate}
	\item Xét $\triangle ABC$ vuông ở $A$, ta có
	\immini
	{ $$AB = BC\cdot\sin C = 4{,}5\cdot\sin 35^{\circ} \approx 2{,}58.$$
	Tương tự, $AC = BC\cdot\cos C = 4{,}5\cdot\cos 35^{\circ} =\approx 3{,}69$.\\ 
	Do $\widehat{B} + \widehat{C} = 90^{\circ}$ nên
	$$\widehat{B} = 90^{\circ} - \widehat{C} = 90^{\circ} - 35^{\circ} = 55^{\circ}.$$
	}
	{\begin{tikzpicture}[line join = round, line cap = round,>=stealth,font=\footnotesize,scale=0.7]
	\tkzDefPoints{0/0/B} 
	\coordinate (C) at ($(B)+(6,0)$); 
	\tkzDefTriangle[two angles = 55 and 35 ](B,C) \tkzGetPoint{A} 
	\pgfresetboundingbox 
	\tkzDrawPolygon(A,B,C)
	\tkzLabelSegments[color=black,below](B,C){$4{,}5$}
	\tkzDrawPoints[fill=black](A,B,C) 
	\tkzLabelPoints[above](A) 
	\tkzLabelPoints[left](B) 
	\tkzLabelPoints[right](C) 
	\tkzMarkRightAngles(B,A,C)
	\tkzMarkAngles[size=.5,arc=l](A,C,B)
	\tkzLabelAngles[pos=.9](A,C,B){\tiny$35^\circ$}
	\end{tikzpicture}
	}
	\item Xét $\triangle ABC$ vuông ở $A$, ta có
	\immini
	{ $$BC = \dfrac{AB}{\cos B} = \dfrac{3{,}1}{\cos 65^{\circ}} \approx 7{,}34.$$
	Tương tự, $AC = AB\cdot\tan B = 3{,}1\cdot\tan 65^{\circ} =\approx 6{,}65$.\\ 
	Do $\widehat{B} + \widehat{C} = 90^{\circ}$ nên
	$$\widehat{C} = 90^{\circ} - \widehat{B} = 90^{\circ} - 65^{\circ} = 25^{\circ}.$$
	}
	{\begin{tikzpicture}[line join = round, line cap = round,>=stealth,font=\footnotesize,scale=0.8]
	\tkzDefPoints{0/0/B} 
	\coordinate (C) at ($(B)+(6,0)$); 
	\tkzDefTriangle[two angles = 65 and 25 ](B,C) \tkzGetPoint{A} 
	\pgfresetboundingbox 
	\tkzDrawPolygon(A,B,C)
	\tkzLabelSegments[color=black,left](A,B){$3{,}1$}
	\tkzDrawPoints[fill=black](A,B,C) 
	\tkzLabelPoints[above](A) 
	\tkzLabelPoints[left](B) 
	\tkzLabelPoints[right](C) 
	\tkzMarkRightAngles(B,A,C)
	\tkzMarkAngles[size=.5,arc=l](C,B,A)
	\tkzLabelAngles[pos=.95](C,B,A){\tiny $65^\circ$}
	\end{tikzpicture}
	}	
	\end{enumerate}
	}
\end{bt}
%%==========Bài 7
\begin{bt}%4.14. 
	Một cuốn sách khổ $17 \times 24 \mathrm{~cm}$, tức là chiều rộng $17 \mathrm{~cm}$, chiểu dài $24 \mathrm{~cm}$. Gọi $\alpha$ là góc giữa đường chéo và cạnh $17 \mathrm{~cm}$. Tính $\sin \alpha$, $\cos \alpha$ (làm tròn đến chư số thập phân thú hai) và tính số đo $\alpha$ (làm tròn đến độ).
	\loigiai{
	\immini{
	Theo bài ra ta có $\alpha = \widehat{CBD} = \dfrac{DC}{BC} = \dfrac{AB}{BC} = \dfrac{24}{17}$.\\
	Do đó $\alpha\approx 54^\circ 41'$.
	}{
	\begin{tikzpicture}[line join=round, line cap=round, >=stealth,font=\footnotesize, scale=0.7]%Hinh nón
	\tikzset{every node/.style={scale=1}}% thu nhỏ phóng tỏ tex trong hình
	\coordinate (A) at (0,0); 
	\coordinate (B) at (5,0); 
	\coordinate (C) at (5,3.5); 
	\coordinate (D) at (0,3.5); 
	\coordinate (m) at ($(A)!0.5!(B)$);
	\coordinate (n) at ($(B)!0.5!(C)$);
	\draw 
	(A)--(B)--(C)--(D)--(A) (D)--(B)
	;
	\draw[fill=black] 
	(A) circle (0.05) node[below] {$A$}
	(B) circle (0.05) node[below] {$B$}
	(C) circle (0.05) node[above] {$C$}
	(D) circle (0.05) node[above] {$D$}
	($(B)+(120:0.9)$) node {$\alpha$}
	(m) node[below] {$24$ cm}
	(n) node[right] {$17$ cm}
	;
	\draw pic[draw,angle radius=4mm] {angle = C--B--D};
	\end{tikzpicture}
	}
	}
\end{bt}
%%==========Bài 8
\begin{bt}%4.16.
\immini{
	Tìm chiều rộng $d$ của dòng sông trong hình bên (làm tròn đến m).	
}{
	\begin{tikzpicture}[scale=.8, line join=round, line cap=round, >=stealth]	
	\tikzset{every node/.style={scale=1}}% thu nhỏ phóng tỏ tex 
	\coordinate (x) at (0,0);
	\coordinate (x') at (7,0);
	\coordinate (B) at (1,0);
	\coordinate (y) at (0,3);
	\coordinate (y') at (7,3);
	\coordinate (A) at (6,3);
	\coordinate (H) at ($(x)!(A)!(x')$);
	\coordinate (m) at ($(A)!0.5!(H)$);
	\coordinate (n) at ($(B)!0.5!(H)$);
	\fill[cyan!20] (x)--(x')--(y')--(y)--(x);
	\draw (x)--(x') (y')--(y) (A)--(B)
	;
	\draw[dashed] (A)--(H)
	;
	\draw[fill=black] (A) node[above]{$A$} circle (1pt) 
	(B) circle (1pt) node[below]{$B$} 
	(H) circle (1pt) node[below]{$H$} 	
	(m) node[right]{$d$} 
	(n) node[below]{$50$ m} 
	($(A)+(-1,-0.3)$) node {$40^\circ$}
	($(B)+(1.2,0.3)$) node {$40^\circ$}
	;	
	\draw pic[draw,angle radius=2mm] {right angle = A--H--B};
	\draw pic[draw,angle radius=4mm] {angle = H--B--A};
	\draw pic[draw,angle radius=4mm] {angle = y--A--B};
	\end{tikzpicture}
}
	\loigiai{
	\immini{
	Xét tam giác $ABH$ vuông tại $H$, ta có\\
	$\tan \widehat{ABH} = \dfrac{AH}{BH} \Rightarrow d=AH= BH \cdot \widehat{ABH} = 50 \cdot \tan 40^\circ\approx 42$ m.
	}{
	\begin{tikzpicture}[scale=.8, line join=round, line cap=round, >=stealth]	
	\tikzset{every node/.style={scale=1}}% thu nhỏ phóng tỏ tex 
	\coordinate (x) at (0,0);
	\coordinate (x') at (7,0);
	\coordinate (B) at (1,0);
	\coordinate (y) at (0,3);
	\coordinate (y') at (7,3);
	\coordinate (A) at (6,3);
	\coordinate (H) at ($(x)!(A)!(x')$);
	\coordinate (m) at ($(A)!0.5!(H)$);
	\coordinate (n) at ($(B)!0.5!(H)$);
	\fill[cyan!20] (x)--(x')--(y')--(y)--(x);
	\draw (x)--(x') (y')--(y) (A)--(B)
	;
	\draw[dashed] (A)--(H)
	;
	\draw[fill=black] (A) node[above]{$A$} circle (1pt) 
	(B) circle (1pt) node[below]{$B$} 
	(H) circle (1pt) node[below]{$H$} 	
	(m) node[right]{$d$} 
	(n) node[below]{$50$ m} 
	($(A)+(-1,-0.3)$) node {$40^\circ$}
	($(B)+(1.2,0.3)$) node {$40^\circ$}
	;	
	\draw pic[draw,angle radius=2mm] {right angle = A--H--B};
	\draw pic[draw,angle radius=4mm] {angle = H--B--A};
	\draw pic[draw,angle radius=4mm] {angle = y--A--B};
	\end{tikzpicture}
	}	
	}
\end{bt}
%%==========Bài 9
\begin{bt} %4.18. 
	\immini{
	Một bạn muốn tính khoảng cách giữa hai địa điểm $A$, $B$ ở hai bên hồ nước. Biết rẳng các khoảng cách từ một điểm $C$ đến $A$ và đến $B$ là $C A=90 \mathrm{~m}, C B=150 \mathrm{~m}$ và $\widehat{A C B}=120^{\circ}$. Hãy tính $A B$ giúp bạn.	}{
	\begin{tikzpicture}[scale=.4, line join=round, line cap=round, >=stealth]	
	\tikzset{every node/.style={scale=0.85}}% thu nhỏ phóng tỏ tex trong hình
	\definecolor{maungoai}{RGB}{143,227,255}
	\definecolor{mautrong}{RGB}{176,240,255}
	\definecolor{lua}{RGB}{103,201,25}
	\def\vongngoai{(2.7,3.3) .. controls (4.4,5.3) and (8,5.2) ..
	(9.9,4.2) .. controls (14,3) and (15.1,1.3) ..
	(9.8,.1) .. controls (5.6,-0) and (2.9,0) ..
	(1.4,.4) .. controls (-1.4,1.4) and (0.5,2.3) ..
	(2.7,3.3) -- cycle}
	\tikzset{caylua/.pic={
	\fill[lua]
	(1.1,2.1) -- (1.8,2.1) .. controls (1.7,2.7) and
	(2.3,3.1) .. (2.6,3.5) .. controls (2.4,3.5) and
	(2.2,3.4) .. (2,3.2) -- (2.4,4) .. controls
	(2,3.7) and (1.7,3.5) .. (1.5,3.2) .. controls
	(1.3,3.5) and (1,3.7) .. (0.7,3.9) .. controls
	(0.8,3.6) and (1.1,3.4) .. (1.1,3.2) .. controls
	(0.8,3.4) and (0.6,3.5) .. (0.4,3.6) .. controls
	(0.9,3.1) and (1.2,2.6) .. (1.1,2.1) -- cycle;}}
	\path (9,3) pic[rotate=0,scale=.25]{caylua};
	\coordinate (B) at (1,-0.5);
	\coordinate (H) at (15,-0.5);
	\coordinate (A) at (15,5);
	\coordinate (C) at ($(B)!0.7!(H)$);
	\coordinate (m) at ($(B)!.5!(C)$);
	\coordinate (n) at ($(A)!.5!(C)$);
	\fill[maungoai]\vongngoai;
	\fill[mautrong]
	(3.4,3.3) .. controls (6,5.7) and (8,4.6) ..
	(10.4,3.7) .. controls (13.2,2.7) and (12.5,.6) ..
	(9.8,.5) .. controls (7.2,.5) and (3.9,0) ..
	(2.2,.9) .. controls (0.3,2) and (1.9,2.7) ..
	(3.4,3.3) -- cycle;
	\path (0,.2) pic[rotate=0,scale=.4]{caylua};
	\draw (A)--(H)--(B) (A)--(C)
	;
	\draw[dashed] (A)--(B)
	;
	\draw[fill=black] (A) node[above]{A} circle (1pt) 
	(B) circle (1pt) node[left]{B} 
	(H) circle (1pt) node[above right]{H} 	 
	(C) circle (1pt) node[below]{C} 
	(m) node[below]{$150$ m} 
	(n) node[right]{$90$ m} 
	($(C)+(-0.6,1)$) node {$120^\circ$}
	;	
	\draw pic[draw,angle radius=2mm] {right angle = A--H--B};
	\draw pic[draw,angle radius=2mm] {angle = A--C--B};
	\end{tikzpicture}
	}
	\loigiai{
\immini{
	Ta có $\widehat{ACH} = 180^\circ - \widehat{ACB} = 180^\circ - 120^\circ = 60^\circ$.\\
	Xét tam giác $AHC$ vuông tại $H$, ta có\\
	$\sin\widehat{ACH} = \dfrac{AH}{AC} \Rightarrow AH = AC \cdot \sin 60^\circ = 45\sqrt{3}$ (m).\\
	$\cos\widehat{ACH} = \dfrac{CH}{AC} \Rightarrow CH = AC \cdot \cos 60^\circ = 45$ (m).\\
	Suy ra $BH = BC+CH = 195$ (m).\\
	Theo Pythagore ta có \\
	$AB^2=BH^2+AH^2 \Rightarrow AB=\sqrt{BH^2+AH^2} = \sqrt{195^2+3\cdot 45^2} = 210$ (m).
}{
	\begin{tikzpicture}[scale=.4, line join=round, line cap=round, >=stealth]	
	\tikzset{every node/.style={scale=0.85}}% thu nhỏ phóng tỏ tex trong hình
	\definecolor{maungoai}{RGB}{143,227,255}
	\definecolor{mautrong}{RGB}{176,240,255}
	\definecolor{lua}{RGB}{103,201,25}
	\def\vongngoai{(2.7,3.3) .. controls (4.4,5.3) and (8,5.2) ..
	(9.9,4.2) .. controls (14,3) and (15.1,1.3) ..
	(9.8,.1) .. controls (5.6,-0) and (2.9,0) ..
	(1.4,.4) .. controls (-1.4,1.4) and (0.5,2.3) ..
	(2.7,3.3) -- cycle}
	\tikzset{caylua/.pic={
	\fill[lua]
	(1.1,2.1) -- (1.8,2.1) .. controls (1.7,2.7) and
	(2.3,3.1) .. (2.6,3.5) .. controls (2.4,3.5) and
	(2.2,3.4) .. (2,3.2) -- (2.4,4) .. controls
	(2,3.7) and (1.7,3.5) .. (1.5,3.2) .. controls
	(1.3,3.5) and (1,3.7) .. (0.7,3.9) .. controls
	(0.8,3.6) and (1.1,3.4) .. (1.1,3.2) .. controls
	(0.8,3.4) and (0.6,3.5) .. (0.4,3.6) .. controls
	(0.9,3.1) and (1.2,2.6) .. (1.1,2.1) -- cycle;}}
	\path (9,3) pic[rotate=0,scale=.25]{caylua};
	\coordinate (B) at (1,-0.5);
	\coordinate (H) at (15,-0.5);
	\coordinate (A) at (15,5);
	\coordinate (C) at ($(B)!0.7!(H)$);
	\coordinate (m) at ($(B)!.5!(C)$);
	\coordinate (n) at ($(A)!.5!(C)$);
	\fill[maungoai]\vongngoai;
	\fill[mautrong]
	(3.4,3.3) .. controls (6,5.7) and (8,4.6) ..
	(10.4,3.7) .. controls (13.2,2.7) and (12.5,.6) ..
	(9.8,.5) .. controls (7.2,.5) and (3.9,0) ..
	(2.2,.9) .. controls (0.3,2) and (1.9,2.7) ..
	(3.4,3.3) -- cycle;
	\path (0,.2) pic[rotate=0,scale=.4]{caylua};
	\draw (A)--(H)--(B) (A)--(C)
	;
	\draw[dashed] (A)--(B)
	;
	\draw[fill=black] (A) node[above]{A} circle (1pt) 
	(B) circle (1pt) node[left]{B} 
	(H) circle (1pt) node[above right]{H} 	 
	(C) circle (1pt) node[below]{C} 
	(m) node[below]{$150$ m} 
	(n) node[right]{$90$ m} 
	($(C)+(-0.6,1)$) node {$120^\circ$}
	;	
	\draw pic[draw,angle radius=2mm] {right angle = A--H--B};
	\draw pic[draw,angle radius=2mm] {angle = A--C--B};
	\end{tikzpicture}
}
	}
\end{bt}
%%==========Bài 10
\begin{bt} %4.19. 
	\immini{	
	Mặt cắt ngang của một đập ngăn nước có dạng hình thang $A B C D$. Chiều rộng của mặt trên $A B$ của đập là $3 \mathrm{~m}$. Độ dốc của sườn $A D$, tức là $\tan D=1{,}25$. Độ dốc của sườn $B C$, tức là $\tan C=1{,}5$. Chiều cao của đập là $3{,}5 \mathrm{~m}$. Hāy tính chiều rộng $C D$ của chân đập, chiều dài của các sườn $A D$ và $B C$ (làm tròn đến $\mathrm{dm}$).}{
	\begin{tikzpicture}[scale=.5, line join=round, line cap=round, >=stealth]	
	\tikzset{every node/.style={scale=0.85}}% thu nhỏ phóng tỏ tex 
	%	\clip(-1,-1) rectangle (5,5.5);
	\definecolor{maungoai}{RGB}{143,227,255}
	\definecolor{mautrong}{RGB}{176,240,255}
	\definecolor{lua}{RGB}{103,201,25}	
	\coordinate (D) at (0,0);
	\coordinate (C) at (7,0);
	\coordinate (B) at (6,5);
	\coordinate (A) at (1.5,5);
	\coordinate (a) at ($(A)!0.5!(B)$);
	\coordinate (H) at ($(D)!(A)!(C)$);
	\coordinate (b) at ($(A)!0.5!(H)$);
	\draw (A)--(B)--(C)--(D)--(A) (A)--(H) 
	;
	\draw[fill=black] (A) node[left]{A} circle (1pt) 
	(B) circle (1pt) node[above]{$B$} 
	(C) circle (1pt) node[below]{$C$} 	
	(D) circle (1pt) node[below]{$D$} 
	(H) circle (1pt) node[below]{$H$} 
	(a) node[above]{$3$ m}
	(b) node[right]{$3{,}5$ m}
	;	
	\draw pic[draw,angle radius=2mm] {right angle = C--H--A};
	\draw pic[draw,angle radius=4mm] {angle = C--D--A};
	\draw pic[draw,angle radius=3mm] {angle = B--C--D};
	\draw pic[draw,angle radius=4mm] {angle = B--C--D};
	\end{tikzpicture}
	}	
	\loigiai{
	\immini{
	Gọi $K$ là chân đường cao hạ từ $B$ xuống $DC$. Khi đó ta có $BK \perp DC$, do $AB \parallel DC$ nên $BK \perp DC$.\\
	Suy ra $\widehat{ABK} = \widehat{BKH}=90^\circ$, $\widehat{AHK}=90^\circ$ (do $AH\perp DC$). \\
	Suy ra $ABKH$ là hình chữ nhật (Tứ giác có ba góc vuông) suy ra $HK=AB = 3$ m và $BK=AH = 3{,}5$ m (cạnh đối hình chữ nhật).\\
	Xét tam giác $ADH$ vuông tại $H$, ta có \\
	$\tan D = \dfrac{AH}{DH} \Rightarrow DH = \dfrac{AH}{\tan D} = \dfrac{3{,}5}{1{,}25} = 2{,}8$ (m).\\
	Xét tam giác $BCK$ vuông tại $K$, ta có \\
	$\tan C = \dfrac{BK}{KC} \Rightarrow KC = \dfrac{BK}{\tan C} = \dfrac{3{,}5}{1{,}5} \approx 2{,}3$ (m).\\
	Vậy $DC= DH+HK+KC = 2{,}8+3+2{,}3 = 8{,}1$ (m) = $8$m $1$dm.
	}{
	\begin{tikzpicture}[scale=.5, line join=round, line cap=round, >=stealth]	
	\tikzset{every node/.style={scale=0.85}}% thu nhỏ phóng tỏ tex 
	%	\clip(-1,-1) rectangle (5,5.5);
	\definecolor{maungoai}{RGB}{143,227,255}
	\definecolor{mautrong}{RGB}{176,240,255}
	\definecolor{lua}{RGB}{103,201,25}	
	\coordinate (D) at (0,0);
	\coordinate (C) at (7,0);
	\coordinate (B) at (6,5);
	\coordinate (A) at (1.5,5);
	\coordinate (a) at ($(A)!0.5!(B)$);
	\coordinate (H) at ($(D)!(A)!(C)$);
	\coordinate (K) at ($(D)!(B)!(C)$);
	\coordinate (b) at ($(A)!0.5!(H)$);
	%	\fill[mautrong] (x)--(x')--(y')--(y)--(x);
	\draw (A)--(B)--(C)--(D)--(A) (A)--(H) (B)--(K)
	;
	%	\draw[dashed] (A)--(H)
	%	;
	\draw[fill=black] (A) node[left]{A} circle (1pt) 
	(B) circle (1pt) node[above]{$B$} 
	(C) circle (1pt) node[below]{$C$} 	
	(D) circle (1pt) node[below]{$D$} 
	(H) circle (1pt) node[below]{$H$} 
	(K) circle (1pt) node[below]{$K$} 
	(a) node[above]{$3$ m}
	(b) node[right]{$3{,}5$ m}
	%	($(B)+(1.2,0.3)$) node {$40^\circ$}
	;	
	\draw pic[draw,angle radius=2mm] {right angle = C--H--A};
	\draw pic[draw,angle radius=2mm] {right angle = B--K--D};
	\draw pic[draw,angle radius=4mm] {angle = C--D--A};
	\draw pic[draw,angle radius=3mm] {angle = B--C--D};
	\draw pic[draw,angle radius=4mm] {angle = B--C--D};
	\end{tikzpicture}
	}	
	}
\end{bt}
%%==========Bài 11
\begin{bt} %4.20. 
	Trong một buổi tập trận, một tàu ngầm đang ở trên mặt biển bắt đầu di chuyển theo đường thẳng tạo với mặt nước biển một góc $21^{\circ}$ để lặn xuống.
	\begin{center}
	\begin{tikzpicture}[scale=.75, line join=round, line cap=round, >=stealth]	
	\tikzset{every node/.style={scale=0.85}}% thu nhỏ phóng tỏ tex 
	\definecolor{maungoai}{RGB}{143,227,255}
	\definecolor{mautrong}{RGB}{176,240,255}
	\definecolor{lua}{RGB}{103,201,25}
	\coordinate (x) at (0,0);
	\coordinate (x') at (7,0);
	\coordinate (B) at (6,3);
	\coordinate (y) at (0,3);
	\coordinate (y') at (7,3);
	\coordinate (A) at (1,3);
	\coordinate (H) at ($(x)!(B)!(x')$);
	\coordinate (m) at ($(A)!0.5!(B)$);
	\fill[mautrong] (x)--(x')--(y')--(y)--(x);
	\draw (x)--(x') (y')--(y) (A)--(B)--(H)--(A)
	;
	\draw[fill=black] (A) node[above]{$A$} circle (1pt) 
	(B) circle (1pt) node[above]{$B$} 
	(H) circle (6pt) node[below=2mm]{$H$} 	
	(m) node[above]{mặt biển} 
	($(A)+(1.3,-0.3)$) node {$21^\circ$}
	;	
	\draw pic[draw,angle radius=2mm] {right angle = A--B--H};
	\draw pic[draw,angle radius=4mm] {angle = H--A--B};
	\end{tikzpicture}
	\end{center}
	\begin{enumEX}{1}
	\item Khi tàu chuyển động theo hướng đó và đi được $200 \mathrm{~m}$ thì tàu ở độ sâu bao nhiêu so với mặt nước biển? (làm tròn đến $\mathrm{m}$).
	\item Giả sử tốc độ của tàu là $9 \mathrm{~km}/\mathrm{h}$ thì sau bao lâu (tính từ lúc bắt đầu lặn) tàu ở độ sâu $200 \mathrm{~m}$ (tức là cách mặt nước biển $200 \mathrm{~m}$)?	
	\end{enumEX}
	\loigiai{
	\begin{enumEX}{1}
	\item Theo bài ra ta có $AH = 200$ (m).
	\immini{
	Xét tam giác $ABH$ vuông tại $B$, ta có\\
	$BH = AH \cdot \sin \widehat{HAB} = 200\cdot \sin 21^\circ \approx 72$ (m).
	\item Theo bài ra ta có $BH = 200$ (m).\\
	Xét tam giác $ABH$ vuông tại $B$, ta có\\
	$BH = AH \cdot \sin \widehat{HAB} \Rightarrow AH =\dfrac{BH}{\sin 21^\circ} \approx 558$ (m) $= 0{,}588$ (km).\\
	Vậy thời gian để tàu ngầm lặn xuống độ sâu $200$ m \\là $0{,}558 : 9 = 0{,}62$ (giờ) $=3{,}72$ phút. 
	}{
	\begin{tikzpicture}[scale=.75, line join=round, line cap=round, >=stealth]	
	\tikzset{every node/.style={scale=0.85}}% thu nhỏ phóng tỏ tex 
	\definecolor{maungoai}{RGB}{143,227,255}
	\definecolor{mautrong}{RGB}{176,240,255}
	\definecolor{lua}{RGB}{103,201,25}
	\coordinate (x) at (0,0);
	\coordinate (x') at (7,0);
	\coordinate (B) at (6,3);
	\coordinate (y) at (0,3);
	\coordinate (y') at (7,3);
	\coordinate (A) at (1,3);
	\coordinate (H) at ($(x)!(B)!(x')$);
	\coordinate (m) at ($(A)!0.5!(B)$);
	\fill[mautrong] (x)--(x')--(y')--(y)--(x);
	\draw (x)--(x') (y')--(y) (A)--(B)--(H)--(A)
	;
	\draw[fill=black] (A) node[above]{$A$} circle (1pt) 
	(B) circle (1pt) node[above]{$B$} 
	(H) circle (6pt) node[below=2mm]{$H$} 	
	(m) node[above]{mặt biển} 
	($(A)+(1.3,-0.3)$) node {$21^\circ$}
	;	
	\draw pic[draw,angle radius=2mm] {right angle = A--B--H};
	\draw pic[draw,angle radius=4mm] {angle = H--A--B};
	\end{tikzpicture}
	}
	\end{enumEX}	
	}
\end{bt}
%%==========Bài 12
\begin{bt}%Vi du 2. 
	\immini{
	Một bức tường đang xây dở có dạng hình thang vuông $A B C D$, vuông góc ở $A$ và $D, A B=1 \mathrm{~m}$, $C D=4 \mathrm{~m}$, $A D=6 \mathrm{~m}$.
	\begin{enumEX}{1}
	\item Hỏi góc $\alpha$ tạo bỏi đường thẳng $B C$ và mặt đất $A D$ có số đo xấp xỉ bằng bao nhiêu (làm tròn đến phút)?
	\item Tính độ dài cạnh $B C$ (làm tròn đến chữ số thập phân thứ nhất).
	\end{enumEX}}{
	\begin{tikzpicture}[line join=round, line cap=round, >=stealth,font=\footnotesize, scale=0.85]%Hinh nón
	\tikzset{every node/.style={scale=1}}% thu nhỏ phóng tỏ tex trong hình
	\coordinate (D) at (0,0); 
	\coordinate (C) at (0,3); 
	\coordinate (I) at (-6,0); 
	\coordinate (B) at ($(I)!0.36!(C)$);
	\coordinate (A) at ($(I)!0.36!(D)$);
	\coordinate (E) at ($(D)!0.36!(C)$);
	\coordinate (m) at ($(A)!0.5!(D)$);
	\coordinate (n) at ($(C)!0.5!(D)$);
	\coordinate (p) at ($(A)!0.5!(B)$);
	\draw[fill=orange!20] (A)--(B)--(C)--(D)--cycle; %miền kẻ dạng tường gạch chỉ
	\draw[dashed] (E)--(B)--(I)--(A);	
	\draw[fill=black] 
	(A) circle (0.05) node[below] {$A$}
	(B) circle (0.05) node[above] {$B$}
	(C) circle (0.05) node[above] {$C$}
	(D) circle (0.05) node[below] {$D$}
	(E) circle (0.05) node[right] {$E$}
	(I) circle (0.05) node[below] {$I$}
	($(I)+(0.75,0.15)$) node {$\alpha$}
	(m) node[below] {$6$}
	(n) node[right] {$4$}
	(p) node[left] {$1$}
	;
	\draw pic[draw,angle radius=4mm] {angle = E--B--C};
	\draw pic[draw,angle radius=4mm] {angle = D--I--C};
	\end{tikzpicture}
	}
	\loigiai{
	\begin{enumEX}{1}
	\item Gọi I là giao điểm của hai đường thẳng $A D$ và $B C$. \\
	Qua $B$ kẻ đường thẳng song song với $A D$ (mặt đất) cắt $C D$ ở $E$ thì khi đó\\
	$\widehat{E B C}=\widehat{D I C}=\alpha$ (hai góc đồng vị).\\
	Tứ giác $A B E D$ có $B E \parallel A D, A B \parallel D E, \widehat{A}=90^{\circ}$ nên $A B E D$ là hình chữ nhật.\\
	Do đó $B E=A D=6$ (m), $E C=C D-E D=4-1=3$ (m).\\
	Xét $\triangle B C E$ vuông tại $E$, ta có $\tan \alpha=\tan \widehat{E B C}=\dfrac{E C}{B E}=\dfrac{3}{6}=\dfrac{1}{2}$.\\
	Từ đó tính được $\alpha \approx 26^{\circ} 34^{\prime}$.
	\item \textbf{Cách 1.} \\
	Áp dụng định lí Pythagore vào tam giác $B E C$ vuông tại $E$, ta có\\
	$B C^{2}=B E^{2}+E C^{2}=6^{2}+3^{2}=45$. 
	Suy ra $B C=\sqrt{45}=3 \sqrt{5} \approx 6{,}7$ (m).\\
	\textbf{Cách 2.}\\
	Tam giác $B E C$ vuông tại $E$ nên $E C=B C \cdot \sin \alpha$, suy ra \\
	$B C=\dfrac{E C}{\sin \alpha}=\dfrac{3}{\sin 26^{\circ} 34^{\prime}} \approx 6{,}7$ (m).
	\end{enumEX}
	}
\end{bt}
%%==========Bài 13
\begin{bt}
	Cho tam giác $ABC$ cân tại $A$, đường cao $BH$. Biết $\widehat{A} = 50^{\circ}$, $BH = 2{,}3$. Tính chu vi của $\Delta ABC$.
	\loigiai
	{\immini
	{Do giả thiết suy ra $\widehat{B} = \widehat{C}$ nên 
	$\widehat{C} = \dfrac{1}{2}\left(180^{\circ}- \widehat{A} \right) = \dfrac{1}{2}\left(180^{\circ}- 50^{\circ}\right) = 65^{\circ}.$\\
	Xét $\triangle AHB$ vuông tại $H$, ta có 
	$AH = BH\cdot\cot\widehat{HAB} = 2{,}3\cdot\cot 50^{\circ}\approx 1{,}92.$\\
	Tương tự, xét $\triangle CHB$ vuông tại $H$, ta có 
	$CH = BH\cdot\cot\widehat{HCB} = 2{,}3\cdot\cot 65^{\circ}\approx 1{,}07;$
	và $BC = \dfrac{BH}{\sin\widehat{HCB}} = \dfrac{2{,}3}{\sin 65^{\circ}}\approx 2{,}54$.\\	
	Mà $AC = AH + HC\approx 1{,}92 + 1{,}07 = 2{,}99$. Do đó chu vi tam giác $ABC$ bằng 
	$$AB + BC + CA\approx 2\cdot 2{,}99 + 2{,}54 = 8{,}52.$$
	}
	{\begin{tikzpicture}[line join = round, line cap = round,>=stealth,
	font=\footnotesize,scale=.75]
	%Trịnh Văn Luân
	\tkzDefPoints{0/0/B}
	\coordinate (C) at ($(B)+(4,0)$);
	\tkzDefTriangle[two angles = 65 and 65](B,C) \tkzGetPoint{A}
	\tkzDefPointBy[projection= onto C--A](B)
	\tkzGetPoint{H}
	%
	\pgfresetboundingbox
	\tkzDrawPolygon(A,B,C)
	\tkzDrawSegments(B,H)
	\tkzLabelAngles[pos=.85](B,A,C){$50^\circ$}
	\tkzMarkRightAngles[size=0.2](B,H,A)
	\path (B)--(H) node[above,midway,sloped]{$2,3$};
	\tkzDrawPoints[fill=black](A,B,C,H)
	\tkzLabelPoints[above](A)
	\tkzLabelPoints[left](B)
	\tkzLabelPoints[right](C,H)
	\end{tikzpicture}
	}
	}
\end{bt}
%%==========Bài 14
\begin{bt}
	Hình thang $ABCD$ có $\widehat{A} = \widehat{D} = 90^{\circ}$. Biết $AB = 2{,}6$, $CD = 4{,}7$ và $\widehat{C} = 35^{\circ}$. Tính diện tích hình thang.
	\loigiai
	{
	\immini
	{
	Vẽ $BH\perp CD$, do giả thiết suy ra $ABHD$ là hình chữ nhật nên $AB = DH = 2{,}6$.\\
	Mà $CD = DH + HC\Leftrightarrow HC = DC - DH = 4{,}7 - 2{,}6 = 2{,}1$.\\
	Xét $\triangle BHC$ vuông tại $H$, ta có 
	$$BH = HC\cdot\tan\widehat{BCH}= 2{,}1\cdot\tan 35^{\circ}\approx 1{,}5.$$
	Gọi $S$ là diện tích hình thang $ABCD$.\\
	Ta có $S= \dfrac{\left(AB + CD\right)\cdot BH}{2} = \dfrac{\left(2{,}6 + 4{,}7\right)\cdot 1{,}5}{2}\approx 5{,}5.$
	}
	{
	\begin{tikzpicture}[line join = round, line cap = round,>=stealth,
	font=\footnotesize,scale=1.1]
	%Trịnh Văn Luân
	\tkzDefPoints{0/0/D}
	\coordinate (C) at ($(D)+(4.7,0)$);
	\coordinate (A) at ($(D)+(0,2.1)$);
	\coordinate (B) at ($(A)+(2.6,0)$);
	\tkzDefPointBy[projection= onto C--D](B)
	\tkzGetPoint{H}
	%
	\tkzLabelSegment(A,B){$2,6$}
	\draw[|<->|] ([yshift=-0.6cm]D)--([yshift=-0.6cm]C)
	node[below,midway,sloped]{$4,7$};
	\tkzLabelAngles[pos=.7](B,C,H){$35^\circ$}
	\tkzMarkRightAngles[size=0.2](A,D,C D,A,B B,H,C)
	\tkzDrawPolygon(A,B,C,D)
	\tkzDrawSegments(B,H)
	\tkzDrawPoints[fill=black](D,C,A,B,H)
	\tkzLabelPoints[above](A,B)
	\tkzLabelPoints[left](D)
	\tkzLabelPoints[right](C)
	\tkzLabelPoints[below](H)
	\end{tikzpicture} 
	}
	}
\end{bt}
%%==========Bài 15
\begin{bt} Không dùng máy tính hoặc bảng số, hãy 
	\begin{enumerate}
	\item Tính giá trị của biểu thức $M = \sin^220^{\circ} + \cos^230^{\circ} - \sin^240^{\circ} - \sin^250^{\circ} + \cos^260^{\circ} + \sin^270^{\circ}$.
	\item Sắp xếp các tỉ số lượng giác sau theo thứ tự tăng dần $\sin 41^{\circ}$; $\cos 58^{\circ}$; $\cot 49^{\circ}$; $\cos 75^{\circ}$; $\sin 25^{\circ}$.
	\end{enumerate}
	\loigiai
	{	\begin{enumerate}
	\item Ta có $\sin^270^{\circ} = \sin^2\left(90^{\circ}- 20^{\circ}\right) = \cos^220^{\circ}$.\\
	Tương tự $\sin^250^{\circ} = \sin^2\left(90^{\circ}- 40^{\circ}\right) = \cos^240^{\circ}$ và $\cos^260^{\circ} = \cos^2\left(90^{\circ}- 30^{\circ}\right) = \sin^230^{\circ}$.\\
	Do đó 
	\allowdisplaybreaks
	\begin{eqnarray*}
	M &=& \sin^220^{\circ} + \cos^230^{\circ} - \sin^240^{\circ} - \cos^240^{\circ} + \sin^260^{\circ} + \cos^220^{\circ}\\
	&=& \left(\sin^220^{\circ} + \cos^220^{\circ}\right) + \left(\cos^230^{\circ} + \sin^230^{\circ}\right) - \left(\sin^240^{\circ} + \cos^240^{\circ}\right) = 1 + 1 - 1 = 1.
	\end{eqnarray*}
	\item Ta có $\cos 58^{\circ} = \sin 32^{\circ}$, $\cos 49^{\circ} = \sin 41^{\circ}$ và $\cos 75^{\circ} = \sin 25^{\circ}$.\\
	Mà $\sin 25^{\circ} < \sin 32^{\circ}< \sin 41^{\circ}$ mà $\cos 49^{\circ} < \cot 49^{\circ}$ nên 
	$$\sin 25^{\circ} = \cos 75^{\circ} < \cos 58^{\circ}< \sin 41^{\circ} = \cos 49^{\circ} < \cot 49^{\circ}.$$
	Vậy $\sin 25^{\circ} = \cos 75^{\circ} < \cos 58^{\circ} < \sin 41^{\circ} < \cot 49^{\circ}.$
	\end{enumerate}
	}
\end{bt}