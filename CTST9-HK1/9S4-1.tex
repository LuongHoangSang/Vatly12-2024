\setcounter{section}{0}
\section{TỈ SỐ LƯỢNG GIÁC CỦA GÓC NHỌN}
\subsection{Trọng tâm kiến thức}
\begin{tomtat}
\subsubsection{Khái niệm tỉ số lượng giác của góc nhọn}
\begin{boxdn}
\immini{
Cho góc nhọn $\alpha$. Xét tam giác $ABC$ vuông tại $A$ có góc nhọn $B$ bằng $\alpha$. Ta có:
\begin{itemize}
	\item Tỉ số giữa cạnh đối và cạnh huyền gọi là sin của $\alpha$, kí hiệu $\sin \alpha$.
	\item Tỉ số giữa cạnh kề và cạnh huyền gọi là côsin của $\alpha$, kí hiệu $\cos \alpha$.
	\item Tỉ số giữa cạnh đối và cạnh kề của góc $\alpha$ gọi là tang của $\alpha$, kí hiệu $\tan \alpha$.
	\item Tỉ số giữa cạnh kề và cạnh đối của góc $\alpha$ gọi là côtang của $\alpha$, kí hiệu $\cot \alpha$.
\end{itemize}}{
\begin{tikzpicture}[scale=0.65]
	\path 
	(0,0) coordinate (B)
	(3,0) coordinate (A)
	(3,4) coordinate (C);
	\draw (0,0) node[left]{$B$}--(1.5,0) node[below]{cạnh kề}--(3,0) node[right]{$A$}--(3,4) node[above]{$C$}--(0,0);
	\draw (C)--(B)node[midway,sloped,above]{cạnh huyền};
	\draw (A)--(C)node[midway,sloped,below]{cạnh đối};
	\pic[draw, angle radius=3mm]{right angle=B--A--C};
	\pic[draw, angle radius=3mm]{angle=A--B--C};
	\draw (0.3,0.3) node[right]{$\alpha$};
\end{tikzpicture}}
\end{boxdn}
\begin{note}
	\begin{enumEX}[\itemCI]{4}
	\item $\sin \alpha = \dfrac{\text{cạnh đối}}{\text{cạnh huyền}}$;
	\item $\cos \alpha = \dfrac{\text{cạnh kề}}{\text{cạnh huyền}}$;
	\item $\tan \alpha = \dfrac{\text{cạnh đối}}{\text{cạnh kề}}$; 
	\item $\cot \alpha = \dfrac{\text{cạnh kề}}{\text{cạnh đối}}$;
	\item $\cot \alpha = \dfrac{1}{\tan \alpha}$;
	\item! $\sin \alpha$, $\cos \alpha$, $\tan \alpha$, $\cot \alpha$ gọi là các tỉ số lượng giác của góc nhọn $\alpha$.
	\end{enumEX}
\end{note}
\begin{note}
	sin, côsin của góc nhọn luôn dương và bé hơn $1$ vì trong tam giác vuông, cạnh huyền dài nhất.
\end{note}
Ta có bảng các giá trị lượng giác đặc biệt:
\begin{center}
\renewcommand\arraystretch{2.25}
\begin{tabular}{|c|c|c|c|c|}
	\hline
	$\alpha$ & $\sin \alpha$ & $\cos \alpha$ &$\tan \alpha$ & $\cot \alpha$ \\ 
	\hline
	$30^\circ$ & $\dfrac{1}{2}$ & $\dfrac{\sqrt{3}}{2}$ & $\dfrac{\sqrt{3}}{3}$ & $\sqrt{3}$\\
	\hline
	$45^\circ$ & $\dfrac{\sqrt{2}}{2}$ & $\dfrac{\sqrt{2}}{2}$ & $1$ & $1$\\
	\hline
	$60^\circ$ & $\dfrac{\sqrt{3}}{2}$ & $\dfrac{1}{2}$ & $\sqrt{3}$ & $\dfrac{\sqrt{3}}{3}$\\
	\hline
\end{tabular}
\end{center}
\subsubsection{Tỉ số lượng giác của hai góc phụ nhau}
\begin{boxdl}
	\immini{
	Nếu hai gọc phụ nhau thì sin góc này bằng côsin góc kia, tang góc này bằng côtang góc kia.
	\begin{note}
	Cho $\alpha$ và $\beta$ là hai góc phụ nhau, khi đó\\
	$\sin \alpha = \cos \beta$, $\cos \alpha = \sin \beta$, $\tan \alpha = \cot \beta$, $\cot \alpha = \tan \beta$.
	\end{note}}{
	\begin{tikzpicture}[>=stealth,line join=round,line cap=round,font=\footnotesize,scale=0.7]
	\path 
	(0,0) coordinate[label=left:$A$] (A)
	(5,0) coordinate[label=right:$B$] (B);
	\coordinate (D) at ($(B)!1!-30: (A)$) ;
	\coordinate (E) at ($(A)!1!60: (B)$) ;
	\coordinate[label=above:$C$] (C) at (intersection of B--D and A--E);
	\draw (A)--(B)--(C)--cycle;
	\pic[draw, angle radius=2mm]{right angle=B--C--A};
	\pic[draw,double, angle radius=5mm]{angle=C--B--A} node[below] at ($(B) + (145:1.4)$){$\beta$};
	\pic[draw, angle radius=4mm]{angle=B--A--C} node[below] at ($(A) + (35:1.2)$){$\alpha$};
	\end{tikzpicture}
	}
\end{boxdl}
Tổng quát, với góc nhọn $\alpha$ ta có:
\begin{enumEX}[\itemCI]{4}
		\item $\sin (90^\circ-\alpha)=\cos\alpha$;
		\item $\cos (90^\circ-\alpha)=\sin\alpha$;
		\item $\tan (90^\circ-\alpha)=\cot\alpha$;
		\item $\cot (90^\circ-\alpha)=\tan\alpha$.
\end{enumEX}
\subsubsection{Sử dụng máy tính cầm tay tính tỉ số lượng giác của một góc nhọn}
\begin{boxdn}
	Bảng tóm tắt cách tính tỉ số lượng giác của góc nhọn
	\begin{center}
%		\renewcommand{\arraystretch}{1.8}
		\begin{tabular}{|c|c|}
			\hline
			Để tính & Thứ tự các nút\\
			\hline
			$\sin\alpha$ & \sink $\alpha$ \key{)}\key{=}\\
			\hline
			$\cos\alpha$ & \cosk $\alpha$ \key{)}\key{=}\\
			\hline
			$\tan\alpha$ & \tank $\alpha$ \key{)}\key{=}\\
			\hline
			$\cot\alpha$ & \key{1}\divk\tank $\alpha$ \key{)}\key{=}\\
			\hline
		\end{tabular}
	\end{center}
\end{boxdn}
\begin{note}
	Để tìm góc $\alpha$ khi biết $\cot \alpha$, ta có thể tìm $\tan$ của góc $(90^\circ -\alpha)$ (vì $\tan(90^\circ - \alpha) = \cot \alpha$) rồi suy ra $\alpha$.
\end{note}
\end{tomtat}
%%%%%%%%%%%%%%%%%%%
\subsection{Các dạng bài tập}
%===================
\begin{dang}{Sử dụng MTCT tính tỉ số lượng giác, tính góc}
\end{dang}
\begin{vd}
	Sử dụng MTCT tính các tỉ số lượng giác và làm tròn kết quả đến chữ số thập phân thứ ba:
	\begin{listEX}[2]
	\item
	$\sin 27^\circ$, 
	$\cos 32^\circ 15'$, 
	$\tan 52^\circ 12'$ 
	và $\cot 35^\circ 23'$.
	\item
	$\sin 40^\circ 54'$,
	$\cos 52^\circ 15'$,
	$\tan 69^\circ 36'$
	và $\cot 25^\circ 18'$.
	\end{listEX}
	\loigiai{
	\begin{listEX}
	\item
	\begin{tabular}{|l|l|l|}
	\hline
	Để tính & Bấm phím & Kết quả \\ \hline
	$\sin 27^\circ$ & \sink \key{2} \key{7} \key{x} \key{=} & $0{,}4539904997$ \\ \hline
	$\cos 32^\circ 15'$ & \cosk \key{3} \key{2} \key{x} \key{1} \key{5} \key{x} \key{=} & $0{,}8457278217$ \\ \hline
	$\tan 52^\circ 12'$ & \tank \key{5} \key{2} \key{x} \key{1} \key{2} \key{x} \key{=} & $1{,}289192232$ \\ \hline
	$\cot 35^\circ 23'$ & \tank \key{3} \key{5} \key{x} \key{2} \key{3} \key{x} \key{=} \key{u} \key{=} & $1{,}408003909$ \\ \hline
	\end{tabular} \\
	Làm tròn đến chữ số thập phân thứ ba ta được $\sin 27^\circ \approx 0{,}454$; $\cos 32^\circ 15'\approx 0{,}846$; $\tan 52^\circ 12' \approx 1{,}289$; $\cot 35^\circ 23' \approx 1{,}408$.\\
	Lưu ý, $\cot 35^\circ 23' = \dfrac{1}{\tan 35^\circ 23'}$.
	\item
	\begin{tabular}{|l|l|l|}
	\hline
	Để tính & Bấm phím & Kết quả \\ \hline
	$\sin 40^\circ 54'$ & \sink \key{4} \key{0} \key{x} \key{5} \key{4} \key{x} \key{=} & $0{,}6547408137$ \\ \hline
	$\cos 52^\circ 15'$ & \cosk \key{5} \key{2} \key{x} \key{1} \key{5} \key{x} \key{=} & $0{,}61221728$ \\ \hline
	$\tan 69^\circ 36'$ & \tank \key{6} \key{9} \key{x} \key{3} \key{6} \key{x} \key{=} & $2{,}688918967$ \\ \hline
	$\cot 25^\circ 18'$ & \tank \key{2} \key{5} \key{x} \key{1} \key{8} \key{x} \key{=} \key{u} \key{=} & $2{,}115516356$ \\ \hline
	\end{tabular} \\
	Làm tròn đến chữ số thập phân thứ ba ta được $\sin 40^\circ 54' \approx 0{,}655$; $\cos \cos 52^\circ 15' \approx 0{,}612$; $\tan 69^\circ 36' \approx 2{,}689$; $\cot 25^\circ 18' \approx 2{,}116$.
	\end{listEX}
	}
\end{vd}
\begin{vd}
	Dùng MTCT, tìm các góc (làm tròn đến phút) biết
	\begin{listEX}
	\item 
	$\sin \alpha _1 = 0{,}3214$, 
	$\cos \alpha _2 =0{,}4321$, 
	$\tan \alpha _3 = 1{,}2742$ 
	và $\cot \alpha _4 = 1{,}5384$.
	\item
	$\sin \alpha_1 = 0{,}3782$,
	$\cos \alpha_1 = 0{,}6251$,
	$\tan \alpha_1 = 2{,}154$
	và $\cot \alpha_1 = 3{,}253$.
	\end{listEX}
	\loigiai{
	\begin{listEX}
	\item
	\begin{tabular}{|l|l|l|l|}
	\hline
	Biết & Bấm phím & Kết quả & Bấm tiếp \key{x} \\ \hline
	$\sin \alpha _1 = 0{,}3214$ & \key{q} \sink \key{0} \key{.} \key{3} \key{2} \key{1} \key{4} \key{=} & $18{,}74761209$ & $18^\circ 44' 51{,}4"$ \\ \hline
	$\cos \alpha _2 =0{,}4321$ & \key{q} \cosk \key{0} \key{.} \key{4} \key{3} \key{2} \key{1} \key{=} & $64{,}39909458$ &$64^\circ 23' 56{,}74"$ \\ \hline
	$\tan \alpha _3 = 1{,}2742$ & \key{q} \tank \key{1} \key{.} \key{2} \key{7} \key{4} \key{2} \key{=} & $51{,}87495892$ & $51^\circ 52' 29{,}85"$ \\ \hline
	$\cot \alpha _4 = 1{,}5384$ & \key{q} \tank \key{1} \key{.} \key{6} \key{3} \key{8} \key{4} \key{u} \key{=} & $33{,}02491482$ & $33^\circ 1' 29{,}69"$ \\ \hline
	\end{tabular} \\
	Làm tròn đến phút ta được: $\alpha _1 \approx 18^\circ 45'$, $\alpha _2 \approx 64^\circ 24'$, $\alpha _3 \approx 51^\circ 52'$, $\alpha _4 \approx 33^\circ 1'$.
	\item
	\begin{tabular}{|l|l|l|l|}
	\hline
	Biết & Bấm phím & Kết quả & Bấm tiếp \key{x} \\ \hline
	$\sin \alpha = 0{,}3782$ & \key{q} \sink \key{0} \key{.} \key{3} \key{7} \key{8} \key{2} \key{=} & $22{,}222231$ & $22^\circ 13' 20{,}03"$ \\ \hline
	$\cos \alpha = 0{,}6251$ & \key{q} \cosk \key{0} \key{.} \key{6} \key{2} \key{5} \key{1} \key{=} & $51{,}31047244$ &$51^\circ 18' 37{,}7"$ \\ \hline
	$\tan \alpha = 2{,}154$ & \key{q} \tank \key{2} \key{.} \key{1} \key{5} \key{4} \key{=} & $65{,}09679426$ & $65^\circ 5' 48{,}46"$ \\ \hline
	$\cot \alpha = 3{,}253$ & \key{q} \tank \key{3} \key{.} \key{2} \key{5} \key{3} \key{u} \key{=} & $17{,}08787556$ & $17^\circ 5' 16{,}35"$ \\ \hline
	\end{tabular} \\
	Làm tròn đến phút ta được:
	$\alpha_1 \approx 22^\circ 13'$,
	$\alpha_2 \approx 51^\circ 19'$,
	$\alpha_3 \approx 65^\circ 6'$,
	$\alpha_4 \approx 17^\circ 5'$.
	\end{listEX}
	}
\end{vd}
%=================
\begin{dang}{Tính tỉ số lượng giác của góc nhọn trong một tam giác vuông}
%	\begin{itemize}
%	\item Tính độ dài cạnh thứ ba (theo định lý Pytago).
%	\item Tính các tỉ số lượng giác của một góc nhọn (theo định nghĩa).
%	\item Suy ra các tỉ số lượng giác của góc nhọn còn lại theo định lí về tỉ số lượng giác của hai góc phụ nhau.
%	\end{itemize}
\end{dang}
\begin{vd}
	\immini{Cho hình thoi $ABCD$ có hai đường chéo cắt nhau tại điểm $O$.
	\begin{listEX}
	\item Tỉ số $\dfrac{OB}{AB}$ là sin của góc nhọn nào? Tỉ số $\dfrac{OB}{BC}$ là côsin của góc nhọn nào?
	\item Viết tỉ số lượng giác của mỗi góc nhọn sau: $\tan \widehat{OCD}$, $\cot \widehat{OAD}$.
	\end{listEX}}
	{\begin{tikzpicture}[line join = round, line cap = round,>=stealth,font=\footnotesize,scale=0.7]
	\def \a{3};
	\path 
	(0,0) coordinate (O)
	(-\a,0)	coordinate (A)	
	(\a,0)coordinate (C)	
	($(O)!0.5!90:(C)$)coordinate (B)
	($(O)!0.5!-90:(C)$)coordinate (D)
	;
	\draw (A)--(B)--(C)--(D)--cycle (A)--(C) (B)--(D); 
	\foreach \x/\g in {O/45,A/180,B/90,C/0,D/-90}\fill[black] (\x) circle (1pt) ($(\x)+(\g:4mm)$) node{$\x$};
	\foreach \a/\b/\c in {C/O/A}{\draw pic[draw,angle radius=2mm] {right angle = \a--\b--\c};}	
	\end{tikzpicture}}
	\loigiai{
	Hình thoi $ABCD$ có hai đường chéo cắt nhau tại điểm $O$ nên $AC$ vuông góc với $BD$ tại $O$.
	\begin{listEX}[2]
	\item Tam giác $OAB$ vuông tại $O$ nên $\dfrac{OB}{AB}=\sin \widehat{OAB}$.\\
	Tam giác $OBC$ vuông tại $O$ nên $\dfrac{OB}{BC}=\cos \widehat{OBC}$.
	\item Tam giác $OCD$ vuông tại $O$ nên tan $\widehat{OCD}=\dfrac{OD}{OC}$.\\
	Tam giác $OA D$ vuông tại $O$ nên cot $\widehat{OAD}=\dfrac{OA}{OD}$.
	\end{listEX}
	}
\end{vd}
\begin{vd}
\immini{
	Tính các tỉ số lượng giác của góc $\alpha$ trong tam giác $ABC$ ở hình vẽ bên.
}{
\begin{tikzpicture}[declare function={a=2;},line join=round,line cap=round,font=\footnotesize,scale=1]
	\path 
	(0:a) coordinate (C)
	(180:a) coordinate (B)
	(60:a) coordinate (A)
	($(B)!0.5!(C)$) coordinate (D)
	;
	\path pic["\scriptsize$\alpha$", angle eccentricity=2,draw,angle radius=7pt]{angle= C--B--A};
	\path pic[draw,angle radius=5pt]{right angle= B--A--C};
	\draw 
	(A)--(B) node[midway,above, sloped] {$12$}
	--(C) node[midway, below, sloped] {$15$}--cycle node[midway,right] {$9$}
	;
	\foreach \t/\g in {A/90,B/180,C/0}{
	\path (\t) node[shift={(\g:7pt)}]{$ \t $};
	}
\end{tikzpicture}
}
	\loigiai{
	Xét tam giác $ABC$ vuông tại $A$, $\widehat{B}=\alpha$, ta có\\
	$\sin\alpha=\dfrac{AC}{BC}=\dfrac{9}{15}=0{,}6$;\qquad $\cos\alpha=\dfrac{AB}{BC}=\dfrac{12}{15}=0{,}8$;\\
	$\tan\alpha=\dfrac{AC}{AB}=\dfrac{9}{12}=0{,}75$;\qquad $\cot\alpha=\dfrac{AB}{AC}=\dfrac{12}{9}=\dfrac{4}{3}$.
	}
\end{vd}
\begin{vd}
	Tính các tỉ số lượng giác của góc nhọn $A$ trong mỗi tam giác vuông $ABC$ có $\widehat{B}=90^\circ$ ở hình sau.
	\begin{center}
	\begin{tikzpicture}[declare function={a=2.5;},line join=round,line cap=round,font=\footnotesize,scale=1]
	\path (0:0) coordinate (B)
	(0:a) coordinate (C)
	(B)+(90:0.7*a) coordinate (A)
	($(B)!0.5!(C)$) coordinate (D) node[below,shift={(-90:0.6cm)}] {a)}
	;
	\path pic[draw,angle radius=5pt]{right angle= A--B--C};
	\path pic[draw,angle radius=7pt]{angle= B--A--C};
	\draw (A)--(B) node[midway,left] {$3$}
	--(C) node[midway,below] {$4$}
	--cycle node[midway,right,yshift=3pt] {$5$}
	;
	\foreach \t/\g in {A/90,B/-90,C/-90}{
	\path (\t) node[shift={(\g:7pt)},font=\scriptsize]{$ \t $};
	}
	\end{tikzpicture} \hspace*{0.6cm}
	\begin{tikzpicture}[declare function={a=2;},line join=round,line cap=round,font=\footnotesize,scale=1]
	\path 
	(0:a) coordinate (C)
	(180:a) coordinate (B)
	(150:a) coordinate (A)
	($(B)!0.5!(C)$) coordinate (D) node[shift={(-90:1cm)}] {b)}
	;
	\path pic[angle eccentricity=2,draw,angle radius=12pt]{angle= A--C--B};
	\path pic[draw,angle radius=5pt]{right angle= B--A--C};
	\draw 
	(A)--(B) node[midway,left] {$1$}
	--(C) node[midway, below] {$\sqrt{17}$}--cycle node[midway, above] {$4$}
	;
	\foreach \t/\g/\i in {A/90/B,B/-90/C,C/-90/A}{
	\path (\t) node[shift={(\g:7pt)}]{$ \i $};
	}
	\end{tikzpicture}
	\begin{tikzpicture}[declare function={a=2.5;},line join=round,line cap=round,font=\footnotesize,scale=1]
	\path (0:0) coordinate (B)
	(0:a) coordinate (C)
	(C)+(-90:0.85*a) coordinate (A)
	(A)+(180:0.5*a) coordinate (O) node[shift={(-90:10pt)}] {c)}
	;
	\path pic[draw,angle radius=5pt]{right angle= A--C--B};
	\path pic[draw,angle radius=7pt]{angle= C--A--B};
	\draw (A)--(B) node[midway,left,xshift=-3pt] {$3$}
	--(C) 
	--cycle node[midway,right,yshift=4pt] {$2$}
	;
	\foreach \t/\g/\i in {A/-90/A,B/90/C,C/90/B}{
	\path (\t) node[shift={(\g:7pt)}]{$ \i $};
	}
	\end{tikzpicture} \hspace*{0.6cm}
	\begin{tikzpicture}[declare function={a=2.5;},line join=round,line cap=round,font=\footnotesize,scale=1]
	\path (0:0) coordinate (B)
	(180:a) coordinate (C)
	(B)+(90:0.7*a) coordinate (A)
	($(B)!0.5!(C)$) coordinate (D) node[below,shift={(-90:0.6cm)}] {d)}
	;
	\path pic[draw,angle radius=5pt]{right angle= A--B--C};
	\path pic[draw,angle radius=7pt]{angle= B--C--A};
	\draw (A)--(B) node[midway,right] {$\sqrt{6}$}
	--(C) node[midway,below] {$\sqrt{10}$}
	--cycle 
	;
	\foreach \t/\g/\i in {A/90/C,B/-90/B,C/-90/A}{
	\path (\t) node[shift={(\g:7pt)}]{$ \i $};
	}
	\end{tikzpicture}
	\end{center}
	\loigiai{
	Xét $\triangle ABC$ vuông tại $B$ (Hình a), ta có
	\begin{enumEX}[\itemCI]{4}
	\item $\sin A=\dfrac{BC}{AC}=\dfrac{4}{5}$;
	\item $\cos A=\dfrac{AB}{AC}=\dfrac{3}{5}$;
	\item $\tan A=\dfrac{BC}{AB}=\dfrac{4}{3}$;
	\item $\cot A=\dfrac{AB}{BC}=\dfrac{3}{4}$.
	\end{enumEX}
	Xét $\triangle ABC$ vuông tại $B$ (Hình b), ta có
	\begin{enumEX}[\itemCI]{4}
	\item $\sin A=\dfrac{BC}{AC}=\dfrac{1}{\sqrt{17}}$;
	\item $\cos A=\dfrac{AB}{AC}=\dfrac{4}{\sqrt{17}}$;
	\item $\tan A=\dfrac{BC}{AB}=\dfrac{1}{4}$;
	\item $\cot A=\dfrac{AB}{BC}=\dfrac{4}{1}$.
	\end{enumEX}
	Xét $\triangle ABC$ vuông tại $B$ (Hình c), ta có	$BC=\sqrt{3^2-2^2}=\sqrt{5}$;
	\begin{enumEX}[\itemCI]{4}
	\item $\sin A=\dfrac{BC}{AC}=\dfrac{\sqrt{5}}{3}$;
	\item $\cos A=\dfrac{AB}{AC}=\dfrac{2}{3}$;
	\item $\tan A=\dfrac{BC}{AB}=\dfrac{\sqrt{5}}{2}$;
	\item $\cot A=\dfrac{AB}{BC}=\dfrac{2}{\sqrt{5}}$.
	\end{enumEX}
	Xét $\triangle ABC$ vuông tại $B$ (Hình d), ta có $AC=\sqrt{\left(\sqrt{10}\right)^2-\left(\sqrt{6}\right)^2}=2$;
	\begin{enumEX}[\itemCI]{4}
	\item $\sin A=\dfrac{BC}{AC}=\dfrac{\sqrt{6}}{2}$;
	\item $\cos A=\dfrac{AB}{AC}=\dfrac{\sqrt{10}}{2}$;
	\item $\tan A=\dfrac{BC}{AB}=\dfrac{\sqrt{6}}{\sqrt{10}}$;
	\item $\cot A=\dfrac{AB}{BC}=\dfrac{\sqrt{10}}{\sqrt{6}}$.
	\end{enumEX}
	}
\end{vd}
\begin{vd} %Ví dụ 1
	Cho tam giác $ABC$ vuông tại $A$, có $AB=3$ cm, $AC=4$ cm. Hãy tính các tỉ số lượng giác $\sin \alpha$, $\cos \alpha$, $\tan \alpha$ với $\alpha = \widehat{B}$.
	\loigiai{
	\immini{
	Xét $\triangle ABC$ vuông tại $A$, $\widehat{B}=\alpha$.\\
	Theo định lí Pythagore, ta có: $BC^2=AC^2 + AB^2 = 4^2 +3^2 = 25$ nên $BC=5$ (cm).\\
	Theo định nghĩa của tỉ số lượng giác sin, côsin , tang, ta có\\
	$\sin \alpha = \dfrac{AC}{BC} = \dfrac{4}{5}$, $\cos \alpha = \dfrac{AB}{BC} = \dfrac{3}{5}$, $\tan \alpha = \dfrac{AC}{AB} = \dfrac{4}{3}$.}{
\begin{tikzpicture}[>=stealth,line join=round,line cap=round,font=\footnotesize,scale=0.65]
	\path 
	(0,0) coordinate (B)
	(3,0) coordinate (A)
	(3,4) coordinate (C);
	\draw (0,0) node[left]{$B$}--(1.5,0) node[below]{$3$ cm}--(3,0) node[right]{$A$}--(3,2) node[right]{$4$ cm}--(3,4) node[above]{$C$}--(0,0);
	\pic[draw, angle radius=3mm]{right angle=B--A--C};
	\pic[draw, angle radius=5mm]{angle=A--B--C} node at ($(B)+(25:0.8)$){$\alpha$};
\end{tikzpicture}
	}
	} 
\end{vd}
\begin{vd} %Luyện tập 1
	Cho tam giác $ABC$ vuông tại $A$ có $AB=5$ cm, $AC=12$ cm. Hãy tính các tỉ số lượng giác của góc $B$.
	\loigiai{
	\immini{
	Xét $\triangle ABC$ vuông tại $A$\\
	Theo định lí Pythagore, ta có \\
	$BC^2=AC^2 + AB^2 = 12^2 +5^2 = 169$ nên $BC=13$ (cm).\\
	Theo định nghĩa của tỉ số lượng giác sin, côsin , tang, côtang ta có\\
	$\sin \alpha = \dfrac{AC}{BC} = \dfrac{12}{13}$, $\cos \alpha = \dfrac{AB}{BC} = \dfrac{5}{13}$,\\
	$\tan \alpha = \dfrac{AC}{AB} = \dfrac{12}{5}$, $\cot \alpha = \dfrac{AB}{AC} = \dfrac{5}{12}$.
	}{
	\begin{tikzpicture}[>=stealth,line join=round,line cap=round,font=\footnotesize,xscale=0.5,yscale=0.25]
	\path 
	(0,0) coordinate (A)
	(5,0) coordinate (B)
	(0,12) coordinate (C);
	\draw (0,0) node[left]{$A$}--(2.5,0) node[below]{$5$ cm}--(5,0) node[right]{$B$}--(0,12) node[above]{$C$}--(0,6) node[left]{$12$ cm}--(0,0);
	\pic[draw, angle radius=3mm]{right angle=B--A--C};
	\pic[draw, angle radius=3mm]{angle=C--B--A};
	\end{tikzpicture}}}
\end{vd}
\begin{vd}%[9H1Y2]
	\immini
	{
	Tính tỉ số lượng giác của góc $B$ trong hình bên.
	}
	{
	\begin{tikzpicture}[>=stealth,line join=round,line cap=round,font=\footnotesize,scale=1]
	\tkzDefPoints{0/0/A}
	\coordinate (B) at ($(A)+(0,2)$);
	\coordinate (C) at ($(A)+(4,0)$);
	\pgfresetboundingbox
	\tkzMarkRightAngles[size=0.25](B,A,C)
	\tkzDrawSegments(A,B B,C C,A)
	\tkzDrawPoints[fill=black](A,B,C)
	%
	\pgfresetboundingbox
	\path (A)--(B) node[left,midway]{$5$};
	\path (A)--(C) node[below,midway]{$12$};
	\tkzMarkAngles[size=.4,arc=l](A,B,C)
	\tkzLabelPoints[above](B)
	\tkzLabelPoints[below](A,C)
	\end{tikzpicture}
	}
	\loigiai
	{
	Ta có $BC^2 = AB^2 + AC^2 = 5^2 + 12^2 = 169 \Rightarrow BC = 13$.\\
	Do đó $\sin B =\dfrac{AC}{BC}=\dfrac{12}{13}$; \hspace{0.5cm} $\cos B =\dfrac{AB}{BC}=\dfrac{5}{13}$; \hspace{0.5cm} $\tan B =\dfrac{AC}{AB}=\dfrac{12}{5}$; \hspace{0.5cm} $\cot B =\dfrac{AB}{AC}=\dfrac{5}{12}$.
	}
\end{vd}
\begin{vd}
	Cho tam giác $MNP$ vuông tại $M$, $MN=3$ cm, $MP=4$ cm. Tính các tỉ số lượng giác của góc $P$.
	\loigiai{
	\immini{Trong tam giác $MNP$ vuông tại $M$, ta có
	$NP=\sqrt{MN^2+MP^2}=\sqrt{3^2+4^2}=\sqrt{25}=5.$\\
	Xét tam giác $MNP$ vuông tại $M$, ta có\\
	$\begin{aligned} 
	&\sin P=\dfrac{MN}{NP}=\dfrac{3}{5};\quad \cos P=\dfrac{MP}{NP}=\dfrac{4}{5}\\ 
	& \tan P=\dfrac{MN}{MP}=\dfrac{3}{4};\quad
	\cot P=\dfrac{MP}{MN}=\dfrac{4}{3}.
	\end{aligned}$
	}
	{\begin{tikzpicture}[line join = round, line cap = round,>=stealth,font=\footnotesize,scale=0.7]
	\def \a{3};
	\path 
	(0,0) coordinate (O)
	(-\a,0)	coordinate (N)	
	(\a,0)coordinate (P)	
	($(O)!1!110:(P)$)coordinate (M)
	;
	\draw (M)--(N)--(P)--cycle; 
	\foreach \x/\g in {M/90,N/-90,P/-90}\fill[black] (\x) circle (1pt) ($(\x)+(\g:3mm)$) node{$\x$};
	\foreach \a/\b/\c in {N/M/P}{\draw pic[draw,angle radius=2mm] {right angle = \a--\b--\c};}
	\end{tikzpicture}}
	}
\end{vd}
\begin{vd}%[9H1B2]
	$\triangle ABC$ vuông tại $A$ có $BC = 2AB$. Tính các tỉ số lượng giác của góc $C$.
	\loigiai
	{
	Ta đặt $AB = m$ thì $BC = 2m$, suy ra $AC^2 = BC^2 - AB^2 = 4m^2 - m^2 = 3m^2 \Rightarrow AC = m \sqrt{3}$.
	\immini
	{
	Ta có $\begin{aligned}[t]
	&\sin C =\dfrac{AB}{BC}=\dfrac{m}{2m}=\dfrac{1}{2}; \\
	& \cos C =\dfrac{AC}{BC}=\dfrac{m\sqrt{3}}{2m}=\dfrac{\sqrt{3}}{2}; \\
	& \tan C =\dfrac{AB}{AC}= \dfrac{m}{m \sqrt{3}}=\dfrac{1}{\sqrt{3}}; \\
	& \cot C = \dfrac{AC}{AB} = \dfrac{m \sqrt{3}}{m} = \sqrt{3}.
	\end{aligned}$
	}
	{
	\begin{tikzpicture}[>=stealth,line join=round,line cap=round,font=\footnotesize,scale=0.8]
	\tkzDefPoints{0/0/A}
	\coordinate (B) at ($(A)+(0,3)$);
	\coordinate (C) at ($(A)+(4,0)$);
	\pgfresetboundingbox
	\tkzMarkRightAngles[size=0.25](B,A,C)
	\tkzDrawSegments(A,B B,C C,A)
	\tkzDrawPoints[fill=black](A,B,C)
	%
	\pgfresetboundingbox
	\tkzMarkAngles[size=.5,arc=l](B,C,A)
	\tkzLabelPoints[above](B)
	\tkzLabelPoints[below](A,C)
	\end{tikzpicture}
	}
	}
\end{vd}
\begin{vd}%[9H1B2]
	Tam giác $ABC$ cân tại $A$, có $BC = 6$, đường cao $AH = 4$. Tính các tỉ số lượng giác của góc $B$. 
	\loigiai
	{
	Ta có $BH = 6 : 2 = 3$; $AB = \sqrt{4^2 + 3^2} = 5$.
	\immini
	{
	Do đó $\begin{aligned}[t]
	&\sin B =\dfrac{AH}{AB}=\dfrac{4}{5}= 0,8; \\
	& \cos B =\dfrac{BH}{AB}=\dfrac{3}{5}= 0,6;\\
	& \tan B = \dfrac{AH}{AB} = \dfrac{4}{3}; \\
	& \cot B = \dfrac{BH}{AH} = \dfrac{3}{4} = 0,75.
	\end{aligned}$
	}
	{
	\begin{tikzpicture}[>=stealth,line join=round,line cap=round,font=\footnotesize,scale=0.6]
	\tkzDefPoints{0/0/B}
	\coordinate (C) at ($(B)+(6,0)$);
	\tkzDefTriangle[two angles = 55 and 55](B,C) \tkzGetPoint{A}
	\tkzDefPointBy[projection = onto B--C](A) \tkzGetPoint{H}
	%
	\pgfresetboundingbox
	\tkzMarkRightAngles[size=0.25](A,H,C)
	\tkzDrawSegments(A,B B,C C,A A,H)
	\tkzDrawPoints[fill=black](A,B,C,H)
	\tkzLabelPoints[above](A)
	\tkzLabelPoints[above left](H)
	\tkzLabelPoints[left](B)
	\tkzLabelPoints[right](C)
	\draw[|<->|] ([yshift=-0.6]B)--([yshift=-0.6]C) node[below,midway,sloped]{$6$};
	%
	\path (A)--(H) node[right,midway]{$4$};
	\end{tikzpicture}
	}
	}
\end{vd}
\begin{vd}%[9H1B2]
	\immini
	{
	Tính $\tan C$ trong hình bên.
	}
	{
	\begin{tikzpicture}[line join = round, line cap = round,>=stealth,font=\footnotesize,scale=0.8] 
	%
	\tkzDefPoints{0/0/B} 
	\coordinate (C) at ($(B)+(6,0)$); 
	\tkzDefTriangle[two angles = 60 and 30](B,C) \tkzGetPoint{A}
	\tkzDefPointBy[projection = onto C--B](A) \tkzGetPoint{H}
	\pgfresetboundingbox 
	\tkzMarkRightAngles[size=.25](B,A,C A,H,C) 	
	\tkzDrawPolygon(A,B,C)
	\tkzDrawSegments(A,H)
	\tkzLabelSegments[color=black,left](A,B){$6$}
	\tkzLabelSegments[color=black,below](H,B){$3$}
	\tkzMarkAngles[size=.5,arc=l](A,C,B)
	\tkzDrawPoints[fill=black](A,B,C,H) 
	\tkzLabelPoints[above](A) 
	\tkzLabelPoints[left](B) 
	\tkzLabelPoints[right](C)
	\tkzLabelPoints[below](H) 
	\end{tikzpicture}
	}
	\loigiai
	{
	Ta có $AH^2 = AB^2 - BH^2 = 6^2 - 3^2 = 27 \Rightarrow AH = 3 \sqrt{3}$.\\
	Do đó $\tan C = \cot B = \dfrac{BH}{AH} = \dfrac{3}{3 \sqrt{3}} = \dfrac{1}{\sqrt{3}}$.
	}
\end{vd}
\begin{vd}%[9H1B2]
	\immini
	{
	Tính $\sin M + \cos N $ trong hình bên.
	}
	{
	\begin{tikzpicture}[line join = round, line cap = round,>=stealth,font=\footnotesize,scale=1.1] 
	%
	\tkzDefPoints{0/0/M} 
	\coordinate (N) at ($(M)+(4,0)$); 
	\tkzDefTriangle[two angles = 63 and 27](M,N) \tkzGetPoint{O}
	%\tkzDrawAltitude(M,N)(O) \tkzGetPoint{H}
	\coordinate (H) at ($(M)!(O)!(N)$);
	\tkzDrawSegments(O,H)
	\tkzDefPointBy[projection = onto M--N](O) \tkzGetPoint{H}
	\pgfresetboundingbox 
	\tkzMarkRightAngles[size=.2](M,O,N O,H,N) 
	\tkzDrawPolygon(M,O,N)
	\tkzDrawSegments(O,H)
	\tkzLabelSegments[color=black,below](M,H){$1$}
	\tkzLabelSegments[color=black,below](H,N){$3$}
	\tkzMarkAngles[size=.7,arc=ll](O,N,M)
	\tkzMarkAngles[size=.5,arc=l](N,M,O)
	\tkzDrawPoints[fill=black](M,N,O,H) 
	\tkzLabelPoints[above](O) 
	\tkzLabelPoints[left](M) 
	\tkzLabelPoints[right](N)
	\tkzLabelPoints[below](H) 
	\end{tikzpicture}
	}
	\loigiai
	{
	Ta có $OH^2 = HM \cdot HN = 1 \cdot 3 = 3 \Rightarrow OH = \sqrt{3}$; $OM = \sqrt{1 + 3} = 2$.\\
	Do đó $\sin M = \dfrac{OH}{OM} = \dfrac{\sqrt{3}}{2}$.\\
	Mặt khác $\cos N = \sin M = \dfrac{\sqrt{3}}{2}$ nên $\sin M + \cos N = \sqrt{3}$.
	}
\end{vd}
\begin{vd}%[9H1Y2]
	Tam giác $ABC$ vuông tại $A$, $AB = 1,5$; $BC = 3,5$. Tính tỉ số lượng giác của góc $C$ rồi suy ra các tỉ số lượng giác của góc $B$.
	\loigiai
	{
	Ta có $AC^2 = BC^2 - AB^2 = 3,5^2 - 1,5^2 = 10 \Rightarrow AC = \sqrt{10}$.
	\immini
	{
	Do đó $\begin{aligned}[t]
	&\cos B =\sin C =\dfrac{AB}{BC}=\dfrac{1,5}{3,5}\approx 0,4286; \\
	&\sin B =\cos C =\dfrac{AC}{BC}=\dfrac{\sqrt{10}}{3,5}\approx 0,9035; \\
	&\cot B =\tan C =\dfrac{AB}{AC}=\dfrac{1,5}{\sqrt{10}}\approx 0,4743; \\
	&\tan B =\cot C =\dfrac{AC}{AB}=\dfrac{\sqrt{10}}{1,5}\approx 2,1082.
	\end{aligned}$
	}
	{
	\begin{tikzpicture}[line join = round, line cap = round,>=stealth,scale=1]
	\tkzDefPoints{0/0/A}
	\coordinate (B) at ($(A)+(0,2)$);
	\coordinate (C) at ($(A)+(4,0)$);
	\pgfresetboundingbox
	\tkzMarkRightAngles[size=0.25](B,A,C)
	\tkzDrawSegments(A,B B,C C,A)
	\tkzDrawPoints[fill=black](A,B,C)
	%
	\pgfresetboundingbox
	\path (A)--(B) node[left,midway]{$1,5$};
	\path (B)--(C) node[above right,midway]{$3,5$};
	\tkzMarkAngles[size=.5,arc=l](B,C,A)
	\tkzLabelPoints[above](B)
	\tkzLabelPoints[below](A,C)
	\end{tikzpicture}
	}
	}
\end{vd}
\begin{vd}
	Hãy viết các tỉ số lượng giác sau thành tỉ số lượng giác của góc nhỏ hơn $45^\circ$:\\
	$\sin 60^\circ$, $\cos 75^\circ$, $\sin 52^\circ 30'$, $\tan 80^\circ$, $\cot 82^\circ$.
	\loigiai{
	Ta có:\\
	$\sin 60^\circ = \cos(90^\circ - 60^\circ) = \cos 30^\circ$;\\
	$\cos 75^\circ = \sin (90^\circ - 75^\circ) = \sin 15^\circ$;\\
	$\sin 52^\circ 30'=\cos(90^\circ - 52^\circ 30') = \cos 37^\circ 30'$;\\
	$\tan 80^\circ = \cot(90^\circ - 80^\circ) = \cot 10^\circ$;\\
	$\cot 82^\circ = tan(90^\circ - 82^\circ) = \tan 8^\circ$.
	}
\end{vd}
\begin{vd}
	\immini{Tia nắng chiếu qua điểm $B$ của nó tòa nhà tạo với mặt đất một góc $x$ và tạo với cạnh $AB$ của tòa nhà một góc $y$ (Hình $9$). Cho biết $\cos x\approx 0{,}78$ và $\cot x\approx 1{,}25$. Tính $\sin y$ và $\tan y$ (kết quả làm tròn đến hàng phần trăm).}{
	\begin{tikzpicture}[>=stealth,font=\footnotesize, yscale=.3,xscale=0.75]
	\coordinate (O) at (0,0) ;
	\def\nhathap{5} %so tang nha 
	\def\kc{10} %khoang cach 2 nha
	\def\dai{1} %chieu dai 1 khoi
	\def\cao{2} %chieu cao 1 khoi
	\coordinate (H) at (\kc,0);
	\newcommand{\xaynha}[2]{
	\draw[very thick] (#1,#2)rectangle(#1+\dai,#2+\cao);
	\draw[very thick,fill=black] (#1,#2)rectangle(#1+\dai,#2+0.25*\cao);
	\draw[very thick] (#1+0.2*\dai,#2+0.25*\cao)rectangle(#1+0.8*\dai,#2+0.9*\cao);
	}
	% Xay nha
	\foreach \j in {1,...,\nhathap}{
	\foreach \i in {0,...,1}{
	\xaynha{\i*1.1}{\j*\cao}
	}
	}
	\draw[line width=0.1cm] (-0.1,\cao*\nhathap+\cao)--(2*\dai*1.1,\cao*\nhathap+\cao);
	\draw ($(0,2)+(2.1,0)$) coordinate (A)--($(90:12)+(2.1,0)$) coordinate (B)--(5,2)coordinate (C)--cycle
	;
	\foreach \x/\g in {A/-90,B/90,C/-90}\fill[red] (\x) circle (1pt)+(\g:6mm) node[black]{$ \x $};
	\path
	(A)--(C)%node[below,midway]{Mặt sàn}
	(A)--(C)
	(A)--(B)
	(B)++(30:1.25cm)%node{Sân thượng}
	;
	\path pic[draw,angle radius=5pt]{right angle= B--A--C};
	\path pic["$x$", angle eccentricity=2,double,draw,angle radius=9pt]{angle= B--C--A};
	\path pic["$y$", angle eccentricity=2,draw,angle radius=9pt]{angle= A--B--C};
	\end{tikzpicture}
	}
	\loigiai{
	Do góc $x$ và góc $y$ là hai góc phụ nhau nên $\sin y=\cos x\approx 0{,}78$ và $\tan y=\cot x\approx 1{,}25$.
	}
\end{vd}
\begin{vd}
	\immini{Hình bên mô tả một chiếc thang có chiều dài $AB=4$ m được đặt dựa vào tường, khoảng cách từ chân thang đến chân tường là $BH=1{,}5$ m. Tính góc tạo bởi cạnh $AB$ và phương nằm ngang trên mặt đất (làm tròn kết quả đến hàng đơn vị của độ).}{
	\begin{tikzpicture}[xscale=0.8, yscale=0.6, font=\footnotesize, >=stealth]
	\path
	(-1,0) coordinate (P)+(-80:1) node{ }
	(P)+(60:5) coordinate (Q)
	(P)+(-30:0.2) coordinate (P1)+(-90:.3) node{$B$}
	(Q)+(-30:0.2) coordinate (Q1)+(10:.37) node{$A$}
	(P)+(150:0.9) coordinate (PP)
	(Q)+(150:0.9) coordinate (QQ)
	(P1)+(150:0.9) coordinate (PP1)
	(Q1)+(150:0.9) coordinate (QQ1)
	(4,0) coordinate (SD)
	(SD)+(150:10) coordinate (ST)
	(Q1)+(-90:1) coordinate (QT)
	(Q)+(0:1) coordinate (Q2)
	(intersection of SD--ST and Q1--QT) coordinate (C)+(-90:.3) node{$H$}
	;
	\draw (-2,-0.5) rectangle (2,5.5);
	\clip (-2,-0.5) rectangle (2,5.5);
	\draw (SD)--(ST) (P1)--(C)--(Q1);
	\draw (Q1)--++(-30:2) (Q1)--++(150:3);
	\draw (Q2)--++(-30:2) (Q2)--++(150:3);
	\pic[draw,angle radius=10]{right angle=P1--C--Q1};
	\foreach \b in {0.1,0.2,...,0.9} {
	\path 
	($(P)!\b!(Q)$) coordinate (m)
	(m)+(50:0.2)coordinate (m1)
	(m)+(150:0.7)coordinate (m2)
	(m1)+(150:0.7)coordinate (m3);
	\draw[fill=cyan!50](m)--(m1)--(m3)--(m2)--(m);}
	\draw[fill=cyan] (P)--(Q)--(Q1)--(P1)--(P);
	\draw[fill=cyan] (PP)--(QQ)--(QQ1)--(PP1)--(PP);
	\end{tikzpicture}}
	\loigiai{
	Ta có, góc tạo bởi cạnh $AB$ và phương nằm ngang trên mặt đất là $\widehat{A B H}$.\\ Xét tam giác $A B H$ vuông tại $H$, ta có $\cos \widehat{A B H}=\dfrac{B H}{A B}=\dfrac{1{,}5}{4}=0{,}375$.\\
	Vậy $\widehat{A B H} \approx 68^{\circ}$.
	}
\end{vd}
\begin{vd}
	\immini{Treo quả cầu kim loại nhỏ vào giá thí nghiệm bằng sợi dây mảnh nhẹ không dãn. Khi quả cầu đứng yên tại vị trí cân bằng, dây treo có phương thẳng đứng. Kéo quả cầu khỏi vị trí cân bằng một đoạn nhỏ rồi buông ra thì quả cầu sẽ chuyển động qua lại quanh vị trí cân bằng. Khi kéo quả cầu khỏi vị trí cân bằng, giả sử tâm $A$ của quả cầu cách $B$ một khoảng $AB=60$ cm và cách vị trí cân bằng một khoảng $AH=20$ cm (Hình 9). Tính số đo góc $\alpha$ tạo bởi sợi dây $BA$ và vị trí cân bằng (làm tròn kết quả đến hàng đơn vị của độ).}{\begin{tikzpicture}[xscale=0.85,transform shape]
	\definecolor{burlywood}{rgb}{0.87, 0.72, 0.53}%da tay
	\definecolor{blanchedalmond}{rgb}{1.0, 0.92, 0.8}%màu móng
	\definecolor{apricot}{rgb}{0.98, 0.81, 0.69}%màu gỗ bút
	\definecolor{ao(english)}{rgb}{0.0, 0.5, 0.0}%màu bút
	\definecolor{dimgray}{rgb}{0.41, 0.41, 0.41}%màu đinh
	\clip (-2.5,.5) rectangle (3.5,6.5);
	\tikzset{tay_ve/.pic={
	\def\T0{ 
	(-1.75,.65)
	..controls +(-85:.7) and +(150:.2) ..(-1.15,-.3)--(-1,-.13)
	..controls +(-60:.35) and +(-40:.25) ..(-.72,0.05)
	..controls +(-35:.55) and +(-40:.35) ..(-.5,0.3)
	..controls +(80:2) and +(60:1) ..cycle
	;
	}
	\fill[burlywood!85] \T0;
	\draw \T0;
	\draw (-1.3,.6)
	..controls +(45:.1) and +(170:.2) ..(-.7,.85)
	(-1.3,.4)
	..controls +(45:.1) and +(170:.1) ..(-.7,.65);
	\def\T1{ 
	(-.8,2)
	..controls +(-130:.65) and +(125:1.2) ..(-1.6,.32)
	..controls +(-40:.2) and +(125:.2) ..(-1.2,-.28)
	..controls +(-20:.5) and +(-35:.2) ..(-1.3,.55)
	..controls +(80:.15) and +(-75:.15) ..(-1.32,.9)%1
	..controls +(-10:.05) and +(-155:.05) ..(-.9,1.2)%--cycle
	;
	}
	\draw (-1.32,.9)%1
	..controls +(150:.05) and +(-35:.05) ..(-1.45,1);
	\fill[burlywood] \T1;
	\draw \T1;
	\def\T2{ 
	(-.9,1.2)
	..controls +(-40:.1) and +(-150:.2) ..(-.45,1)
	..controls +(-100:.4) and +(80:.45) ..(-1.25,0)
	..controls +(-80:.6) and +(-170:.45) ..(.1,.6)
	..controls +(10:.3) and +(-130:.5) ..(.8,1)
	..controls +(50:.3) and +(-150:.3) ..(1.3,1.5)--(.7,2.2)
	..controls +(-150:.1) and +(35:.1) ..(.4,2)
	..controls +(-150:.1) and +(-10:.1) ..(-.8,2)
	;
	}
	\fill[burlywood] \T2;
	\draw \T2;
	%móng tay
	\draw[fill=blanchedalmond] (-1.05,.27)
	..controls +(10:.05) and +(85:.05) .. (-.85,.05)
	..controls +(-120:.45) and +(-145:.45) .. cycle;
	\draw[ball color=red,opacity=1,draw=black] (-1.17,.2)
	..controls +(-120:.25) and +(-152:.4) .. (-.9,-.1)
	..controls +(-100:.15) and +(-20:.3) .. (-1.3,-.37)
	..controls +(160:.3) and +(-180:.45) .. cycle
	;
	}}
	\fill[gray!20!black,draw=black] (-2.5,1.1)--(3,1.1)--(3,.7)--(-2.5,.7)--cycle;
	\fill[burlywood,draw=black] (-2,2.1)--(3.3,2.1)--(3,1.1)--(-2.5,1.1)--cycle;
	\fill[gray!70!black,draw=black] (3.3,1.7)--(3.3,2.1)--(3,1.1)--(3,.7)--cycle;
	\draw[dashed] (.55,2.2) circle (1.4mm);
	\path (.55,6) coordinate (B)
	(-.8,2.5) coordinate (A)
	(.55,2.5) coordinate (H)
	;
	\fill[gray!90!black,draw=black] (.85,1.4)--(.9,1.8)--(2.1,1.8)--(2.2,1.4)--cycle;
	\fill[gray!50!black,draw=black] (.85,1.2)--(.85,1.4)--(2.2,1.4)--(2.2,1.2)--cycle;
	\fill[gray!50!black,draw=black] (-2.5,1.1)--(3,1.1)--(3,.7)--(-2.5,.7)--cycle;
	\draw[gray!70!black,line width=.5mm] (.2,6)--(2,6);
	\draw[gray!70!black,line width=.5mm] (1.5,6.8)--(1.5,1.65);
	\draw[dashed] (A)--(H)--(B);
	\draw[blue,line width=.3mm] (A)--(B);
	\node at (A) [below]{\tiny $A$};
	\node at (B) [below left]{\tiny $B$};
	\node at (H) [right]{\tiny $H$};
	\node at (.8,5) [left,rotate=90]{\tiny Vị trí cân bằng};
	\node at (H) [right]{\tiny $H$};
	\node at (-1.1,4) [right]{\tiny $60$ cm};
	\node at (-.5,2.3) [right]{\tiny $20$ cm};
	\draw pic["\tiny $\alpha$", draw=black, angle eccentricity=1.5, angle radius=.2cm, color=blue]
	{angle=H--A--B}; 
	\draw pic[draw=black, angle eccentricity=2, angle radius=0.15cm]
	{right angle=B--H--A};	
	\path %(0,0)pic[scale=1]{tay_ve};
	(-1.25,2.25)pic[scale=.42,yscale=-1,rotate=140]{tay_ve};
	\end{tikzpicture}}
	\loigiai{
	Xét tam giác $ABH$ vuông tại $H$, ta có:
	$
	\sin \alpha=\dfrac{A H}{A B}=\dfrac{20}{60}=\dfrac{1}{3}.
	$\\
	Do đó $\alpha \approx 19^{\circ}$.\\
	Vậy góc $\alpha$ tạo bởi sợi dây $BA$ và vị trí cân bằng có số đo khoảng $19^{\circ}$.
	}
\end{vd}
%===================
\begin{dang}{Tính các cạnh trong một tam giác vuông sử dụng tỉ số lượng giác của góc nhọn}
\end{dang}
\begin{vd} %Ví dụ 2
	Cho tam giác $ABC$ vuông tại $A$ có $\widehat{C} = 30^\circ$ và $BC=a$. Tính các cạnh $AB$, $AC$ theo $a$.
	\loigiai{
	\immini{
	Ta có $\sin C = \dfrac{AB}{BC}$, suy ra $AB=BC \cdot \sin C = a \cdot \sin 30^\circ$.\\
	Theo bảng trên, $\sin 30^\circ = \dfrac{1}{2}$ nên $AB=\dfrac{a}{2}$.\\
	Tương tự, ta có $\cos C = \dfrac{AC}{BC}$, suy ra $AC=BC \cdot cos C = a \cdot \cos 30^\circ =\dfrac{a\sqrt{3}}{2}$.
	}{
	\begin{tikzpicture}[>=stealth,line join=round,line cap=round,font=\footnotesize,scale=0.65]
	\path 
	(0,0) coordinate[label=left:$A$] (A)
	(5,0) coordinate[label=right:$B$] (B)
	(0,1) coordinate (E);
	\coordinate (D) at ($(B)!1!-30: (A)$) ;
	\coordinate[label=left:$C$] (C) at (intersection of B--D and A--E);
	\draw (A)--(B)--(C)--cycle;
	\pic[draw, angle radius=3mm]{right angle=B--A--C};
	\pic[draw, angle radius=5mm]{angle=C--B--A} node at ($(B) + (165:1.2)$){$30^\circ$};
	\end{tikzpicture}}
	}
\end{vd}
\begin{vd}
	Cho tam giác $ABC$ vuông tại $A$ có $\widehat{C} = 45^\circ$ và $AB=c$. Tính $BC$ và $AC$ theo $c$.
	\loigiai{
	\immini{
	Xét $\Delta ABC$ vuông tại $A$ có $\sin C = \dfrac{AB}{BC}$\\
	Suy ra $BC= \dfrac{AB}{\sin C} = \dfrac{c}{\sin 45^\circ} = c\sqrt{2}$.\\
	Xét $\Delta ABC$ vuông tại $A$ có $\widehat{B} + \widehat{C} = 90^\circ$.\\
	Do đó $\widehat{B} = \widehat{C} = 45^\circ$.\\
	Suy ra $\Delta ABC$ vuông cân tại $A$.\\
	Suy ra $AC=AB=c$.
	}{
	\begin{tikzpicture}[>=stealth,line join=round,line cap=round,font=\footnotesize,scale=0.5]
	\path 
	(0,0) coordinate[label=left:$A$] (A)
	(5,0) coordinate[label=right:$B$] (B)
	(0,5) coordinate[label=left:$C$] (C);
	\draw (A)--(B)--(C)--cycle;
	\pic[draw, angle radius=3mm]{right angle=B--A--C};
	\pic[draw, angle radius=3mm]{angle=A--C--B} node[below] at ($(C) - (125:0.9)$){$45^\circ$};
	\end{tikzpicture}}}
\end{vd}
\begin{vd}
	\immini{Tìm chiều cao của tháp canh trong hình bên (kết quả là tròn đến hàng phần trăm).}{
	\begin{tikzpicture}[declare function={a=2.3;},>=stealth, line join=round,line cap=round,font=\footnotesize,scale=0.85]
	\path (0:0) coordinate (B)
	(180:a) coordinate (C)
	(C)+(60:2*a) coordinate (A)
	($(B)!0.5!(C)$) coordinate (D)
	% khai báo vẽ tháp canh
	(180:0.2*a) coordinate (T) %node {T}
	(0:0.2*a) coordinate (D) %node {D}
	(90:1.1*a) coordinate (X) %node {x} % nháp
	(X)+(180:0.13*a) coordinate (E) %node {E}
	(X)+(0:0.13*a) coordinate (F) %node {F}
	% thân chòi
	(X)+(180:0.2*a) coordinate (E1) %node {E1}
	(X)+(0:0.2*a) coordinate (F1) %node {F1}
	(X)+(90:0.45*a) coordinate (Y) %node {Y}
	(Y)+(180:0.3*a) coordinate (Y1) %node {Y1}
	(Y)+(0:0.3*a) coordinate (Y2) %node {Y2}
	($(Y1)!(E1)!(Y2)$) coordinate (E2) %node {E2}
	($(Y1)!(F1)!(Y2)$) coordinate (F2) %node {F2}
	%=========================
	% lang can
	(X)+(180:0.26*a) coordinate (H1) %node {h1}
	(X)+(0:0.26*a) coordinate (H2)% node {h2}
	(H1)+(90:0.1*a) coordinate (H3)% node {h3}
	($(H2)+(H3)-(H1)$) coordinate (H4)% node {h4}
	;
	% thanh chéo ,0.6/T6/X6,0.7/T7/X7,0.8/T8/X8,0.9/T9/X9
	\foreach \i/\t/\x in {0.2/T1/X1,0.4/T2/X2,0.6/T3/X3,0.8/T4/X4}{
	\path ($(E)!\i!(T)$) coordinate (\t);
	\path ($(F)!\i!(D)$) coordinate (\x);
	\draw [thick,black!60!red] (\t)--(\x);
	}
	\draw[thick,black!60!red] (T1)--(X2) (X1)--(T2) (T3)--(X4) (X3)--(T4);
	%==================================
	% chân chòi
	\filldraw[red,opacity=0.65] (A)--(Y1)--(Y2)--cycle;
	\filldraw[orange!70!green,opacity=0.65] (E2)--(E1)--(F1)--(F2)--cycle;
	%==========================
	% vẽ lang can
	\foreach \i/\t/\x in {0/A0/B0,0.2/A1/B1,0.4/A2/B2,0.6/A3/B3,0.8/A4/B4,1/A5/B5}{
	\path ($(H1)!\i!(H2)$) coordinate (\t);
	\path ($(H3)!\i!(H4)$) coordinate (\x);
	\draw[thick,brown] (\t)--(\x);
	}
	\draw[thick,brown] (H1)--(H2)--(H4)--(H3)--cycle;
	%=============
	\draw [thick,black!60!red] 
	(T)--(E)
	(D)--(F)
	;
	%cửa sổ
	\path 
	($(X)!0.5!(F2)$)+(90:0.04*a) coordinate (M) %node {m}
	(M)+(90:0.06*a) coordinate (M1)
	(M)+(0:0.06*a) coordinate (M2)
	(M)+(-90:0.06*a) coordinate (M3)
	(M)+(180:0.06*a) coordinate (M4)
	;
	\draw [blue!70] (M2) rectangle (M1)
	(M3) rectangle (M2)
	(M3) rectangle (M4)
	(M4) rectangle (M1)
	;
	%====================
	\path pic[draw,angle radius=5pt]{right angle= A--B--C};
	\path pic["$60^\circ$", angle eccentricity=2.2, draw,angle radius=8pt]{angle= B--C--A};
	\draw[thick,blue] (A)--(B) 
	--(C) node[midway,below] {$5{,}8$ m}
	--cycle 
	;
	\foreach \t/\g in {A/90,B/-90,C/-90}{
	\path (\t) node[shift={(\g:7pt)}]{$ \t$};
	}
	\end{tikzpicture}}
	\loigiai{
	Xét $\triangle ABC$ vuông tại $B$, ta có\\	
	$\tan C=\dfrac{AB}{CB}$, suy ra $AB=CB\tan C$ hay $AB=5{,}8\cdot\tan 60^\circ=5{,}8\cdot\sqrt{3}\approx 10{,}05 $ (m).\\
	Vậy chiều cao của tháp canh gần bằng $10{,}05$ mét.
	}
\end{vd}
%==================
\begin{dang}{\boldmath Dựng góc nhọn $\alpha$ biết một tỉ số lượng giác của góc đó bằng $\frac{m}{n}$}
	Dựng một tam giác vuông có cạnh là $m$ và $n$ ($m$ và $n$ tương ứng là cạnh góc vuông và cạnh huyền nếu tỉ số lượng giác đã cho là $\sin$ hoặc $\cos$; $m$ và $n$ là hai cạnh góc vuông nếu tỉ số lượng giác đã cho là $\tan$ hoặc $\cot$) rồi vận dụng định nghĩa để nhận ra góc $\alpha$.
\end{dang}
\begin{vd}%[9H1B2]
	Dựng góc $\alpha$, biết $\sin \alpha = 0,25$.
	\loigiai{
	\immini{
	Ta có $0,25 = \dfrac{1}{4}$.
	\begin{itemize}
	\item Dựng góc vuông $xOy$;
	\item Trên cạnh $Ox$ đặt $OA = 1$;
	\item Dựng đường tròn $(A;\ 4)$ cắt cạnh $Oy$ tại $B$.
	\end{itemize}
	Khi đó $\widehat{ABO} = \alpha \ \left( \mbox{vì }\sin \alpha = \dfrac{OA}{AB} = \dfrac{1}{4} \right)$.
	}{
	\begin{tikzpicture}[line join = round, line cap = round,>=stealth,font=\footnotesize,scale=1.3] 
	%
	\tkzDefPoints{0/0/O, 0/1/A, 1/0/E, 0/-3/S} 
	\tkzInterLC(O,E)(A,S) \tkzGetPoints{D}{B}
	\tkzDrawArc[R,delta=20,dashed,color=black](O,1)(80,90)
	\pgfresetboundingbox
	\tkzDrawArc[R,dashed,color=black](A,4)(340,355)
	\draw[->] (0,0)--(5,0) node[below] at (5,-0.2){$x$};
	\draw[->] (0,0)--(0,2) node[right] at (0.2,2){$y$};
	\tkzDrawSegments(A,B)
	\tkzMarkRightAngles[size=0.2](A,O,B)
	\tkzLabelSegments[color=black,left](O,A){$1$}
	\tkzLabelSegments[color=black,above](A,B){$4$}
	\tkzMarkAngles[size=.9,arc=l](A,B,O)
	\tkzLabelAngles[pos=1.2](A,B,O){$\alpha$}
	\tkzDrawPoints[fill=black](A,B) 
	\tkzLabelPoints[below left](O) 
	\tkzLabelPoints[above left](A)
	\tkzLabelPoints[below right](B) 
	\end{tikzpicture}
	}
	}
\end{vd}
\begin{vd}%[9H1B2]
	Dựng góc $\alpha$, biết $\cos \alpha = 0,75$.
	\loigiai
	{
	Ta có $0,75 = \dfrac{3}{4}$.
	\immini
	{
	\begin{itemize}
	\item Dựng góc vuông $xOy$;
	\item Trên cạnh $Oy$ đặt $OB = 3$;
	\item Dựng đường tròn $(B;\ 4)$ cắt cạnh $Ox$ tại $A$.
	\end{itemize}
	Khi đó $\widehat{ABO} = \alpha \ \left( \mbox{vì }\cos \alpha = \dfrac{OB}{AB} = \dfrac{3}{4} \right)$.
	}
	{
	\begin{tikzpicture}[line join = round, line cap = round,>=stealth,font=\footnotesize,scale=0.9] 
	%
	\tkzDefPoints{0/0/O, 3/0/B, 0/1/E, -1/0/S} 
	\tkzInterLC(O,E)(B,S) \tkzGetPoints{A}{D}
	\tkzDrawArc[R,dashed,color=black](O,3)(-10,10)
	\pgfresetboundingbox
	\tkzDrawArc[R,dashed,color=black](B,4)(130,145)
	\draw[->] (0,0)--(4,0) node[below] at (4,-0.2){$x$};
	\draw[->] (0,0)--(0,3) node[above] at (0.2,3){$y$};
	\tkzDrawSegments(A,B)
	\tkzMarkRightAngles[size=0.2](A,O,B)
	\tkzLabelSegments[color=black,below](O,B){$3$}
	\tkzLabelSegments[color=black,above](A,B){$4$}
	\tkzMarkAngles[size=.5,arc=l](A,B,O)
	\tkzLabelAngles[pos=0.7](A,B,O){$\alpha$}
	\tkzDrawPoints[fill=black](A,B) 
	\tkzLabelPoints[below left](O) 
	\tkzLabelPoints[above left](A)
	\tkzLabelPoints[below right](B) 
	\end{tikzpicture}
	}	
	}
\end{vd}
\begin{vd}%[9H1B2]
	Dựng góc $\alpha$, biết $\tan \alpha = 1,5$.
	\loigiai
	{
	\immini
	{Ta có $1,5 = \dfrac{3}{2}$.
	\begin{itemize}
	\item Dựng góc vuông $xOy$;
	\item Trên cạnh $Ox$ đặt $OA = 3$;
	\item Trên cạnh $Oy$ đặt $OB = 2$.
	\end{itemize}
	Khi đó $\widehat{ABO} = \alpha \ \left( \mbox{vì }\tan \alpha = \dfrac{OA}{OB} = \dfrac{3}{2} \right)$.
	}
	{
	\begin{tikzpicture}[line join = round, line cap = round,>=stealth,font=\footnotesize,scale=0.85] 
	%
	\tkzDefPoints{0/0/O, 2/0/B, 0/3/A} 
	\tkzDrawArc[R,dashed,color=black](O,2)(-10,10)
	\pgfresetboundingbox
	\tkzDrawArc[R,dashed,color=black](O,3)(85,95)
	\draw[->] (0,0)--(3,0) node[below] at (3,-0.2){$x$};
	\draw[->] (0,0)--(0,4) node[above] at (0.2,4){$y$};
	\tkzDrawSegments(A,B)
	\tkzMarkRightAngles[size=0.2](A,O,B)
	\tkzLabelSegments[color=black,below](O,B){$3$}
	\tkzLabelSegments[color=black,above](A,B){$4$}
	\tkzMarkAngles[size=.5,arc=l](A,B,O)
	\tkzLabelAngles[pos=0.7](A,B,O){$\alpha$}
	\tkzDrawPoints[fill=black](A,B) 
	\tkzLabelPoints[below left](O) 
	\tkzLabelPoints[above left](A)
	\tkzLabelPoints[below right](B) 
	\end{tikzpicture}
	}	
	}
\end{vd}
\begin{vd}%[9H1B2]
	Dựng góc $\alpha$, biết $\cot \alpha = 2$.
	\loigiai
	{
	\immini
	{
	\begin{itemize}
	\item Dựng góc vuông $xOy$;
	\item Trên cạnh $Ox$ đặt $OA = 1$;
	\item Trên cạnh $Oy$ đặt $OB = 2$.
	\end{itemize}
	Khi đó $\widehat{ABO} = \alpha \ \left( \mbox{vì }\cot \alpha = \dfrac{OB}{OA} = 2 \right)$.
	}
	{
	\begin{tikzpicture}[line join = round, line cap = round,>=stealth,font=\footnotesize,scale=1.5] 
	%
	\tkzDefPoints{0/0/O, 2/0/B, 0/1/A} 
	\tkzDrawArc[R,dashed,color=black](O,2)(-10,10)
	\pgfresetboundingbox
	\tkzDrawArc[R,dashed,color=black](O,1)(70,105)
	\draw[->] (0,0)--(3,0) node[below] at (3,-0.2){$x$};
	\draw[->] (0,0)--(0,2) node[above] at (0.2,2){$y$};
	\tkzDrawSegments(A,B)
	\tkzMarkRightAngles[size=0.2](A,O,B)
	\tkzLabelSegments[color=black,left](O,A){$1$}
	\tkzLabelSegments[color=black,below](O,B){$2$}
	\tkzMarkAngles[size=.5,arc=l](A,B,O)
	\tkzLabelAngles[pos=0.7](A,B,O){$\alpha$}
	\tkzDrawPoints[fill=black](A,B) 
	\tkzLabelPoints[below left](O) 
	\tkzLabelPoints[above left](A)
	\tkzLabelPoints[below right](B) 
	\end{tikzpicture}
	}	
	}
\end{vd}
%==================
\begin{dang}{Tính giá trị của biểu thức lượng giác với các góc đặc biệt}
	\begin{itemize}
	\item Sử dụng bảng giá trị các tỉ số lượng giác của các góc $30^\circ$; $45^\circ$; $60^\circ$.
	\item Sử dụng tỉ số lượng giác của hai góc phụ nhau.
	\end{itemize}
\end{dang}
\begin{vd}
	Tính giá trị của mỗi biểu thức sau:
	\begin{listEX}[3]
	\item $\sin^2 45^{\circ}+\cos ^2 45^{\circ}$;
	\item $\tan 30^{\circ} \cdot \cot 30^{\circ}$;
	\item $\dfrac{\sin 30^\circ\cdot\cos 60^\circ}{\tan 45^\circ}$
	\end{listEX}
	\loigiai{
	\begin{listEX}
	\item $\sin^2 45^{\circ}+\cos^2 45^{\circ}=\left(\dfrac{\sqrt{2}}{2}\right)^2+\left(\dfrac{\sqrt{2}}{2}\right)^2=\dfrac{1}{2}+\dfrac{1}{2}=1$.
	\item $\tan 30^{\circ} \cdot \cot 30^{\circ}=\dfrac{\sqrt{3}}{3} \cdot \sqrt{3}=1$.
	\item $\dfrac{\sin 30^\circ\cdot\cos 60^\circ}{\tan 45^\circ}=\dfrac{\dfrac{1}{2}\cdot\dfrac{1}{2}}{1}=\dfrac{1}{4}$.
	\end{listEX}
	}
\end{vd}	
\begin{vd}
	Tính giá trị của các biểu thức sau
	\begin{enumEX}{2}
	\item $A=\dfrac{2\cos 45^\circ}{\sqrt{2}}+\sqrt{3}\tan 30^\circ$;
	\item $B=\dfrac{2\sin 60^\circ}{\sqrt{3}}-\cot 45^\circ$.
	\end{enumEX}
	\loigiai{
	\begin{listEX}
	\item Ta có $A=\dfrac{2\cos 45^\circ}{\sqrt{2}}+\sqrt{3}\tan 30^\circ=\dfrac{2\cdot\dfrac{\sqrt{2}}{2}}{\sqrt{2}}+\sqrt{3}\cdot \dfrac{\sqrt{3}}{3}=1+1=2$;
	\item Ta có $B=\dfrac{2\sin 60^\circ}{\sqrt{3}}-\cot 45^\circ=\dfrac{2\cdot\dfrac{\sqrt{3}}{2}}{\sqrt{3}}-1=1-1=0$.
	\end{listEX}
	}
\end{vd}
\begin{vd}%[9H1Y2]
	Tính giá trị của biểu thức
	\begin{enumEX}{2}
	\item $M= 4\cos^245^{\circ} + \sqrt{3}\cot30^{\circ} - 16\cos^360^{\circ}$;
	\item $N =\dfrac{2\sin 30^{\circ} - \sin 60^{\circ}}{\cos^230^{\circ} - \cos 60^{\circ}}$.
	\end{enumEX}
	\loigiai
	{
	\begin{enumEX}{2}
	\item \allowdisplaybreaks $\begin{aligned}[t]
	M &= 4\cos^245^{\circ} + \sqrt{3}\cot30^{\circ} - 16\cos^360^{\circ}\\
	&= 4\cdot\left(\dfrac{\sqrt{2}}{2}\right)^2 + \sqrt{3}\cdot\sqrt{3} - 16\cdot\left(\dfrac{1}{2}\right)^3\\
	&= 2 + 3 - 2 = 3.
	\end{aligned}$
	\item \allowdisplaybreaks $\begin{aligned}[t]
	N &=\dfrac{2\sin 30^{\circ} - \sin 60^{\circ}}{\cos^230^{\circ} - \cos 60^{\circ}}\\
	&= \dfrac{2\cdot\dfrac{1}{2} - \dfrac{\sqrt{3}}{2}}{\left(\dfrac{\sqrt{3}}{2}\right)^2 - \dfrac{1}{2}}=\dfrac{1 - \dfrac{\sqrt{3}}{2}}{\dfrac{3}{4} - \dfrac{1}{2}}= 4 - 2\sqrt{3}.
	\end{aligned}$
	\end{enumEX}
	}
\end{vd}
\begin{vd}
	Tính:
	\begin{listEX}[4]
	\item $\sin 61^{\circ}-\cos 29^{\circ}$;
	\item $\cos 15^{\circ}-\sin 75^{\circ}$;
	\item $\tan 28^{\circ}-\cot 62^{\circ}$;
	\item $\cot 47^{\circ}-\tan 43^{\circ}$.
	\end{listEX}
	\loigiai{
	\begin{listEX}[2]
	\item $\sin 61^{\circ}-\cos 29^{\circ}=\sin 61^{\circ}-\sin 61^{\circ}=0$;
	\item $\cos 15^{\circ}-\sin 75^{\circ}=\cos 15^{\circ}-\cos 15^{\circ}=0$;
	\item $\tan 28^{\circ}-\cot 62^{\circ}=\tan 28^{\circ}-\tan 28^{\circ}=0$;
	\item $\cot 47^{\circ}-\tan 43^{\circ}=\cot 47^{\circ}-\cot 47^{\circ}=0$.
	\end{listEX}
	}
\end{vd}
\begin{vd}%[9H1Y2]
	Tính giá trị của biểu thức
	\begin{listEX}
	\item $P =\sin^2 30^{\circ} - \sin^2 40^{\circ} - \sin^2 50^{\circ} + \sin^2 60^{\circ}$;
	\item $Q =\cos^2 25^{\circ} - \cos^2 35^{\circ} + \cos^2 45^{\circ} - \cos^2 55^{\circ} + \cos^2 65^{\circ}$.
	\end{listEX}
	\loigiai
	{
	\begin{listEX}
	\item \allowdisplaybreaks $\begin{aligned}[t]
	P &=\sin^230^{\circ} - \sin^240^{\circ} - \sin^250^{\circ} + \sin^260^{\circ}\\
	&= \left(\sin^2 30^{\circ} + \sin^2 60^{\circ}\right) - \left(\sin^2 40^{\circ} + \sin^2 50^{\circ}\right)\\
	&= \left(\sin^2 30^{\circ} + \cos^2 30^{\circ}\right) - \left(\sin^2 40^{\circ} + \cos^2 40^{\circ}\right)\\
	&= 1 - 1 = 0.
	\end{aligned}$
	\item \allowdisplaybreaks $\begin{aligned}[t]
	Q &=\cos^2 25^{\circ} - \cos^2 35^{\circ} + \cos^2 45^{\circ} - \cos^2 55^{\circ} + \cos^2 65^{\circ}\\
	&=\left(\cos^2 25^\circ + \cos^2 65^{\circ}\right) - \left(\cos^2 35^{\circ} + \cos^2 55^{\circ}\right) + \cos^2 45^{\circ}\\
	&= \left(\cos^2 25^{\circ} + \sin^2 25^{\circ}\right) - \left(\cos^2 35^{\circ} + \sin^2 35^{\circ}\right) + \left(\dfrac{\sqrt{2}}{2}\right)^2\\
	&=1 - 1 + \dfrac{1}{2}=\dfrac{1}{2}.
	\end{aligned}$
	\end{listEX}
	}
\end{vd}
%===================
\begin{dang}{So sánh các tỉ số lượng giác mà không dùng máy tính hoặc bảng số}
	Dùng định lí tỉ số lượng giác của hai góc phụ nhau (nếu cần) và căn cứ vào những tính chất sau:
	\begin{itemize}
	\item Khi góc nhọn $\alpha$ tăng từ $0^\circ$ đến $90^\circ$ thì
	\begin{itemize}
	\item $\sin \alpha$ tăng và $\tan \alpha$ tăng;
	\item $\cos \alpha$ giảm và $\cot \alpha$ giảm.
	\end{itemize}
	\item $\sin \alpha < \tan \alpha$; $\cos \alpha < \cot \alpha$.
	\end{itemize}
\end{dang}
\begin{vd}
	So sánh
	\begin{enumEX}{4}
	\item $\sin 25^\circ$ và $\cos 65^\circ$;
	\item $ \cos 25^\circ$ và $\sin 65^\circ$;
	\item $\tan 25^\circ$ và $\cot 65^\circ$;
	\item $\cot 25^\circ$ và $\tan 65^\circ$.
	\end{enumEX}
	\loigiai{
	Ta có
	\begin{enumEX}{2}
	\item $\sin 25^\circ=\cos (90^\circ-25^\circ)=\cos 65^\circ$;
	\item $\cos 25^\circ=\sin (90^\circ-25^\circ)=\sin 65^\circ$;
	\item $\tan 25^\circ=\cot (90^\circ-25^\circ)=\cot 65^\circ$;
	\item $\cot 25^\circ=\tan (90^\circ-25^\circ)=\tan 65^\circ$.
	\end{enumEX}
	}
\end{vd}
\begin{vd}
	\begin{listEX}
	\item So sánh $\sin 72^\circ$ và $\cos 18^\circ$; $ \cos 72^\circ$ và $\sin 18^\circ$; $\tan 72^\circ$ và $\cot 18^\circ$.
	\item Cho biết $\sin 18^\circ\approx 0{,}31$; $\tan 18^\circ\approx 0{,}32$. Tính $\cos 72^\circ$ và $\cot 72^\circ$.
	\end{listEX}
	\loigiai{
	\begin{listEX}
	\item 
	$\sin 72^\circ=\cos (90^\circ-72^\circ)=\cos 18^\circ$;\\
	$\cos 72^\circ=\sin (90^\circ-72^\circ)=\sin 18^\circ$;\\
	$\tan 72^\circ=\cot (90^\circ-72^\circ)=\cot 18^\circ$.
	\item
	$\sin 18^\circ=\cos 72^\circ\approx 0{,}31$;\\
	$\tan 18^\circ=\cot 72^\circ\approx 0{,}32$.
	\end{listEX}
	}
\end{vd}
\begin{vd}%[9H1B2]
	So sánh hai số $m$ và $n$, biết $m = \dfrac{\sin 50^\circ}{\cos 65^\circ}$; $n = \dfrac{\cot 70^\circ}{\tan 35^\circ}$.
	\loigiai
	{
	Ta có $m=\dfrac{\sin 50^{\circ}}{\cos 65^{\circ}}=\dfrac{\sin 50^{\circ}}{\sin 25^{\circ}}>\dfrac{\sin 25^{\circ}}{\sin 25^{\circ}}= 1$; \tagEX{1}
	\hspace{0.5cm}$n = \dfrac{\cot 70^{\circ}}{\tan 35^{\circ}}=\dfrac{\tan 20^{\circ}}{\tan 35^{\circ}} < \dfrac{\tan 35^{\circ}}{\tan 35^{\circ}}= 1$. \tagEX{2}
	Từ ($1$) và ($2$) suy ra $m > n$.
	}
\end{vd}
\begin{vd}%[9H1B2]
	Sắp xếp các tỉ số lượng giác sau theo thứ tự tăng dần
	\begin{enumEX}{2}
	\item $\sin 70^{\circ},\cos 30^{\circ},\cos 40^{\circ},\sin 51^{\circ}$;
	\item $\cos 34^{\circ},\sin 57^{\circ},\cot 32^{\circ}$.
	\end{enumEX}
	\loigiai
	{
	\begin{listEX}
	\item Ta có $\cos 30^{\circ}=\sin 60^{\circ}$; $\cos 40^{\circ}=\sin 50^{\circ}$.\\
	Vì $\sin 50^{\circ}<\sin 51^{\circ}<\sin 60^{\circ}<\sin 70^{\circ}$ nên $\cos 40^{\circ}<\sin 51^{\circ}<\cos 30^{\circ}<\sin 70^{\circ}$.
	\item Ta có $\cos 34^{\circ}=\sin 56^{\circ}$; $\cot 32^{\circ} = \tan 58^{\circ}$.\\
	Vì $\sin 56^{\circ} < \sin 57^{\circ} < \sin 58^{\circ} < \tan 58^{\circ}$ nên $\cos 34^{\circ} < \sin 57^{\circ} < \cot 32^{\circ}$.
	\end{listEX}
	}
\end{vd}
\begin{vd}%[9H1B2]
	Sắp xếp các tỉ số lượng giác sau theo thứ tự tăng dần
	\begin{enumEX}{2}
	\item $\cot40^{\circ},\sin 40^{\circ},\cot43^{\circ},\tan42^{\circ}$;
	\item $\tan52^{\circ},\cot63^{\circ},\tan72^{\circ},\cot31^{\circ},\sin 27^{\circ}$.
	\end{enumEX}
	\loigiai
	{
	\begin{listEX}
	\item Ta có $\cot40^{\circ}=\tan50^{\circ}$; $\cot43^{\circ}=\tan47^{\circ}$.\\
	Vì $\sin 40^{\circ}<\tan40^{\circ}<\tan42^{\circ}<\tan47^{\circ}<\tan50^{\circ}$ nên $\sin 40^{\circ}<\tan42^{\circ}<\cot43^{\circ}<\cot40^{\circ}$.
	\item Ta có $\cot63^{\circ}=\tan27^{\circ}$; $\cot31^{\circ}=\tan59^{\circ}$.\\
	Vì $\sin 27^{\circ}<\tan27^{\circ}<\tan52^{\circ}<\tan59^{\circ}<\tan72^{\circ}$ nên $\sin 27^{\circ}<\cot63^{\circ}<\tan52^{\circ}<\cot31^{\circ}<\tan72^{\circ}$.
	\end{listEX}
	}
\end{vd}
\begin{vd}%[9H1B2]
	Cho $25^{\circ} < \alpha < 50^{\circ}$, hãy sắp xếp các tỉ số lượng giác sau theo thứ tự giảm dần: $\sin\alpha ;\ \cos\left(\alpha + 40^{\circ}\right) ;\ \tan\left(\alpha + 10^{\circ}\right)$.
	\loigiai
	{
	Vì $25^{\circ}<\alpha < 50^{\circ}$ nên $\alpha + 10^{\circ}>\alpha > 50^{\circ} - \alpha$.\\
	Mặt khác góc $50^{\circ} - \alpha$ phụ với góc $a + 40^{\circ}$.\\
	Ta có $\tan\left(\alpha + 10^{\circ}\right) >\sin\left(\alpha + 10^{\circ}\right) >\sin\alpha >\sin\left(50^{\circ} - \alpha\right)$,
	do đó $\tan\left(\alpha + 10^{\circ}\right) >\sin\alpha >\cos\left(\alpha + 40^{\circ}\right)$.
	}
\end{vd}
%%%%%%%%%%%%%%%%%%%
\subsection{Bài tập vận dụng}
%%%%% tính tỉ số LG
\begin{bt}
	Cho tam giác $ABC$ vuông tại $A$. Tính các tỉ số lượng giác của góc $B$ trong mỗi trường hợp sau
	\begin{enumEX}{2}
	\item $BC=5$ cm; $AB=3$ cm;
	\item $BC=13$ cm; $AC=12$ cm;
	\item $BC=5\sqrt{2}$ cm; $AB=5$ cm;
	\item $AB=a\sqrt{3}$; $AC=a$.
	\end{enumEX}
	\loigiai{
	\begin{listEX}
	\item Áp dụng định lí Pythagore ta được $AC=\sqrt{5^2-3^2}=4$ cm. Tỉ số lượng giác của $\widehat{B}$ là\\
	$\sin B=\dfrac{AC}{BC}=\dfrac{4}{5}$;\quad $\cos B=\dfrac{AB}{BC}=\dfrac{3}{5}$;\quad $\tan B=\dfrac{AC}{AB}=\dfrac{4}{3}$;\quad $\cot B=\dfrac{AB}{AC}=\dfrac{3}{4}$.
	\item Áp dụng định lí Pythagore ta được $AB=\sqrt{13^2-12^2}=5$ cm. Tỉ số lượng giác của $\widehat{B}$ là\\
	$\sin B=\dfrac{AC}{BC}=\dfrac{12}{13}$;\quad $\cos B=\dfrac{AB}{BC}=\dfrac{5}{13}$;\quad $\tan B=\dfrac{AC}{AB}=\dfrac{12}{5}$;\quad $\cot B=\dfrac{AB}{AC}=\dfrac{5}{12}$.
	\item Áp dụng định lí Pythagore ta được $AC=\sqrt{\left(5\sqrt{2}\right)^2-5^2}=5$ cm. Ta có tam giác $ABC$ là tam giác vuông cân nên tỉ số lượng giác của $\widehat{B}$ là\\
	$\sin B=\cos B=\dfrac{AB}{BC}=\dfrac{5}{5\sqrt{2}}\approx0{,}71$;\quad $\tan B=\cot B=\dfrac{AB}{AC}=\dfrac{5}{5}=1$.
	\item Ta có $\cot B=\dfrac{AB}{AC}=\dfrac{a\sqrt{3}}{a}=\sqrt{3}$ nên $\widehat{B}=30^\circ$. Do đó\\
	$\sin B=\sin 30^\circ=\dfrac{1}{2}$;\quad $\cos B=\cos 30^\circ=\dfrac{\sqrt{3}}{2}$;\quad $\tan B=\tan 30^\circ=\dfrac{\sqrt{3}}{3}$.
	\end{listEX}
	}
\end{bt}
\begin{bt}
	Cho tam giác $ABC$ vuông tại $A$ có $AC=4$ cm, $BC=6$ cm. Tính các tỉ số lượng giác của góc $B$.
	\loigiai{
	\immini{Xét tam giác $ABC$ vuông tại $A$, ta có
	$$
	\begin{aligned}
	& AB=\sqrt{BC^2-AC^2}=2\sqrt{5}.\\
	& \sin \widehat{B}=\dfrac{AC}{BC}=\dfrac{4}{6}=\dfrac{2}{3}; 
	\cos \widehat{B}=\dfrac{AB}{BC}=\dfrac{2\sqrt{5}}{6} =\dfrac{\sqrt{5}}{3};\\
	& \tan \widehat{B}=\dfrac{AC}{AB}=\dfrac{4}{2\sqrt{5}}=\dfrac{2}{\sqrt{5}};
	\cot \widehat{B}=\dfrac{AB}{AC}=\dfrac{2\sqrt{5}}{4}=\dfrac{\sqrt{5}}{2}.
	\end{aligned}
	$$
	}{\begin{tikzpicture}[scale=0.8, font=\footnotesize, >=stealth]
	\coordinate (B) at (-3,0);
	\coordinate (C) at (2,0);
	\coordinate (D) at ($(B) + (60:1)$);
	\coordinate (A) at ($(B)!(C)!(D)$);
	\draw(A)--(B)--(C)--cycle;
	\foreach \i/\g in {A/90,B/-135,C/0}{\draw[fill=black](\i) circle (1.0pt) ($(\i)+(\g:3mm)$) node[scale=1]{$\i$};}
	\tkzMarkRightAngles(B,A,C)
	\tkzMarkAngles[arc=l,size=0.5 cm](C,B,A)
	\end{tikzpicture}}
	}
\end{bt}
\begin{bt}
	Cho tam giác $A B C$ vuông tại $A$ có $AB=2$ cm, $A C=3$ cm. Tính các tỉ số lượng giác của góc $C$.
	\loigiai{
	\immini{Xét tam giác $ABC$ vuông tại $A$, ta có
	$$
	\begin{aligned}
	& BC=\sqrt{AB^2+AC^2}=\sqrt{13}.\\
	& \sin \widehat{C}=\dfrac{AB}{BC}=\dfrac{2}{\sqrt{13}}; 
	\cos \widehat{C}=\dfrac{AC}{BC}=\dfrac{3}{\sqrt{13}};\\
	& \tan \widehat{C}=\dfrac{AB}{AC}=\dfrac{2}{3}; 
	\cot \widehat{C}=\dfrac{AC}{AB}=\dfrac{3}{2}.
	\end{aligned}
	$$}{\begin{tikzpicture}[scale=0.8, font=\footnotesize, >=stealth]
	\coordinate (B) at (-3,0);
	\coordinate (C) at (2,0);
	\coordinate (D) at ($(B) + (60:1)$);
	\coordinate (A) at ($(B)!(C)!(D)$);
	\draw(A)--(B)--(C)--cycle;
	\foreach \i/\g in {A/90,B/-90,C/-90}{\draw[fill=black](\i) circle (1.0pt) ($(\i)+(\g:3mm)$) node[scale=1]{$\i$};}
	\tkzMarkRightAngles(B,A,C)
	\tkzMarkAngles[arc=l,size=0.5 cm](A,C,B)
	\end{tikzpicture}}
	}
\end{bt}
\begin{bt}
	Cho tam giác $MNP$ có $MN=5$ cm, $MP=12$ cm, $NP=13$ cm. Chứng minh tam giác $MNP$ vuông tại $M$. Từ đó, tính các tỉ số lượng giác của góc $N$.
	\loigiai{
	\immini{
	Ta có $\heva{&NP^2=169\\&MN^2+MP^2=169}\Rightarrow NP^2=MN^2+MP^2$.\\
	Vậy tam giác $MNP$ vuông tại $M$.\\
	Xét tam giác $MNP$ vuông tại $M$, ta có
	$$
	\begin{aligned}
	& \sin \widehat{N}=\dfrac{MP}{NP}=\dfrac{12}{13}; 
	\cos \widehat{N}=\dfrac{MN}{NP}=\dfrac{5}{13};\\
	& \tan \widehat{N}=\dfrac{MP}{MN}=\dfrac{12}{5}; 
	\cot \widehat{N}=\dfrac{NM}{MP}=\dfrac{5}{12}.
	\end{aligned}
	$$
	}{\begin{tikzpicture}[scale=0.8, font=\footnotesize, >=stealth]
	\coordinate (N) at (-3,0);
	\coordinate (P) at (3,0);
	\coordinate (D) at ($(N) + (60:1)$);
	\coordinate (M) at ($(N)!(P)!(D)$);
	\draw(M)--(N)--(P)--cycle;
	\foreach \i/\g in {M/90,N/-90,P/-90}{\draw[fill=black](\i) circle (1.0pt) ($(\i)+(\g:3mm)$) node[scale=1]{$\i$};}
	\tkzMarkRightAngles(N,M,P)
	\tkzMarkAngles[arc=l,size=0.5 cm](P,N,M)
	\end{tikzpicture}}
	}
\end{bt}
\begin{bt}
	Cho tam giác $ABC$ vuông tại $A$. Tính các tỉ số lượng giác sin, côsin, tang, côtang của các góc nhọn $B$ và $C$ khi biết:
	\begin{listEX}[2]
	\item $AB=8$ cm, $BC = 17$ cm;
	\item $AC = 0{,}9$ cm, $AB = 1{,}2$ cm.
	\end{listEX}
	\loigiai{
	\begin{listEX}
	\item 
%	\immini{
	Xét $\triangle ABC$ vuông tại $A$.\\
	Theo định lí Pythagore, ta có:\\
	$BC^2 = AB^2 + AC^2$\\
	$AC^2 = BC^2 - AB^2 = 17^2 - 8^2 = 225$\\
	$AC= 15$ (cm). \\
	Theo định nghĩa tỉ số lượng giác, ta có:\\
	$\sin B = \dfrac{AC}{BC} = \dfrac{15}{17}$, $\cos B = \dfrac{AB}{BC} = \dfrac{8}{17}$, 
	$\tan B = \dfrac{AC}{AB} = \dfrac{15}{8}$, $\cot B = \dfrac{AB}{AC} = \dfrac{8}{15}$.\\
	$\sin C = \dfrac{AB}{BC} = \dfrac{8}{17}$, $\cos C = \dfrac{AC}{BC} = \dfrac{15}{17}$, 
	$\tan C = \dfrac{AB}{AC} = \dfrac{8}{15}$, $\cot C = \dfrac{AC}{AB} = \dfrac{15}{8}$.
%	}{
%	\begin{tikzpicture}[line join = round, line cap = round, scale=.8]
%	\coordinate[label=left:$A$] (A) at (0,0);
%	\coordinate[label=right:$B$] (B) at (4,0);
%	\coordinate[label=above:$C$] (C) at (0,7.5);
%	\draw (A)--(B)--(C)--cycle;
%	\draw (A)--(B)node[midway,sloped,below]{$8$ cm};
%	\draw (C)--(B)node[midway,sloped,above]{$17$ cm};
%	\pic[draw, angle radius=3mm]{right angle=B--A--C};
%	\pic[draw,double, angle radius=5mm]{angle=C--B--A};
%	\pic[draw, angle radius=5mm]{angle=A--C--B};
%	\end{tikzpicture}
%	}
	\item 
%	\immini{
	Xét $\triangle ABC$ vuông tại $A$.\\
	Theo định lí Pythagore, ta có:\\
	$BC^2 = AB^2 + AC^2 = 1{,}2^2 + 0{,}9^2 = 2{,}25$\\
	$BC=1{,}5$ (cm).\\
	Theo định nghĩa tỉ số lượng giác, ta có:\\
	$\sin B = \dfrac{AC}{BC} = \dfrac{0{,}9}{1{,}5} = \dfrac{3}{5}$, $\cos B = \dfrac{AB}{BC} = \dfrac{1{,}2}{1{,}5} = \dfrac{4}{5}$, 
	$\tan B = \dfrac{AC}{AB} = \dfrac{0{,}9}{1{,}2} = \dfrac{3}{4}$, $\cot B = \dfrac{AB}{AC} = \dfrac{1{,}2}{0{,}9} = \dfrac{4}{3}$.\\
	$\sin C = \dfrac{AB}{BC} = \dfrac{1{,}2}{1{,}5} = \dfrac{4}{5}$, $\cos C = \dfrac{AC}{BC} = \dfrac{0{,}9}{1{,}5} = \dfrac{3}{5}$, 
	$\tan C = \dfrac{AB}{AC} = \dfrac{1{,}2}{0{,}9} = \dfrac{4}{3}$, $\cot C = \dfrac{AC}{AB} = \dfrac{0{,}9}{1{,}2} = \dfrac{3}{4}$.
%	}{
%	\begin{tikzpicture}[line join = round, line cap = round, scale=3]
%	\coordinate[label=left:$A$] (A) at (0,0);
%	\coordinate[label=right:$B$] (B) at (1.2,0);
%	\coordinate[label=above:$C$] (C) at (0,0.9);
%	\draw (A)--(B)--(C)--cycle;
%	\draw (A)--(B)node[midway,sloped,below]{$1{,}2$ cm};
%	\draw (A)--(C)node[midway,sloped,above]{$0{,}9$ cm};
%	\pic[draw, angle radius=3mm]{right angle=B--A--C};
%	\pic[draw,double, angle radius=5mm]{angle=C--B--A};
%	\pic[draw, angle radius=5mm]{angle=A--C--B};
%	\end{tikzpicture}
%	}
	\end{listEX}}
\end{bt}
\begin{bt}
	\immini{
	Hình bên mô tả tia nắng mặt trời dọc theo $AB$ tạo với phương nằm ngang trên mặt đất một góc $\alpha=\widehat{ABH}$. Sử dụng máy tính cầm tay, tính số đo góc $\alpha$ (làm tròn kết quả đến hàng đơn vị của độ), biết $AH=2$m, $BH=5$m.
	}{
	\begin{tikzpicture}[font=\scriptsize]
	\coordinate (B) at (0,0);
	\coordinate (H) at (5,0);
	\coordinate (A) at (5,2);
	\draw(B)+(0:.5) arc (0:22:0.5) node[midway, shift={(10:.3)}]{$\alpha$};
	\draw(A)--(B)--(H) node[midway, below]{$5$m}--(A) node[midway, left]{$2$m};
	\foreach \p/\g in {A/90,B/-90,H/-90}\draw[fill=black] (\p) circle (1pt)node[shift={(\g:.3)},scale=1]{$\p$};
	\foreach \i/\j/\k/\t in {H/A/B/6}{
	\def\dgiua{\i}\def\dmot{\j}\def\dhai{\k}\def\tyso{\t pt}
	\draw ($(\dgiua)!\tyso!(\dmot)$)--($(\dgiua)!2!($($(\dgiua)!\tyso!(\dmot)$)!.5!($(\dgiua)!\tyso!(\dhai)$)$)$)--($(\dgiua)!\tyso!(\dhai)$);}
	\end{tikzpicture}
	}
	\loigiai{
	Xét tam giác vuông $ABH$, ta có:
	$\tan\alpha=\dfrac{AH}{BH}=\dfrac{2}{5}$.\\
	Do đó $\alpha=21{,}8^\circ.$ 
	}
\end{bt}
\begin{bt}
	\immini{Tia nắng chiếu qua nóc của một tòa nhà hợp với mặt đất một góc $\alpha$. Cho biết tòa nhà cao $21$ m và bóng của nó trên mặt đất dài $15$ m. Tính góc $\alpha$.(kết quả làm tròn đến độ).}{
	\begin{tikzpicture}[>=stealth, line join=round, line cap=round, font=\footnotesize, yscale=.25]
	\coordinate (O) at (0,0);
	\def\nhathap{5} %so tang nha 
	\def\kc{10} %khoang cach 2 nha
	\def\dai{1} %chieu dai 1 khoi
	\def\cao{2} %chieu cao 1 khoi
	\coordinate (H) at (\kc,0);
	\newcommand{\xaynha}[2]{
	\draw[very thick] (#1,#2)rectangle(#1+\dai,#2+\cao);
	\draw[very thick,fill=black] (#1,#2)rectangle(#1+\dai,#2+0.25*\cao);
	\draw[very thick] (#1+0.2*\dai,#2+0.25*\cao)rectangle(#1+0.8*\dai,#2+0.9*\cao);
	}
	% Xay nha
	\foreach \j in {1,...,\nhathap}{
	\foreach \i in {0,...,1}{
	\xaynha{\i*1.1}{\j*\cao}
	}
	}
	\draw[line width=0.1cm] (-0.1,\cao*\nhathap+\cao)--(2*\dai*1.1,\cao*\nhathap+\cao);
	\draw (0,2) coordinate (A)--(90:12) coordinate (B)--(-3,2)coordinate (C)--cycle
	;
	\foreach \x/\g in {A/-90,B/90,C/-90}\fill[red] (\x) circle (1pt)+(\g:7mm) node[black]{$ \x $};
	\path
	(A)--(C)%node[below,midway]{Mặt sàn}
	(A)--(C)node[above,midway]{$15$ m}
	(A)--(B)node[left,midway]{$21$ m}
	(B)++(30:1.25cm)%node{Sân thượng}
	;
	\path pic["$\alpha$", angle eccentricity=2,draw,angle radius=9pt]{angle= A--C--B};
	\end{tikzpicture}}
	\loigiai{
	Ta có $\tan\alpha=\dfrac{21}{15}$, sử dụng máy tính ta tính được $\alpha\approx54^\circ28'$.\\
	Vậy $\alpha\approx54^\circ28'$.
	}
\end{bt}
\begin{bt}
	\immini{Một cái thang dài $12$ m được đặt dựa vào một bức tường sao cho chân thang cách tường $7$ m. Tính góc $\alpha$ tạo bởi thang và tường.}{
	\begin{tikzpicture}[declare function={r=2.5;},>=stealth, line join=round,line cap=round,font=\footnotesize,scale=0.7]
	\path 
	(0,0) coordinate (A)
	(180:r) coordinate (B)
	(90:1.65*r) coordinate (C)
	(0:r*0.35) coordinate (D)
	($(C)+(D)-(A)$) coordinate (E) % Vẽ tường phía trên
	(110:r*2) coordinate (F)
	($(E)+(F)-(C)$) coordinate (G)
	($(C)!0.2!(F)$) coordinate (C1) % vẽ thang
	($(B)+(C1)-(C)$) coordinate (B1)
	;
	\draw (B)--(A) node[pos=0.6,below] {$7\ \mathrm{m}$}--(C);
	\draw[thick] (C)--(B) ;
	\draw[thin,pattern=grid] (C)--(F)--(G)--(E)--cycle;
	\draw[thin,pattern=bricks] (A)--(D)--(E)--(C)--cycle;
	\draw[thick] (C1)--(B1) node[midway,sloped, above] {$12\ \mathrm{m}$};
	\path pic["$\alpha$", angle eccentricity=1.5,draw,angle radius=15pt]{angle= B--C--A};
	\foreach \i/\t/\x in {0.1/T1/X1,0.2/T2/X2,0.3/T3/X3,0.4/T4/X4,0.5/T5/X5,0.6/T6/X6,0.7/T7/X7,0.8/T8/X8,0.9/T9/X9}{
	\path ($(B)!\i!(C)$) coordinate (\t);
	\path ($(B1)!\i!(C1)$) coordinate (\x);
	\draw (\t)--(\x);
	}
	\foreach \t/\g in {A/-90,B/-90,C/70}{
	\path (\t) node[shift={(\g:12pt)}]{$ \t $};
	}
	\end{tikzpicture}}
	\loigiai{
	Ta có $\sin\alpha=\dfrac{7}{12}$, sử dụng máy tính ta tính được $\alpha\approx35^\circ41'$.\\
	Vậy góc tạo bởi thang và tường gần bằng $35^\circ41'$.	
	}
\end{bt}
\begin{bt}
	Cho hình chữ nhật có chiều dài và chiều rộng lần lượt là $3$ và $\sqrt{3}$. Tính góc giữa đường chéo và cạnh ngắn hơn của hình chữ nhật (Sử dụng bảng giá trị lượng giác).
	\loigiai{
	\immini{
	Hình chữ nhật $ABCD$ có $AB=3$, $AD=\sqrt{3}$.\\
	Xét $\Delta ABD$ vuông tại $A$.\\
	Ta có $\tan \widehat{ADB} = \dfrac{AB}{AD} = \dfrac{3}{\sqrt{3}}=\sqrt{3}$.\\
	Khi đó $\widehat{ADB} = 60^\circ$.\\
	Vậy góc giữa đường chéo và cạnh ngắn hơn bằng $60^\circ$.
	}{
	\begin{tikzpicture}[line join = round, line cap = round, scale=1.2]
	\coordinate[label=left:$A$] (A) at (0,0);
	\coordinate[label=right:$B$] (B) at (3,0);
	\coordinate (M) at ($(B)!1!90: (A)$);
	\coordinate (N) at ($(A)!1!30: (B)$);
	\coordinate[label=right:$C$] (C) at (intersection of A--N and B--M);
	\coordinate[label=left:$D$] (D) at ($(A)+(C)-(B)$);
	\draw (A)--(B)--(C)--(D)--(A);
	\draw (B)--(D);
	\draw (A)--(B) node[midway,sloped,below]{$3$ cm};
	\draw (A)--(D)node[midway,sloped,below]{$\sqrt{3}$ cm};
	\end{tikzpicture}
	}}
\end{bt}
\begin{bt}
	Cho tam giác vuông có một góc nhọn $60^\circ$ và cạnh kề với góc $60^\circ$ bằng 3 cm. Hãy tính cạnh đối của góc này.
	\loigiai{
	\immini{
	Xét $\triangle ABC$ vuông tại $A$, có $\widehat{B}=60^\circ$, $AB=3$ cm.\\
	Khi đó $\tan B = \dfrac{AC}{AB}$.\\
	$AC = AB \cdot \tan B = 3 \cdot \tan 60^\circ = 3\sqrt{3}$ (cm).\\
	Vậy cạnh đối của góc $60^\circ$ bằng $3\sqrt{3}$ cm.
	}{
	\begin{tikzpicture}[scale=0.8]
	\path 
	(0,0) coordinate[label=left:$B$] (B)
	(5,0) coordinate[label=right:$C$] (C);
	\coordinate (D) at ($(B)!1!60: (C)$) ;
	\coordinate (E) at ($(C)!1!-30: (B)$) ;
	\coordinate[label=above:$A$] (A) at (intersection of B--D and C--E);
	\draw (A)--(B)--(C)--cycle;
	\draw (A)--(B)node[midway,sloped,above]{$3$ cm};
	\pic[draw, angle radius=2mm]{right angle=B--A--C};
	\pic[draw, angle radius=5mm]{angle=C--B--A} node[below] at ($(B) + (35:1.3)$){$60^\circ$};
	\end{tikzpicture}
	}}
\end{bt}
%%=====Bài 3
\begin{bt}
	Cho tam giác vuông có một góc nhọn $30^\circ$ và cạnh đối với góc này bằng 5 cm. Tính độ dài cạnh huyền của tam giác.
	\loigiai{
	\immini{
	Xét $\triangle ABC$ vuông tại $A$, có $\widehat{C}=30^\circ$, $AB=5$ cm.\\
	Khi đó $\sin C = \dfrac{AB}{BC}$.\\
	$BC = AB : \sin C = 5 \cdot \sin 30^\circ = 10$ (cm).\\
	Vậy cạnh huyền của tam giác bằng $10$ cm.
	}{
	\begin{tikzpicture}[scale=0.8]
	\path 
	(0,0) coordinate[label=left:$B$] (B)
	(5,0) coordinate[label=right:$C$] (C);
	\coordinate (D) at ($(B)!1!60: (C)$) ;
	\coordinate (E) at ($(C)!1!-30: (B)$) ;
	\coordinate[label=above:$A$] (A) at (intersection of B--D and C--E);
	\draw (A)--(B)--(C)--cycle;
	\draw (A)--(B)node[midway,sloped,above]{$5$ cm};
	\pic[draw, angle radius=2mm]{right angle=B--A--C};
	\pic[draw, angle radius=5mm]{angle=A--C--B} node[below] at ($(C) + (151:1.3)$){$30^\circ$};
	\end{tikzpicture}
	}}
\end{bt}
%%%%%%%%%% MTCT
\begin{bt}
	Sử dụng máy tính cầm tay, tính tỉ số lượng giác của các góc sau
	\begin{enumEX}{3}
	\item $26^\circ$;	
	\item $72^\circ$;
	\item $81^\circ27'$
	\end{enumEX}
	\loigiai{
	\begin{listEX}
	\item $\sin 26^\circ\approx0{,}44$;\quad $\cos26^\circ\approx 0{,}9$; \quad $\tan 26^\circ\approx0{,}49$;\quad $\cot 26^\circ\approx 2{,}05$.
	\item $\sin 72^\circ\approx0{,}95$;\quad $\cos72^\circ\approx 0{,}31$; \quad $\tan 72^\circ\approx3{,}08$;\quad $\cot 72^\circ\approx 0{,}32$.
	\item $\sin 81^\circ27'\approx0{,}99$;\quad $\cos81^\circ27'\approx 0{,}15$; \quad $\tan 81^\circ27'\approx6{,}65$;\quad $\cot 81^\circ27'\approx 0{,}15$.
	\end{listEX}
	}
\end{bt}
%%=====Bài 5
\begin{bt}
	Sử dụng máy tính cầm tay, tìm góc nhọn $\alpha$ trong mỗi trường hợp sau đây
	\begin{enumEX}{2}
	\item $\cos\alpha=0{,}6$;
	\item $\tan\alpha=\dfrac{3}{4}$.
	\end{enumEX}
	\loigiai{
	\begin{listEX}
	\item Ta có $\cos\alpha=0{,}6$ sử dụng máy tính ta tìm được $\alpha\approx53^\circ8'$;
	\item Ta có $\tan\alpha=\dfrac{3}{4}$ sử dụng máy tính ta tìm được $\alpha\approx36^\circ52'$;
	\end{listEX}
	}
\end{bt}

\begin{bt}
	Dùng MTCT, tính (làm tròn đến chữ số thập phân thứ ba):
	\begin{listEX}[4]
	\item $\sin 40^\circ 12'$;
	\item $\cos 52^\circ 54'$;
	\item $\tan 63^\circ 36'$;
	\item $\cot 25^\circ 18'$.
	\end{listEX}
	\loigiai{
	\begin{center}
	\begin{tabular}{|l|l|l|}
	\hline
	Để tính & Bấm phím & Kết quả \\ \hline
	$\sin 40^\circ 12'$ & \sink \key{4} \key{0} \key{x} \key{1} \key{2} \key{x} \key{=} & $0{,}6454576877$ \\ \hline
	$\cos \cos 52^\circ 54'$ & \cosk \key{5} \key{2} \key{x} \key{5} \key{4} \key{x} \key{=} & $0{,}6032079877$ \\ \hline
	$\tan 63^\circ 36'$ & \tank \key{6} \key{3} \key{x} \key{3} \key{6} \key{x} \key{=} & $2{,}014486937$ \\ \hline
	$\cot 25^\circ 18'$ & \tank \key{2} \key{5} \key{x} \key{1} \key{8} \key{x} \key{=} \key{u} \key{=} & $2{,}115516356$ \\ \hline
	\end{tabular}
	\end{center}
	Làm tròn đến chữ số thập phân thứ ba ta được:\\
	\begin{listEX}[4]
	\item $\sin 40^\circ 12' \approx 0{,}645$. 
	\item $\cos 52^\circ 54' \approx 0{,}603$.
	\item $\tan 69^\circ 36' \approx 2{,}014$.
	\item $\cot 25^\circ 18' \approx 2{,}116$.
	\end{listEX}
	}
\end{bt}
%===== Bài 7
\begin{bt}
	Dùng MTCT, tìm số đo của góc nhọn $x$ (làm tròn đến phút), biết rằng:
	\begin{listEX}[4]
	\item $\sin x = 0{,}2368$;
	\item $\cos x = 0{,}6224$;
	\item $\tan x = 1{,}236$;
	\item $\cot x = 2{,}154$.
	\end{listEX}
	\loigiai{
	\begin{center}
	\begin{tabular}{|l|l|l|l|}
	\hline
	Biết & Bấm phím & Kết quả & Bấm tiếp \key{x} \\ \hline
	$\sin x = 0{,}2368$ & \key{q} \sink \key{0} \key{.} \key{2} \key{3} \key{6} \key{8} \key{=} & $13{,}6977504$ & $13^\circ 41' 51{,}9"$ \\ \hline
	$\cos x = 0{,}6224$ & \key{q} \cosk \key{0} \key{.} \key{6} \key{2} \key{2} \key{4} \key{=} & $51{,}50839221$ &$51^\circ 30' 30{,}21"$ \\ \hline
	$\tan x = 1{,}236$ & \key{q} \tank \key{1} \key{.} \key{2} \key{3} \key{6} \key{=} & $51{,}02501186$ & $51^\circ 1' 30{,}04"$ \\ \hline
	$\cot x = 2{,}154$ & \key{q} \tank \key{2} \key{.} \key{1} \key{5} \key{4} \key{u} \key{=} & $24{,}90320574$ & $24^\circ 54' 11{,}54"$ \\ \hline
	\end{tabular}
	\end{center}
	Làm tròn đến phút ta được:
	\begin{enumEX}{4}
	\item $x \approx 13^\circ 42'$;
	\item $x \approx 51^\circ 31'$;
	\item $x \approx 51^\circ 2'$;
	\item $x \approx 24^\circ 54'$.
	\end{enumEX}
	}
\end{bt}
%%%%%%%%%%
\begin{bt}
	\begin{listEX}
	\item Viết các tỉ số lượng giác sau thành tỉ số lượng giác của các góc nhỏ hơn $45^\circ$.
	\begin{center} 
	$\sin 55^\circ$, $\cos 62^\circ$, $\tan 57^\circ$, $\cot 64^\circ$.
	\end{center}
	\item Tính $\dfrac{\tan25^\circ}{\cot 65^\circ}$, $\tan 34^\circ - \cot 56^\circ$.
	\end{listEX}
	\loigiai{
	\begin{listEX}
	\item $\sin 55^\circ = \cos (90^\circ - 55^\circ) = \cos 35^\circ$.\\
	$\cos 62^\circ = \sin (90^\circ - 62^\circ) = \sin 28^\circ$.\\
	$\tan 57^\circ = \cot (90^\circ - 57^\circ) = \cot 33^\circ$.\\
	$\cot 64^\circ = \tan (90^\circ - 64^\circ) = \tan 26^\circ$.
	\item $\dfrac{\tan 25^\circ}{\cot 65^\circ} = \dfrac{\cot (90^\circ - 25^\circ)}{\cot 65^\circ} = \dfrac{\cot 65^\circ}{\cot 65^\circ} = 1$.\\
	$\tan 34^\circ - \cot 56^\circ = \tan 34^\circ - \tan (90^\circ - 56^\circ) = \tan 34^\circ - \tan 34^\circ = 0$.
	\end{listEX}
	}
\end{bt}
\begin{bt}
	Hãy viết các tỉ số lượng giác sau thành tỉ số lượng giác của các góc nhỏ hơn $45^\circ$
	\begin{enumEX}{3}
	\item $\sin 60^\circ$;
	\item $\cos 75^\circ$;
	\item $\tan 80^\circ$.
	\end{enumEX}
	\loigiai{
	\begin{enumEX}{3}
	\item $\sin 60^\circ=\cos 30^\circ$;
	\item $\cos 75^\circ=\sin 15^\circ$;
	\item $\tan 80^\circ=\cot 10^\circ$.
	\end{enumEX}
	}
\end{bt}
\begin{bt}
	Mỗi tỉ số lượng giác sau đây bằng tỉ số lượng giác nào của góc $63^\circ$? Vì sao?
	\begin{enumEX}{4}
	\item $\sin27^\circ$;
	\item $\cos27^\circ$;
	\item $\tan27^\circ$;
	\item $\cot27^\circ$.
	\end{enumEX}
	\loigiai{
	Ta thấy $27^\circ+63^\circ$ nên theo định lí về hai góc phụ nhau ta có
	\begin{multicols}{2}
	\begin{listEX}
	\item $\sin27^\circ=\cos63^\circ$;
	\item $\cos27^\circ=\sin63^\circ$;
	\item $\tan27^\circ=\cot63^\circ$;
	\item $\cot27^\circ=\tan63^\circ$.
	\end{listEX}
	\end{multicols}
	}
\end{bt}
%%%%%%%%%% Tính gt biểu thức
\begin{bt}
	Sử dụng tỉ số lượng giác của hai góc phụ nhau, tính giá trị biểu thức 
	\begin{center}
	$A=\sin25^\circ+\cos25^\circ-\sin65^\circ-\cos65^\circ.$
	\end{center}
	\loigiai{
	Vì $25^\circ+65^\circ=90^\circ$ nên ta có
	\begin{align*}
	A&=\sin25^\circ+\cos25^\circ-\sin65^\circ-\cos65^\circ\cr
	&=\sin25^\circ+\cos25^\circ-\cos25^\circ-\sin25^\circ\cr
	&=0.
	\end{align*}
	}
\end{bt}
\begin{bt}
	Tính giá trị của các biểu thức sau
	\begin{enumEX}{2}
	\item $A=\dfrac{\sin 30^\circ\cdot\cos 30^\circ}{\cot 45^\circ}$;
	\item $B=\dfrac{\tan 30^\circ}{\cos 45^\circ\cdot\cos 60^\circ}$.
	\end{enumEX}
	\loigiai{
	\begin{listEX}
	\item $A=\dfrac{\sin 30^\circ\cdot\cos 30^\circ}{\cot 45^\circ}=\dfrac{\dfrac{1}{2}\cdot\dfrac{\sqrt{3}}{2}}{1}=\dfrac{\sqrt{3}}{4}$;
	\item $B=\dfrac{\tan 30^\circ}{\cos 45^\circ\cdot\cos 60^\circ}=\dfrac{\dfrac{\sqrt{3}}{3}}{\dfrac{\sqrt{2}}{2}\cdot\dfrac{1}{2}}=\dfrac{2\sqrt{6}}{3}$.
	\end{listEX}
	}
\end{bt}
\begin{bt}
	Cho góc nhọn $\alpha$. Biết rằng, tam giác $ABC$ vuông tại $A$ sao cho $\widehat{B}=\alpha$.
	\begin{listEX}
	\item Biểu diễn các tỉ số lượng giác của góc nhọn $\alpha$ theo $AB,BC,CA$.
	\item Chứng minh $\sin^2\alpha+\cos^2\alpha=1$; $\tan\alpha=\dfrac{\sin\alpha}{\cos\alpha}$; $\cot\alpha=\dfrac{\cos\alpha}{\sin\alpha}$; $\tan\alpha\cdot\cot\tan\alpha=1$. \\
	Từ đó tính giá trị biểu thức $S=\sin^235^\circ+\cos^235^\circ$; $T=\tan61^\circ\cdot\cot61^\circ$.
	\end{listEX}
	\loigiai{
	\immini{
	\begin{listEX}
	\item Ta có $\sin\alpha=\dfrac{AC}{BC}$, $\cos\alpha=\dfrac{AB}{BC}$, $\tan\alpha=\dfrac{AC}{AB}$, $\cot\alpha=\dfrac{AB}{AC}$.
	\item Ta có (sử dụng thêm định lí Pythagore)
	\begin{align*}
	\sin^2\alpha+\cos^2\alpha&=\dfrac{AC^2}{BC^2}+\dfrac{AB^2}{BC^2}=\dfrac{AC^2+AB^2}{BC^2}\cr
	&=\dfrac{BC^2}{BC^2}=1.
	\end{align*}
	\end{listEX}
	}{
	\begin{tikzpicture}[font=\scriptsize]
	\coordinate (B) at (0,0);
	\coordinate (A) at (4,0);
	\coordinate (C) at (4,3);
	\draw(B)+(0:.5) arc (0:37:0.5) node[midway, shift={(10:.3)}]{$\alpha$};
	\draw(A)--(B)--(C)--(A);
	\foreach \p/\g in {A/-90,B/-90,C/90}\draw[fill=black] (\p) circle (1pt)node[shift={(\g:.3)},scale=1]{$\p$};
	\foreach \i/\j/\k/\t in {A/C/B/6}{
	\def\dgiua{\i}\def\dmot{\j}\def\dhai{\k}\def\tyso{\t pt}
	\draw ($(\dgiua)!\tyso!(\dmot)$)--($(\dgiua)!2!($($(\dgiua)!\tyso!(\dmot)$)!.5!($(\dgiua)!\tyso!(\dhai)$)$)$)--($(\dgiua)!\tyso!(\dhai)$);}
	\end{tikzpicture}
	}
	Ta có
	\begin{align*}
	\dfrac{\sin\alpha}{\cos\alpha}&=\dfrac{AC}{BC}:\dfrac{AB}{BC}=\dfrac{AC}{BC}\cdot\dfrac{BC}{AB}=\dfrac{AC}{AB}=\tan\alpha.\cr
	\dfrac{\cos\alpha}{\sin\alpha}&=\dfrac{AB}{BC}:\dfrac{AC}{BC}=\dfrac{AB}{BC}\cdot\dfrac{BC}{AC}=\dfrac{AB}{AC}=\cot\alpha.\cr
	\tan\alpha\cdot \cot \alpha&=\dfrac{AC}{AB}\cdot\dfrac{AB}{AC}=1.\cr
	\end{align*}
	Áp dụng các kết quả trên, ta có ngay $S=1$ và $T=1$.
	}
\end{bt}