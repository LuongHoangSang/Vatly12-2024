\let\lesson\undefined
\newcommand{\lesson}{\phantomlesson{Bài 7: Phương trình trạng thái của khí lí tưởng}}
\chapter[Phương trình Clapeyron - Mendeleev]{Phương trình Clapeyron - Mendeleev}
\section{Lý thuyết}
\textbf{Phương trình Clapeyron - Mendeleev:}
$$pV=n RT=\dfrac{m}{M}RT$$
trong đó:
\begin{itemize}
	\item $p$: áp suất khí, đơn vị trong hệ SI là $\si{\pascal}$;
	\item $V$: thể tích khí, đơn vị trong hệ SI là $\si{\meter^3}$;
	\item $n$: số mole khí, đơn vị trong hệ SI là $\si{\mole}$;
	\item $m$: khối lượng khí, đơn vị trong hệ SI là $\si{\kilo\gram}$;
	\item $M$: khối lượng mole, đơn vị trong hệ SI là $\si{\kilo\gram/\mole}$;
	\item $R=\SI{8.31}{\joule/\left(\mole\cdot\kelvin\right)}$: hằng số khí lí tưởng;
	\item $T$: nhiệt độ tuyệt đối, đơn vị trong hệ SI là $\si{\kelvin}$.
	
\end{itemize}
\luuy{$$\SI{1}{atm}=\SI{101325}{\pascal}.$$
	$$\SI{1}{at}=\SI{9.81e4}{\pascal}.$$}
\section{Mục tiêu bài học - Ví dụ minh hoạ}
\begin{dang}{Xác định được mối liên hệ giữa các đại lượng $p$, $V$, $T$, $m$, $ M$}
	\viduii{2}
	{Một bình dung tích $\SI{10}{\liter}$ chứa $\SI{2}{\gram}$ khí hydrogen ở $\SI{27}{\celsius}$. Tính áp suất khí trong bình.
		
	
	
}
{\hide{Áp dụng phương trình Clapeyron - Mendeleev:
	$$pV=\dfrac{m}{M}RT$$
	$$\Rightarrow p=\dfrac{mRT}{M V}=\dfrac{\left(\SI{2}{\gram}\right)\cdot\left(\SI{8.31}{\joule/\mole\cdot\kelvin}\right)\cdot\left(\SI{300}{\kelvin}\right)}{\left(\SI{2}{\gram/\mole}\right)\cdot\left(\SI{10E-3}{\meter^3}\right)}=\SI{249300}{\pascal}.$$


}}
\viduii{3}
{Có $\SI{10}{\gram}$ khí oxygen ở $\SI{47}{\celsius}$, áp suất $\SI{2.1}{at}$. Sau khi đun nóng đẳng áp thì thể tích khí là $\SI{10}{\liter}$. Tìm
	\begin{enumerate}[label=\alph*)]
		\item thể tích khí trước khi đun.
		\item nhiệt độ khí sau khi đun.
		\item khối lượng riêng của khí trước và sau khi đun.
	\end{enumerate}

}
{\hide{\begin{center}
		\begin{tabular}{C{4cm} C{3cm} C{4cm}}
			\colorbox{yellow}{\textcolor{red}{\textbf{Trạng thái 1}}} & $\xrightarrow[ n=\text{const}]{p_1=p_2}$ & \colorbox{yellow}{\textcolor{red}{\textbf{Trạng thái 2}}}\\
			$p_1=\SI{2.1}{at}=\SI{2.06E5}{\pascal}$ & &$p_2=p_1$\\
			$V_1=?$ & & $V_2=\SI{10}{\liter}$\\
			$T_1=\SI{320}{\kelvin}$ & & $T_2=?$
		\end{tabular}
	\end{center}
	\begin{enumerate}[label=\alph*)]
		\item Áp dụng phương trình Clapeyron - Mendeleev:
		$$p_1V_1=\dfrac{m}{M}RT_1$$
		$$\Rightarrow V_1=\dfrac{mRT_1}{M p_1}=\dfrac{\left(\SI{10}{\gram}\right)\cdot\left(\SI{8.31}{\gram/\mole}\right)\cdot\left(\SI{320}{\kelvin}\right)}{\left(\SI{32}{\gram/\mole}\right)\cdot\left(\SI{2.06E5}{\pascal}\right)}\approx\SI{4E-3}{\meter^3}=\SI{4}{\liter}.$$
		\item Áp dụng định luật Charles:
		$$\dfrac{V_1}{T_1}=\dfrac{V_2}{T_2}$$
		$$\Rightarrow T_2=\dfrac{V_2T_1}{V_1}=\dfrac{\left(\SI{10}{\liter}\right)\cdot\left(\SI{320}{\kelvin}\right)}{\SI{4}{\liter}}=\SI{800}{\kelvin}\Rightarrow t_2=\SI{527}{\celsius}.$$
		\item Khối lượng riêng của khí trước khi đun:
		$$\rho_1=\dfrac{m}{V_1}=\dfrac{\SI{10}{\gram}}{\SI{4}{\liter}}=\SI{2.5}{\gram/\liter}.$$
		Khối lượng riêng của khí sau khi đun:
		$$\rho_2=\dfrac{m}{V_2}=\dfrac{\SI{10}{\gram}}{\SI{10}{\liter}}=\SI{1}{\gram/\liter}.$$
	\end{enumerate}

}}

\viduii{3}
{Hai bình cầu có thể tích $V_1=\SI{100}{\centi\meter^3}$, $V_2=\SI{200}{\centi\meter^3}$ được nối bằng một ống nhỏ cách nhiệt. Ban đầu hệ có nhiệt độ $t_0=\SI{27}{\celsius}$ và chứa khí oxygen ở áp suất $p_0=\SI{760}{\milli\meter Hg}$. Sau đó, bình $V_1$ giảm nhiệt độ xuống đến $\SI{0}{\celsius}$ còn bình $V_2$ tăng nhiệt độ lên đến $\SI{100}{\celsius}$. Tính áp suất khí trong các bình lúc sau.

}
{\hide{\begin{center}
		\begin{tabular}{|L{3cm}|C{3cm}|C{3cm}|C{3cm}|}
			\hline
			&$p$& $V$&$T$\\
			\hline
			\textbf{Ban đầu} & $p_0=\SI{760}{\milli\meter Hg}$ & $V_0=V_1+V_2=\SI{300}{\centi\meter^3}$ & $\SI{300}{\kelvin}$\\
			\hline
			\textbf{Bình 1 lúc sau} & $p=?$ & $V_1=\SI{100}{\centi\meter^3}$ & $T_1=\SI{273}{\kelvin}$\\
			\hline
			\textbf{Bình 2 lúc sau} & $p=?$ & $V_2=\SI{200}{\centi\meter^3}$ &  $T_2=\SI{373}{\kelvin}$\\
			\hline
		\end{tabular}
	\end{center}
	Từ phương trình Clapeyron - Mendeleev:
$$pV=n RT\Rightarrow n=\dfrac{pV}{RT}.$$
Mà: $n=n_1+n_2$
\begin{eqnarray*}
	&\Leftrightarrow& \dfrac{p_0V_0}{RT_0}=\dfrac{pV_1}{RT_1}+\dfrac{pV_2}{RT_2}\\
	&\Rightarrow& p=\dfrac{p_0V_0}{T_0\left(\dfrac{V_1}{T_1}+\dfrac{V_2}{T_2}\right)}=\dfrac{\left(\SI{760}{\milli\meter Hg}\right)\cdot\left(\SI{300}{\centi\meter^3}\right)}{\left(\SI{300}{\kelvin}\right)\cdot\left(\dfrac{\SI{100}{\centi\meter^3}}{\SI{273}{\kelvin}}+\dfrac{\SI{200}{\centi\meter^3}}{\SI{373}{\kelvin}}\right)}\approx\SI{842.11}{\milli\meter Hg}.
\end{eqnarray*}
}}

\end{dang}
\begin{dang}{Giải được bài toán thay đổi thông số trạng thái khi khối lượng khí thay đổi}
	\viduii{3}
	{Bình chứa khí nén ở nhiệt độ $\SI{27}{\celsius}$, $\SI{40}{at}$. Một nửa lượng khí trong bình thoát ra và nhiệt độ hạ xuống đến $\SI{12}{\celsius}$. Tìm áp suất của khí còn lại trong bình.
	
}
{\hide{\begin{center}
		\begin{tabular}{C{4cm} C{3cm} C{4cm}}
			\colorbox{yellow}{\textcolor{red}{\textbf{Trạng thái 1}}} & $\xrightarrow[ n_1\neq n_2]{V=\text{const}}$ & \colorbox{yellow}{\textcolor{red}{\textbf{Trạng thái 2}}}\\
			$p_1=\SI{40}{at}$ & &$p_2=?$\\
			$V_1=V$ & & $V_2=V$\\
			$T_1=\SI{300}{\kelvin}$ & & $T_2=\SI{285}{\kelvin}$\\
			$n_1=n$&&$n_2=\dfrac{n }{2}$
		\end{tabular}
	\end{center}
Áp dụng phương trình Clapeyron - Mendeleev:
\begin{equation}
	\label{eq:13P-1}
	p_1V=n RT_1
\end{equation}
\begin{equation}
	\label{eq:13P-2}
	p_2V=\dfrac{n}{2} RT_2
\end{equation}
Từ phương trình (\ref{eq:13P-1}) và (\ref{eq:13P-2}):
$$\dfrac{p_2}{p_1}=\dfrac{1}{2}\cdot\dfrac{T_2}{T_1}\Rightarrow p_2=\dfrac{T_2p_1}{2T_1}=\dfrac{\left(\SI{285}{\kelvin}\right)\cdot\left(\SI{40}{at}\right)}{2\cdot\left(\SI{300}{\kelvin}\right)}=\SI{19}{at}.$$
}}
	
\viduii{3}
{Một bình kín có thể tích $\SI{0.4}{\meter^3}$, chứa khí ở nhiệt độ $\SI{27}{\celsius}$ và áp suất $\SI{1.5}{atm}$. Khi mở nắp, áp suất khí còn $\SI{1}{atm}$ và nhiệt độ $\SI{0}{\celsius}$.
	\begin{enumerate}[label=\alph*)]
		\item Tìm thể tích khí thoát ra khỏi bình ở nhiệt độ $\SI{0}{\celsius}$ và áp suất $\SI{1}{atm}$.
		\item Tìm khối lượng khí còn lại trong bình và khối lượng khí thoát ra khỏi bình, biết khối lượng riêng của khí ở điều kiện tiêu chuẩn là $\rho_0=\SI{1.2}{\kilogram/\meter^3}$.
	\end{enumerate}

}
{\hide{\begin{enumerate}[label=\alph*)]
		\item Quá trình biến đổi trạng thái của khí trong bình:
		\begin{center}
			\begin{tabular}{C{4cm} C{3cm} C{4cm}}
				\colorbox{yellow}{\textcolor{red}{\textbf{Trạng thái 1}}} & $\xrightarrow[ ]{ n=\text{const}}$ & \colorbox{yellow}{\textcolor{red}{\textbf{Trạng thái 2}}}\\
				$p_1=\SI{1.5}{atm}$ & &$p_2=\SI{1}{atm}$\\
				$V_1=\SI{0.4}{\meter^3}$ & & $V_2=?$\\
				$T_1=\SI{300}{\kelvin}$ & & $T_2=\SI{273}{\kelvin}$
			\end{tabular}
		\end{center}
	Áp dụng phương trình trạng thái khí lí tưởng:
	$$\dfrac{p_1V_1}{T_1}=\dfrac{p_2V_2}{T_2}$$
	$$\Rightarrow V_2=\dfrac{p_1V_1}{T_1}\cdot\dfrac{T_2}{p_2}=\dfrac{\left(\SI{1.5}{atm}\right)\cdot\left(\SI{0.4}{\meter^3}\right)}{\SI{300}{\kelvin}}\cdot\dfrac{\left(\SI{273}{\kelvin}\right)}{\SI{1}{atm}}=\SI{0.546}{\meter^3}.$$
	Thể tích khí thoát ra khỏi bình ở nhiệt độ $\SI{0}{\celsius}$ và áp suất $\SI{1}{atm}$:
	$$\Delta V=V_2-V_1=\SI{0.146}{\meter^3}.$$
	\item Khối lượng khí còn lại trong bình:
	$$m=\rho_0V_1=\left(\SI{1.2}{\kilogram/\meter^3}\right)\cdot\left(\SI{0.4}{\meter^3}\right)=\SI{0.48}{\kilogram}.$$
	Khối lượng khí thoát ra khỏi bình:
	$$\Delta m=\rho_0\Delta V=\left(\SI{1.2}{\kilogram/\meter^3}\right)\cdot\left(\SI{0.146}{\meter^3}\right)=\SI{0.1752}{\kilogram}.$$
	\end{enumerate}

}}
\end{dang}