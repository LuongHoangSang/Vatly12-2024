\section*{ÔN TẬP CHƯƠNG I}
\subsection{Câu trắc nghiệm nhiều phương án lựa chọn.}
\textit{Thí sinh trả lời từ câu 1 đến câu 18. Mỗi câu hỏi thí sinh chọn một phương án.}
\Opensolutionfile{ans}[ans/G12Y24B7TN]
% ===================================================================
\begin{ex}
	Quy ước dấu nào sau đây phù hợp với định luật I của nhiệt động lực học?
	\choice
	{Vật nhận công $A<0$; vật nhận nhiệt $Q<0$.}
	{Vật thực hiện công $A>0$; vật truyền nhiệt lượng $Q<0$.}
	{\True Vật nhận công $A>0$; vật nhận nhiệt lượng $Q>0$.}
	{Vật thực hiện công $A>0$; vật truyền nhiệt lượng $Q>0$.}
	\loigiai{}
\end{ex}
% ===================================================================
\begin{ex}
	Ở nhiệt độ phòng, chất nào sau đây không tồn tại ở thể lỏng?
	\choice
	{Rượu.}
	{\True Nhôm.}
	{Thuỷ ngân.}
	{Nước.}
	\loigiai{}
\end{ex}
% ===================================================================
\begin{ex}
	Vật nào sau đây có cấu trúc tinh thể?
	\choice
	{\True Chiếc cốc thuỷ tinh.}
	{Hạt muối ăn.}
	{Viên kim cương.}
	{Miếng thạch anh.}
	\loigiai{}
\end{ex}
% ===================================================================
\begin{ex}
Điều nào sau đây là \textbf{sai} khi nói về sự đông đặc?	
	\choice
	{Sự đông đặc là quá trình chuyển từ thể lỏng sang thể rắn.}
	{\True Với một chất rắn, nhiệt độ đông đặc luôn nhỏ hơn nhiệt độ nóng chảy.}
	{Trong suốt quá trình đông đặc, nhiệt độ của vật không thay đổi.}
	{Nhiệt độ đông đặc của các chất thay đổi theo áp suất bên ngoài.}
	\loigiai{}
\end{ex}
% ===================================================================
\begin{ex}
	Biểu thức nào sau đây là biểu thức chuyển đổi đúng đơn vị nhiệt độ từ $\si{\celsius}$ sang thang $\si{\kelvin}$?
	\choice
	{\True $\xsi{T}{\left(\si{\kelvin}\right)}=\xsi{t}{\left(\si{\celsius}\right)}+273$.}
	{$\xsi{T}{\left(\si{\kelvin}\right)}=\xsi{t}{\left(\si{\celsius}\right)}-273$.}
	{$\xsi{T}{\left(\si{\kelvin}\right)}=\dfrac{9}{5}\xsi{t}{\left(\si{\celsius}\right)}+273$.}
	{$\xsi{T}{\left(\si{\kelvin}\right)}=\dfrac{9}{5}\xsi{t}{\left(\si{\celsius}\right)}-273$.}
	\loigiai{}
\end{ex}
% ===================================================================
\begin{ex}
	Trong thang nhiệt độ Celsius, nhiệt độ không tuyệt đối là
	\choice
	{$\SI{100}{\celsius}$.}
	{\True $\SI{-273}{\celsius}$.}
	{$\SI{0}{\celsius}$.}
	{$\SI{-32}{\celsius}$.}
	\loigiai{}
\end{ex}
% ===================================================================
\begin{ex}
	Đơn vị nhiệt nóng chảy riêng của vật rắn là
	\choice
	{$\si{\joule}$.}
	{$\si{\joule/\kelvin}$.}
	{\True $\si{\joule/\kilogram}$.}
	{$\si{\joule/\left(\kilogram\cdot\kelvin\right)}$.}
	\loigiai{}
\end{ex}

% ===================================================================
\begin{ex}
	Kết luận nào sau đây \textbf{không đúng} với thang nhiệt độ Celsius?
	\choice
	{Đơn vị đo nhiệt độ là $\si{\celsius}$.}
	{Chọn mốc nhiệt độ nước đá đang tan ở áp suất $\SI{1}{atm}$ là $\SI{0}{\celsius}$.}
	{Chọn mốc nhiệt độ nước sôi ở áp suất $\SI{1}{atm}$ là $\SI{100}{\celsius}$.}
	{\True $\SI{1}{\celsius}$ tương ứng với $\SI{273}{\kelvin}$.}
	\loigiai{}
\end{ex}
\Closesolutionfile{ans}