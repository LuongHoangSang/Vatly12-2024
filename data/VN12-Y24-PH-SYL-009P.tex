\let\lesson\undefined
\newcommand{\lesson}{\phantomlesson{Bài 5.}}
\section{Bài tập trắc nghiệm}
\begin{enumerate}[label=\bfseries Câu \arabic*:, leftmargin=1.7cm]
	\item \mkstar{1}\\
	Tính chất nào sau đây \textbf{không phải} là tính chất của chất khí?
	\begin{mcq}
		\item Có hình dạng và thể tích riêng.
		\item Có các phân tử chuyển động hỗn loạn không ngừng.
		\item Có thể nén được dễ dàng.
		\item Có khối lượng riêng nhỏ hơn so với chất rắn và chất lỏng.
	\end{mcq}
	\hideall{
		\textbf{Đáp án A.}
	}
	
	\item \mkstar{1}\\
	Chọn phương án \textbf{sai}. Số Avogadro là
	\begin{mcq}
		\item số phân tử (hay nguyên tử) có trong $\SI{22.4}{\text{lít}}$ khí ở điều kiện tiêu chuẩn $\left(\SI{0}{\celsius}, \SI{1}{atm}\right)$.
		\item số phân tử (hay nguyên tử) có trong 1 mol chất.
		\item số phân tử (hay nguyên tử) có trong 1 đơn vị khối lượng chất.
		\item số nguyên tử có trong $\SI{12}{\gram}$ $\ce{^{12}C}$.
	\end{mcq}
	\hideall{
		\textbf{Đáp án C.}
	}
	
	\item \mkstar{1}\\
	Gọi $\xsi{M}{(\gram/\mole)}$ là khối lượng mol nguyên tử, $N_\text{A}$ là số Avogadro. Biểu thức xác định số phân tử hay nguyên tử chứa trong $\xsi{m}{(\gram)}$ của chất đó là
	\begin{mcq}(4)
		\item $N=MmN_\text{A}$.
		\item $N=\dfrac{MN_\text{A}}{m}$.
		\item $N=\dfrac{mN_\text{A}}{M}$.
		\item $N=\dfrac{N_\text{A}}{mM}$.
	\end{mcq}
	\hideall{
		\textbf{Đáp án C.}
	}
	
	\item \mkstar{1}\\
	Điền vào chỗ trống.\\
	Chất khí trong đó các phân tử được coi là \dots và chỉ tương tác khi \dots được gọi là khí lí tưởng.
	\begin{mcq}(2)
		\item chất điểm; va chạm.
		\item vật rắn; va chạm.
		\item chất điểm; ở gần nhau.
		\item vật rắn; ở gần nhau.
	\end{mcq}
	\hideall{
		\textbf{Đáp án A.}
	}
	
	\item \mkstar{1}\\
	Nhận xét nào sau đây về các phân tử khí lí tưởng là \textbf{không đúng}?
	\begin{mcq}
		\item Có thể tích riêng không đáng kể.
		\item Có lực tương tác không đáng kể khi không va chạm.
		\item Có khối lượng không đáng kể.
		\item Có vận tốc càng lớn khi nhiệt độ phân tử càng cao.
	\end{mcq}
	\hideall{
		\textbf{Đáp án C.}
	}

\item \mkstar{2}\\
Chọn câu \textbf{sai}. Số Avogadro có giá trị bằng
\begin{mcq}
	\item số nguyên tử chứa trong $\SI{4}{\gram}$ helium.
	\item số phân tử chứa trong $\SI{16}{\gram}$ oxygen.
	\item số phân tử chứa trong $\SI{18}{\gram}$ nước lỏng.
	\item số nguyên tử chứa trong $\SI{22.4}{\liter}$ khí trơ ở $\SI{0}{\celsius}$ và áp suất $\SI{1}{atm}$.
\end{mcq}
\hideall{
	\textbf{Đáp án B.}
}

\item \mkstar{2}\\
Một bình kín chứa $N=\SI{3.01E23}{}$ phân tử khí helium. Khối lượng helium chứa trong bình là
\begin{mcq}(4)
	\item $\SI{0.5}{\gram}$.
	\item $\SI{1}{\gram}$.
	\item $\SI{2}{\gram}$.
	\item $\SI{4}{\gram}$.
\end{mcq}
\hideall{
	\textbf{Đáp án C.}
}

\item \mkstar{2}\\
Cho biết khối lượng riêng của không khí ở điều kiện tiêu chuẩn là $\SI{1.29}{\kilogram/\meter^3}$. Coi không khí như một chất khí thuần nhất, khối lượng mol của không khí là
\begin{mcq}(4)
	\item $\SI{0.041}{\kilogram/\mole}$.
	\item $\SI{0.029}{\kilogram/\mole}$.
	\item $\SI{0.023}{\kilogram/\mole}$.
	\item $\SI{0.026}{\kilogram/\mole}$.
\end{mcq}
\hideall{
	\textbf{Đáp án B.}\\
	$$\rho=\dfrac{m}{V}=\dfrac{m}{n\cdot\left(\SI{22.4E-3}{\meter^3}\right)}\Rightarrow M=\dfrac{m}{n}=\SI{0.029}{\kilogram/\mole}.$$
}
\end{enumerate}

\section{Trắc nghiệm đúng/sai}
\begin{enumerate}[label=\bfseries Câu \arabic*:, leftmargin=1.7cm]
	\item \mkstar{1}\\
	Nhận định các phát biểu sau về đặc điểm của chuyển động Brown.
	\begin{enumerate}[label=\alph*)]
		\item Chuyển động Brown tuân theo một số quy luật nhất định.
		\item Quỹ đạo của chuyển động Brown là những đường gấp khúc.
		\item Chuyển động Brown là chuyển động của các hạt nhẹ trong chất rắn, lỏng, khí.
		\item Nhiệt độ càng cao thì các phân tử khí chuyển động càng hỗn loạn.
	\end{enumerate}
\hideall{
\begin{enumerate}[label=\alph*)]
	\item Sai. Chuyển động Brown không theo quy luật.
	\item Đúng.
	\item Sai. Chuyển động Brown là chuyển động của các hạt nhẹ trong chất lỏng và chất khí.
	\item Đúng.
\end{enumerate}
}
	
	\item \mkstar{1}\\
	Nhận định các phát biểu sau về nội dung thuyết động học phân tử chất khí.
	\begin{enumerate}[label=\alph*)]
		\item Chất khí gồm tập hợp nhiều các phân tử dao động nhiệt quanh các vị trí cân bằng của chúng.
		\item Nhiệt độ càng cao thì các phân tử khí chuyển động nhiệt càng nhanh.
		\item Kích thước các phân tử rất nhỏ so với khoảng cách trung bình giữa chúng.
		\item Áp suất khí được tạo ra bởi sự va chạm giữa các phân tử khí với nhau.
	\end{enumerate}
\hideall{
\begin{enumerate}[label=\alph*)]
	\item Sai. Các phân tử khí chuyển động hỗn loạn, không ngừng.
	\item Đúng.
	\item Đúng.
	\item Sai. Sự va chạm của các phân tử khí với thành bình gây ra áp suất lên thành bình.
\end{enumerate}
}

\item \mkstar{1}\\
Nhận định các phát biểu sau đây về đặc điểm của khí lí tưởng
\begin{enumerate}[label=\alph*)]
	\item Các phân tử khí được coi là chất điểm nên người ta có thể bỏ qua khối lượng các phân tử khí.
	\item Thể tích khối khí bằng thể tích bình chứa trừ đi thể tích riêng của các phân tử.
	\item Các phân tử khí chỉ tương tác với nhau khi va chạm.
	\item Nội năng của khối khí bằng tổng động năng chuyển động nhiệt của các phân tử khí và chỉ phụ thuộc vào nhiệt độ.
\end{enumerate}
\hideall{
\begin{enumerate}[label=\alph*)]
	\item Sai. Các phân tử khí được coi là chất điểm nên người ta có thể bỏ qua kích thước các phân tử khí.
	\item Sai. Thể tích khối khí bằng thể tích bình chứa.
	\item Đúng.
	\item Đúng.
\end{enumerate}
}

\item \mkstar{2}\\
Một người xịt nước hoa ở đầu phòng thì người ở cuối phòng vẫn nghe được mùi hương của nước hoa.
\begin{enumerate}[label=\alph*)]
	\item Hiện tượng trên được gọi là sự khuếch tán.
	\item Các phân tử nước hoa chuyển động thành dòng từ đầu phòng sang cuối phòng chỉ nhờ vào đối lưu.
	\item Hiện tượng trên chứng tỏ nhiệt độ ở đầu phòng cao hơn nhiệt độ ở cuối phòng.
	\item Nhiệt độ trong phòng càng cao thì người ở cuối phòng càng sớm nhận ra mùi nước hoa.
\end{enumerate}
\hideall{
\begin{enumerate}[label=\alph*)]
	\item Đúng.
	\item Sai. Sự dao động nhiệt của các phân tử khí và phân tử nước hoa làm cho chúng khuếch tán vào nhau.
	\item Sai. Các phân tử khí chuyển động hỗn loạn, không ngừng về mọi phía nên không thể khẳng định được nhiệt độ nơi nào cao hơn.
	\item Đúng. Nhiệt độ càng cao, các phân tử chuyển động càng nhanh làm tốc độ khuếch tán diễn ra càng nhanh.
\end{enumerate}
}



\end{enumerate}
\section{Bài tập tự luận}
\begin{enumerate}[label=\bfseries Câu \arabic*:, leftmargin=1.7cm]
	
	\item\mkstar{2}\\
	Mùi hôi từ các bãi rác thải là một vấn nạn đối với cư dân sống xung quanh. Khi thời tiết càng nắng nóng thì mùi hôi bốc ra càng nồng nặc và càng bay xa (ngay cả trong điều kiện không có gió). Dựa vào thuyết động học phân tử chất khí, hãy giải thích điều này và đề xuất biện pháp hạn chế tình trạng trên.
	\hideall{
Thời tiết càng nóng thì quá trình phân huỷ các chất hữu cơ diễn ra càng nhanh và sinh ra nhiều khí có mùi hôi như: $\ce{H_2S}, \ce{NH_3}, \ce{CH_4}, \ce{SO_2},\dots$. Bên cạnh đó, nhiệt độ càng cao thì các phân tử khí chuyển động nhiệt càng nhanh, do đó các phân tử khí này càng dễ khuếch tán vào không khí và bay đi xa hơn.\\
Biện pháp hạn chế: Thường xuyên thu gom, xử lý rác thải, nâng cao ý thức cộng đồng.
}

\item \mkstar{2}\\
Đun một nồi nước trên bếp, khi nước sôi nắp nồi thường bị đẩy lên. Hãy giải thích điều này.
\hideall{
Khi đun nước đến nhiệt độ sôi, nước sẽ bay hơi tạo thành hơi nước. Bên cạnh đó, nhiệt độ càng cao làm cho các phân tử khí và hơi nước trong nồi chuyển động càng nhanh. Sự gia tăng mật độ khí và tốc độ chuyển động nhiệt của các phân tử khí và hơi nước trong nồi làm gia tăng số lượt va chạm của các phân tử khí lên nắp nồi $\rightarrow$ tăng áp suất khí tác dụng lên nắp và làm nắp bị đẩy lên.
}

\item \mkstar{2}\\
Xác định số phân tử chứa trong
\begin{enumerate}[label=\alph*)]
	\item $\SI{0.2}{\kilogram}$ nước.
	\item $\SI{1}{\kilogram}$ không khí nếu như không khí có $\SI{22}{\percent}$ là khí $\ce{O_2}$ và $\SI{78}{\percent}$ là khí $\ce{N_2}$.
\end{enumerate}
\hideall{
\begin{enumerate}[label=\alph*)]
	\item $N=\dfrac{m}{M_{\ce{H_2O}}}\cdot N_\text{A}\approx\SI{6.68E24}{\text{phân tử}}.$
	\item $N=\SI{22}{\percent}\cdot\dfrac{m}{M_{\ce{O_2}}}\cdot N_\text{A}+\SI{78}{\percent}\cdot\dfrac{m}{M_{\ce{N_2}}}\cdot N_\text{A}\approx\SI{2.1E25}{\text{phân tử}}.$
\end{enumerate}
}

\item \mkstar{3}\\
Coi Trái Đất là một khối cầu bán kính $\SI{6400}{\kilo\meter}$, nếu lấy toàn bộ số phân tử nước trong $\SI{1.0}{\gram}$ hơi nước trải đều trên bề mặt Trái Đất thì mỗi mét vuông trên bề mặt Trái Đất có bao nhiêu phân tử nước? Biết khối lượng mol của phân tử nước khoảng $\SI{18}{\gram/\mole}$.
\hideall{
	$$\eta=\dfrac{N}{S}=\dfrac{\dfrac{m}{M}N_\text{A}}{4\pi R^2}=\SI{64.98E6}{\text{phân tử}/\meter^2}.$$
}
\end{enumerate}
