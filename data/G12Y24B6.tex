\section{BÀI TẬP VẬT LÝ NHIỆT}
\subsection{ÔN TẬP LÝ THUYẾT}
\subsubsection{Khi nội năng của vật biến đổi chỉ bằng cách truyền nhiệt}
\begin{itemize}
	\item Nếu quá trình truyền nhiệt chỉ làm thay đổi nhiệt độ của vật, không làm vật chuyển thể thì:
	$$\Delta U=Q; \quad Q=mc\Delta t\quad \text{và}\quad Q_\text{thu}=Q_\text{toả}.$$
	\item Nếu quá trình truyền nhiệt làm vật chuyển từ thể này sang thể khác ở nhiệt độ không đổi thì:
	$$\Delta U=Q; \quad Q=\lambda m; Q=Lm \quad \text{và}\quad Q_\text{thu}=Q_\text{toả}.$$
\end{itemize}
\subsubsection{Khi nội năng của vật biến đổi bằng cả hai cách truyền nhiệt và thực hiện công}
$$\Delta U=Q+A$$
Các công thức tính công cơ học:
\begin{itemize}
	\item Công ngoại lực: $A=Fs\cos\alpha$;
	\item Công ngoại lực: $A=W_\text{đ2}-W_\text{đ1}$;
	\item Công ngoại lực:
	$A=\calP t$;
	\item Công lực thế: $A=W_\text{t1}-W_\text{t2}$.
\end{itemize}
\subsection{VÍ DỤ MINH HOẠ}
\begin{dang}{Vận dụng định luật I nhiệt động lực học}
\end{dang}
\begin{vd}
Để làm nguội một sản phẩm bằng đồng có khối lượng $\SI{50}{\gram}$ đã bị nung nóng, người thợ thủ công thả nó vào một bình nhiệt lượng kế đựng $\SI{400}{\gram}$ nước ở $\SI{20}{\celsius}$ từ độ cao cách mặt nước trong bình $\SI{5}{\centi\meter}$. Khi sản phẩm đã nằm yên ở đáy bình và trong bình có cân bằng nhiệt thì nhiệt độ của nước là $\SI{22}{\celsius}$.
		\begin{enumerate}[label=\alph*)]
			\item Dựa vào cơ sở nào để biết nội năng của nước đã biến đổi? Tính độ biến thiên nội năng này.
			\item Nội năng của nước được biến đổi bằng những cách nào? Tính độ biến thiên nội năng của nước gây ra bởi mỗi cách.
			\item Sản phẩm được nung nóng tới nhiệt độ bao nhiêu?
		\end{enumerate}
		Biết độ cao của nước trong bình nhiệt lượng kế là $\SI{20}{\centi\meter}$, nhiệt dung riêng của nước là $\SI{4190}{\joule/\left(\kilogram\cdot\kelvin\right)}$, của đồng là $\SI{380}{\joule/\left(\kilogram\cdot\kelvin\right)}$; lấy $g=\SI{9.81}{\meter/\second^2}$. Bỏ qua sự truyền năng lượng của nước, sản phẩm cho bình nhiệt lượng kế và môi trường xung quanh. Coi lượng nước trong bình không đổi và lực đẩy Archimedes là không đáng kể.
\loigiai{
			Đây là bài toán về sự biến đổi nội năng bằng cả hai cách: nhận công và truyền nhiệt.\\
			\textbf{	Xác định trạng thái ban đầu và cuối của hệ vật:}\\
			Vì bỏ qua mọi sự truyền năng lượng cho nhiệt lượng kế và môi trường xung quanh nên hệ trao đổi năng lượng chỉ có nước trong nhiệt lượng kế và sản phẩm.
			\begin{itemize}
				\item Trạng thái đầu: Nước ở nhiệt độ $t_1=\SI{20}{\celsius}$; sản phẩm ở độ cao $h_1=\SI{0.25}{\meter}$ so với đáy bình và nhiệt độ $\xsi{t}{\celsius}$.
				\item Trạng thái cuối: Nước ở nhiệt độ $t_2=\SI{22}{\celsius}$; sản phẩm ở độ cao $h_2=\SI{0}{\meter}$ và nhiệt độ $\SI{22}{\celsius}$.
			\end{itemize}
			\begin{enumerate}[label=\alph*)]
				\item Dựa vào sự tăng nhiệt độ của nước mà biết nội năng của nước tăng.\\
				Do nước muốn tăng nhiệt độ $\Delta t_n=t_2-t_1$ thì phải nhận thêm nhiệt lượng $Q_n=m_nc_n\Delta t_n=\left(\SI{0.4}{\kilogram}\right)\cdot\left[\SI{4190}{\joule/\left(\kilogram\cdot\kelvin\right)}\right]\cdot\left(\SI{22}{\celsius}-\SI{20}{\celsius}\right)=\SI{3352}{\joule}$, nên độ tăng nội năng của nước là:
				$$\Delta U=Q_n=\SI{3352}{\joule}.$$
				\item Nội năng của nước tăng lên bằng hai cách là nhận công của sản phẩm khi rơi và nhận nhiệt của sản phẩm truyền cho.\\
				Sản phẩm rơi vào nước bị nước cản nên phải thực hiện công để chống lại sức cản của nước. Do bỏ qua tác dụng của lực đẩy Archimedes, nên công do sản phẩm thực hiện bằng công trọng lực của sản phẩm:
				$$A=W_\text{t1}-W_\text{t2}=mg\left(h_1-h_2\right)=\left(\SI{0.05}{\kilogram}\right)\cdot\left(\SI{9.81}{\meter/\second^2}\right)\cdot\left(\SI{0.25}{\meter}-\SI{0}{\meter}\right)\approx\SI{0.12}{\joule}.$$
				Độ tăng nội năng của nước do nhận được công là $\SI{0.12}{\joule}$.\\
				Sản phẩm bị nung nóng nên khi rơi vào nước nó truyền nhiệt cho nước. Theo định luật I nhiệt động lực học, ta tính được nhiệt lượng do sản phẩm toả ra:
				$$Q=\Delta U-A=\SI{3352}{\joule}-\SI{0.12}{\joule}=\SI{3351.88}{\joule}.$$
				Độ tăng nội năng của nước do truyền nhiệt là $\SI{3351.88}{\joule}.$
				\item Độ biến thiên nhiệt độ của sản phẩm:
				$$Q_s=m_sc_s\Delta t_s\Rightarrow \Delta t_s=\dfrac{Q_s}{m_sc_s}$$
				Vì nhiệt lượng do nước nhận được bằng nhiệt lượng do sản phẩm toả ra nên $Q_s=-Q=-\SI{3351.88}{\joule}.$
				\begin{eqnarray*}
					&	\Rightarrow& t_s-t_0=\dfrac{-\SI{3351.88}{\joule}}{\left(\SI{0.05}{\kilogram}\right)\cdot\left(\SI{380}{\joule/\kilogram\cdot\kelvin}\right)}\approx-\SI{176.41}{\celsius}
					\\
					&\Rightarrow& t_0=t_s+\SI{176.41}{\celsius}=\SI{198.41}{\celsius}.
				\end{eqnarray*}
		\end{enumerate}}
\end{vd}
\begin{dang}{Bài toán về hiệu suất truyền nhiệt}
\end{dang}
\begin{vd}
	Dùng bếp điện để đun một ấm nhôm khối lượng $\SI{600}{\gram}$ đựng 1,5 lít nước ở nhiệt độ $\SI{20}{\celsius}$. Sau 35 phút đã có $\SI{20}{\percent}$ lượng nước trong ấm hoá hơi ở nhiệt độ $\SI{100}{\celsius}$. Tính nhiệt lượng trung bình mà bếp điện cung cấp cho ấm nước trong mỗi giây, biết chỉ có $\SI{75}{\percent}$ nhiệt lượng mà bếp toả ra được dùng vào việc đun ấm nước. Biết nhiệt dung riêng của nhôm là $\SI{880}{\joule/\left(\kilogram\cdot\kelvin\right)}$, của nước là $\SI{4190}{\joule/\left(\kilogram\cdot\kelvin\right)}$; nhiệt hoá hơi riêng của nước ở nhiệt độ sôi $\SI{100}{\celsius}$ là $\SI{2.26E6}{\joule/\kilogram}$, khối lượng riêng của nước là $\SI{1}{\kilogram/\text{lít}}$.
	\loigiai{Gọi:
			\begin{itemize}
				\item $m_1$, $c_1$ lần lượt là khối lượng ấm nhôm và nhiệt dung riêng của nhôm;
				\item $m_2$, $c_2$ lần lượt là khối lượng của nước và nhiệt dung riêng của nước.
			\end{itemize}
			Khối lượng nước trong ấm:
			$$m_2=V_2D_2=\left(\SI{1.5}{\text{lít}}\right)\cdot\left(\SI{1.5}{\kilogram/\text{lít}}\right)=\SI{1.5}{\kilogram}.$$
			Nhiệt lượng ấm nhôm thu vào để tăng nhiệt độ từ $\SI{20}{\celsius}$ lên $\SI{100}{\celsius}$:
			$$Q_1=m_1c_1\Delta t=\left(\SI{0.6}{\kilogram}\right)\cdot\left[\SI{880}{\joule/\left(\kilogram\cdot\kelvin\right)}\right]\cdot\left(\SI{100}{\celsius}-\SI{20}{\celsius}\right)=\SI{42240}{\joule}.$$
			Nhiệt lượng nước thu vào để tăng nhiệt độ từ $\SI{20}{\celsius}$ lên $\SI{100}{\celsius}$ và $\SI{20}{\percent}$ lượng nước hoá thành hơi:
			$$Q_2=m_2c_2\Delta t+\SI{20}{\percent}m_2L=\left(\SI{1.5}{\kilogram}\right)\cdot\left[\SI{4190}{\joule/\left(\kilogram\cdot\kelvin\right)}\right]\cdot\left(\SI{80}{\celsius}\right)+\SI{20}{\percent}\cdot\left(\SI{1.5}{\kilogram}\right)\cdot\left(\SI{2.26E6}{\joule/\kilogram}\right)=\SI{1180.8}{\kilo\joule}.$$
			Nhiệt lượng do bếp điện cung cấp:
			$$Q=\dfrac{Q_1+Q_2}{H}=\SI{1630.72}{\kilo\joule}.$$
			Nhiệt lượng trung bình mà bếp điện cung cấp cho ấm nước trong mỗi giây:
			$$\calP=\dfrac{Q}{t}=\dfrac{\SI{1630.72}{\kilo\joule}}{35\cdot\SI{60}{\second}}\approx\SI{776.5}{\watt}.$$
		}
	
\end{vd}
	
% ========================================
\begin{vd}
\begin{enumerate}[label=\alph*)]
			\item Tính nhiệt lượng cần thiết để $\SI{2}{\kilogram}$ nước đá ở $\SI{-10}{\celsius}$ hoá hơi hoàn toàn ở nhiệt độ sôi, cho biết:
			\begin{itemize}
				\item nhiệt dung riêng của nước đá là $\SI{1800}{\joule/\left(\kilogram\cdot\kelvin\right)}$;
				\item nhiệt dung riêng của nước $\SI{4200}{\joule/\left(\kilogram\cdot\kelvin\right)}$;
				\item nhiệt nóng chảy riêng của nước đá là $\SI{34E4}{\joule/\kilogram}$;
				\item nhiệt hoá hơi riêng của nước là $\SI{23E5}{\joule/\kilogram}$.
			\end{itemize}
			\item Nếu dùng một bếp dầu hoả có hiệu suất $\SI{80}{\percent}$, người ta phải đốt cháy hoàn toàn bao nhiêu lít dầu để cho $\SI{2}{\kilogram}$ nước đá ở $\SI{-10}{\celsius}$ biến thành hơi.\\
			Cho biết:
			\begin{itemize}
				\item khối lượng riêng của dầu hoả là $\SI{800}{\kilogram/\meter^3}$;
				\item năng suất toả nhiệt của dầu hoả là $\SI{44E6}{\joule/\kilogram}$.
			\end{itemize}
		\end{enumerate}
\loigiai{\begin{enumerate}[label=\alph*)]
				\item Quá trình nước đá $\SI{-10}{\celsius}$ hoá hơi hoàn toàn ở nhiệt độ sôi trải qua 4 giai đoạn:
				\begin{itemize}
					\item Nước đá thu nhiệt để tăng nhiệt độ từ $\SI{-10}{\celsius}$ lên $\SI{0}{\celsius}$: 
					$$Q_1=mc_\text{đ}\left(0-t_0\right)=\left(\SI{2}{\kilogram}\right)\cdot\left[\SI{1800}{\joule/\left(\kilogram\cdot\kelvin\right)}\right]\cdot\left(\SI{10}{\celsius}\right)=\SI{36}{\kilo\joule}.$$
					\item Nước đá thu nhiệt để nóng chảy hoàn toàn ở $\SI{0}{\celsius}$:
					$$Q_2=m\lambda=\left(\SI{2}{\kilogram}\right)\cdot\left(\SI{34E4}{\joule/\kilogram}\right)=\SI{680}{\kilo\joule}.$$
					\item Nước thu nhiệt để tăng nhiệt độ từ $\SI{0}{\celsius}$ đến $\SI{100}{\celsius}$:
					$$Q_3=mc_\text{n}\left(\SI{100}{\celsius}-\SI{0}{\celsius}\right)=\left(\SI{2}{\kilogram}\right)\cdot\left[\SI{4200}{\joule/\left(\kilogram\cdot\kelvin\right)}\right]\cdot\left(\SI{100}{\celsius}\right)=\SI{840}{\kilo\joule}.$$
					\item Nước hoá hơi hoàn toàn ở $\SI{100}{\celsius}$:
					$$Q_4=mL=\left(\SI{2}{\kilogram}\right)\cdot\left(\SI{23E5}{\joule/\kilogram}\right)=\SI{4600}{\kilo\joule}.$$
				\end{itemize}
				Tổng nhiệt lượng đá cần thu vào để hoá hơi hoàn toàn ở $\SI{100}{\celsius}$:
				$$Q=Q_1+Q_2+Q_3+Q_4=\SI{6156}{\kilo\joule}.$$
				\item Nhiệt lượng bếp dầu cần cung cấp:
				$$Q_\text{tp}=\dfrac{Q}{H}=\dfrac{\SI{6156}{\kilo\joule}}{\SI{80}{\percent}}=\SI{7695}{\kilo\joule}.$$
				Khối lượng dầu cần đốt để tạo ra nhiệt lượng như trên:
				$$m=\dfrac{Q_\text{tp}}{q}=\dfrac{\SI{7695E3}{\joule}}{\SI{44E6}{\joule/\kilogram}}\approx\SI{0.175}{\kilogram}.$$
				Thể tích dầu cần đốt:
				$$V=\dfrac{m}{D}=\dfrac{\SI{0.175}{\kilogram}}{\SI{800}{\kilogram/\meter^3}}\approx\SI{2.19E-4}{\meter^3}=\SI{0.219}{\text{lít}}.$$
			\end{enumerate}
		}
	
\end{vd}
	\begin{dang}{Bài tập về động cơ nhiệt}
Động cơ nhiệt là thiết bị biến đổi nhiệt lượng thành công.\\
			\begin{itemize}
				\item \textbf{Nguyên tắc hoạt động của động cơ nhiệt}\\
				Tác nhân nhận nhiệt lượng $Q_1$ từ nguồn nóng biến một phần nhiệt lượng nhận được này thành công $A$ và toả phần nhiệt lượng $Q_2$ còn lại cho nguồn lạnh.
				\begin{center}
					\includegraphics[scale=0.25]{figs/VN12-Y24-PH-SYL-007-1}
				\end{center}
				\item \textbf{Hiệu suất của động cơ nhiệt}\\
				$$H=\dfrac{A}{Q_1}=\dfrac{Q_1-Q_2}{Q_1}=1-\dfrac{Q_2}{Q_1}.$$
\begin{luuy}
	Động cơ nhiệt không thể chuyển đổi toàn bộ nhiệt lượng nhận được thành công $\left(H<1\right)$.\\
					Hiệu suất cực đại của động cơ nhiệt:
					$$H_\text{max}=\dfrac{T_1-T_2}{T_1}.$$
\end{luuy}
			\end{itemize}
\end{dang}
% =======================================
\begin{vd}
Một động cơ nhiệt làm việc sau một thời gian thì tác nhân đã nhận từ nguồn nóng nhiệt lượng $Q_1=\SI{1.5E6}{\joule}$, truyền cho nguồn lạnh nhiệt lượng $Q_2=\SI{1.2E6}{\joule}$. Hiệu suất thực của động cơ nhiệt này là bao nhiêu?
	\loigiai{
			Hiệu suất thực của động cơ nhiệt:
			$$H=1-\dfrac{Q_2}{Q_1}=1-\dfrac{\SI{1.2E6}{\joule}}{\SI{1.5E6}{\joule}}=\SI{20}{\percent}.$$
		}
\end{vd}
% ===============================================
\begin{vd}
Máy hơi nước công suất $\SI{10}{\kilo\watt}$ tiêu thụ $\SI{10}{\kilogram}$ than đá trong $\SI{1}{\text{giờ}}$. Biết hơi nước vào và ra cylanh có nhiệt độ $\SI{227}{\celsius}$ và $\SI{100}{\celsius}$. Năng suất toả nhiệt của than đá là $\SI{3.6E7}{\joule/\kilogram}$. Tính hiệu suất thực của máy và của một động cơ nhiệt lí tưởng làm việc với nhiệt độ nguồn nóng và nguồn lạnh nói trên.
\loigiai{
			Hiệu suất của một động cơ nhiệt lí tưởng hoạt động giữa hai nguồn nhiệt $\SI{227}{\celsius}$ và $\SI{100}{\celsius}$:
			$$T_1=t_1+273=\SI{500}{\kelvin};\quad T_2=t_2+273=\SI{373}{\kelvin}.$$
			$$H_\text{max}=\dfrac{T_1-T_2}{T_1}=\SI{25.4}{\percent}.$$
			Hiệu suất thực của động cơ:
			$$H=\dfrac{A}{Q}=\dfrac{\calP t}{mq}=\dfrac{\left(\SI{10E3}{\watt}\right)\cdot\left(\SI{3600}{\second}\right)}{\left(\SI{10}{\kilogram}\right)\cdot\left(\SI{3.6E7}{\joule/\kilogram}\right)}=\SI{10}{\percent}.$$}

\end{vd}
\subsection{BÀI TẬP TRẮC NGHIỆM}
\Opensolutionfile{ans}[ans/G12Y24B6TN]
% ===================================================================
\begin{ex}
	Một động cơ nhiệt lí tưởng thực hiện một công $\SI{5}{\kilo\joule}$ đồng thời truyền cho nguồn lạnh nhiệt lượng $\SI{15}{\kilo\joule}$. Hiệu suất của động cơ nhiệt này là
	\choice
	{$\SI{33.33}{\percent}$}
	{$\SI{75}{\percent}$}
	{\True $\SI{25}{\percent}$}
	{$\SI{66.67}{\percent}$}
	\loigiai{$$H=\dfrac{A}{Q_1}=\dfrac{A}{Q_2+A}=\SI{25}{\percent}.$$
	}
\end{ex}
% ===================================================================
\begin{ex}
	Một động cơ nhiệt làm việc giữa hai nguồn nhiệt. Nhiệt lượng tác nhân nhận của nguồn nóng trong một chu trình là $\SI{2400}{\joule}$. Hiệu suất của động cơ nhiệt là $\SI{25}{\percent}$. Nhiệt lượng tác nhân truyền cho nguồn lạnh trong một chu trình là
	
	\choice
	{$\SI{1200}{\joule}$}
	{$\SI{2400}{\joule}$}
	{$\SI{600}{\joule}$}
	{\True $\SI{1800}{\joule}$}
	\loigiai{$$H=1-\dfrac{Q_2}{Q_1}\Rightarrow Q_2=\SI{1800}{\joule}.$$
	}
\end{ex}
% ===================================================================
\begin{ex}
	Một động cơ nhiệt nhả cho nguồn lạnh $\SI{80}{\percent}$ nhiệt lượng mà nó thu được từ nguồn nóng. Hiệu suất của động cơ nhiệt này là
	
	\choice
	{\True $\SI{20}{\percent}$}
	{$\SI{37}{\percent}$}
	{$\SI{50}{\percent}$}
	{$\SI{80}{\percent}$}
	\loigiai{Hiệu suất của động cơ nhiệt:
		$$H=1-\dfrac{Q_2}{Q_1}=\SI{20}{\percent}.$$
	}
\end{ex}
% ===================================================================
\begin{ex}
	Người ta phải tốn $\SI{150}{\gram}$ dầu hoả để đun sôi được $\SI{4.5}{\text{lít}}$ nước ở nhiệt độ ban đầu $\SI{20}{\celsius}$. Cho biết khối lượng riêng của nước là $\SI{1}{\kilogram/\text{lít}}$, nhiệt dung riêng của nước là $\SI{4200}{\joule/\left(\kilogram\cdot\kelvin\right)}$, năng suất toả nhiệt của dầu hoả là $\SI{44E6}{\joule/\kilogram}$. Hiệu suất của bếp đun là
	
	\choice
	{\True $\SI{22.9}{\percent}$}
	{$\SI{2.29}{\percent}$}
	{$\SI{12.9}{\percent}$}
	{$\SI{26.9}{\percent}$}
	\loigiai{Hiệu suất của bếp đun:
		$$H=\dfrac{m_\text{n}c\Delta t}{m_\text{d}q}=\SI{22.9}{\percent}.$$
	}
\end{ex}
% ===================================================================
\begin{ex}
	Để đun sôi một lượng nước bằng bếp dầu có hiệu suất $\SI{30}{\percent}$, phải dùng hết 1 lít dầu. Để đun sôi cũng lượng nước trên với bếp dầu có hiệu suất $\SI{20}{\percent}$ thì phải dùng
	
	\choice
	{2 lít dầu}
	{0,5 lít dầu}
	{\True 1,5 lít dầu}
	{3 lít dầu}
	\loigiai{$$V_2=\dfrac{V_1H_1}{H_2}=\SI{1.5}{\liter}.$$
	}
\end{ex}
% ===================================================================
\begin{ex}
	Khi dùng lò có hiệu suất $H_1$ để làm chảy một lượng quặng, phải đốt hết $\xsi{m_1}{\left(\kilogram\right)}$ nhiên liệu có năng suất toả nhiệt $q_1$. Nếu dùng lò có hiệu suất $H_2$ để làm chảy lượng quặng trên thì phải đốt hết $m_2=\xsi{3m_1}{\left(\kilogram\right)}$ nhiên liệu có năng suất toả nhiệt $q_2=0,5q_1$. Hệ thức liên hệ giữa $H_1$ và $H_2$ là
	
	\choice
	{$H_1=H_2$}
	{$H_1=2H_2$}
	{$H_1=3H_2$}
	{\True $H_1=1,5H_2$}
	\loigiai{$$H=\dfrac{Q}{mq}$$
		$$\Rightarrow \dfrac{H_1}{H_2}=\dfrac{m_2q_2}{m_1q_1}=\dfrac{3}{2}.$$}
\end{ex}
% ===================================================================
\begin{ex}
	Một động cơ nhiệt có hiệu suất $\SI{25}{\percent}$ và công suất $\SI{30}{\kilo\watt}$. Nhiệt lượng mà động cơ toả ra cho nguồn lạnh trong 5 giờ làm việc liên tục là
	
	\choice
	{$\SI{176E7}{\joule}$}
	{$\SI{194E7}{\joule}$}
	{$\SI{213E7}{\joule}$}
	{\True $\SI{162E7}{\joule}$}
	\loigiai{$$H=\dfrac{A}{A+Q_2}=\dfrac{\calP t}{\calP t+Q_2}\Rightarrow Q_2=\SI{162E7}{\joule}.$$}
\end{ex}
% ===================================================================
\begin{ex}
	Một đầu máy diezen xe lửa có công suất $\SI{3E6}{\watt}$ và có hiệu suất $\SI{25}{\percent}$. Cho biết năng suất toả nhiệt của nhiên liệu là $\SI{4.2E7}{\joule/\kilogram}$. Nếu đầu máy chạy hết công suất thì khối lượng nhiên liệu tiêu thụ trong mỗi giờ \textbf{gần giá trị nào nhất} sau đây?
	
	\choice
	{$\SI{2489}{\kilogram}$}
	{$\SI{1429}{\kilogram}$}
	{\True $\SI{1028}{\kilogram}$}
	{$\SI{1056}{\kilogram}$}
	\loigiai{Khối lượng nhiên liệu tiêu thụ trong 1 giờ khi động cơ xe lửa hoạt động hết công suất:
		$$m=\dfrac{\calP t}{qH}\approx\SI{1028.57}{\kilogram}.$$
	}
\end{ex}
% ===================================================================
\begin{ex}
	Một máy bơm sau khi tiêu thụ hết $\SI{8}{\kilogram}$ dầu thì đưa được $\SI{700}{\meter^3}$ nước lên cao $\SI{8}{\meter}$. Biết năng suất toả nhiệt của dầu dùng cho  máy bơm này là $\SI{4.6E7}{\joule/\kilogram}$. Xem rằng nước được đưa lên cao một cách đều đặn, khối lượng riêng của nước $\SI{1000}{\kilogram/\meter^3}$, gia tốc trọng trường $g=\SI{10}{\meter/\second^2}$. Hiệu suất của máy bơm là
	
	\choice
	{\True $\SI{15.22}{\percent}$}
	{$\SI{24.46}{\percent}$}
	{$\SI{1.52}{\percent}$}
	{$\SI{2.45}{\percent}$}
	\loigiai{Hiệu suất máy bơm:
		$$H=\dfrac{\rho Vgh}{qm_\text{d}}\approx\SI{15.22}{\percent}.$$
	}
\end{ex}
% ===================================================================
\begin{ex}
	Một ô tô chạy $\SI{100}{\kilo\meter}$ với lực kéo không đổi $\SI{700}{\newton}$ thì tiêu thụ hết $\SI{6}{\text{lít}}$ xăng. Biết năng suất toả nhiệt của xăng là $\SI{4.6E7}{\joule/\kilogram}$, khối lượng riêng của xăng là $\SI{700}{\kilogram/\meter^3}$. Hiệu suất của động cơ ô tô là
	
	\choice
	{$\SI{25}{\percent}$}
	{\True $\SI{36}{\percent}$}
	{$\SI{0.36}{\percent}$}
	{$\SI{17.75}{\percent}$}
	\loigiai{Hiệu suất của động cơ ô tô:
		$$H=\dfrac{A}{Q}=\dfrac{Fs}{\rho_\text{xăng}Vq}=\SI{36.23}{\percent}.$$
	}
\end{ex}
% ===================================================================
\begin{ex}
	Một nhà máy điện tiêu thụ $\SI{0.35}{\kilogram}$ nhiên liệu cho mỗi $\SI{1}{\kilo\watt\hour}$ điện năng. Cho biết năng suất toả nhiệt của nhiên liệu trên là $\SI{42}{\mega\joule/\kilogram}$. Hiệu suất của động cơ nhiệt dùng trong nhà máy điện \textbf{gần nhất} với giá trị nào sau đây?
	
	\choice
	{$\SI{38}{\percent}$}
	{$\SI{15}{\percent}$}
	{$\SI{20}{\percent}$}
	{\True $\SI{24}{\percent}$}
	\loigiai{Hiệu suất của động cơ nhiệt dùng trong nhà máy:
		$$H=\dfrac{A}{Q_\text{tp}}=\dfrac{\left(\SI{E3}{\watt}\right)\cdot\left(\SI{3600}{\second}\right)}{\left(\SI{0.35}{\kilogram}\right)\cdot\left(\SI{42E6}{\joule/\kilogram}\right)}\approx\SI{24.5}{\percent}.$$}
\end{ex}
% ===================================================================
\begin{ex}
	Một chiếc xe máy hoạt động với công suất $\SI{3.2}{\kilo\watt}$ và chuyển động đều với tốc độ $\SI{45}{\kilo\meter/\hour}$. Hiệu suất của động cơ là $\SI{25}{\percent}$, năng suất toả nhiệt của xăng là $\SI{4.6E7}{\joule/\kilogram}$, khối lượng riêng của xăng là $\SI{700}{\kilogram/\meter^3}$. Với 2 lít xăng thì xe máy đi được bao nhiêu $\si{\kilo\meter}$?
	
	\choice
	{$\SI{100.6}{\kilo\meter}$}
	{\True $\SI{63}{\kilo\meter}$}
	{$\SI{45}{\kilo\meter}$}
	{$\SI{54}{\kilo\meter}$}
	\loigiai{Thời gian động cơ xe máy hoạt động được khi tiêu thụ $\SI{2}{\text{lít}}$ xăng:
		$$t=\dfrac{q\rho V}{\calP}\cdot H=\SI{5031.25}{\second}=\SI{1.39}{\hour}.$$
		Quãng đường xe máy đi được:
		$$s=vt\approx\SI{62.89}{\kilo\meter}.$$
	}
\end{ex}
% ===================================================================
\begin{ex}
	Một động cơ ô tô hoạt động với công suất $\SI{20}{\kilo\watt}$ và ô tô chuyển động đều với tốc độ $\SI{72}{\kilo\meter/\hour}$. Ô tô tiêu thụ $\SI{20}{\text{lít}}$ xăng thì chạy được quãng đường $\SI{200}{\kilo\meter}$. Biết khối lượng riêng của xăng là $\SI{700}{\kilogram/\meter^3}$, năng suất toả nhiệt của xăng là $\SI{4.6E7}{\joule/\kilogram}$. Hiệu suất của động cơ ô tô khi đó là
	
	\choice
	{\True $\SI{31}{\percent}$}
	{$\SI{61}{\percent}$}
	{$\SI{63}{\percent}$}
	{$\SI{36}{\percent}$}
	\loigiai{Hiệu suất của ô tô:
		$$H=\dfrac{\calP\cdot\dfrac{s}{v}}{\rho Vq}\approx\SI{31.05}{\percent}.$$
	}
\end{ex}
% ===================================================================
\begin{ex}
	Dùng một bếp dầu hoả để đun sôi 2 lít nước từ $\SI{15}{\celsius}$ thì mất $\SI{10}{\text{phút}}$. Biết rằng chỉ có $\SI{40}{\percent}$ nhiệt lượng do dầu toả ra làm nóng nước. Lấy nhiệt dung riêng của nước là $\SI{4190}{\joule/\left(\kilogram\cdot\kelvin\right)}$, năng suất toả nhiệt của dầu hoả là $\SI{46E6}{\joule/\kilogram}$, khối lượng riêng của nước là $\SI{1}{\gram/\centi\meter^3}$. Lượng dầu hoả cần dùng trong mỗi phút là
	\choice
	{$\SI{0.619}{\gram}$}
	{$\SI{0.619}{\kilogram}$}
	{\True $\SI{3.87}{\gram}$}
	{$\SI{3.87}{\kilogram}$}
	\loigiai{Khối lượng dầu cần dùng trong mỗi phút:
		$$m=\dfrac{Q}{qH}=\dfrac{mc\Delta t}{qH}=\SI{3.87E-3}{\kilogram}=\SI{3.87}{\gram}.$$
	}
\end{ex}
% ===================================================================
\begin{ex}
	Người ta cần nấu chảy 10 tấn đồng trong lò nung dùng dầu làm nhiên liệu đốt. Cho biết nhiệt độ ban đầu, nhiệt độ nóng chảy, nhiệt dung riêng và nhiệt nóng chảy riêng của đồng lần lượt là $\SI{13}{\celsius}$, $\SI{1083}{\celsius}$, $\SI{380}{\joule/\left(\kilogram\cdot\kelvin\right)}$, $\SI{1.8E5}{\joule/\kilogram}$. Nhiệt lượng toả ra khi đốt cháy $\SI{1}{\kilogram}$ dầu là $\SI{4.6E7}{\joule/\kilogram}$. Nếu hiệu suất nung của lò là $\SI{30}{\percent}$ thì khối lượng dầu cần dùng là
	\choice
	{\True $\SI{425.1}{\kilogram}$}
	{$\SI{127.5}{\kilogram}$}
	{$\SI{38.3}{\kilogram}$}
	{$\SI{432.2}{\kilogram}$}
	\loigiai{Nhiệt lượng đồng cần thu vào để nóng chảy hoàn toàn:
		$$Q=mc\Delta t+m\lambda=\SI{58.66E8}{\joule}.$$
		Khối lượng xăng cần dùng:
		$$m=\dfrac{Q}{qH}=\SI{425.1}{\kilogram}.$$
	}
\end{ex}
% ===================================================================
\begin{ex}
Một ấm nhôm có khối lượng $m_\text{b}=\SI{600}{\gram}$ chứa $V=\SI{1.5}{\text{lít}}$ nước ở $t_1=\SI{20}{\celsius}$, sau đó đun bằng bếp điện. Sau thời gian $t=\SI{35}{\text{phút}}$ thì đã có $\SI{20}{\percent}$ khối lượng nước đã hoá hơi ở nhiệt độ sôi $t_2=\SI{100}{\percent}$. Biết rằng, $\SI{75}{\percent}$ nhiệt lượng mà bếp cung cấp được dùng vào việc đun nước. Cho biết nhiệt dung riêng của nước là $c_\text{n}=\SI{4190}{\joule/\left(\kilogram\cdot\kelvin\right)}$, của nhôm là $c_\text{b}=\SI{880}{\joule/\left(\kilogram\cdot\kelvin\right)}$, nhiệt hoá hơi riêng của nước ở $\SI{100}{\celsius}$ là $L=\SI{2.26E6}{\joule/\kilogram}$, khối lượng riêng của nước là $D=\SI{1}{\kilogram/\text{lít}}$. Công suất cung cấp nhiệt của bếp điện \textbf{gần giá trị nào nhất} sau đây?	
	\choice
	{\True $\SI{776}{\watt}$}
	{$\SI{796}{\watt}$}
	{$\SI{786}{\watt}$}
	{$\SI{876}{\watt}$}
	\loigiai{Khối lượng nước $$m_\text{n}=VD=\SI{1.5}{\kilogram}.$$
		Tổng nhiệt lượng nước và ấm thu vào:
		$$Q=\left(m_\text{n}c_\text{n}+m_\text{b}c_\text{b}\right)\cdot\left(t_2-t_1\right)+\SI{20}{\percent}m_\text{n}L=\SI{1223040}{\joule}.$$
		Điện năng tiêu thụ của ấm:
		$$A=\dfrac{Q}{H}=\SI{1630720}{\joule}.$$
		Công suất của ấm điện:
		$$\calP=\dfrac{A}{t}\approx\SI{776.5}{\watt}.$$	}
\end{ex}
\Closesolutionfile{ans}
	\subsection{TRẮC NGHIỆM ĐÚNG/SAI}
	\setcounter{ex}{0}
	% ================================
	\begin{ex}
		\immini{
	Cầu chì là linh kiện được sử dụng để bảo vệ thiết bị và lưới điện tránh sự cố ngắn mạch, hạn chế tình trạng cháy, nổ.
	\begin{enumerate}[label=\alph*)]
		\item Cầu chì có thể bảo vệ mạch điện dựa trên sự phụ thuộc của điện trở kim loại theo nhiệt độ.
		\item Dây chảy trong cầu chì thường được làm từ kim loại có nhiệt độ nóng chảy cao.
		\item Khi cường độ dòng điện qua mạch tăng vượt hạn, dây chì sẽ nóng chảy trước.
		\item Khi dây chảy trong cầu chì bị đứt, ta có thể nối cầu chì bằng dây sắt.
	\end{enumerate}	
	}{
\includegraphics[scale=0.2]{figs/VN12-Y24-PH-SYL-007P-1}
}
\loigiai{
\begin{enumerate}[label=\alph*)]
	\item Sai. Khi dòng điện qua mạch vượt hạn, dây chảy trong cầu chì nóng chảy trước và là hở mạch.
	\item Sai. Dây chảy được làm từ kim loại có nhiệt độ nóng chảy thấp.
	\item Đúng.
	\item Sai. Sắt là kim loại có nhiệt độ nóng chảy cao, khi cường độ dòng điện tăng quá lớn nhưng dây sắt nóng chảy chậm nên không bảo vệ được mạch điện.
\end{enumerate}
}
		\end{ex}
	\subsection{BÀI TẬP TỰ LUẬN}
	\setcounter{ex}{0}
% ===================================================================
\begin{ex}
	Một bếp dầu đun sôi 1 lít nước đựng trong ấm bằng nhôm khối lượng $m_2=\SI{300}{\gram}$ thì sau thời gian $t_1=\SI{10}{\minute}$ nước sôi. Nếu dùng bếp trên để đun 2 lít nước trong cùng điều kiện thì sau bao lâu nước sôi? Cho nhiệt dung riêng của nước và nhôm lần lượt là $c_1=\SI{4200}{\joule/\left(\kilogram\cdot\kelvin\right)}$, $c_2=\SI{880}{\joule/\left(\kilogram\cdot\kelvin\right)}$. Biết nhiệt do bếp dầu cung cấp một cách đều đặn.
	\loigiai{Vì bếp dầu cung cấp nhiệt lượng một cách đều đặn nên
		$$\dfrac{t_2}{t_1}=\dfrac{\left(m_2c_2+m'_1c_1\right)\Delta t}{\left(m_2c_2+m_1c_1\right)\Delta t}=\dfrac{m_2c_2+m'_1c_1}{m_2c_2+m_1c_1}\Rightarrow t_2\approx\SI{19.4}{\minute}.$$}
\end{ex}
% ===================================================================
\begin{ex}
	Khi thả một quả cầu nhôm khối lượng $\SI{500}{\gram}$ vào $\SI{2}{\kilogram}$ nước ở $\SI{25}{\celsius}$ thì nhiệt độ của chúng sau khi cân bằng nhiệt là $\SI{30}{\celsius}$. Hỏi nhiệt độ ban đầu của quả cầu nhôm là bao nhiêu? Biết nhiệt lượng hao phí trong trường hợp này bằng $\SI{20}{\percent}$ nhiệt lượng do nước thu vào. Biết nhiệt dung riêng của nhôm là $\SI{880}{\joule/\left(\kilogram\cdot\kelvin\right)}$, nhiệt dung riêng của nước là $\SI{4200}{\joule/\left(\kilogram\cdot\kelvin\right)}$. Bỏ qua sự hoá hơi của nước ngay khi tiếp xúc với quả cầu.
	
	\loigiai{Khi hệ cân bằng nhiệt, tổng nhiệt lượng trao đổi trong hệ bằng 0:
		\begin{eqnarray*}
			&&Q_1+Q_2+Q_\text{hp}=0\\
			&\Leftrightarrow& m_1C_1\left(t_\text{cb}-t_1\right)+1,2m_2C_2\left(t_\text{cb}-t_2\right)=0\\
			&\Rightarrow& t_1\approx\SI{144.55}{\celsius}.
	\end{eqnarray*}}
\end{ex}
% ===================================================================
\begin{ex}
	Động cơ nhiệt lí tưởng làm việc giữa hai nguồn nhiệt $\SI{27}{\celsius}$ và $\SI{127}{\celsius}$. Nhiệt lượng tác nhân nhận từ nguồn nóng trong một chu trình là $\SI{2400}{\joule}$. Tính:
	\begin{enumerate}[label=\alph*)]
		\item hiệu suất của động cơ.
		\item công thực hiện trong một chu trình.
		\item nhiệt lượng truyền cho nguồn lạnh trong một chu trình.
	\end{enumerate}
	
	\loigiai{\begin{enumerate}[label=\alph*)]
			\item $H=\dfrac{T_1-T_2}{T_1}=\SI{25}{\percent}.$
			\item $A=H\cdot Q_1=\SI{600}{\joule}$.
			\item $Q_2=Q_1-A=\SI{1800}{\joule}$.
		\end{enumerate}
	}
\end{ex}
% ===================================================================
\begin{ex}
	Máy hơi nước công suất $\SI{1}{\kilo\watt}$ tiêu thụ $\SI{10}{\kilogram}$ than đá trong 1 giờ. Biết hơi nước vào và ra cylanh có nhiệt độ $\SI{227}{\celsius}$ và $\SI{100}{\celsius}$. Năng suất toả nhiệt của than đá là $\SI{3.6E7}{\joule/\kilogram}$. Tính hiệu suất thực của máy và của một động cơ nhiệt lý tưởng làm việc giữa hai nhiệt độ nói trên.
	
	\loigiai{Hiệu suất của máy:
		$$H=\dfrac{A}{Q}=\dfrac{\calP t}{mq}=\SI{10}{\percent}.$$
		Hiệu suất của động cơ nhiệt lí tưởng:
		$$H_\text{max}=\dfrac{T_1-T_2}{T_1}=\SI{25.4}{\percent}.$$
	}
\end{ex}
% ===================================================================
\begin{ex}
	Một động cơ hơi nước lí tưởng là động cơ nhiệt có hiệu suất cực đại, hoạt động với nguồn nóng là lò hơi có nhiệt độ $\SI{500}{\kelvin}$. Nước được đưa vào lò hơi và được đun nóng để chuyển thể thành hơi nước. Hơi nước này làm piston chuyển động. Nhiệt độ của nguồn lạnh là nhiệt độ bên ngoài của không khí, bằng $\SI{300}{\kelvin}$.
	\begin{enumerate}[label=\alph*)]
		\item Tính công của động cơ hơi nước thực hiện khi lò hơi cung cấp cho tác nhân một nhiệt lượng bằng $\SI{6.5E3}{\joule}$.
		\item Giả sử muốn tăng hiệu suất này lên $\SI{45}{\percent}$ phải tăng nhiệt độ lò hơi lên một lượng bằng bao nhiêu?
	\end{enumerate}
	
	\loigiai{\begin{enumerate}[label=\alph*)]
			\item $H=\dfrac{A}{Q_1}=\dfrac{T_1-T_2}{T_1}\Rightarrow A=\SI{2600}{\joule}$.
			\item $H_\text{max}=\dfrac{T_1-T_2}{T_1}\Rightarrow T_1\approx\SI{545.46}{\kelvin}.$
		\end{enumerate}
	}
\end{ex}
% ===================================================================
\begin{ex}
	Búa máy 10 tấn rơi từ độ cao $\SI{2.3}{\meter}$ xuống một cọc sắt khối lượng $\SI{200}
	{\kilogram}$. Biết $\SI{40}{\percent}$ động năng của búa biến thành nhiệt làm nóng cọc sắt. Hỏi búa rơi bao nhiêu lần thì cọc tăng nhiệt độ thêm $\SI{20}{\celsius}$. Cho rằng cọc không toả nhiệt ra môi trường và nhiệt dung riêng của sắt là $\SI{0.46}{\kilo\joule/\left(\kilogram\cdot\kelvin\right)}$.
	\loigiai{\begin{itemize}
			\item Động năng của búa ngay trước khi va chạm với cọc: $W_\text{đ}=Mgh$.
			\item Nhiệt lượng cọc thu được sau mỗi lần búa rơi: $Q_0=0,4W_\text{đ}=0,4Mgh$.
			\item Nhiệt lượng cọc thu được sau $n$ lần búa rơi: $Q=nQ_0=mc\Delta t$.
		\end{itemize}
		Suy ra:
		$$n=\dfrac{mc\Delta t}{0,4Mgh}=20.$$
	}
\end{ex}

